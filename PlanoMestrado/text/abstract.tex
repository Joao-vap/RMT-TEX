The theory of Random Matrices finds its way in a diverse range of areas of Mathematics and Physics, and in particular to statistical mechanics. In the study of the spectra of eigenvalues of some random matrices ensembles, a physical analogy regarding thermodynamical systems becomes evident and the use of physical motivations in its study is of great importance. An example of such motivation is the study of the partition function, which holds in thermodynamics a great deal of information on the system at hand. Recent developments have been intensely unraveling new critical phenomena in random matrix theory, through a deep understanding of asymptotic expansions of the associated partition functions in the thermodynamic limit. The main goal of this work is to understand these developments in the context of normal random matrices, their relevance and implications in the theory of random matrices and beyond.
