The study of Random Matrices demonstrates applicability in a diverse range of areas in mathematics and Physics, emphasizing the study of statistical mechanics. In the study of the spectra of eigenvalue on some random matrices ensembles, an physical analogy regarding thermodynamical systems becomes evident and the use of physical motivations in its study of great importance. An example of such motivation is the study of the partition function, witch holds in thermodynamics a great deal of information on the system as hand and its properties. Recent developments have been intensely changing what is know on the thermodynamical limit asymptotic of these partition functions, especially for systems that regard ensembles of matrices of interest in modern research. The main goal of this work is to understand this developments and its implication and importance in the theory of random matrices.
