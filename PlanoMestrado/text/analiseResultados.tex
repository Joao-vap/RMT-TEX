Espera-se que com a obtenção de resultados, estes possam ser apresentados em seminários para alunos da graduação e em congressos voltados à iniciação científica. Os produtos computacionais, nominalmente as simulações realizadas para os gases, serão disponibilizados no \textit{GitHub} com licença livre de uso e distribuição. 

Por último, o projeto é também de desenvolvimento do aluno em um âmbito maior. Praticando desde a revisão de literatura à reprodução e geração de resultados para um problema atual e relevante. Além disso, da prática do uso e escrita da linguagem científica e dos métodos de produção do conhecimento, típicos do nicho trabalhado, e gerais da matemática e ciências.