O estudo de Matrizes Aleatórias demonstra aplicabilidade em uma gama diversa de áreas na matemática e na física, com destaque no estudo de mecânica estatística. No estudo da medida do espectro de autovalores de alguns ensembles de matrizes, analogias físicas à sistemas termodinâmicos se tornam evidentes e algumas motivações físicas podem ser tomadas. Um exemplo disso é a descrição da função partição, que em termodinâmica codifica grande detalhe sobre o sistema e suas propriedades. Desenvolvimentos recentes tem progredido intensamente na compreensão da expansão assintótica, no limite termodinâmico, da função partição de alguns sistemas termodinâmicos compatíveis com ensembles de interesse. O objetivo deste trabalho é estudar tais desenvolvimentos e entender suas implicações e importância dentro da teoria de matrizes aleatórias.