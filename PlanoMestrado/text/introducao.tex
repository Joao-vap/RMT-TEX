A random matrix is a matrix in witch the entries are random variables, not necessarily independent nor equally distributed. The algebraic-geometric properties in a matrix, such as its natural multiplication and spectral decomposition, makes this representation especially useful. It is also important to remember that many complex systems use a matrix representation. Correlation matrices and operators, especially in physics, are big reasons why this representation is important. Studying these matrices we can deduce properties from the system of interest, either that being the eigenvalues and eigenvectors of operators describing an atomic nuclei \cite{Dyson} or the description of market shares in highly correlated systems \cite[Chapter~2]{fabozziquantitative}. Either way, using an random matrix approach is relevant by the same reason random variables proved themselves crucial: It possibilitates an statistical description of phenomena and systems. That is the purpose of studying Random Matrix Theory (RMT).

The possible random matrices are subdivided by what we call ensembles. Such as in physics, where an ensemble is defined by its possible microstates, in RMT an ensemble is defined by a set of matrices that share some macroscopical properties. We can divide these ensembles in two main characterization, invariance by rotation and independence of entries. We will focus mainly on ensembles we define rationally invariant, that is, for any $\matriz{M}$ and $\matriz{M'}$ have same probability if they have the same eigenvalues. For each ensemble we can associate a measure to its realizations. For example, the intersection between ensembles invariant by rotation and with independent entries sits uniquely the Gaussian Ensembles. For these ensembles we can write the joint probability density function
\begin{equation}
		\p(\mmany{\lambda}{N}) = \frac{1}{Z_{N, \beta}} \ee^{-\beta_N \mathcal{H}_N(\vec{\lambda})},
	\label{Equation: Gaussian measure}
\end{equation}
where $Z_{N, \beta}$ is the canonical partition function, such that $\p$ is a measure. The factor $\beta_N = \beta N^2$ is though as the inverse temperature and the Hamiltonian $\mathcal{H}_N$ is expressed $$\mathcal{H}_N(\vec{\lambda}) = \frac{1}{N}\sum_{i = 1}^{N} \frac{\lambda_i^2}{2} + \frac{1}{N^2} \sum_{i < j} \log{\frac{1}{|\lambda_i - \lambda_j|}}, \ \ \  \lambda_i \mapsto \lambda_i \sqrt{\beta N}.$$ More generally there ia a natural extension of this measure for other invariant ensembles where the Hamiltonian is expressed $$\mathcal{H}_N(\vec{\lambda}) = \frac{1}{N}\sum_{i = 1}^{N} V(\lambda_i) + \frac{1}{N^2} \sum_{i < j} \log{\frac{1}{|\lambda_i - \lambda_j|}}, \ \ \  V(\lambda_i) \mapsto \beta NV(\lambda_i).$$

Considering such a measure is natural to make an analogy to the well know Coulomb Gas. Under the right conditions, the Coulomb Gas is the Gibbs-Boltzmann probability measuregiven in $(R^d)^N$. This measure $\p_N$ models an interacting gas of charged particles, at $\mmany{x}{N} \in \Se$ of dimension $d$ in $\R^n$ ambient space, under the influence of an external potential. Its measure its given by 
\begin{equation}
	\p_N(\mmany{x}{N}) = \frac{e^{-\beta N^2 \Hf_N(\mmany{x}{N})}}{Z_{N,\beta}},
	\label{Equação: Medida Gas de Coulomb}
\end{equation}
where $$\Hf_N(\vec{x}) = \frac{1}{N} \sum_{i = 1}^{N} \V(x) + \frac{1}{2N^2} \sum_{i \neq j} \g(x_i - x_j)$$ is the Hamiltonian or energy of the system and $\g(x_i - x_j)$ is the Coulomb Kernel of interaction. This analogy indicates some possible approaches for the study of such systems, for now we can use thermodynamical arguments to describe the configurations of eigenvalues of random matrices. With this in mind we turn to the partition function. In the beginning of its book, Feynmann states \cite{feynmanstatistical} that the key principle of the statistical mechanics is that a system states with energy $E$, in equilibrium, has probability given by a function $\frac{1}{Z} \ee^{\frac{-E}{kT}}$ where $Z$ is the partition function. In a more general sense there is a a relation between the partition function and the free energy of the system, that, in itself, relates to the entropy. Knowing well the partition function is a way to describe with great detail the macroproperties of such a system.

There is a great effort in the community of random matrix theory to study expansions of the partition function of systems such of Coulomb Gases. Recently, some advance has been made in the works of Sug-Soo Byun et al. \cite{Byun_2023}. They derived large-N expansions up to the $O(1)$-terms for both $Z_N$, related to the Coulomb Gas under complex and radially symmetric potential and $\tilde{Z}_N$, for its counterpart in the upper-half plane. They have also notice the expansion dependence on whether the limiting spectrum is an annulus or a disc, witch seems to not have been properly considered in the previous work on the matter. These recent developments are a major step in a continuous effort of more than 20 years in the field and can have many consequences. 

As was said, the study of the partition function is a major way to describe thermodynamical systems such that of a Coulomb Gas. With these new developments it is expected that many systems could be better understood and described by the asymptotic given. In that way this poses a great opportunity for development in the field. More than that, by the great connection that the field of random matrices holds with many other fields is possible that by the study of this work some suggestions on the behavior of many other related problems become plausible.





