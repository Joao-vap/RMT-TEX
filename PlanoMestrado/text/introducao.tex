A random matrix is a matrix in which the entries are random variables, not necessarily independent nor equally distributed. The algebraic-geometric properties in a matrix, such as its natural multiplication and spectral decomposition, makes this representation especially useful. Naturally, many complex systems use a matrix representation. Correlation matrices and operators, especially in physics, are some of many major reasons why this representation is important. Studying these matrices we can deduce properties from the system of interest, either that being the eigenvalues and eigenvectors of operators describing an atomic nuclei \cite{Dyson} or the description of market shares in highly correlated systems \cite[Chapter~2]{fabozziquantitative}. Either way, using an random matrix approach is relevant by the same reason random variables proved themselves crucial: it unlocks relevant statistical description of phenomena and systems.

In the research proposed here, we will focus on the study of {\it normal random matrices}, which consists of the space of normal matrices, equipped with a proper probability distribution. In our case, such probability distribution is given by a Gibbs-type measure, which we also impose to be invariant under the action of the unitary group. All in all, such invariance imposes a trivialization of the eigenvectors, which turn out to be simply Haar-distributed on the unitary group. In turn, all the relevant information on the matrix model is contained in their eigenvalues, and their joint probability density function is explicitly given by
%
\begin{equation}
		\p(\mmany{\lambda}{N}) = \frac{1}{Z_{N, \beta}} \ee^{-\beta_N \mathcal{H}_N(\vec{\lambda})},\quad (\lambda_1,\hdots,\lambda_N)\in \mathbb C^N
	\label{Equation: Gaussian measure}
\end{equation}
where $Z_{N, \beta}$ is the canonical partition function, such that $\p$ is a probability density with respect to the Lebesgue measure on $\mathbb C^N$. The factor $\beta_N = \beta N^2$ is though of as the inverse temperature and the factor $\mathcal H_N$, called the Hamiltonian, is given by
%
$$\mathcal{H}_N(\vec{\lambda}) = \frac{1}{N}\sum_{i = 1}^{N} V(\lambda_i) + \frac{1}{N^2} \sum_{i < j} \log{\frac{1}{|\lambda_i - \lambda_j|}}, \quad (\lambda_1,\hdots,\lambda_N)\in \mathbb C^N$$
%
where $V:\mathbb C\to \mathbb C$ is a suitably regular function, acting as a confining potential on the eigenvalues $\lambda_1,\hdots, \lambda_N$. 

Considering such a measure is natural to make an analogy to the well know Coulomb Gas. Under appropriate conditions, the Coulomb Gas is a Gibbs-Boltzmann probability measure given in $(\mathbb R^d)^N$. This measure $\p_N$ models an interacting gas of charged particles, at $\mmany{x}{N} \in \Se$ of dimension $d$ in $\R^n$ ambient space, under the influence of an external potential. Its measure its given by 
\begin{equation}
	\p_N(\mmany{x}{N}) = \frac{e^{-\beta N^2 \Hf_N(\mmany{x}{N})}}{Z_{N,\beta}},
	\label{Equação: Medida Gas de Coulomb}
\end{equation}
where $$\Hf_N(\vec{x}) = \frac{1}{N} \sum_{i = 1}^{N} \V(x) + \frac{1}{2N^2} \sum_{i \neq j} \g(x_i - x_j)$$ is the Hamiltonian or energy of the system and $\g(x_i - x_j)$ is the Coulomb Kernel of interaction. In this analogy, eigenvalues $\lambda_1,\hdots, \lambda_N$ may be seen as particles under the influence of the Gibbs law, and thermodynamical arguments may be employed to describe the configurations of eigenvalues of random matrices. With this in mind we turn to the partition function. In the beginning of its book, Feynman states \cite{feynmanstatistical} that the key principle of the statistical mechanics is that the probability that a system attains a state with energy $E$, in equilibrium, is given by a function $\frac{1}{Z} \ee^{\frac{-E}{kT}}$ where $Z$ is the partition function. In a more general sense there is a relation between the partition function and the free energy of the system, that, in itself, relates to the entropy. Knowing well the partition function is a way to describe with great detail the macro-properties of such a system.

There is a great effort in the community of random matrix theory to study expansions of the partition function of systems such of Coulomb Gases. Recently, some advance has been made in the works of Sug-Soo Byun et al. \cite{Byun_2023}. They derived large-$N$ asymptotic expansions up to the $O(1)$-terms for  $Z_{N,\beta}$ when $V$ is a radially symmetric potential, in the form
%
$$
Z_{N,\beta=2}=-I^VN^2+\frac{1}{2}N\log N+E^V N+G\log N+F^V + o(1),\quad N\to \infty.
$$
%
In such expansion, for a large class of potentials $V$ the first term $I^Q$ was known for a long time, being given by the energy of the associated equilibrium measure. The coefficient $1/2$ of the order $N\log N$, as well as the entropy term $E^V$, were known from recent work of Leblé and Serfaty \cite{leblé2017large}, also for rather large classes of $V$. The remaining terms in this expansion are new, and they are computed exploring the radially symmetry imposed on the potential. Strikingly, the term $G$ appears to be universal in $V$, depending solely on the connectivity of the limiting spectrum of eigenvalues. More precisely, they noticed that $G$ depends on whether the limiting spectrum is an annulus or a disc which, surprisingly, seems to not have been observed in the vast previous literature on the matter. 

As was said, the study of the partition function is a major way to describe thermodynamical systems such that of a Coulomb Gas. With these new developments it is expected that many systems could be better understood and described by the asymptotic given. In that way this poses a great opportunity for development in the field. More than that, by the great connection that the field of random matrices holds with many other fields is possible that by the study of this work some suggestions on the behavior of many other related problems become plausible.
