%% main.tex <- USPSC-TCC-modelo.tex
% ---------------------------------------------------------------
% USPSC: Modelo de Trabalho Academico (tese de doutorado, dissertacao de
% mestrado e trabalhos monograficos em geral) em conformidade com 
% ABNT NBR 14724:2011: Informacao e documentacao - Trabalhos academicos -
% Apresentacao
%----------------------------------------------------------------
%% Esta é uma customização do abntex2-modelo-trabalho-academico.tex de v-1.9.5 laurocesar 
%% para as Unidades do Campus USP de São Carlos:
%% EESC - Escola de Engenharia de São Carlos
%% IAU - Instituto de Arquitetura e Urbanismo
%% ICMC - Instituto de Ciências Matemáticas e de Computação
%% IFSC - Instituto de Física de São Carlos
%% IQSC - Instituto de Química de São Carlos
%%
%% Este trabalho utiliza a classe USPSC.cls que é mantida pela seguinte equipe:
%% 
%% Coordenação e Programação:
%%   - Marilza Aparecida Rodrigues Tognetti - marilza@sc.usp.br (PUSP-SC)
%%   - Ana Paula Aparecida Calabrez - aninha@sc.usp.br (PUSP-SC)
%% Normalização:
%%   - Brianda de Oliveira Ordonho Sigolo - brianda@usp.br (IAU)
%%   - Eduardo Graziosi Silva - edu.gs@sc.usp.br (EESC)
%%   - Eliana de Cássia Aquareli Cordeiro - eliana@iqsc.usp.br (IQSC)
%%   - Flávia Helena Cassin - cassinp@sc.usp.br (EESC)
%%   - Maria Cristina Cavarette Dziabas - mcdziaba@ifsc.usp.br (IFSC)
%%   - Regina Célia Vidal Medeiros - rcvmat@icmc.usp.br (ICMC)
%%
%% O USPSC-modelo.tex e USPSC-TCC-modelo.tex utilizam diversos arquivos relacionado em 
%% 2.1 Pacote USPSC: Classe USPSC e modelos de trabalhos acadêmicos	do Tutorial do Pascote 
%%  USPSC para modelos de trabalhos de acadêmicos em LaTeX - versão 3.1


%----------------------------------------------------------------
%% Sobre a classe abntex2.cls:
%% abntex2.cls, v-1.9.5 laurocesar
%% Copyright 2012-2015 by abnTeX2 group at https://www.abntex.net.br/ 
%%
%----------------------------------------------------------------

\documentclass[
% -- opções da classe memoir --
12pt,		% tamanho da fonte
openright,	% capítulos começam em pág ímpar (insere página vazia caso preciso)
%twoside,  % para impressão em anverso (frente) e verso. Oposto a oneside
oneside, % para impressão em páginas separadas (somente anverso)
a4paper,			% tamanho do papel. 
% -- opções da classe abntex2 --
chapter=TITLE,		% títulos de capítulos convertidos em letras maiúsculas
% -- opções do pacote babel --
english,			% idioma adicional para hifenização
french,				% idioma adicional para hifenização
spanish,			% idioma adicional para hifenização
brazil				% o último idioma é o principal do documento
% {USPSC-classe/USPSC} configura o cabeçalho contendo apenas o número da página
]{USPSC-classe/USPSC}
%]{USPSC-classe/USPSC1}
% Inclua % antes de ]{USPSC-classe/USPSC} e retire a % antes de %]{USPSC-classe/USPSC1} para utilizar o 
% cabeçalho diferenciado para as páginas pares e ímpares:
%- páginas ímpares: com seções ou subseções e o número da página
%- páginas pares: com o número da página e o título do capítulo 
% ---
% ---
% Pacotes básicos - Fundamentais 
% ---
\usepackage[T1]{fontenc}		% Seleção de códigos de fonte.
\usepackage[utf8]{inputenc}		% Codificação do documento (conversão automática dos acentos)
\usepackage{lmodern}			% Usa a fonte Latin Modern
% Para utilizar a fonte Times New Roman, inclua uma % no início do comando acima  "\usepackage{lmodern}"
% Abaixo, tire a % antes do comando  \usepackage{times}
%\usepackage{times}		    	% Usa a fonte Times New Roman	
% Para usar a fonte , lembre-se de tirar a % do comando %\renewcommand{\ABNTEXchapterfont}{\rmfamily}, localizado mais abaixo, logo após "Outras opções para nota de rodapé no Sistema Numérico" 					
\usepackage{lastpage}			% Usado pela Ficha catalográfica
\usepackage{indentfirst}		% Indenta o primeiro parágrafo de cada seção.
\usepackage{color}				% Controle das cores
\usepackage{graphicx}			% Inclusão de gráficos
\usepackage{float} 				% Fixa tabelas e figuras no local exato
\usepackage{lipsum}  
\usepackage{tikz}				% Para escrever reações químicas e outros
%\usetikzlibrary{positioning}
\usetikzlibrary{shapes,positioning,calc,quotes}
\tikzstyle{subrotina} = [rectangle, draw,text centered, minimum width=5em, inner sep=5pt]

\tikzstyle{funcao} = [rounded rectangle, draw, text centered, minimum width=7em, inner sep=5pt]

\tikzstyle{random} = [rectangle, text centered, inner sep=5pt, minimum width=5em]

\tikzstyle{loop} = [rectangle, draw, align=left]

\tikzstyle{fluxo} = [draw, thick, -latex]

\tikzstyle{meiofluxo} = [draw, -]

\tikzstyle{chamada} = [draw, dashed, <->]
%\tikzset{
%	basic box/.style={
%		shape=rectangle, rounded corners, align=center, draw=#1, fill=#1!25},
%	header node/.style={
%		node family/width=header nodes,
%		font=\strut\Large\ttfamily,
%		text depth=+.3ex, fill=white, draw},
%	header/.style={%
%		inner ysep=+1.5em,
%		append after command={
%			\pgfextra{\let\TikZlastnode\tikzlastnode}
%			node [header node] (header-\TikZlastnode) at (\TikZlastnode.north) {#1}
%			% the next node contains both \tikzlastnode and its header
%			% this is needed so that h- can be used to connect lines
%			node [span=(\TikZlastnode)(header-\TikZlastnode)]
%			at (fit bounding box) (h-\TikZlastnode) {}
%		}
%	},
%	fat blue line/.style={ultra thick, blue}
%}
\usepackage{microtype} 			% para melhorias de justificação
\usepackage{pdfpages}
\usepackage{makeidx}            % para gerar índice remissivo
\usepackage{hyphenat}          % Pacote para retirar a hifenizacao do texto
\usepackage[absolute]{textpos} % Pacote permite o posicionamento do texto
\usepackage{eso-pic}           % Pacote para incluir imagem de fundo
\usepackage{makebox}  
\usepackage{amsmath}
\usepackage[superscript]{cite} % agrupa citações numéricas consecutivas
\usepackage[num, abnt-emphasize=bf, abnt-thesis-year=both, abnt-repeated-author-omit=no, abnt-last-names=abnt, abnt-etal-cite, abnt-etal-list=3, abnt-etal-text=it, abnt-and-type=e, abnt-doi=doi, abnt-url-package=none, abnt-verbatim-entry=no]{abntex2cite} 
\bibliographystyle{USPSC-classe/abntex2-num-USPSC}

\renewcommand{\thefootnote}{\fnsymbol{footnote}}  %Comando para inserção de símbolos em nota de rodapé

\renewcommand{\footnotesize}{\small} %Comando para diminuir a fonte das notas de rodapé

% ---
% Pacotes adicionais, usados apenas no âmbito do Modelo Canônico do abnteX2
% ---
\usepackage{lipsum}				% para geração de dummy text
% ---

% pacotes de tabelas
\usepackage{multicol}	% Suporte a mesclagens em colunas
\usepackage{multirow}	% Suporte a mesclagens em linhas
\usepackage{longtable}	% Tabelas com várias páginas
\usepackage{threeparttablex}    % notas no longtable
\usepackage{array}

\usepackage{physics}
\usepackage{bbold}
\usepackage{subcaption}

% ----
% Compatibilização com a ABNT NBR 6023:2018
% Para tirar <> da URL
%\DeclareFieldFormat{url}{\bibstring{urlfrom}\addcolon\addspace\url{#1}}
\usepackage{USPSC-classe/ABNT6023-2018}
% As demais compatibilizações estão nos arquivos abntex2-alf-USPSC.bst,abntex2-alfeng-USPSC.bst, abntex2-num-USPSC.bst e abntex2-numeng-USPSC.bst, dependendo do idioma do textos e se o sistemas de chamada for autor-data ou numérico, conforme explicitado acima.
% ----

% ---
% DADOS INICIAIS - Define sigla com título, área de concentração e opção do Programa 
\siglaunidade{IFSC-TCC}
\programa{BFCp}

% Configurações de aparência do PDF final
% alterando o aspecto da cor azul
\definecolor{blue}{RGB}{41,5,195}


% informações do PDF
\makeatletter
\hypersetup{
	%pagebackref=true,
	pdftitle={\@title}, 
	pdfauthor={\@author},
	pdfsubject={\imprimirpreambulo},
	pdfcreator={LaTeX with abnTeX2},
	pdfkeywords={abnt}{latex}{abntex}{USPSC}{trabalho acadêmico}, 
	colorlinks=true,       		% false: boxed links; true: colored links
	linkcolor=black,          	% color of internal links
	citecolor=black,        		% color of links to bibliography
	filecolor=black,      		% color of file links
	urlcolor=black,
	%Para habilitar as cores dos links, retire a % antes dos comandos abaixo e inclua a % antes das 4 linhas de comando acima 
	%linkcolor=blue,            	% color of internal links
	%citecolor=blue,        		% color of links to bibliography
	%filecolor=magenta,      		% color of file links
	%urlcolor=blue,
	bookmarksdepth=4	
}
\makeatother
% --- 
% --- 
% Espaçamentos entre linhas e parágrafos 
% --- 

% O tamanho do parágrafo é dado por:
\setlength{\parindent}{1.3cm}

% Controle do espaçamento entre um parágrafo e outro:
\setlength{\parskip}{0.2cm}  % tente também \onelineskip

% ---
% compila o sumário e índice
\makeindex
% ---

% ----------------

\newcommand\underrel[2]{\mathrel{\mathop{#2}\limits_{#1}}}

\newcommand{\matriz}[1]{\hat#1}

\newcommand{\many}[2]{$#1_1, #1_2, \dots, #1_#2$}

\newcommand{\cmany}[3]{$#1_1 #3 #1_2 #3 \dots #3 #1_#2$}

\newcommand{\mmany}[2]{ #1_1, #1_2, \dots, #1_#2 }

\newcommand{\mcmany}[3]{#1_1 #3 #1_2 #3 \dots #3 #1_#2}

\newcommand{\set}[1]{\{#1\}}

\newcommand{\cjgt}[1]{\overline{#1}}
\DeclareMathOperator{\diag}{diag}
\DeclareMathOperator{\sign}{sign}
\DeclareMathOperator{\ai}{Ai}
\DeclareMathOperator{\re}{Re}
\DeclareMathOperator{\im}{Im}
\DeclareMathOperator{\Df}{D}
\DeclareMathOperator{\Ee}{E}
\DeclareMathOperator{\h}{h_1}
\DeclareMathOperator{\f}{f}
\DeclareMathOperator{\U}{U}
\DeclareMathOperator{\W}{W}
\DeclareMathOperator{\K}{K}
\DeclareMathOperator{\Hf}{\mathcal{H}}
\DeclareMathOperator{\Qf}{Q}
\DeclareMathOperator{\Gl}{\mathcal{L}}
\DeclareMathOperator{\g}{g}
\DeclareMathOperator{\V}{V}
\newcommand{\iu}{\mathrm{i}\mkern1mu}
\renewcommand{\Im}{\mathop{\textrm Im}}
\DeclareMathOperator{\ee}{e}
\DeclareMathOperator{\supp}{supp}
\newcommand{\N}{\mathbb{N}}
\newcommand{\C}{\mathbb{C}}
\newcommand{\R}{\mathbb{R}}
\newcommand{\Z}{\mathbb{Z}}
\newcommand{\D}{\mathbb{D}}
\newcommand{\Q}{\mathbb{Q}}
\newcommand{\J}{J} %Jacobiano
\newcommand{\Id}{\mathbb{1}}
\newcommand{\p}{p} %medida
\newcommand{\E}{\mathbb{E}}
\newcommand{\Se}{\mathbb{S}}
\newcommand{\He}{\mathbb{H}}
\newcommand{\boh}{\mathit{o}}
\newcommand{\Boh}{\mathcal{O}}
\newcommand{\bbp}{\bm K_{\mathrm{BBP}}}
\newcommand{\ii}{\mathrm{i}}
\newcommand*{\deff}{\mathrel{\vcenter{\baselineskip0.5ex \lineskiplimit0pt
			\hbox{\scriptsize.}\hbox{\scriptsize.}}}%
	=}
\newcommand*{\revdeff}{=\mathrel{\vcenter{\baselineskip0.5ex \lineskiplimit0pt
			\hbox{\scriptsize.}\hbox{\scriptsize.}}}%
}


% MATH DECLARATIONS
\newtheorem{lemma}{Lema}[section]
\newtheorem{thm}[lemma]{Teorema}
\newtheorem{claim}[lemma]{Afirmação}
\newtheorem{cor}[lemma]{Corolário}
\newtheorem{definition}[lemma]{Definição}
\newtheorem{conjecture}[lemma]{Conjectura}
\newtheorem{prop}[lemma]{Proposição}
\newtheorem{assumption}[lemma]{Assumpção}
\numberwithin{equation}{section} %numeracao dentro de secoes

% PROOF ENV
\makeatletter
\newenvironment{proof}[1][Demonstração]{\par
	\pushQED{\qed}%
	\normalfont \topsep6\p@\@plus6\p@\relax
	\trivlist
	\item\relax
	{\itshape
		#1\@addpunct{.}}\hspace\labelsep\ignorespaces
}{%
	\popQED\endtrivlist\@endpefalse
}
\makeatother

%-----------

% ----
% Início do documento
% ----
\begin{document}

% Seleciona o idioma do documento (conforme pacotes do babel)
\selectlanguage{brazil}
% Se o idioma do texto for inglês, inclua uma % antes do 
%      comando \selectlanguage{brazil} e 
%      retire a % antes do comando abaixo
%\selectlanguage{english}

% Retira espaço extra obsoleto entre as frases.
\frenchspacing 

% --- Formatação dos Títulos
\renewcommand{\ABNTEXchapterfontsize}{\fontsize{12}{12}\bfseries}
\renewcommand{\ABNTEXsectionfontsize}{\fontsize{12}{12}\bfseries}
\renewcommand{\ABNTEXsubsectionfontsize}{\fontsize{12}{12}\normalfont}
\renewcommand{\ABNTEXsubsubsectionfontsize}{\fontsize{12}{12}\normalfont}
\renewcommand{\ABNTEXsubsubsubsectionfontsize}{\fontsize{12}{12}\normalfont}


% ----------------------------------------------------------
% ELEMENTOS PRÉ-TEXTUAIS
% ----------------------------------------------------------
% ---
% Capa
% ---
\imprimircapa
% ---
% Folha de rosto
% (o * indica impressão em anverso (frente) e verso )
% ---
\imprimirfolhaderosto*
%\imprimirfolhaderosto
% ---
% ---
% Inserir a ficha catalográfica em pdf
% ---
% A biblioteca da sua Unidade lhe fornecerá um PDF com a ficha
% catalográfica definitiva. 
% Quando estiver com o documento, salve-o como PDF no diretório
% do seu projeto como fichacatalografica.pdf e inclua o arquivo
% utilizando o comando abaixo:

%\includepdf{USPSC-TA-PreTextual/USPSC-fichacatalografica.pdf}

% Se você optar por elaborar a ficha catalográfica, deverá 
% incluir uma % antes da linha % antes
% do comando \include{USPSC-TA-PreTextual/USPSC-fichacatalografica} 
% e retirar o % do comando abaixo
\include{USPSC-TA-PreTextual/USPSC-fichacatalografica}
% As informações que compõem a ficha catalográfica estão 
% definidas no arquivo USPSC-pre-textual-UUUU.tex
% ---

% ---
% ---
% Inserir errata
% ---

%\include{USPSC-TA-PreTextual/USPSC-Errata}

% ---

% ---
% Inserir folha de aprovação
% ---

% A Folha de aprovação é um elemento obrigatório da NBR 4724/2011 (seção 4.2.1.3). 
% Após a defesa/aprovação do trabalho, gere o arquivo folhadeaprovacao.pdf da página assinada pela banca 
% e iclua o arquivo utilizando o comando abaixo:
%\includepdf{USPSC-TA-PreTextual/USPSC-folhadeaprovacao.pdf}
% Alternativa para a Folha de Aprovação:
% Se for a sua opção elaborar uma folha de aprovação, insira uma % antes do comando acima que inclui o arquivo folhadeaprovacao.pdf,
% tire o % do comando abaixo e altere o arquivo folhadeaprovacao.tex conforme suas necessidades
%\include{folhadeaprovacao}
%\includepdf{USPSC-TA-PreTextual/USPSC-PaginaEmBranco.pdf}

% ---
% Dedicatória
% ---
%\include{USPSC-TA-PreTextual/USPSC-Dedicatoria}
% ---

% ---
% Agradecimentos
% ---
%\include{USPSC-TA-PreTextual/USPSC-Agradecimentos}
% ---

% ---
% Epígrafe
% ---
%%% USPSC-Epigrafe.tex
\begin{epigrafe}
    \vspace*{\fill}
	\begin{flushright}
		\textit{"En remontant chez moi pour y passer la soirée à travailler de mon mieux, je me disais que le monde n'est pas construit pour l'équilibre. Le monde est désordre. L'équilibre n'est pas la règle, c'est l'exception."\\
		G.Duhamel, Maitres, 1937}
	\end{flushright}
\end{epigrafe}
% ---
% ---

% A T E N Ç Ã O
% Se o idioma do texto for em inglês, o abstract deve preceder o resumo
% resumo em português
%
% Resumo
% ---
%% USPSC-Resumo.tex
\setlength{\absparsep}{18pt} % ajusta o espaçamento dos parágrafos do resumo		
\begin{resumo}
	\begin{flushleft} 
			\setlength{\absparsep}{0pt} % ajusta o espaçamento da referência	
			\SingleSpacing 
			\imprimirautorabr~~\textbf{\imprimirtituloresumo}.	\imprimirdata. \pageref{LastPage}p. 
			%Substitua p. por f. quando utilizar oneside em \documentclass
			%\pageref{LastPage}f.
			\imprimirtipotrabalho~-~\imprimirinstituicao, \imprimirlocal, \imprimirdata. 
 	\end{flushleft}
\OnehalfSpacing 	
		
O estudo do espectro de matrizes aleatórias demonstra aplicabilidade em uma gama diversa de áreas da física, matemática à computação e engenharia. Estaremos interessados em estudar os principais ensembles da Teoria de Matrizes Aleatórias e suas medidas de equilíbrio, entender a analogia com Gases de Coulomb e, com essa ferramenta, realizar simulações que nos permitam calcular médias de funções de interesse. Discutiremos quais métodos são importantes para a simulação do problema de gases de coulomb e quais suas limitações além das impostas pelo problema de escalabilidade e de singularidades. Nos resultados, mostramos que o método de \textit{Langevin Monte Carlo} tem boa performance e consegue replicar as medidas de modelos conhecidos, e um menos discutido, de matrizes aleatórias, com boa precisão. 

 \textbf{Palavras-chave}: Gases de Coulomb. Matrizes Aleatórias. 
\end{resumo}
% ---

% Abstract
% ---
%\include{USPSC-TA-PreTextual/USPSC-Abstract}
% ---

% ---
% inserir lista de figurass
% ---
%\pdfbookmark[0]{\listfigurename}{lof}
%\listoffigures*
%\cleardoublepage
% ---

% ---
% inserir lista de tabelas
% ---
%\pdfbookmark[0]{\listtablename}{lot}
%\listoftables*
%\cleardoublepage
% ---

% ---
% inserir lista de quadros
% ---
%\pdfbookmark[0]{\listofquadroname}{loq}
%\listofquadro*
%\cleardoublepage
% ---

% ---
% inserir lista de abreviaturas e siglas
% ---
%\include{USPSC-TA-PreTextual/USPSC-AbreviaturasSiglas}
% ---

% ---
% inserir lista de símbolos
% ---
%\include{USPSC-TA-PreTextual/USPSC-Simbolos}
% ---
% ---
% inserir o sumario
% ---
\pdfbookmark[0]{\contentsname}{toc}
\tableofcontents*
\cleardoublepage
% ---
% ----------------------------------------------------------
% ELEMENTOS TEXTUAIS
% ----------------------------------------------------------
\textual
% Os capítulos são inseridos como arquivos externos 

% ---
% Capítulo 1 - Introdução
% ---

\chapter{Introdução}
\label{Capitulo: Intro}

De acordo com a mecânica quântica, níveis de energia de uma sistema são descritos pelos autovalores de seu operador hermitiano associado, o hamiltoniano $\Hf$. Por simplicidade, usualmente toma-se truncamentos do espaço de Hilbert no qual opera $\Hf$, o tornando finito. Em geral, a influência das outras dimensões são desconsideradas ou aproximadas sobre o espaço restante. Com o Hamiltoniano finito, caracterizar o sistema físico é o equivalente à resolver o problema de autoenergias $\Hf \Psi_i = E_i \Psi_i$. Esta abordagem obteve muito sucesso na descrição de estados excitados de baixa energia para núcleos atômicos pesados, por exemplo. Contudo, é irrazoável descrever, ou ainda resolver, Hamiltonianos para a descrição de níveis de excitações mais altos.

%Na descrição de sistemas complexos, como núcleos atômicos pesados, $\Hf$ pode não ser completamente descrito pela teoria ou complicado. Somado à necessidade do cálculo explícito de grandezas tal qual autoenergias, considera-se truncamentos do espaço de Hilbert onde opera $\Hf$, representado por matriz de dimensão finita. Caracterizar o sistema físico é equivalente à resolver o problema de autovalores $\Hf \Psi_i = E_i \Psi_i$.

Pela dificuldade apresentada, Wigner, em seu estudo de núcleos atômicos, sugere uma abordagem alternativa, uma mecânica estatística para o problema de autovalores. Tal teoria descreveria, estocasticamente, o perfil da estrutura energética nucleica ao invés de detalhar seus níveis. Buscava-se, em algum sentido, uma universalidade, uma descrição que fosse, dada complexidade o suficiente, independente dos detalhes em $\Hf$. A teoria foi prontamente seguida por, dentre outros, Gaudin, Mehta \cite{MehtaGaudin}, e Dyson \cite{Dyson}, que avançaram em sua descrição. Esse desenvolvimento é o início da chamada Teoria de Matrizes Aleatórias (RMT, \textit{Random Matrix Theory}) e hoje desempenha importante papel na descrição estatística de sistemas com alta complexidade representados com matrizes de simetria induzida pela natureza do problema descrito.

%Hoje, suas aplicações são extensas em campos de alta complexidade ou com descrição matricial, principalmente quando há estrutura, como matrizes de correlação ou operadores físicos.
Para ensembles invariantes (de matrizes equivalentes por rotação), uma importante analogia se apresenta, a de Gases de Coulomb. Pensando os $N$ autovalores como partículas de um gás com interagente sob potencial externo, podemos usar de noções físicas para derivar, por exemplo, as densidades de autovalores no limite termodinâmico ($N \rightarrow \infty$). A analogia permite, por variação do potencial externo, dimensões do sistema e núcleo de interação, explorar ensembles com entradas correlacionadas, de difícil construção direta. Contudo, nem sempre soluções analíticas são possível para as equações diferencias que descrevem a dinâmica destes gases. Por isso, recorre-se à simulações numéricas que, ainda assim, são delicadas de tratar. A dinâmica tem alta complexidade temporal e as singularidades dificultam a invariância do hamiltoniano. Ainda assim, existem abordagens que permitem tornar a simulação da dinâmica suficientemente acurada. Simulações como esta permitem, de forma direta, uma exploração numérica holística de casos analiticamente complicados e visualização de fenômenos, medidas e funções outrossim inacessíveis.


\chapter{Matrizes Aleatórias}
% -
% C1S1 - Distribuição de autovalores
% - 
\section{Distribuição de autovalores}

Seja $\Se$ um conjunto tal como $\R, \C, \He $ (Reais, Complexos e Quaterniônicos). Consideremos inicialmente uma matriz $\matriz{M} \in \mathcal{M}_{\Se}(N)$ espaço de matrizes $N \cross N$, ou seja, de $N^2$ entradas, sejam elas reais, complexas ou quaterniônicas. Se tomamos o elemento de matriz $M_{i,j}$ $\forall i, j \in \Z$, com $1 \leq i, j \leq N$, como variável aleatória de distribuição arbitrária, podemos expressar a densidade de probabilidade conjunta (jpdf) como $$\p(\hat{M}) \dd M = \p(M_{1,1}, \dots, M_{N,N}) \prod_{i,j=1}^{N} \dd M_{i,j}.$$

Não lidaremos, contudo, com uma classe tão ampla de matrizes. Considere a decomposição $\matriz{M} = \matriz{O} \matriz{D} \matriz{O}^{-1}$, onde $\matriz{D} = \diag(\mmany{\lambda}{N})$. Estamos especialmente interessados no caso onde $\matriz{O} \in V_N(\Se^N)$, espaço denominado variedade de Stiefel. Isso implica que $ \matriz{O} \matriz{O}^* = \Id$. Tomamos $\matriz{O}$ matriz ortogonal, unitária ou simplética, a depender de $\Se$, o que resulta em autovalores $\lambda \in \R$. Isso pode ser motivado fisicamente, por exemplo, se pensarmos que queremos tomar grandezas reais.

Para o subespaço tomado vale que $\cjgt{M_{i,j}} = M_{j, i}$. Este vínculo reflete na dimensão do subespaço escolhido, com valor dependente de $\Se$. A transformação tomada tem ainda Jacobiano $\J(\matriz{M} \rightarrow \{ \vec{\lambda}, \matriz{O} \} )$. Com estes fatos podemos reescrever a jpdf como 
\begin{equation}
	 \p(\hat{M}) \dd M = \p \left( M_{1,1}(\vec{\lambda}, \matriz{O}), \cdots, M_{N,N}(\vec{\lambda}, \matriz{O}) | \J(\matriz{M} \rightarrow \{ \vec{\lambda}, \matriz{O} \} ) \right) \dd O \prod_{i=1}^{N} \lambda_i.
\label{Equation: p(lambda, O)}
\end{equation}

Aqui, ressalta-se que estamos interessados em distribuições de autovalores. Para calcular $\p(\mmany{\lambda}{N})$ devemos integrar os termos à direita da Equação \ref{Equation: p(lambda, O)} sobre o subespaço $V_N(\Se^N)$. Isso nem sempre é fácil ou possível. Para garantir a integrabilidade, tomaremos \textit{ensembles} de matrizes aleatórias onde o jpdf de suas entradas pode ser escrito exclusivamente como função dos autovalores, ou seja $$\p(\mmany{\lambda}{N}, \matriz{O}) \equiv \p \left( M_{1,1}(\vec{\lambda}), \cdots, M_{N,N}(\vec{\lambda}) | \J(\matriz{M} \rightarrow \{ \vec{\lambda} \} ) \right).$$

Ensembles com esta propriedade são denominados invariantes por rotação. Esta escolha implica que quaisquer duas matrizes $\matriz{M}, \matriz{M'~}$ que satisfaçam a relação de equivalência $\matriz{M} = \matriz{U} \matriz{M'} \matriz{U}^{-1}$ tem mesma probabilidade. Nesta relação, $\matriz{U}$ é simétrica, hermitiana ou simplética respectivamente quando $\Se = \R,\C,\He $. Considere o teorema \cite{AlanThesis}.
\begin{thm}
	Tome $\matriz{M} \in M_{\R}(N),  M_{\C}(N),  M_{\He}(N)$ simétrica, hermitiana ou autodual, respectivamente. Se  $\matriz{M}$ tem jpdf da forma $\phi(\matriz{M})$, invariante sobre transformações de similaridade ortogonal, a jpdf dos $N$ autovalores ordenados de $\matriz{M}$, $\mcmany{\lambda}{N}{\geq}$, é $$ C_{N}^{(\beta)} \phi(\matriz{D}) \prod_{i < j} (\lambda_i - \lambda_j)^{\beta}$$ onde $C_{N}^{\beta}$ é constante e $\beta = 1, 2, 3$ corresponde respectivamente à $\matriz{M} \in M_{\R}(N),  M_{\C}(N),  M_{\He}(N)$. 
	\label{Teorema: Invariante}
\end{thm}

Logo, desde que tomemos um ensemble de matrizes aleatórias com a jpdf das entradas apropriado, podemos reescrever a distribuição em função dos autovalores com \ref{Teorema: Invariante}. Vale ainda observar que, pelo Lema de Weyl \cite{weyl1946classical}, uma jpdf invariante pode ser expressa totalmente por $\p(\matriz{M})= \phi \left( \Tr(\matriz{M}), \Tr(\matriz{M}^2), \cdots, \Tr(\matriz{M}^N) \right)$. Tomada esta expressão, podemos escrever
\begin{equation}
	\p_{ord}(\mmany{\lambda}{N}) = C_{N}^{\beta} \phi{\left( \sum_i^N \lambda_i, \cdots, \sum_i^N \lambda_i^N \right)} \prod_{i < j} (\lambda_i - \lambda_j)^{\beta}
	\label{Equation: p-ord}
\end{equation}

%Essa expressão será usada em breve. Aqui, é mais natural entender o teorema quando se entende a constante $C_N^{\beta}$ como relacionada à integração $\int_{V_N(\Se^N)} \dd O$ e quando se enuncia o lema:

%\begin{lemma}
%	\[
%	\J(\matriz{M} \rightarrow \{ \vec{\lambda}, \matriz{O} \}) = \prod_{j > k} (\lambda_j - \lambda_k)^\beta
%	\]
%	Onde $\beta = 1,2,4$ respectivamente quando $M_{i,j} \in \R, \C, \He $.
%	\label{Lema: Jacobiano}
%\end{lemma}

% -
% C1S2 - Emsembles Gaussianos
% - 
\section{Ensembles Gaussianos}
\label{Section: Ensembles Gaussianos}

Dentre os muitos ensembles em RMT, os Gaussianos são notórios. São eles o \textit{Gaussian Orthogonal Ensemble (GOE)} ($\beta=1$), \textit{Gaussian Unitary Ensemble (GUE)} ($\beta=2$) e \textit{Gaussian Sympletic Ensemble (GSE)} ($\beta=4$). Notemos primeiramente que o nome é relacionado à escolha de $\Se$. Mais explicitamente, o nome é dado em relação à se $\matriz{O}$, tal que $\matriz{M} = \matriz{O}\matriz{D}\matriz{O}^*$, é ortogonal, unitário ou simplético. É natural então pensar nos ensembles \textit{GOE}, \textit{GUE} e \textit{GSE} como matrizes $\matriz{M} \in \mathcal{M}_{\Se}(N)$ onde 
$$
\mathcal{M}_{\Se}(N) \ni M_{i,j} \sim
\begin{cases}
	\mathcal{N}_{\Se}(0,1/2) &  \ \text{para} \ i \neq j,\\
	\mathcal{N}_{\Se}(0,1) & \ \text{para} \ i = j.
\end{cases}
$$

Os três ensembles gaussianos compartilham de uma propriedade exclusiva - são os únicos ensembles com entradas independentes e, simultaneamente, jpdf rotacionalmente invariante. Tomemos, por simplicidade, $\matriz{U} \in \mathcal{M}_{\R}(N)$, matriz real simétrica, do GOE. Para esta, sabendo as entradas independentes, podemos escrever $$\p(\matriz{U}) = \prod_{i=1}^{N}\frac{\exp{\frac{U_{i,i}^2}{2}}}{\sqrt{2\pi}} \prod_{i<j} \frac{\exp{U_{i,i}^2}}{\sqrt{\pi}} = 2^{-N/2} \pi^{-N(N + 1)/4} \exp{-\frac{1}{2} \Tr{U^2}}.$$

Note que essa jpdf satisfaz as condições do Teorema \ref{Teorema: Invariante} e, especialmente, é da forma que propomos na Equação \ref{Equation: p-ord}, logo, $$ \p_{ord}(\mmany{\lambda}{N}) = \frac{1}{Z_{N, \beta = 1}^{(ord)}} \exp{-\frac{1}{2} \sum_{i = 1}^{N} \lambda_i^2} \prod_{i < j} (\lambda_i - \lambda_j).$$ 
De forma análoga, podemos deduzir mais geralmente para $\beta = 1,2,4$ que
\begin{equation}
	\begin{split}
		\p(\mmany{\lambda}{N}) 
		&= \frac{1}{ N! Z_{N, \beta}^{(ord)}} \exp{- \left(\sum_{i = 1}^{N} \frac{\lambda_i^2}{2} - \sum_{i < j} \log{|\lambda_i - \lambda_j|^{\beta}} \right)}, \\
		&= \frac{1}{Z_{N, \beta}} \ee^{-\beta_N \mathcal{H}_N(\vec{\lambda})},
	\end{split}
\label{Equation: medida Gaussian}
\end{equation}
onde $Z_{N, \beta}$ é função de partição canônica para autovalores desordenados\footnote{Usa-se do fator de contagem de Boltzmann \cite[Capítulo~3]{landau2013statistical} para escrever $ Z_{N, \beta} = N! Z_{N, \beta}^{(ord)}$.}, normalizante da expressão \ref{Equation: medida Gaussian}. O fator $\beta_N = \beta N^2$ é pensado como a temperatura inversa. Definimos ainda o Hamiltoniano $$\mathcal{H}_N(\vec{\lambda}) = \frac{1}{N}\sum_{i = 1}^{N} \frac{\lambda_i^2}{2} + \frac{1}{N^2} \sum_{i < j} \log{\frac{1}{|\lambda_i - \lambda_j|}}, \ \ \  \lambda_i \mapsto \lambda_i \sqrt{\beta N}.$$
% Sabemos então, que a partir dessa função podemos retirar importantes propriedades estatísticas (macroscópicas) do sistema de autovalores dos ensembles Gaussianos.



% -
% C1S3 - Gases de Coulomb (Log Gas?)
% - 
\section{Gases de Coulomb}
\label{Section: Gases de Coulomb}

Sob as devidas condições, o gás de coulomb $\p_N$ \cite{ChafaCoulombMeasure} é a medida de probabilidade de Boltzmann-Gibbs dada em $(R^d)^N$ por 
\begin{equation}
	\dd \p_N(\mmany{x}{N}) = \frac{e^{-\beta N^2 \Hf_N(\mmany{x}{N})}}{Z_{N,\beta}} \mcmany{\dd x}{N}{},
	\label{Equação: Medida Gas de Coulomb}
\end{equation}
onde $\Hf_N(\vec{x}) = \frac{1}{N} \sum_{i = 1}^{N} \V(x) + \frac{1}{2N^2} \sum_{i \neq j} \g(x_i - x_j)$ é usualmente chamado hamiltoniano ou energia do sistema.

A medida $\p_N$ modela um gás de partículas indistinguíveis com carga nas posições $\mmany{x}{N} \in \Se$ de dimensão $d$ em $\R^n$ \textit{ambient space}. As partículas estão sujeitas a um potencial externo $\V \colon \Se \mapsto \R$ e interagem por $\g \colon \Se \mapsto (-\infty, \infty]$. A temperatura inversa é $\beta N^2$. Assumiremos, para que valha a definição \ref{Equação: Medida Gas de Coulomb}, que $V, \ \g \ \text{e} \ \beta$ são tais que a constante de normalização (função partição) $Z_{N, \beta} < \infty \ \forall \ N$\footnote{Note que $\p_N$ é um modelo de interações estáticas e não há campos magnéticos considerados.}. Tome $\R^n$ com $n \geq 2$, sabemos que, para $x \neq 0$ o núcleo de interação coulombiana (função de Green) vale $$
	g(\vec{x}) =
	\begin{cases}
			\log \frac{1}{|\vec{x}|} \ \ \text{se} \ n = 2,\\
			\frac{1}{|x|^{n-2}} \ \ \text{se} \ n \geq 3.
	\end{cases}
$$ onde $g$ é solução da equação de Poisson dada por $$
	- \nabla g(\vec{x}) = c\delta_0 \ \ \text{com} \ c = 
	\begin{cases}
		2\pi \ \ \text{para} \ n = 2,\\
		(n-2) |S^{n-1}| \ \ \text{para} \ n \geq 3.
	\end{cases}
$$

Se lembramos da expressão \ref{Equation: medida Gaussian}, perceberemos que, para o devido $\V(x)$, podemos tomar $d=1$ e $n = 2$ para recuperar a medida dos ensembles gaussianos 
\begin{equation}
	\p_N(\vec{x}) = \frac{e^{-\beta_N \Hf_N(\vec{x})}}{Z_{N,\beta}}, \ \ \Hf_N(\vec{x}) = \frac{1}{N} \sum_{i = 1}^{N} \V(x_i) + \frac{1}{N^2} \sum_{i < j} \log{\frac{1}{|x_i - x_j|}}.
	\label{Equation: Medida Log V}
\end{equation}
Estamos tratando de partículas no plano confinadas à uma reta. Para algum potencial arbitrário, além da devida escolha de $n$ e $d$, cairemos em outros ensembles de matrizes.

% -
% C1S4 - Medidas de Equilíbrio
% - 
\section{Medidas de Equilíbrio}
\label{Seção: Medida}
O conjunto de pontos no espaço de fase ou ainda, os microestados, determinam um \textit{ensemble estatístico}. De mesma forma, um conjunto de matrizes determina um ensemble em RMT. Podemos relacionar o conjunto de microestados dos autovalores $\{\vec{\lambda}\}$ com as configurações do sistema de $N$ partículas descrito na Seção \ref{Section: Gases de Coulomb}. Notando que tratamos do ensemble canônico, um argumento termodinâmico nos indica então que devemos minimizar a energia livre $E^V_{N,\beta} = \log{Z_{N, \beta}}.$

Consideraremos $\V$ sob condições tais que seja denominado um potencial admissível \cite{ChafaCoulombMeasure}. Com isso, se $\mu_{V}(\vec{\lambda})$ é medida de probabilidade no espaço das possíveis configurações de autovalores, $Z_{N, \beta}$ será finita e existirá $\mu_{V}^* = \arg \inf {\mathcal{H}_N(\vec{\lambda})}$ medida de equilíbrio única no limite termodinâmico $N \rightarrow \infty$. Para determinar a medida de equilíbrio de \ref{Equation: Medida Log V} \cite{RMT-firstcourse-Potters}, queremos satisfazer o sistema de equações
\begin{equation}
	\frac{\partial \mathcal{H}}{\partial \lambda_i} = 0 \ \implies \ \V'(\lambda_i) = \frac{1}{N} \sum_{1 = j \neq i}^{N} \frac{1}{\lambda_i - \lambda_j} \ \ \text{para} \ i = 1, \cdots, N.
	\label{Equação: Sistema minimizante}
\end{equation} 
Usaremos o denominado \textit{resolvent}. Considere a função complexa $$G_N(z) = \frac{1}{N} \Tr{\left(z\Id - \matriz{M}\right)^{-1}} = \frac{1}{N} \sum_{i=1}^{N} \frac{1}{z - \lambda_i},$$ onde $\matriz{M}$ é matriz aleatória com autovalores $\{\mmany{\lambda}{N}\}$. Note que $G_N(z)$ é uma função complexa aleatória com polos em $\lambda_i$. Não trivialmente, podemos reescrever \ref{Equação: Sistema minimizante} como $$\V'(z) G_N(z) - \Pi_N(z) = \frac{G_N^2(z)}{2} + \frac{G'_N(z)}{2N},$$ onde $$\Pi_N(z) = \frac{1}{N} \sum_{i = 1}^{N} \frac{\V'(z) - \V'(\lambda_i)}{z - \lambda_i}$$ é um polinômio de grau $\deg{\V'(z)} - 1 = k - 1$. Resolver explicitamente para $N$ constante pode não ser simples ou mesmo possível. Em geral, tomaremos a assintótica $N \to \infty$ de $G_N(z)$, nesse limite temos a transformada de Stieltjes\footnote{Também chamada transformada de Cauchy.}
\begin{equation}
	S^{\mu_V}(z) = \int \frac{\mu^*_V(\lambda)}{z - \lambda} \dd \lambda= \V'(z) \pm \sqrt{\V'(z)^2 - 2 \Pi_{\infty}(z) }.
	\label{Equation: Resolvent}
\end{equation}
com $$\Pi_{\infty}(z) = \int \frac{\V'(z) - \V'(\lambda)}{z - \lambda} \mu^*_V(\lambda) d\lambda.$$ Como consequência da fórmula de Sokhotski-Plemeji, é enunciado o resultado 
\begin{equation}
	\mu^{*}_{V}(x) = \frac{1}{2\pi \ii} \left( S^{\mu_V}_{+} -  S^{\mu_V}_{-}\right) = \frac{1}{\pi} \lim_{\epsilon \to 0^+} \Im{S^{\mu_V}_{+}(x + \ii\epsilon)}.
	\label{Equation: p(lambda)}
\end{equation}
%Considere ainda um potencial $\V(x)$ convexo. Neste caso, tomamos naturalmente $\mu^*_V(x)$ não nula apenas em um intervalo $(\lambda_{-}, \lambda_{+})$. Sabemos que o comportamento não analítico surge da raiz quadrada, tal que se definirmos $\Df(z) := \V'(z)^2 - 2 \Pi_{\infty}(z)$ polinômio de grau $2k$, $\{\lambda_{-}, \lambda_{+}\}$ são suas raízes e o polinômio tem valor negativo em algum intervalo. Equivalentemente $D(z) = (z-\lambda_{-})(z - \lambda_{+}) \Qf^2(z),$ onde $\Qf(z)$ é polinômio de grau $k-1$. Com essas definições podemos escrever que, por \ref{Equation: p(lambda)},
%\begin{equation}
%	\mu^*_V(x) =\frac{\Qf(x)}{\pi} \sqrt{(\lambda_{+} - x)(x - \lambda_{-})}, \ \ \text{para} \ \  \lambda_{-} \leq x \leq \lambda_{+}.
%\end{equation}
Em geral, restaria para cada potencial, balancear o sistema de $k+2$ equações dada por
\[
\left( S^{\mu_{V}} - \V' \right)^2 = \left( V' \right)^2 - 2 \Pi_{\infty}
\]




% -
% C1S5 - Potenciais notáveis
% - 
\section{Potenciais notáveis}

Consideraremos a mudança de variável $\V(x) \mapsto \beta N \V(x)$ tal que possamos escrever $\p(\{ \lambda_i\}) \propto \ee^{-\beta N \mathcal{H}_N(\{ \lambda_i\})}$ com $$\mathcal{H}_N(\{ \lambda_i\}) = \sum_{i = 1}^{N} V(x) + \frac{1}{2N} \sum_{i \neq j} \log{|\lambda_i - \lambda_j|}.$$ Com essa mudança, consideremos os seguintes potencias.


%Um resultado importante enuncia \cite{deiftorthogonal}:

%\begin{thm}
%	Para $V(x) = t x^{2m}$ com $t>0$, vale que $$ \p_V(x) = - \frac{m t}{\pi} \sqrt{x^2 - a^2} + h(x) $$ no suporte $\supp(-a, a)$. Onde, $$ a = \left( mt \prod_{l=1}^{m} \frac{2l - 1}{2l} \right)$$ e $$h(x) = x^{2m-2} + \sum_{j=1}^{m-1} x^{2m - 2 - 2j} a^{2j} \prod_{l=1}^{j}.$$
%	\label{Teorema: Medida V(x)}
%\end{thm}


\subsection{Potenciais Quadráticos}

O caso de potencial quadrático $$V(x) = \frac{x^2}{2}$$ descreve o caso dos ensembles gaussianos, onde é fácil determinar que $$V'(z) = z \ \implies \ \Pi(z) = 1$$ e, por isso, 

% $$\mathcal{H}(\vec{\lambda}) = \frac{1}{2}\sum_{i = 1}^{N} \lambda_i^2 - \frac{1}{2N}\sum_{i \neq j} \log{|\lambda_i - \lambda_j|}.$$ É fácil determinar que $$V'(z) = z \ \implies \ \Pi(z) = 1$$ e, por isso, 
%Note o fator $N\beta$. Tomaremos, eventualmente, $N$ suficientemente grande para notar efeitos assintóticos em N. Com isso, estaremos explorando o limite de temperatura zero. Minimizemos $\mathcal{H}(\vec{\lambda})$\footnote{Seguindo passos de \cite{IntroRM}}. Para cada $\lambda_i$, escrevemos

%\begin{equation}
%	\frac{\partial \mathcal{H}(\vec{\lambda})}{\partial \lambda_i} = 0 \implies \lambda_i = \frac{1}{N} \sum_{i \neq j} \frac{1}{\lambda_i - \lambda_j}.
%	\label{Equation: Saddle}
%\end{equation}

%\noindent Multiplicando \ref{Equation: Saddle} por  $1/(N (z - \lambda_i))$, onde $z \in \C \setminus \{\lambda_i\}$ e somando sobre todos autovalores, teremos

%\begin{equation*}
%	\frac{1}{N} \sum_{i=1}^{N} \frac{\lambda_i}{z - \lambda_i} = \frac{1}{N} \sum_{i=1}^{N} \sum_{i \neq j} \frac{1}{\lambda_i - \lambda_j} \frac{1}{N(z - \lambda_i)}.
%	\label{Equation: Post-Saddle}
%\end{equation*}

%\noindent é possível reescrever ainda (não trivialmente) a expressão \ref{Equation: Post-Saddle} retomando a definição \ref{Equation: def G}. Ficaremos com:

%\begin{equation}
%	\frac{1}{2} G_N^2(z) + \frac{1}{2N} G_N'(z) = -1 + z G_N(z)
%	\label{Equation: S1 G(z)}
%\end{equation}

%Temos uma equação diferencial nas mãos. Contudo, o termo em \ref{Equation: S1 G(z)} com a derivada está sendo dividido por $N$. Lembremos que, pela ordem de $\lambda_i$ devemos ter também que $G_N(z)$ tem ordem $\Boh(1)$. Logo, sua derivada divida por $N$ não terá a ordem dominante. Naturalmente, quando tomamos o limite $N \rightarrow \infty$ ficaremos com $${G_{\infty}^{(med)}}^2 (z) - 2z G_{\infty}^{(med)}(z) + 2 = 0.$$ Equação algébrica que pode ser resolvida diretamente, resultando

\begin{equation}
	G_{\infty}^{(med)}(z) = z \pm \sqrt{z^2 - 2}
	\label{Equation: G gauss}.
\end{equation}

Nosso problema chega ao fim pois definimos o \textit{resolvent}. Resta agora invocar a Equação \ref{Equation: p(lambda)} utilizando de \ref{Equation: G gauss} para descobrir que

\begin{equation*}
	\p(x) = \pm \frac{\sign(-x)}{\pi \sqrt{2}} \sqrt{|x^2 - 2| - x^2 + 2}.
\end{equation*}

\noindent Ou ainda, no suporte $\supp(-\sqrt{2}, \sqrt{2})$,

\begin{equation}
	\p(x) = \frac{1}{\pi} \sqrt{2 - x^2}.
\end{equation}

Esse resultado é bem conhecido e a medida encontrada denominada Semi-Círculo de Wigner. Note que isso vale para qualquer $\beta$, a diferença é notada somente quando $N$ é suficientemente pequeno..


\subsection{Potencial Mônico}

Considere o potencial

\[
V(x) = \frac{t}{2\alpha} x^{2\alpha},
\]
onde $t > 0$ é escala e $\alpha \in \Z$. A medida de equilíbrio para $\alpha = 1$ é o semi-círculo de Wigner podemos validar na figura com a distribuição em vermelho. Sabemos também que o suporte $[-a, a]$ da densidade é dado por

\[
a = \left( \frac{t}{2} \prod_{j=1}^{\alpha} \frac{2j-1}{2j} \right)^{-\frac{1}{2\alpha}}.
\]


\subsection{Potencial Quártico}

Para Considere o potencial

\begin{equation}
	V(x) = \frac{x^4}{4} + t \frac{x^2}{2}.
	\label{Equação: Quartico}	
\end{equation}

\noindent Aqui observaremos, a depender de $t$, pela primeira vez a separação do suporte da função. Teremos um ponto crítico em $t=-2$ onde o suporte se separa nos intervalos $[-b_t, -a_t]$ e $[a_t, b_t]$ para $t < -2$. Para $t > -2$ o suporte é um único intervalo $[-b_t, b_t]$. Definiremos a medida nos dois casos,

\begin{itemize}
	\item \(t > -2\)
	\[
	\supp \mu_V = [-b_t, b_t], \ \ \frac{\dd \mu_V}{\dd x}(x) = \frac{1}{2\pi} (x^2 + c_t^2) \sqrt{b_t^2 - x^2},
	\]
	
	com $c_t^2 \deff\frac{1}{2} b_t^2 + t \deff \frac{1}{3} (2t + \sqrt{t^2 + 12})$.
	
	\item \(t < -2\)
	\[
	\supp \mu_V = [-b_t, -a_t] \cup [a_t, b_t], \ \ \frac{\dd \mu_V}{\dd x}(x) = \frac{1}{2\pi} |x| \sqrt{(x^2 - a_t^2)(b_t^2 - x^2)},
	\]
	
	com $ a_t \deff \sqrt{-2-t}, b_t \deff \sqrt{2-t}$.
\end{itemize}

% ---
% Capítulo 2 - Simulações e Algoritmos
% ---

\chapter{Simulações e Algoritmos}
\label{Capitulo: Simulações}

A medida $\mu$ de Boltzmann-Gibbs descreve o denominado ensemble canônico. Médias sobre suas configurações, microestados, são usadas para inferir informações macroscópicas do sistema. Sistemas dinâmicos que amostrem da medida $\mu$ são denominados termostatos e são notoriamente difíceis de se construir ergoticamente com processos dinâmicos determinísticos, portanto, uma teoria de equações diferenciais estocásticas foi desenvolvida. Usualmente, para o ensemble canônico, uma escolha natural de processo é a denominada \textit{Langevin Dynamics} \cite[Capítulo~6]{leimmolecular}, especialmente sua versão cinética. Muitas vezes as equações usadas não são diretamente integráveis e, por isso, se recorre a métodos numéricos. O caso cinético pode ser separado em duas dinâmicas. Para a integração da primeira, chamada Hamiltoniana, utilizamos o esquema de Verlet \cite{Verlet}. Para a segunda parte, denominada flutuação-dissipação, resolve-se analiticamente por se tratar de processo de Ornstein-Uhlenbeck de variância explícita. Apesar das qualidades dos métodos citados, a discretização pode introduzir instabilidade numérica e, para amenizar seus efeitos, introduz-se um passo de Metropolis \cite[Apêndice~C]{leimmolecular}. As escolhas supracitadas são descritas em \cite{Chafa2018} e é denominada \textit{Langevin Monte Carlo}.


% -
% C2S1 - Introdução ao algoritmo
% - 

\section{Dinâmica de \textit{Langevin Monte Carlo}}

Nosso objetivo com a simulação é determinar a esperança de uma função de interesse $\zeta(q,p)$, dado um ensemble. Pela teoria ergódica, sob algumas condições e no limite adequado, a média espacial $\langle \zeta \rangle_{\mu}$ é igual a média temporal $$\langle \zeta \rangle_t \approx \frac{1}{\tau} \sum_{k=1}^{\tau} \zeta(q_k, p_k),$$ onde $(q_k, p_k)$ podem ser obtidos por meio de uma dinâmica que preserve dada distribuição de Gibbs-Boltzmann. Para fazer o modelo ergódico, ou seja, garantir que a simulação - e nossas amostras - não esteja restrita a um subconjunto do espaço de fase, tomaremos uma dinâmica, um termostato, estocástica. Isso usualmente garante que o sistema possa convergir para sua medida invariante (única). Um esquema comumente utilizado é a dinâmica de Langevin\footnote{Poderíamos ter explorado outras dinâmicas similares tais como as dinâmicas de \textit{Dissipative Particle} \cite{DPD} ou \textit{Nose-Hoover} \cite{Hoover}.}.

Denote $q$, com $q \in \R^{(dN)}$, posição generalizada associada as $N$ partículas. A Equação \eqref{Equação: Medida Gas de Coulomb} é medida invariante do processo de difusão de Markov solução da equação diferencial estocástica
\begin{equation}
	\dd q_t = -\alpha_N \nabla \Hf_N(q_t) \dd t + \sqrt{2\frac{\gamma_N \alpha_N}{\beta_N}} \dd W_t,
	\label{Equação: Langevin Overdamped}
\end{equation}
onde $(W_t)_{t>0}$ é processo de Wiener, $\gamma_N > 0$ é constante de atrito e $\alpha_N$ é escala temporal. Isso seria suficiente e é chamado \textit{Overdamped Langevin}, contudo, tomaremos sua extensão cinética. Usaremos $q$ como variável de interesse e $p$, com $p \in \R^{(dN)}$, variável de momento generalizado, para flexibilizar a dinâmica. Considere $\U_N \colon \R^{(dN)} \rightarrow \R$ energia cinética generalizada tal que $\ee^{-\beta_N \U_N}$ seja Lebesgue integrável. Para uma energia da forma $\Ee_N(q,p) = \Hf_N(q) + \U_N(p)$, seja $(q_t, p_t)_{t\geq0}$ processo de difusão em $\R^{dN} \times \R^{dN}$ solução da equação diferencial estocástica
\begin{equation}
\begin{cases}
	\dd q_t = \alpha_N \nabla U_N (p_t) \dd t, \\
	\dd p_t = -\alpha_N \nabla \Hf_N(q_t) \dd t - \gamma_N \alpha_N \nabla U_N(p_t) \dd t + \sqrt{2\frac{\gamma_N \alpha_N}{\beta_N}} \dd B_t,
\end{cases}
\label{Equação: EqDif - Dinamica Langevin}
\end{equation}
onde $\beta_N$ é temperatura inversa e $\Hf_N \colon \R^{(dN)} \rightarrow \R$ é como na Distribuição \eqref{Equation: Medida Log V}. \cite{Stoltz2018} Esse processo deixa invariante $\p(q,p) = \p_q \otimes \p_p = \ee^{-\beta_N \Ee_N(q,p)}/Z'_N$ e admite o gerador infinitesimal 
\[
	\Gl = \Gl_{\Hf} + \Gl_{\U},
\]
\[
 \Gl_{\Hf} = -\alpha_N \nabla\Hf_N(q) \cdot \nabla_p + \alpha_N \nabla \U_N(p) \cdot \nabla_q, \ \ \ \ \Gl_{\U} = \frac{\gamma_N\alpha_N}{\beta_N} \Delta_p - \gamma_N \alpha_N \nabla \U_N(p) \cdot \nabla_p.
\]

Denomina-se $\Gl_{\Hf}$ a parte hamiltoniana e $\Gl_{\U}$ a parte de flutuação-dissipação. Tomaremos $\U_N(p) = \frac{1}{2} |p|^2$ tal que $\U_N(p)$ é energia cinética usual. Um esquema análogo é possível para energias cinéticas generalizadas. \cite{Stoltz2018} Além disso, $(B_t)_{t>0}$ é processo browniano. Para simular $(q_t,p_t)_{t \geq 0}$ integramos a Equação \eqref{Equação: EqDif - Dinamica Langevin}, contudo, isso pode não ser possível analiticamente, levando a recorrer a métodos numéricos para amostragem.

% -
% C2S2 - Algoritmo Híbrido de Monte Carlo
% - 

%
\section{Algoritmo Híbrido de Monte Carlo}

O algoritmo híbrido de Monte Carlo é baseado no algoritmo anterior mas adicionando uma variável de momento para melhor explorar o espaço. Defina $E = \mathbb{R}^{\dd N}$ e deixe $U_N : E \rightarrow \mathbb{R}$ ser suave para que $\ee^{-\beta_N U_N}$ seja Lebesgue integrável. Seja ainda $(X_t, Y_t)_{t>0}$ o processo de difusão em $E \times E$ solução de

\[
\begin{cases}
	\dd X_t = \alpha_N \nabla U_N (Y_t) \dd t, \\
	\dd Y_t = \alpha_N \nabla H_N(X_t) \dd t - \gamma_N \alpha_N \nabla U_N(Y_t) \dd t + \sqrt{2\frac{\gamma_N \alpha_N}{\beta_N} \dd B_t},
\end{cases}
\]
onde $(B_t)_{t>0}$ é o movimento browniano em $E$ e $\gamma_N > 0$ parâmetro representando atrito.

Quando $U_N(y) = \frac{1}{2}|y|^2$ temos $Y_t = \dd X_t/\dd t$ e teremos que $X_t$ e $Y_t$ poderão ser interpretados como posição e velocidade do sistema de $N$ pontos em $S$ no tempo $t$. Nesse caso, $U_n$ é energia cinética

% -
% C2S2 - Discretização
% - 

\section{Discretização}
\label{Seção: Discretização}

Para integrar $\Gl$, faremos separadamente a operação sobre $\Gl_{\Hf}$ e $\Gl_{\U}$. A dinâmica hamiltoniana é reversível, o que é importante no algoritmo para garantir que mantém-se a medida invariante. Ainda mais, preserva o volume do espaço de fase, de forma que não precisamos calcular o jacobiano da matriz que define a transformação da dinâmica. Essas duas propriedades podem ser mantidas quando discretizada a dinâmica pelo método de Verlet \cite{Chafa2018}\cite{leimmolecular}. A dinâmica deveria também manter o Hamiltoniano constante, contudo, discretizada, podemos garantir somente que ele se mantenha quase constante. Para lidar com esse fato, discute-se a implementação de um passo de Metropolis na próxima seção. Para $\Delta t > 0$, a partir do estado $(q_k, p_k)$, o esquema lê-se
\begin{equation}
\begin{cases}
	\tilde{p}_{k+\frac{1}{2}} = \tilde{p}_k - \nabla \Hf_N(q_k) \alpha_N \frac{\Delta t}{2}, \\
	\tilde{q}_{k+1} = q_k + \tilde{p}_{k + \frac{1}{2}} \alpha_N \Delta t, \\
	\tilde{p}_{k+1} = \tilde{p}_{k+\frac{1}{2}} - \nabla \Hf_N(q_{k+1}) \alpha_N \frac{\Delta t}{2}.
\end{cases}
\label{Equation: Verlet}
\end{equation}
Um esquema análogo é possível para energias cinéticas generalizadas \cite{Stoltz2018}. Outros métodos tais quais \textit{Euler-Maruyama} (EM) \cite[Capítulo~7]{leimmolecular} podem ser utilizados para o mesmo fim. Nos método que temos interesse o erro associado à discretização deve ir à zero quando $\Delta t$ vai à zero. Para EM, o erro por passo, local, é da ordem de $\Boh{(\Delta t^2)}$ e o erro final, global, $\Boh{(Delta t)}$, Já para o esquema escolhido, temos erro local de  $\Boh{(\Delta t^3)}$ e global de  $\Boh{(\Delta t^2)}$. Essa diferença vem do fato da discretização usada ser reversível \cite[Capítulo~5]{handbookmontecarlo}. 

Nos resta integrar $\Gl_{\U}$, o qual, para a energia cinética usual, consiste em um processo de Ornstein-Uhlenbeck de variância explícita $$dx_t = - \xi x_t dt + \sigma dB_t$$ onde $\xi, \sigma > 0$ são parâmetros e $B_t$ é processo browniano. Note que para $\alpha > 0$ substituiremos parcialmente o momento das variáveis, se $\alpha = 0$ retomaríamos \ref{Equação: Langevin Overdamped}. Este processo não é muito melhor, contudo, do que um \textit{Random Walk Metropolis} \cite[Capítulo~5]{handbookmontecarlo} já que o momento seria completamente substituído. Este processo pode ser resolvido a partir da fórmula de Mehler e obtêm-se
\begin{equation}
\tilde{p}_k = \eta p_k + \sqrt{\frac{1-\eta^2}{\beta_N}} G_k, \ \ \ \eta = \ee^{-\gamma_N \alpha_N \Delta t}.
\label{Equation: Mehler}
\end{equation}
Onde $G_k$ é variável aleatória Gaussiana usual.



% -
% C2S3 - Discretização
% - 

\section{Passo de Seleção - Metropolis}

Muitos algoritmos utilizam  de um passo de seleção para estabilizar sua dinâmica e otimizar a convergência e a amostragem da variável de interesse. Dentre eles citemos por exemplo o \textit{Metropolis-Adjusted Langevin Algorith} (MALA) \cite[Anexo~C]{leimmolecular}. Outros métodos com a amenização do processo também teriam efeito sobre a estilização da dinâmica. Para o método de metrópolis, um importante aspecto é manter a quantidade de rejeições baixa para não atrapalhar a eficiência do programa, o que influencia no tamanho do passo temporal decidido. Apesar disso, seu uso pode levar à descrita melhor estabilidade numérica.

Usaremos um esquema de Metropolis onde procuraremos, definindo uma probabilidade de aceite de cada atualização, evitar passos redundantes ou irrelevantes. Propõe-se então que, a partir da atualização para a posição $\tilde{q}_{k+1}$ \cite{Chafa2018}, se calcule a probabilidade
\begin{equation}
P_k = 1 \wedge \frac{\K(\tilde{q}_{k+1}, q_k) \ee^{-\beta_N \Hf_N(\tilde{q}_{k+1})}}{\K(q_k, \tilde{q}_{k+1}) \ee^{-\beta_N \Hf_N(q_{k})}},
\label{Equation: Pk}
\end{equation}
onde o núcleo $K(x, y)$ é simétrico \cite{Chafa2018} para o caso do \textit{Hybrid Monte Carlo} e, por se cancelar, não será discutido adiante. Atribua agora às coordenadas generalizadas $(q_{k+1}, p_{k+1})$ valor da seguinte forma
\begin{equation}
	(q_{k+1}, p_{k+1}) =
\begin{cases}
	(\tilde{q}_{k+1}, \tilde{p}_{k+1}) \ \text{com probabilidade} \ P_k, \\
	(q_k, -\tilde{p}_{k}) \ \text{com probabilidade} \ 1-P_k; \\
\end{cases}
\label{Equation: Metropolis}
\end{equation}
De forma a garantir a conservação da energia para o sistema e otimizar a exploração do espaço de fase.





% ---
% Capítulo 3 -Implementação e Resultados
% ---

\chapter{Implementação e Resultados}
\label{Capitulo: Resultados}

 Simular Gases de Coulomb é especialmente interessante quando não há modelos de Matrizes conhecidos ou disponíveis para o $\Hf$ escolhido. Escrevemos então um programa que faz uso da interpretação dos Gases de Coulomb para simular medidas de interesse em RMT. Consideraremos nossas partículas em um subespaço $S$ de dimensão $d$ em $\mathbb{R}^N$ de forma que nosso espaço de fase $\Omega$ será de dimensão $dN$. O campo externo será denominado $V : S \mapsto \mathbb{R}$ e o núcleo de interação entre as partículas $\W : S \mapsto (-\infty, \infty]$. Reunindo os resultados do capítulo passado sob essas condições, temos, por completeza, o algoritmo completo descrito em \cite{Chafa2018}. Dada uma condição inicial $(q_k, p_k)$, para cada $k\geq0$, realizamos os seguintes passos
\begin{enumerate}
	\item Baseado em \ref{Equation: Mehler}, atualize as velocidades com
	\begin{equation}
	\tilde{p}_k = \eta p_k + \sqrt{\frac{1-\eta^2}{\beta_N}} G_k, \ \eta = \ee^{-\gamma_N \alpha_N \Delta t};
	\label{Equation: Alg Mehler}
	\end{equation}
	\item Utilizando do esquema de \ref{Equation: Verlet}, calcule os termos
	\begin{equation}
	\begin{cases}
		\tilde{p}_{k+\frac{1}{2}} = \tilde{p}_k - \nabla H_N(q_k) \alpha_N \frac{\Delta t}{2}, \\
		\tilde{q}_{k+1} = q_k + \tilde{p}_{k + \frac{1}{2}} \alpha_N \Delta t, \\
		\tilde{p}_{k+1} = \tilde{p}_{k+\frac{1}{2}} - \nabla H_N(q_{k+1}) \alpha_N \frac{\Delta t}{2};
		\label{Equation: Alg Verlet}
	\end{cases}
	\end{equation}
	\item Pela definição \ref{Equation: Pk}, tome
	\begin{equation}
	P_k = 1 \wedge \exp{\left[ -\beta_N \left(  H_N(\tilde{q}_{k+1}) + \frac{\tilde{p}^2_{k+1}}{2} - H_N(q_k) - \frac{\tilde{p}^2_k}{2} \right)\right] };
	\label{Equação: Alg Pk}
	\end{equation}
	\item Defina, a partir de \ref{Equation: Metropolis}, 
	\begin{equation}
	(q_{k+1}, p_{k+1}) = 
	\begin{cases}
		(\tilde{q}_{k+1}, \tilde{p}_{k+1}) \ \text{com probabilidade} \ P_k, \\
		(q_k, -\tilde{p}_{k}) \ \text{com probabilidade} \ 1-P_k; \\
	\end{cases}
	\label{Equation: Alg Metro}
	\end{equation}
\end{enumerate}
Resta agora apresentar a implementação e os resultados obtidos.

% -
% C3S1 - A implementação
% - 
\section{A implementação}

Restringiremos o subespaço $\Se$ à $\R$ tal que $q_i \in \R$. Isso vem do fato de que estamos, nesse trabalho, interessados na simulação de partículas (ou autovalores) reais. Casos em mais dimensões são igualmente de interesse na teoria e o leitor interessado pode se referir à \cite{tao2008random}, por exemplo. Consideraremos ainda um núcleo de interação $\W = g$ coulombiano. Por isso, retomamos medida da forma \ref{Equação: Medida Gas de Coulomb} usual de gases de coulomb. A esquemática da implementação se encontra na Figura \ref{Figura: Implementação}. Podemos entender melhor a relação entre as sub-rotinas e funções em referência à Tabela \ref{Table: Funcoes e Subrotinas}.

\begin{figure}[ht]
	\centering
	\begin{tikzpicture}[font=\small,thick]
		
		% Start block
		\node[subrotina] (INIT) {INIT};
		
		% -------------------------------------------------------------------		
		
		\node[subrotina,
		left=0.7cm of INIT] (LabelSubrotina) {Subrotinas};
		
		\node[funcao,
		below=0.1cm of LabelSubrotina] (LabelFunção) {Funções};
		
		% -------------------------------------------------------------------		
		
		\node[funcao,
		below=0.5cm of INIT, xshift=2cm] (Hold) {H};
		
		\node[funcao,
		right=0.5cm of Hold, yshift=0.5cm] (Wold) {W};
		
		\node[funcao,
		right=0.5cm of Hold, yshift=-0.5cm] (Vold) {V};
		
		
		\node[loop,
		below=2cm of INIT,
		minimum width=6cm,
		xshift=2cm,
		] (LOOP) {
			\begin{tikzpicture}
				
				\node[subrotina,
				] (L2) {L2-OrnsUhlen};
				
				\node[funcao,
				below=0.5cm of L2
				] (Gauss) {Gauss};
				
				\node[subrotina,
				right=2cm of L2] (L1) {L1-Verlet};
				
				\node[subrotina,
				below=0.5cm of L1] (GradH) {GradH};
				
				\node[subrotina,
				below=0.5cm of GradH, xshift=1cm] (GradW) {GradW};
				
				\node[subrotina,
				below=0.5cm of GradH, xshift=-1cm] (GradV) {GradV};
								
				\node[subrotina,
				below=4cm of L2, xshift=-0.5cm] (Metro) {Metropolis};
				
				\node[funcao,
				below=0.5cm of Metro
				] (Problog) {ProbLog};
				
				\node[funcao,
				right=1cm of Problog] (H) {H};
				
				\node[funcao,
				right=0.5cm of H, yshift=0.5cm] (W) {W};
				
				\node[funcao,
				right=0.5cm of H, yshift=-0.5cm] (V) {V};
				
				\node[random,
				above=0.5cm of Metro, xshift=-1.3cm] (aceito) {$q_k = \tilde{q}_{k_1}$ \\ $p_k = \tilde{p}_{k_1}$};
				
				\node[random,
				above=0.5cm of Metro, xshift=1.3cm] (negado) {$q_k = q_k$ \\ $p_k = -p_k$};
				
				
				% ---------------------------------------------------------------------
				
				\path [fluxo] (L2) -- (L1);
				\path [fluxo]  (L2) ++(-3cm, 0cm) -- (L2);
				\path [chamada] (L2) -- (Gauss);
				\path [chamada] (L1) -- (GradH);
				\path [chamada] (GradH) -- (GradV);
				\path [chamada] (GradH) -- (GradW);
				\path [fluxo]  (L1) --++(3cm, 0cm) |- (Metro);
				\path [chamada] (Metro) -- (Problog);
				\path [chamada] (Problog) -- (H);
				\path [chamada] (H) -- (W);
				\path [chamada] (H) -- (V);
				\path [meiofluxo] (Metro) -- (aceito);
				\path [meiofluxo] (Metro) -- (negado);
				\path [meiofluxo] (negado) -- ++(0cm, 1.5cm) -- ++(-2.6cm, 0cm);
				\path [meiofluxo] (aceito) -- ++(0cm, 1.45cm);
				\path [fluxo] (aceito)++(0cm, 1.45cm) -- ++(0cm, 1.75cm);
				
			\end{tikzpicture}
		};
	
		\node[random,
		left=0.3cm of LOOP,
		yshift=2cm,
		rotate=90
		] (do) {DO k = 1, nsteps};
		
		\path [fluxo] (INIT) -- ++(0cm, -2.3cm);
		\path [chamada] (INIT) ++(0cm, -1.1cm) -- (Hold);
		\path [chamada] (Hold) -- (Vold);
		\path [chamada] (Hold) -- (Wold);
		
	\end{tikzpicture}
\caption{Implementação do algoritmo \textit{Langevin Monte Carlo} (LMC). Setas sólidas indicam o fluxo do programa. Setas tracejadas indicam chamadas de funções dentro do bloco. A descrição das funções se encontra na Tabela \ref{Table: Funcoes e Subrotinas}.}
\label{Figura: Implementação}
\end{figure}

\begin{table}[ht]
	\centering
	\begin{tabular}{ |p{2.6cm}||p{12cm}|  }
		\hline
		\multicolumn{2}{|c|}{Lista de Funções e Subrotinas} \\
		\hline
		\hline
		Nome & Descrição \\ 
		\hline
		\hline
		Init   		  	 & 
		Modifica ${p}_{k}$ vetor $[N\cross m]$, global, uniforme no cubo em $R^d$ e ${q}_{k}, G_H$, vetores $[N\cross m]$, globais, nulos. \\
		\hline
		L1-OrnsUhlen 	 & 
		Modifica $\tilde{p}_k$, vetor $[N\cross m]$, global, por $\Gl_U$ segundo \ref{Equation: Alg Mehler}. \\
		\hline
		L2-Verlet  	 	 & 
		Modifica $\tilde{p}_{k_1},\tilde{q}_{k_1}$ vetores $[N\cross m]$, globais, por $\Gl_{\Hf}$ segundo \ref{Equation: Alg Verlet}.	\\
		\hline
		GradH         	 & 
		Modifica $G_H$, vetor $[N\cross m$], global, gradiente do Hamiltoniano.					\\
		\hline
		GradW        	 &
		Modifica $G_{W_i}$, escalar, global, gradiente de $W$ núcleo de interação.	\\
		\hline
		GradV  	      	 &
		Modifica $G_{V_i}$, escalar, global, gradiente de $\V$ potencial.		                    \\
		\hline
		ProbLog       		 &
		Retorna $P_K$, escalar, local, probabilidade de aceite de \ref{Equação: Alg Pk}. \\
		\hline
		H              	 &
		Retorna $H$, escalar, local, hamiltoniano em $k$.	 							\\
		\hline
		V  	      			 &
		Retorna $V_i$, escalar, local, potencial de $q_i$.								\\
		\hline
		W         	  		 & 
		Retorna $W_{i,j}$, escalar, local, interação entre $q_i,q_j$ 							\\
		\hline
		Metropolis     	 & 
		Modifica ${p}_{k},{q}_{k}$, vetores $[N\cross m]$, globais por \ref{Equation: Alg Metro}.								\\
		\hline
	\end{tabular}
	\caption{ Descrição das funções e subrotinas utilizadas na implementação do programa.}
	\label{Table: Funcoes e Subrotinas}
\end{table}

 Alguns detalhes são importantes de notar. O gerador de variáveis aleatórias gaussianas, necessário em \ref{Equation: Alg Mehler} foi implementado utilizando do algoritmo de \textit{Box-Muller} \cite{NormalVariable}. Para além disso, o ajuste de variáveis é notoriamente um dos aspectos complicados do algoritmo implementado. Precisamos de uma holística par ajustar $\Delta t, \alpha_N \ \text{e} \ \gamma_N$. No escopo do nosso programa, $\Delta t$ e $\alpha_N$ desempenham o mesmo papel e, por isso, tomaremos $\alpha_N = 1$ e decidiremos sobre o valor de $\Delta t$. Seguindo a recomendação de \cite[Capítulo~5]{handbookmontecarlo}, tomaremos $\Delta t = \Delta\tilde{t} + X$, onde $X$ é variável aleatória de média $0$ e variância $\sigma^2$ pequena. Essa escolha ajuda a acelerar a convergência em casos exóticos, que queremos evitar. Lembramos ainda que $\Delta \tilde{t}$ é melhor quando é da ordem de $N^{-\frac{1}{4}}$, isto é, é pequeno o suficiente para manter a razão de aceite do passo de Metropolis alta e grande o suficiente para não desacelerar a convergência do algoritmo. Já $\gamma_N$ definirá o quanto substituiremos o momento anterior das partículas e o quanto utilizaremos do passo aleatório. Aqui, sabemos apenas que tornar $\eta$ próximo demais de $0$, ou de $1$ para todos efeitos, desacelera intensamente a convergência. Faremos com que $\gamma_N \alpha_N \Delta t \approx 0.5$.
 
 %Para além dos ajustes, cada simulação é identificada pelo Hamiltoniano, ou seja, pelo potencial $V$ e pelas dimensões $d, n$, respectivamente do espaço que as partículas estão restritas e do que elas existem.




% -
% C3S2 - Validação em distribuições conhecidas
% - 
\section{Potenciais de medida conhecida}

Podemos validar a execução do programa e qualidade da medida gerada utilizando de potenciais bem descritos na literatura. Para isso, retomaremos os resultados da Seção \ref{Section: Potencias}. Foi comentado que modelos de matrizes aleatórias são úteis em simulações das medidas quando um modelo está disponível. A família de ensembles gaussianos são modelos que mostramos ser bem representados como matrizes em \ref{Section: Ensembles Gaussianos}. Tomar a medida dos ensembles gaussianos é o equivalente na simulação descrita a tomar 
\begin{equation}
d = 1, \ \  n = 2, \ \ \V(x)=\frac{|x|^2}{2}, \ \ W(x) = g(x) = \log{|x|}, \ \ \beta_N = \beta N^2, \ \ \beta = 1,2,4.
\label{Equation: Parametros Gaussian}
\end{equation}
O resultado da simulação para \ref{Equation: Parametros Gaussian} está na Figura \ref{Figura: Gaussian}. Apresentamos ainda na coluna da esquerda os resultados, para $N=10$, da densidade gerada pela simulação equivalente com matrizes para os três modelos ($\beta = 1,2,4$). Na coluna central, representa-se uma comparação com o Semi-Círculo de Wigner, configuração de equilíbrio para os três modelos quando $N$ é grande o suficiente. Note que os valores foram escalados por $\sqrt{2 \beta}$ para melhor visualização. Finalmente, na coluna da direita apresentamos a distribuição do maior autovalor. Um resultado importante  \cite{Tracy} enuncia que existem $z_{N}^{(\beta)}$ e $s_N^{(\beta)}$ tais que $$\lim_{N \to \infty} \mathbb{P}_{\beta,N,V} \left( \frac{\lambda_{max} - z_{N}^{(\beta)}}{s_N^{(\beta)}} \leq x \right) = F_{\beta}(x),$$ onde $F_{\beta}(x)$ é a densidade acumulada de Tracy-Widow. Mostraremos a concordância desse resultado com a simulação na coluna da direita.
\begin{figure}[ht!]
	\includegraphics[width=\textwidth]{Assets/validationGaussianTracy.png}
	\caption{Densidade para ensembles gaussianos, \ref{Equation: Parametros Gaussian}. Tomou-se $\Delta t = 0.3$ e $nsteps = 5\cdot10^6$ passos, registrando a cada $1000$ iterações a partir de $nsteps/5$. À esquerda da figura, em azul, a densidade da amostragem de $4\cdot10^3$ matrizes do ensemble. No centro, o Semi-Círculo de Wigner, medida de equilíbrio. Na direita, apresenta-se a densidade de $\lambda_{max}$ normalizado e sua mediada esperada.}
	\label{Figura: Gaussian}
\end{figure}

Indo além dos modelos gaussianos podemos retomar as descrições dos potenciais mônico em \ref{Equação: Mônico} e as duas situações para o potencial quártico \ref{Equação: Quartico +} e \ref{Equação: Quartico -}. Respectivamente, estes modelos equivalem a tomar na simulação os parâmetros
\begin{equation}
	d = 1, \ \  n = 2, \ \ \V(x)= t |x|^{2m}, \ \ W(x) = g(x) = \log{|x|}, \ \ \beta_N = \beta N^2, \ \ \beta = 2.
	\label{Equation: Parametros Monico}
\end{equation}
\begin{equation}
	d = 1, \ \  n = 2, \ \ \V(x)=\frac{|x|^4}{4} + t \frac{|x|^2}{2}, \ \ W(x) = g(x) = \log{|x|}, \ \ \beta_N = \beta N^2, \ \ \beta = 2.
	\label{Equation: Parametros Quartico}
\end{equation}
O caso mônico se reduz ao gaussiano se $m=1$. Os resultados para ambos os potenciais estão explicitados na Figura \ref{Figura: Quartic Monic} para alguns parâmetros interessantes de $t$ e $m$.
\begin{figure}[ht!]
	\includegraphics[width=\textwidth]{Assets/validationQuarticMonic-alt.png}
	\caption{Potencial Quártico \ref{Equation: Parametros Quartico} e Mônico \ref{Equation: Parametros Monico}, respectivamente à esquerda e direita. Tomou-se $\Delta t = 0.1$, $N=100$, e $nsteps = 5\cdot10^6$ passos. Registra-se a cada $1000$ iterações a partir de $nsteps/5$. No Quártico, simula-se $t=-1,-2,-3$. No Mônico fixa-se $t=1$ e simula-se $m=1,3,5$.}
	\label{Figura: Quartic Monic}
\end{figure}



% -
% C3S3 - Outros Potenciais?
% - 
%\input{USPSC-TA-Textual/C3-ImplementacaoEResultados/OutrosPotenciais.tex}


% ---
% Capítulo 4 - Conclusão
% ---

\chapter{Conclusão}
\label{Capitulo: Conclusão}

\lipsum[2-5]



% ----------------------------------------------------------
% ELEMENTOS PÓS-TEXTUAIS
% ----------------------------------------------------------
\postextual
% ----------------------------------------------------------

% -----------------------------------------------------------
% Referências bibliográficas
% ----------------------------------------------------------
\bibliography{USPSC-bib/USPSC-modelo-references}


% ----------------------------------------------------------
% Glossário
% ----------------------------------------------------------
%
% Consulte o manual da classe abntex2 para orientações sobre o glossário.
%
%\glossary

% ----------------------------------------------------------
% Apêndices
% ----------------------------------------------------------
%\include{USPSC-TA-PosTextual/USPSC-Apendices}

% ----------------------------------------------------------
% Anexos
% ----------------------------------------------------------
%\include{USPSC-TA-PosTextual/USPSC-Anexos}

%---------------------------------------------------------------------
% INDICE REMISSIVO
%--------------------------------------------------------------------
%\include{USPSC-TA-PosTextual/USPSC-IndicesRemissivos}

%---------------------------------------------------------------------

\end{document}
