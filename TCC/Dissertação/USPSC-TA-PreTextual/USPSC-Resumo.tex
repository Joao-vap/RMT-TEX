%% USPSC-Resumo.tex
\setlength{\absparsep}{18pt} % ajusta o espaçamento dos parágrafos do resumo		
\begin{resumo}
	\begin{flushleft} 
			\setlength{\absparsep}{0pt} % ajusta o espaçamento da referência	
			\SingleSpacing 
			\imprimirautorabr~~\textbf{\imprimirtituloresumo}.	\imprimirdata. \pageref{LastPage}p. 
			%Substitua p. por f. quando utilizar oneside em \documentclass
			%\pageref{LastPage}f.
			\imprimirtipotrabalho~-~\imprimirinstituicao, \imprimirlocal, \imprimirdata. 
 	\end{flushleft}
\OnehalfSpacing 			
O estudo de Teoria de Matrizes Aleatórias (RMT) demonstra aplicabilidade em uma gama diversa de áreas, tanto em matemática quanto na física e em computação. Em matemática sua influência se estende de análise complexa à teoria dos números e combinatória. Em computação, sistemas de aprendizado de máquina podem se beneficiar do desenvolvimento da teoria. Finalmente, em física, de alguma forma, distribuições descritas pela RMT capturam o comportamento de grandezas de interesse em sistema de alta complexidade, em especial aqueles que surgem em mecânica estatística, mecânica quântica e em teorias de caos. Substitui-se o estudo analítico microscópico de desenvolvimento destes sistemas por ensembles de matrizes com grandezas médias macroscópicas descritivas da dinâmica. Tanto o é importante, que os anais da teoria se encontram no estudo de níveis energéticos em núcleos atômicos de átomos pesados, pelo estudo do espectro da matriz aleatória representativa do Hamiltoniano do sistema. Em suma, RMT está em uma interessante posição de interdisciplinaridade - por descrever de algum forma uma lei universal de sistemas com complexidade e ou aleatoriedade - e seu estudo se aproveita de recursos de múltiplas áreas se tornando terreno fértil para desenvolvimento matemático-científico. O forte paralelo com mecânica estatística não é coincidência e, particularmente, estudando a densidade espectral de sistemas de matrizes Gaussianas pode-se desenvolver, analogamente, um estudo de sistemas de gases coulombianos. Algumas dificuldades surgem na implementação de simulações baseadas nesta teoria, principalmente na escalabilidade do sistema e no tratamento de possíveis singularidades. Para este tipo de simulação, e dificuldades, muitos algoritmos foram desenvolvidos. Tomaremos especial interesse nos métodos numéricos mais comuns da área, por exemplo, por Algoritmos Híbridos de Monte Carlo que demonstram consistentemente bom desempenho numérico para a simulações de gases iônicos. Utilizando a literatura atual e os algoritmos respectivamente descritos, procurar-se-á desenvolver sólida base teórica e replicar o desempenho e precisão dos métodos disponíveis - e de interesse. Espera-se ainda, poder explorar e estender, em algum nível, os resultados das simulações de gases e relacionados para condições distintas de hamiltoniano, explorando diferentes situações físico-matemáticas.
 

 \textbf{Palavras-chave}: Gases de Coulomb. Matrizes Aleatórias. 
\end{resumo}