%% USPSC-Resumo.tex
\setlength{\absparsep}{18pt} % ajusta o espaçamento dos parágrafos do resumo		
\begin{resumo}
	\begin{flushleft} 
			\setlength{\absparsep}{0pt} % ajusta o espaçamento da referência	
			\SingleSpacing 
			\imprimirautorabr~~\textbf{\imprimirtituloresumo}.	\imprimirdata. \pageref{LastPage}p. 
			%Substitua p. por f. quando utilizar oneside em \documentclass
			%\pageref{LastPage}f.
			\imprimirtipotrabalho~-~\imprimirinstituicao, \imprimirlocal, \imprimirdata. 
 	\end{flushleft}
\OnehalfSpacing 	
		
O estudo do espectro de matrizes aleatórias demonstra aplicabilidade em uma gama diversa de áreas da física, matemática à computação e engenharia. Estaremos interessados em estudar os principais ensembles da Teoria de Matrizes Aleatórias e suas medidas de equilíbrio, entender a analogia com Gases de Coulomb e, com essa ferramenta, realizar simulações que nos permitam calcular médias de funções de interesse. Discutiremos quais métodos são importantes para a simulação do problema de gases de coulomb e quais suas limitações além das impostas pelo problema de escalabilidade e de singularidades. Nos resultados, mostramos que o método de \textit{Langevin Monte Carlo} tem boa performance e consegue replicar as medidas de modelos conhecidos, e um menos discutido, de matrizes aleatórias, com boa precisão. 

 \textbf{Palavras-chave}: Gases de Coulomb. Matrizes Aleatórias. 
\end{resumo}