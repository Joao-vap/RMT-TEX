%% USPSC-Resumo.tex
\setlength{\absparsep}{18pt} % ajusta o espaçamento dos parágrafos do resumo		
\begin{resumo}
	\begin{flushleft} 
			\setlength{\absparsep}{0pt} % ajusta o espaçamento da referência	
			\SingleSpacing 
			\imprimirautorabr~~\textbf{\imprimirtituloresumo}.	\imprimirdata. \pageref{LastPage}p. 
			%Substitua p. por f. quando utilizar oneside em \documentclass
			%\pageref{LastPage}f.
			\imprimirtipotrabalho~-~\imprimirinstituicao, \imprimirlocal, \imprimirdata. 
 	\end{flushleft}
\OnehalfSpacing 	
		
O estudo de Matrizes Aleatórias demonstra aplicabilidade em uma gama diversa de áreas, com destaque no estudo de mecânica estatística, principalmente na simulação de gases. Estudando a densidade espectral de sistemas de matrizes Gaussianas pode-se desenvolver uma analogia que possibilita a simulação de sistemas de gases diversos, como o de Coulomb. Algumas dificuldades surgem na implementação de simulações baseadas nesta teoria, principalmente em escalabilidade do sistema e no tratamento de possíveis singularidades. Para resolver estes problemas, abordou-se na simulação na literatura, dentre outros, o Algoritmo Híbrido de Monte Carlo, de ótimo comportamento numérico. Nosso objetivo é explorar este assunto, as simulações de gases e o algoritmo citado acima além de expandir os potenciais em que foi-se bem documentado o comportamento destas simulações.
 

 \textbf{Palavras-chave}: Gases de Coulomb. Matrizes Aleatórias. 
\end{resumo}