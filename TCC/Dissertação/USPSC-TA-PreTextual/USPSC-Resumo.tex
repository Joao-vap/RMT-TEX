%% USPSC-Resumo.tex
\setlength{\absparsep}{18pt} % ajusta o espaçamento dos parágrafos do resumo		
\begin{resumo}
	\begin{flushleft} 
			\setlength{\absparsep}{0pt} % ajusta o espaçamento da referência	
			\SingleSpacing 
			\imprimirautorabr~~\textbf{\imprimirtituloresumo}.	\imprimirdata. \pageref{LastPage}p. 
			%Substitua p. por f. quando utilizar oneside em \documentclass
			%\pageref{LastPage}f.
			\imprimirtipotrabalho~-~\imprimirinstituicao, \imprimirlocal, \imprimirdata. 
 	\end{flushleft}
\OnehalfSpacing 	
		
		
O estudo do espectro de matrizes é peça fundamental na descrição de sistemas físicos, dentre outras razões, isso se deve ao fato que a representação matricial das equações diferenciais que regem a mecânica do sistema codificam também suas características intrínsecas. Contudo, para sistemas com suficiente complexidade, estudar a dinâmica deterministicamente é contraproducente; quando se sabe montar os operadores, suas soluções são, em geral, instáveis. Uma abordagem alternativa nos indica à Teoria de Matrizes Aleatórias (RMT, \textit{Random Matrix Theory}), que, sob as devidas hipóteses, permite caracterizar estatisticamente as propriedades físicas do sistema considerado. A relevância dos métodos de matrizes aleatórias não se restringe, contudo, à física, fazendo aparições na descrição dos zeros da função zeta de Riemann, em modelos de correlação no mercado financeiro e em inúmeras outras aplicações. Descrevemos alguns dos principais conceitos em RMT, como ensembles e as medidas de matrizes aleatórias, dos quais tratamos com algum detalhe os chamados ensembles invariantes. Com a natural analogia de Gases de Coulomb, descrevemos o comportamento dos autovalores destas matrizes como gases de partículas interagentes e intuímos a noção de medidas de equilíbrio no limite termodinâmico, que explicitamos para alguns ensembles. Com essa ferramenta, a teoria de simulação de moléculas previamente desenvolvida para condições similares nos permite calcular médias de funções de interesse. Discutimos um pouco sobre a metodologia utilizada e algumas de suas alternativas e limitações. Dando atenção para as duas mais salientes, a escalabilidade do problema e suas singularidades. Mostramos que o método de \textit{Langevin Monte Carlo} tem bom desempenho e possibilita a réplica de medidas na reta para ensembles conhecidos. Além disso, em extensões dimensionais do potencial e do espaço da simulação podemos afirmar, com boa segurança, a possibilidade de replicar resultados apenas recentemente explorados na literatura. Com isso, indica-se uma alternativa numérica para a descrição qualitativa de uma ampla gama de modelos de interesse.

 \textbf{Palavras-chave}: Matrizes Aleatórias. Gases de Coulomb. Dinâmica de Langevin.
\end{resumo}