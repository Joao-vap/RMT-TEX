%% USPSC-Resumo.tex
\setlength{\absparsep}{18pt} % ajusta o espaçamento dos parágrafos do resumo		
\begin{resumo}
	\begin{flushleft} 
			\setlength{\absparsep}{0pt} % ajusta o espaçamento da referência	
			\SingleSpacing 
			\imprimirautorabr~~\textbf{\imprimirtituloresumo}.	\imprimirdata. \pageref{LastPage}p. 
			%Substitua p. por f. quando utilizar oneside em \documentclass
			%\pageref{LastPage}f.
			\imprimirtipotrabalho~-~\imprimirinstituicao, \imprimirlocal, \imprimirdata. 
 	\end{flushleft}
\OnehalfSpacing 	
		
O estudo do espectro de matrizes aleatórias demonstra aplicabilidade em uma gama diversa de áreas da física, matemática, computação e engenharia. Estamos interessados em estudar os principais ensembles da Teoria de Matrizes Aleatórias, entender a analogia com Gases de Coulomb e suas medidas de equilíbrio no limite termodinâmico. Com essa ferramenta, realiza-se simulações que nos permitem calcular médias de funções de interesse. Discutimos ainda quais métodos são importantes para a simulação do problema de Gases de Coulomb  considerado e quais suas limitações além das naturalmente impostas pela escalabilidade e singularidade do problema. Mostramos que o método de \textit{Langevin Monte Carlo} tem boa performance, possibilitando a réplica de medidas para modelos em uma dimensão bem descritos e, ainda, em extensões de potencial e dimensão recentemente exploradas na literatura.

 \textbf{Palavras-chave}: Gases de Coulomb. Matrizes Aleatórias. 
\end{resumo}