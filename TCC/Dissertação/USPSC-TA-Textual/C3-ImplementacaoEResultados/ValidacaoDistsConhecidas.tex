\section{Resultados e Discussão}

Simular gases de coulomb é especialmente interessante quando não há modelos de matrizes conhecidos, disponíveis ou simples para o $\Hf$ definido. Podemos, com a simulação de tais gases, calcular a média da função densidade das partículas, ou autovalores. Alternativamente, quando há modelos disponíveis em matrizes aleatórias, essa medida poderia ser tirada diretamente do calculo de seus autovalores.

A família de ensembles gaussianos são modelos que mostramos ser bem representados como matrizes na Seção \ref{Section: Ensembles Gaussianos}. Retorne os resultados da Seção \ref{Section: Potencias}. Tomar a medida dos ensembles gaussianos é o equivalente, na simulação descrita, a tomar 
\begin{equation}
d = 1; \ \  n = 2; \ \ \V(x)=\frac{|x|^2}{2}; \ \ W(x) = g(x) = \log{|x|}; \ \ \beta_N = \beta N^2; \ \ \beta = 1,2,4.
\label{Equation: Parametros Gaussian}
\end{equation}
O resultado da simulação para a configuração \ref{Equation: Parametros Gaussian} é apresentado na Figura \ref{Figura: Gaussian}. Apresentamos por contraste, na coluna da esquerda, os resultados para $N=10$, da densidade gerada pela simulação equivalente com matrizes e pelos gases para os três modelos ($\beta = 1,2,4$). Na coluna central, representa-se a comparação da medida da simulação com o Semi-Círculo de Wigner, configuração de equilíbrio para os três modelos quando $N$ é grande o suficiente. Note que os valores foram escalados por $\sqrt{2 \beta}$. Finalmente, na coluna da direita apresentamos a distribuição do maior autovalor $\lambda_{max}$. Um resultado importante  \cite{Tracy} enuncia que existem $z_{N}^{(\beta)}$ e $s_N^{(\beta)}$ tais que $$\lim_{N \to \infty} \mathbb{P}_{\beta,N,V} \left( \frac{\lambda_{max} - z_{N}^{(\beta)}}{s_N^{(\beta)}} \leq x \right) = F_{\beta}(x),$$ onde $F_{\beta}(x)$ é a densidade acumulada de Tracy-Widow. Mostraremos a concordância desse resultado com a simulação na coluna da direita. Observa-se que os dois modelos à esquerda concordam bem na estimativa da medida para o $N$ usado. No centro, é possível notar que a medida de equilíbrio esperada, o Semi-Círculo de Wigner, é aproximada rapidamente pelo aumento de partículas no sistema. A distribuição do autovalor máximo é mais delicada, contudo, ainda apodemos ver boa correspondência com o resultado esperado pela Tracy-Widow.
\begin{figure}[ht!]
	\centering
	\includegraphics[width=\textwidth]{Assets/validationGaussianTracy.png}
	\caption{Densidade para ensembles gaussianos, \ref{Equation: Parametros Gaussian}. Tomou-se $\Delta t = 0.5$ e $nsteps = 5\cdot10^6$ passos, registrando a cada $1000$ iterações a partir de $nsteps/5$. À esquerda da figura, em azul, a densidade da amostragem de $4\cdot10^3$ matrizes do ensemble. No centro, o Semi-Círculo de Wigner, medida de equilíbrio. Na direita, apresenta-se a densidade de $\lambda_{max}$ normalizado e sua medida esperada.}
	\label{Figura: Gaussian}
\end{figure}

Podemos retomar também as descrições dos potenciais mônico em \ref{Equação: Mônico} e os dois regimes do potencial quártico, \ref{Equação: Quartico +} e \ref{Equação: Quartico -}. Respectivamente, estes modelos equivalem a tomar na simulação os parâmetros
\begin{equation}
	d = 1; \ \  n = 2; \ \ \V(x)= t |x|^{2m}; \ \ W(x) = g(x) = \log{|x|}; \ \ \beta_N = \beta N^2; \ \ \beta = 2.
	\label{Equation: Parametros Monico}
\end{equation}
\begin{equation}
	d = 1; \ \  n = 2; \ \ \V(x)=\frac{|x|^4}{4} + t \frac{|x|^2}{2}; \ \ W(x) = g(x) = \log{|x|}; \ \ \beta_N = \beta N^2; \ \ \beta = 2.
	\label{Equation: Parametros Quartico}
\end{equation}
O caso mônico se reduz ao gaussiano se $m=1$. Os resultados para ambos os potenciais estão explicitados na Figura \ref{Figura: Quartic Monic} para alguns parâmetros interessantes de $t$ e $m$.
\begin{figure}[ht!]
	\centering
	\includegraphics[width=0.95\textwidth]{Assets/validationQuarticMonic-alt.png}
	\caption{Potencial Quártico \ref{Equation: Parametros Quartico} e Mônico \ref{Equation: Parametros Monico}, respectivamente à esquerda e direita. Tomou-se $\Delta t = 0.1$, $N=100$, e $nsteps = 5\cdot10^6$ passos. Registra-se a cada $1000$ iterações a partir de $nsteps/5$. No Quártico, simula-se $t=-1,-2,-3$. No Mônico fixa-se $t=1$ e simula-se $m=1,3,5$.}
	\label{Figura: Quartic Monic}
\end{figure}

Novamente as medidas experimentais parecem convergir para a medida teórica enunciada em todas as configurações testadas. Contudo, isso é explorado e pode ser observado igualmente, com exceção do Mônico, em \cite{Chafa2018}. Em uma situação menos explorada, considere a seguinte configuração de potencial e autovalores complexos ($\R^d = \R^2$) e a representação das medidas simuladas para alguns valores de interesse de $t, a$ na Figura \ref{Figura: Complex},
\begin{equation}
	d = 2; \  n = 2; \  \V(z)=|z|^{2a} - \Re{t z^a};  \ W(x) = g(x) = \log{|x|};  \ \beta_N = \beta N^2;  \ \beta = 2.
	\label{Equation: Complex}
\end{equation}

\begin{figure}[ht]
	\centering
	\includegraphics[width=0.9\textwidth]{Assets/complexPotential.png}
	\caption{Medidas referentes à configuração \ref{Equation: Complex}. Tomou-se $\Delta t = 0.5$ e $nsteps = 2\cdot10^6$ passos, registrando a cada $500$ iterações a partir de $nsteps/5$.}
	\label{Figura: Complex}
\end{figure}

É previsto para esse modelo uma transição de regime, uma separação da medida de equilíbrio, para $t_c \approx \sqrt{\frac{1}{a}}$ \cite{balogh2016orthogonal}. A simulação replicar o comportamento esperado é um bom indicador de que é possível estudar a medida de tal ensemble numericamente, que pode ser explorado posteriormente. Outro fator que corrobora o bom comportamento do modelo é a medida uniforme quando $a=1$, também prevista pela teoria. Esse exemplo demonstra que é possível sem muito esforço replicar a medida, e principalmente o suporte, para potenciais mais complexos estudados em publicações recentes no tema. 

No Capítulo \ref{Capitulo: Intro} é notado que os modelos gaussianos são os únicos em RMT com invariância por rotação e independência das entradas simultaneamente. Gerar matrizes de outros modelos invariantes dependeria de ser capaz de construir entradas correlacionadas devidamente ja que, se tratando de ensembles invariantes, ou seja, de medida igual para quaisquer $M, M'$ tais que $\matriz{M} = \matriz{U} \matriz{M'} \matriz{U}^{-1}$, podemos simular $\matriz{U}$ autovetores uniformemente do espaço correspondente. Isso pois sabemos do teorema espectral que, para as matrizes tomadas, vale a decomposição $\matriz{M} = \matriz{U} \matriz{D} \matriz{U}^{-1}$. Para reconstruir uma elemento do ensemble de interesse nos resta replicar a medida de autovalores, $\matriz{D}$. Isso, de forma interessante, pode ser feito pela simulação descrita de Gases de Coulomb, que replica a medida dos autovalores dos ensemble.

%Outra possibilidade interessante da replicação numérica dessas medidas é que, miniminizada a energia livre $E_{N, V}$, podemos fazer estimativas para constantes da expansão para $\log(Z_{\beta_N})$ proposta em trabalhos recentes, como em \cite{Byun_2023}. Essas estimativas podem dar uma ideia geral do comportamento dessas constantes, de relevante significado físico, para sistemas de interesse.  
