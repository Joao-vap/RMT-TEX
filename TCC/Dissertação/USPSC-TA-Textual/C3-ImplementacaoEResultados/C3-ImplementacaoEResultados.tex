\chapter{Implementação e Resultados}
\label{Capitulo: Resultados}

 Simular Gases de Coulomb é especialmente interessante quando não há modelos de Matrizes conhecidos ou disponíveis para o $\Hf$ escolhido. Escrevemos então um programa que faz uso da interpretação dos Gases de Coulomb para simular medidas de interesse em RMT. Consideraremos nossas partículas em um subespaço $S$ de dimensão $d$ em $\mathbb{R}^N$ de forma que nosso espaço de fase $\Omega$ será de dimensão $dN$. O campo externo será denominado $V : S \mapsto \mathbb{R}$ e o núcleo de interação entre as partículas $\W : S \mapsto (-\infty, \infty]$. Reunindo os resultados do capítulo passado sob essas condições, temos, por completeza, o algoritmo completo descrito em \cite{Chafa2018}. Dada uma condição inicial $(q_k, p_k)$, para cada $k\geq0$, realizamos os seguintes passos
\begin{enumerate}
	\item Baseado em \ref{Equation: Mehler}, atualize as velocidades com
	\begin{equation}
	\tilde{p}_k = \eta p_k + \sqrt{\frac{1-\eta^2}{\beta_N}} G_k, \ \eta = \ee^{-\gamma_N \alpha_N \Delta t};
	\label{Equation: Alg Mehler}
	\end{equation}
	\item Utilizando do esquema de \ref{Equation: Verlet}, calcule os termos
	\begin{equation}
	\begin{cases}
		\tilde{p}_{k+\frac{1}{2}} = \tilde{p}_k - \nabla H_N(q_k) \alpha_N \frac{\Delta t}{2}, \\
		\tilde{q}_{k+1} = q_k + \tilde{p}_{k + \frac{1}{2}} \alpha_N \Delta t, \\
		\tilde{p}_{k+1} = \tilde{p}_{k+\frac{1}{2}} - \nabla H_N(q_{k+1}) \alpha_N \frac{\Delta t}{2};
		\label{Equation: Alg Verlet}
	\end{cases}
	\end{equation}
	\item Pela definição \ref{Equation: Pk}, tome
	\begin{equation}
	P_k = 1 \wedge \exp{\left[ -\beta_N \left(  H_N(\tilde{q}_{k+1}) + \frac{\tilde{p}^2_{k+1}}{2} - H_N(q_k) - \frac{\tilde{p}^2_k}{2} \right)\right] };
	\label{Equação: Alg Pk}
	\end{equation}
	\item Defina, a partir de \ref{Equation: Metropolis}, 
	\begin{equation}
	(q_{k+1}, p_{k+1}) = 
	\begin{cases}
		(\tilde{q}_{k+1}, \tilde{p}_{k+1}) \ \text{com probabilidade} \ P_k, \\
		(q_k, -\tilde{p}_{k}) \ \text{com probabilidade} \ 1-P_k; \\
	\end{cases}
	\label{Equation: Alg Metro}
	\end{equation}
\end{enumerate}
Resta agora apresentar a implementação e os resultados obtidos.

% -
% C3S1 - A implementação
% - 
\section{A implementação}

Restringiremos o subespaço $\Se$ à $\R$ tal que $q_i \in \R$. Isso vem do fato de que estamos, nesse trabalho, interessados na simulação de partículas (ou autovalores) reais. Casos em mais dimensões são igualmente de interesse na teoria e o leitor interessado pode se referir à \cite{tao2008random}, por exemplo. Consideraremos ainda um núcleo de interação $\W = g$ coulombiano. Por isso, retomamos medida da forma \ref{Equação: Medida Gas de Coulomb} usual de gases de coulomb. A esquemática da implementação se encontra na Figura \ref{Figura: Implementação}. Podemos entender melhor a relação entre as sub-rotinas e funções em referência à Tabela \ref{Table: Funcoes e Subrotinas}.

\begin{figure}[ht]
	\centering
	\begin{tikzpicture}[font=\small,thick]
		
		% Start block
		\node[subrotina] (INIT) {INIT};
		
		% -------------------------------------------------------------------		
		
		\node[subrotina,
		left=0.7cm of INIT] (LabelSubrotina) {Subrotinas};
		
		\node[funcao,
		below=0.1cm of LabelSubrotina] (LabelFunção) {Funções};
		
		% -------------------------------------------------------------------		
		
		\node[funcao,
		below=0.5cm of INIT, xshift=2cm] (Hold) {H};
		
		\node[funcao,
		right=0.5cm of Hold, yshift=0.5cm] (Wold) {W};
		
		\node[funcao,
		right=0.5cm of Hold, yshift=-0.5cm] (Vold) {V};
		
		
		\node[loop,
		below=2cm of INIT,
		minimum width=6cm,
		xshift=2cm,
		] (LOOP) {
			\begin{tikzpicture}
				
				\node[subrotina,
				] (L2) {L2-OrnsUhlen};
				
				\node[funcao,
				below=0.5cm of L2
				] (Gauss) {Gauss};
				
				\node[subrotina,
				right=2cm of L2] (L1) {L1-Verlet};
				
				\node[subrotina,
				below=0.5cm of L1] (GradH) {GradH};
				
				\node[subrotina,
				below=0.5cm of GradH, xshift=1cm] (GradW) {GradW};
				
				\node[subrotina,
				below=0.5cm of GradH, xshift=-1cm] (GradV) {GradV};
								
				\node[subrotina,
				below=4cm of L2, xshift=-0.5cm] (Metro) {Metropolis};
				
				\node[funcao,
				below=0.5cm of Metro
				] (Problog) {ProbLog};
				
				\node[funcao,
				right=1cm of Problog] (H) {H};
				
				\node[funcao,
				right=0.5cm of H, yshift=0.5cm] (W) {W};
				
				\node[funcao,
				right=0.5cm of H, yshift=-0.5cm] (V) {V};
				
				\node[random,
				above=0.5cm of Metro, xshift=-1.3cm] (aceito) {$q_k = \tilde{q}_{k_1}$ \\ $p_k = \tilde{p}_{k_1}$};
				
				\node[random,
				above=0.5cm of Metro, xshift=1.3cm] (negado) {$q_k = q_k$ \\ $p_k = -p_k$};
				
				
				% ---------------------------------------------------------------------
				
				\path [fluxo] (L2) -- (L1);
				\path [fluxo]  (L2) ++(-3cm, 0cm) -- (L2);
				\path [chamada] (L2) -- (Gauss);
				\path [chamada] (L1) -- (GradH);
				\path [chamada] (GradH) -- (GradV);
				\path [chamada] (GradH) -- (GradW);
				\path [fluxo]  (L1) --++(3cm, 0cm) |- (Metro);
				\path [chamada] (Metro) -- (Problog);
				\path [chamada] (Problog) -- (H);
				\path [chamada] (H) -- (W);
				\path [chamada] (H) -- (V);
				\path [meiofluxo] (Metro) -- (aceito);
				\path [meiofluxo] (Metro) -- (negado);
				\path [meiofluxo] (negado) -- ++(0cm, 1.5cm) -- ++(-2.6cm, 0cm);
				\path [meiofluxo] (aceito) -- ++(0cm, 1.45cm);
				\path [fluxo] (aceito)++(0cm, 1.45cm) -- ++(0cm, 1.75cm);
				
			\end{tikzpicture}
		};
	
		\node[random,
		left=0.3cm of LOOP,
		yshift=2cm,
		rotate=90
		] (do) {DO k = 1, nsteps};
		
		\path [fluxo] (INIT) -- ++(0cm, -2.3cm);
		\path [chamada] (INIT) ++(0cm, -1.1cm) -- (Hold);
		\path [chamada] (Hold) -- (Vold);
		\path [chamada] (Hold) -- (Wold);
		
	\end{tikzpicture}
\caption{Implementação do algoritmo \textit{Langevin Monte Carlo} (LMC). Setas sólidas indicam o fluxo do programa. Setas tracejadas indicam chamadas de funções dentro do bloco. A descrição das funções se encontra na Tabela \ref{Table: Funcoes e Subrotinas}.}
\label{Figura: Implementação}
\end{figure}

\begin{table}[ht]
	\centering
	\begin{tabular}{ |p{2.6cm}||p{12cm}|  }
		\hline
		\multicolumn{2}{|c|}{Lista de Funções e Subrotinas} \\
		\hline
		\hline
		Nome & Descrição \\ 
		\hline
		\hline
		Init   		  	 & 
		Modifica ${p}_{k}$ vetor $[N\cross m]$, global, uniforme no cubo em $R^d$ e ${q}_{k}, G_H$, vetores $[N\cross m]$, globais, nulos. \\
		\hline
		L1-OrnsUhlen 	 & 
		Modifica $\tilde{p}_k$, vetor $[N\cross m]$, global, por $\Gl_U$ segundo \ref{Equation: Alg Mehler}. \\
		\hline
		L2-Verlet  	 	 & 
		Modifica $\tilde{p}_{k_1},\tilde{q}_{k_1}$ vetores $[N\cross m]$, globais, por $\Gl_{\Hf}$ segundo \ref{Equation: Alg Verlet}.	\\
		\hline
		GradH         	 & 
		Modifica $G_H$, vetor $[N\cross m$], global, gradiente do Hamiltoniano.					\\
		\hline
		GradW        	 &
		Modifica $G_{W_i}$, escalar, global, gradiente de $W$ núcleo de interação.	\\
		\hline
		GradV  	      	 &
		Modifica $G_{V_i}$, escalar, global, gradiente de $\V$ potencial.		                    \\
		\hline
		ProbLog       		 &
		Retorna $P_K$, escalar, local, probabilidade de aceite de \ref{Equação: Alg Pk}. \\
		\hline
		H              	 &
		Retorna $H$, escalar, local, hamiltoniano em $k$.	 							\\
		\hline
		V  	      			 &
		Retorna $V_i$, escalar, local, potencial de $q_i$.								\\
		\hline
		W         	  		 & 
		Retorna $W_{i,j}$, escalar, local, interação entre $q_i,q_j$ 							\\
		\hline
		Metropolis     	 & 
		Modifica ${p}_{k},{q}_{k}$, vetores $[N\cross m]$, globais por \ref{Equation: Alg Metro}.								\\
		\hline
	\end{tabular}
	\caption{ Descrição das funções e subrotinas utilizadas na implementação do programa.}
	\label{Table: Funcoes e Subrotinas}
\end{table}

 Alguns detalhes são importantes de notar. O gerador de variáveis aleatórias gaussianas, necessário em \ref{Equation: Alg Mehler} foi implementado utilizando do algoritmo de \textit{Box-Muller} \cite{NormalVariable}. Para além disso, o ajuste de variáveis é notoriamente um dos aspectos complicados do algoritmo implementado. Precisamos de uma holística par ajustar $\Delta t, \alpha_N \ \text{e} \ \gamma_N$. No escopo do nosso programa, $\Delta t$ e $\alpha_N$ desempenham o mesmo papel e, por isso, tomaremos $\alpha_N = 1$ e decidiremos sobre o valor de $\Delta t$. Seguindo a recomendação de \cite[Capítulo~5]{handbookmontecarlo}, tomaremos $\Delta t = \Delta\tilde{t} + X$, onde $X$ é variável aleatória de média $0$ e variância $\sigma^2$ pequena. Essa escolha ajuda a acelerar a convergência em casos exóticos, que queremos evitar. Lembramos ainda que $\Delta \tilde{t}$ é melhor quando é da ordem de $N^{-\frac{1}{4}}$, isto é, é pequeno o suficiente para manter a razão de aceite do passo de Metropolis alta e grande o suficiente para não desacelerar a convergência do algoritmo. Já $\gamma_N$ definirá o quanto substituiremos o momento anterior das partículas e o quanto utilizaremos do passo aleatório. Aqui, sabemos apenas que tornar $\eta$ próximo demais de $0$, ou de $1$ para todos efeitos, desacelera intensamente a convergência. Faremos com que $\gamma_N \alpha_N \Delta t \approx 0.5$.
 
 %Para além dos ajustes, cada simulação é identificada pelo Hamiltoniano, ou seja, pelo potencial $V$ e pelas dimensões $d, n$, respectivamente do espaço que as partículas estão restritas e do que elas existem.




% -
% C3S2 - Validação em distribuições conhecidas
% - 
\section{Potenciais de medida conhecida}

Podemos validar a execução do programa e qualidade da medida gerada utilizando de potenciais bem descritos na literatura. Para isso, retomaremos os resultados da Seção \ref{Section: Potencias}. Foi comentado que modelos de matrizes aleatórias são úteis em simulações das medidas quando um modelo está disponível. A família de ensembles gaussianos são modelos que mostramos ser bem representados como matrizes em \ref{Section: Ensembles Gaussianos}. Tomar a medida dos ensembles gaussianos é o equivalente na simulação descrita a tomar 
\begin{equation}
d = 1, \ \  n = 2, \ \ \V(x)=\frac{|x|^2}{2}, \ \ W(x) = g(x) = \log{|x|}, \ \ \beta_N = \beta N^2, \ \ \beta = 1,2,4.
\label{Equation: Parametros Gaussian}
\end{equation}
O resultado da simulação para \ref{Equation: Parametros Gaussian} está na Figura \ref{Figura: Gaussian}. Apresentamos ainda na coluna da esquerda os resultados, para $N=10$, da densidade gerada pela simulação equivalente com matrizes para os três modelos ($\beta = 1,2,4$). Na coluna central, representa-se uma comparação com o Semi-Círculo de Wigner, configuração de equilíbrio para os três modelos quando $N$ é grande o suficiente. Note que os valores foram escalados por $\sqrt{2 \beta}$ para melhor visualização. Finalmente, na coluna da direita apresentamos a distribuição do maior autovalor. Um resultado importante  \cite{Tracy} enuncia que existem $z_{N}^{(\beta)}$ e $s_N^{(\beta)}$ tais que $$\lim_{N \to \infty} \mathbb{P}_{\beta,N,V} \left( \frac{\lambda_{max} - z_{N}^{(\beta)}}{s_N^{(\beta)}} \leq x \right) = F_{\beta}(x),$$ onde $F_{\beta}(x)$ é a densidade acumulada de Tracy-Widow. Mostraremos a concordância desse resultado com a simulação na coluna da direita.
\begin{figure}[ht!]
	\includegraphics[width=\textwidth]{Assets/validationGaussianTracy.png}
	\caption{Densidade para ensembles gaussianos, \ref{Equation: Parametros Gaussian}. Tomou-se $\Delta t = 0.3$ e $nsteps = 5\cdot10^6$ passos, registrando a cada $1000$ iterações a partir de $nsteps/5$. À esquerda da figura, em azul, a densidade da amostragem de $4\cdot10^3$ matrizes do ensemble. No centro, o Semi-Círculo de Wigner, medida de equilíbrio. Na direita, apresenta-se a densidade de $\lambda_{max}$ normalizado e sua mediada esperada.}
	\label{Figura: Gaussian}
\end{figure}

Indo além dos modelos gaussianos podemos retomar as descrições dos potenciais mônico em \ref{Equação: Mônico} e as duas situações para o potencial quártico \ref{Equação: Quartico +} e \ref{Equação: Quartico -}. Respectivamente, estes modelos equivalem a tomar na simulação os parâmetros
\begin{equation}
	d = 1, \ \  n = 2, \ \ \V(x)= t |x|^{2m}, \ \ W(x) = g(x) = \log{|x|}, \ \ \beta_N = \beta N^2, \ \ \beta = 2.
	\label{Equation: Parametros Monico}
\end{equation}
\begin{equation}
	d = 1, \ \  n = 2, \ \ \V(x)=\frac{|x|^4}{4} + t \frac{|x|^2}{2}, \ \ W(x) = g(x) = \log{|x|}, \ \ \beta_N = \beta N^2, \ \ \beta = 2.
	\label{Equation: Parametros Quartico}
\end{equation}
O caso mônico se reduz ao gaussiano se $m=1$. Os resultados para ambos os potenciais estão explicitados na Figura \ref{Figura: Quartic Monic} para alguns parâmetros interessantes de $t$ e $m$.
\begin{figure}[ht!]
	\includegraphics[width=\textwidth]{Assets/validationQuarticMonic-alt.png}
	\caption{Potencial Quártico \ref{Equation: Parametros Quartico} e Mônico \ref{Equation: Parametros Monico}, respectivamente à esquerda e direita. Tomou-se $\Delta t = 0.1$, $N=100$, e $nsteps = 5\cdot10^6$ passos. Registra-se a cada $1000$ iterações a partir de $nsteps/5$. No Quártico, simula-se $t=-1,-2,-3$. No Mônico fixa-se $t=1$ e simula-se $m=1,3,5$.}
	\label{Figura: Quartic Monic}
\end{figure}



% -
% C3S3 - Outros Potenciais?
% - 
%\input{USPSC-TA-Textual/C3-ImplementacaoEResultados/OutrosPotenciais.tex}
