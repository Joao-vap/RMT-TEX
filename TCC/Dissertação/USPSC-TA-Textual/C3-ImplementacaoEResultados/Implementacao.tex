\section{A implementação}

Restringiremos o subespaço $\Se$ à $\R^d$, $d = 1, 2$. Usaremos autovalores na reta para validar o funcionamento do algoritmo nos potencias já discutidos. e exploraremos uma distribuição em $\R^2$, mais complexa. Consideraremos um núcleo de interação $\W = g$ coulombiano em $n = 2$. Por isso, retomamos medida da forma \ref{Equação: Medida Gas de Coulomb} usual de gases de coulomb. A esquemática da implementação se encontra na Figura \ref{Figura: Implementação}. Podemos entender melhor a relação entre as sub-rotinas e funções em referência à Tabela \ref{Table: Funcoes e Subrotinas}.

\begin{figure}[ht]
	\centering
	\begin{tikzpicture}[font=\tiny,thick]
		
		% Start block
		\node[subrotina] (INIT) {INIT};
		
		% -------------------------------------------------------------------		
		
		\node[subrotina,
		left=0.7cm of INIT] (LabelSubrotina) {Subrotinas};
		
		\node[funcao,
		below=0.1cm of LabelSubrotina] (LabelFunção) {Funções};
		
		% -------------------------------------------------------------------		
		
		\node[funcao,
		below=0.5cm of INIT, xshift=2cm] (Hold) {H};
		
		\node[funcao,
		right=0.5cm of Hold, yshift=0.5cm] (Wold) {W};
		
		\node[funcao,
		right=0.5cm of Hold, yshift=-0.5cm] (Vold) {V};
		
		
		\node[loop,
		below=2cm of INIT,
		minimum width=6cm,
		xshift=2cm,
		] (LOOP) {
			\begin{tikzpicture}
				
				\node[subrotina,
				] (L2) {L2-OrnsUhlen};
				
				\node[funcao,
				below=0.5cm of L2
				] (Gauss) {Gauss};
				
				\node[subrotina,
				right=2cm of L2] (L1) {L1-Verlet};
				
				\node[subrotina,
				below=0.5cm of L1] (GradH) {GradH};
				
				\node[subrotina,
				below=0.5cm of GradH, xshift=1cm] (GradW) {GradW};
				
				\node[subrotina,
				below=0.5cm of GradH, xshift=-1cm] (GradV) {GradV};
								
				\node[subrotina,
				below=4cm of L2, xshift=-0.5cm] (Metro) {Metropolis};
				
				\node[funcao,
				below=0.5cm of Metro
				] (Problog) {ProbLog};
				
				\node[funcao,
				right=1cm of Problog] (H) {H};
				
				\node[funcao,
				right=0.5cm of H, yshift=0.5cm] (W) {W};
				
				\node[funcao,
				right=0.5cm of H, yshift=-0.5cm] (V) {V};
				
				\node[random,
				above=0.5cm of Metro, xshift=-1.3cm] (aceito) {$q_k = \tilde{q}_{k_1}$ \\ $p_k = \tilde{p}_{k_1}$};
				
				\node[random,
				above=0.5cm of Metro, xshift=1.3cm] (negado) {$q_k = q_k$ \\ $p_k = -p_k$};
				
				
				% ---------------------------------------------------------------------
				
				\path [fluxo] (L2) -- (L1);
				\path [fluxo]  (L2) ++(-3cm, 0cm) -- (L2);
				\path [chamada] (L2) -- (Gauss);
				\path [chamada] (L1) -- (GradH);
				\path [chamada] (GradH) -- (GradV);
				\path [chamada] (GradH) -- (GradW);
				\path [fluxo]  (L1) --++(3cm, 0cm) |- (Metro);
				\path [chamada] (Metro) -- (Problog);
				\path [chamada] (Problog) -- (H);
				\path [chamada] (H) -- (W);
				\path [chamada] (H) -- (V);
				\path [meiofluxo] (Metro) -- (aceito);
				\path [meiofluxo] (Metro) -- (negado);
				\path [meiofluxo] (negado) -- ++(0cm, 1.5cm) -- ++(-2.6cm, 0cm);
				\path [meiofluxo] (aceito) -- ++(0cm, 1.45cm);
				\path [fluxo] (aceito)++(0cm, 1.45cm) -- ++(0cm, 1.75cm);
				
			\end{tikzpicture}
		};
	
		\node[random,
		left=0.3cm of LOOP,
		yshift=2cm,
		rotate=90
		] (do) {DO k = 1, nsteps};
		
		\path [fluxo] (INIT) -- ++(0cm, -2.3cm);
		\path [chamada] (INIT) ++(0cm, -1.1cm) -- (Hold);
		\path [chamada] (Hold) -- (Vold);
		\path [chamada] (Hold) -- (Wold);
		
	\end{tikzpicture}
\caption{Implementação do algoritmo \textit{Langevin Monte Carlo} (LMC). Setas sólidas indicam o fluxo do programa. Setas tracejadas indicam chamadas de funções dentro do bloco. A descrição das funções se encontra na Tabela \ref{Table: Funcoes e Subrotinas}.}
\label{Figura: Implementação}
\end{figure}

\begin{table}[ht]
	\centering
	\begin{tabular}{ |p{2.6cm}||p{12cm}|  }
		\hline
		Nome & Descrição \\ 
		\hline
		\hline
		Init   		  	 & 
		Modifica ${p}_{k}$ vetor $[N\cross m]$, global, uniforme no cubo em $R^d$ e ${q}_{k}, G_H$, vetores $[N\cross m]$, globais, nulos. \\
		\hline
		L1-OrnsUhlen 	 & 
		Modifica $\tilde{p}_k$, vetor $[N\cross m]$, global, por $\Gl_U$ segundo \ref{Equation: Alg Mehler}. \\
		\hline
		L2-Verlet  	 	 & 
		Modifica $\tilde{p}_{k_1},\tilde{q}_{k_1}$ vetores $[N\cross m]$, globais, por $\Gl_{\Hf}$ segundo \ref{Equation: Alg Verlet}.	\\
		\hline
		GradH         	 & 
		Modifica $G_H$, vetor $[N\cross m$], global, gradiente do Hamiltoniano.					\\
		\hline
		GradW        	 &
		Modifica $G_{W_i}$, escalar, global, gradiente de $W$ núcleo de interação.	\\
		\hline
		GradV  	      	 &
		Modifica $G_{V_i}$, escalar, global, gradiente de $\V$ potencial.		                    \\
		\hline
		ProbLog       		 &
		Retorna $P_K$, escalar, local, probabilidade de aceite de \ref{Equação: Alg Pk}. \\
		\hline
		H              	 &
		Retorna $H$, escalar, local, hamiltoniano em $k$.	 							\\
		\hline
		V  	      			 &
		Retorna $V_i$, escalar, local, potencial de $q_i$.								\\
		\hline
		W         	  		 & 
		Retorna $W_{i,j}$, escalar, local, interação entre $q_i,q_j$ 							\\
		\hline
		Metropolis     	 & 
		Modifica ${p}_{k},{q}_{k}$, vetores $[N\cross m]$, globais por \ref{Equation: Alg Metro}.								\\
		\hline
	\end{tabular}
	\caption{ Descrição das funções e subrotinas utilizadas na implementação do programa.}
	\label{Table: Funcoes e Subrotinas}
\end{table}

 Alguns detalhes são importantes de notar. O gerador de variáveis aleatórias gaussianas, necessário em \ref{Equation: Alg Mehler} foi implementado utilizando do algoritmo de \textit{Box-Muller} \cite{NormalVariable}. Para além disso, o ajuste de variáveis é notoriamente um dos aspectos complicados do algoritmo implementado. Precisamos de uma holística par ajustar $\Delta t, \alpha_N \ \text{e} \ \gamma_N$. No escopo do nosso programa, $\Delta t$ e $\alpha_N$ desempenham o mesmo papel e, por isso, tomaremos $\alpha_N = 1$ e decidiremos sobre o valor de $\Delta t$. Seguindo a recomendação de \cite[Capítulo~5]{handbookmontecarlo}, tomaremos $\Delta t = \Delta\tilde{t} + X$, onde $X$ é variável aleatória de média $0$ e variância $\sigma^2$ pequena. Essa escolha ajuda a acelerar a convergência em casos exóticos, que queremos evitar. Lembramos ainda que $\Delta \tilde{t}$ é melhor quando é da ordem de $N^{-\frac{1}{4}}$, isto é, é pequeno o suficiente para manter a razão de aceite do passo de Metropolis alta e grande o suficiente para não desacelerar a convergência do algoritmo. Já $\gamma_N$ definirá o quanto substituiremos o momento anterior das partículas e o quanto utilizaremos do passo aleatório. Aqui, sabemos apenas que tornar $\eta$ próximo demais de $0$, ou de $1$ para todos efeitos, desacelera intensamente a convergência. Faremos, em geral, com que $\gamma_N \alpha_N \Delta t \approx 0.5$.
 
 %Para além dos ajustes, cada simulação é identificada pelo Hamiltoniano, ou seja, pelo potencial $V$ e pelas dimensões $d, n$, respectivamente do espaço que as partículas estão restritas e do que elas existem.


