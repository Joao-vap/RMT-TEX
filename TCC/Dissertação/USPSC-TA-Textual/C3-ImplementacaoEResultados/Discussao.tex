\section{Discussão}

É fácil ver para os casos \ref{Equation: Parametros Gaussian}, \ref{Equation: Parametros Monico}, \ref{Equation: Parametros Quartico} que as medidas explícita pela teoria são concordantes, como esperado, com a medida experimental obtida nas simulações. Contudo, isso é bem explorado e pode ser observado, com exceção do Mônico, em \cite{Chafa2018}. Foi discutido no Capítulo \ref{Capitulo: Intro} que os modelos gaussianos são os únicos em RMT com invariância por rotação e independência das entradas. Fica então a questão de como gerar matrizes de outros modelos se as entradas são correlacionadas. Sabemos do teorema espectral que, para as matrizes tomadas, vale a decomposição $\matriz{M} = \matriz{U} \matriz{D} \matriz{U}^{-1}$, já apresentada no Capítulo \ref{Capitulo: Intro}. Sabemos ainda que trataremos de ensembles invariantes, ou seja, a medida é a mesma para quaisquer $M, M'$ tais que $\matriz{M} = \matriz{U} \matriz{M'} \matriz{U}^{-1}$. Isso implica que podemos simular $\matriz{U}$ autovetores uniformemente do espaço correspondente. Para reconstruir uma elemento do ensemble de interesse nos resta replicar a medida de autovalores, $\matriz{D}$. Isso pode ser feito pela simulação descrita. Reconstruímos elementos dos ensembles a partir da simulação de sua medida com gases de Coulomb.