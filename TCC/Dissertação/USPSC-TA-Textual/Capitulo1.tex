\chapter[Introdução]{Introdução}
\label{Introdução}

Sólitons topológicos desempenham papel fundamental no estudo de fenomênos não lineares em diversas áreas da física, eles aparecem em uma variedade de teorias, como \textit{kinks} em $(1+1)$ dimensões, vórtices em $(2+1)$ dimensões, monopolos magnéticos e \textit{Skyrmions} em $(3+1)$ dimensões e \textit{Instantons} em quatro dimensões euclidianas. Os sólitons topológicos são relevantes para muitos fenômenos não lineares na física de altas energias, matéria condensada e na ciência em geral.

Dentre os sólitons topológicos existe uma classe especial, os chamados sólitons auto-duais. Eles são soluções clássicas das equações de auto-dualidade, que são equações de primeira ordem que implicam nas equações de Euler-Lagrange da teoria. Além disso, em cada setor topológico, isto é, o conjunto das soluções que possuem a mesma carga topológica associada, existe um limite inferior da energia estática, ou ação Euclidiana, e os sóltions auto-duais saturam esse limite. Portanto, os sólitons auto-duais são muito estáveis.

A razão pela qual se realiza apenas uma integração para construir os sólitons auto-duais, ao invés de duas para o caso de sóltions topológicos usuais, não está ligada a conservação de uma quantidade dinamicamente. Em todos os casos que a auto-dualidade funciona, a carga topológica relevante admite uma representação integral, ou seja, existe uma densidade de carga topológica. A invariância da carga sobre qualquer variação suave (homotópica) dos campos leva a identidades, em forma de equações diferenciais de segunda ordem, que são satisfeitas por qualquer configuração regular dos campos, não necessariamente solução da teoria. Entretanto, com a imposição das equações de auto-dualidade essas identidades se tornam as equações de Euler-Lagrange da teoria.

Utilizando o conceito de auto-dualidade generalizada se pode criar, com uma carga topológica, uma grande classe de teorias de campo contendo setores auto-duais \cite{lafBPS}. Em $(1+1)$ dimensões foi possível criar teorias de campo, com qualquer número de campos escalares, contendo sólitons topológicos, generalizando o processo conhecido para teorias com somente um campo escalar, como o modelo de \textit{sine-Gordon} e $\lambda \phi^4$ \cite{laf(1+1)}.

Nesse trabalho, serão revisados os recentes desenvolvimentos e aplicações do conceito de auto-dualidade generalizada proposto em \cite{lafBPS}, de uma forma simples e concisa.\footnote{Será utilizada a convenção de soma em índices repetidos durante todo trabalho.}