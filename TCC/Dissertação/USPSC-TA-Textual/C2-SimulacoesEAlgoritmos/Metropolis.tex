\section{Passo de Seleção - Metropolis}

Muitos algoritmos utilizam  de um passo de seleção para estabilizar sua dinâmica e otimizar a convergência e a amostragem da variável de interesse. Dentre eles citemos por exemplo o \textit{Metropolis-Adjusted Langevin Algorith} (MALA) \cite[Anexo~C]{leimmolecular}. Outros métodos com a amenização do processo também teriam efeito sobre a estilização da dinâmica. Para o método de metrópolis, um importante aspecto é manter a quantidade de rejeições baixa para não atrapalhar a eficiência do programa, o que influencia no tamanho do passo temporal decidido. Apesar disso, seu uso pode levar à descrita melhor estabilidade numérica.

Usaremos um esquema de Metropolis onde procuraremos, definindo uma probabilidade de aceite de cada atualização, evitar passos redundantes ou irrelevantes. Propõe-se então que, a partir da atualização para a posição $\tilde{q}_{k+1}$ \cite{Chafa2018}, se calcule a probabilidade
\begin{equation}
P_k = 1 \wedge \frac{\K(\tilde{q}_{k+1}, q_k) \ee^{-\beta_N \Hf_N(\tilde{q}_{k+1})}}{\K(q_k, \tilde{q}_{k+1}) \ee^{-\beta_N \Hf_N(q_{k})}},
\label{Equation: Pk}
\end{equation}
onde o núcleo $K(x, y)$ é simétrico \cite{Chafa2018} para o caso do \textit{Hybrid Monte Carlo} e, por se cancelar, não será discutido adiante. Atribua agora às coordenadas generalizadas $(q_{k+1}, p_{k+1})$ valor da seguinte forma
\begin{equation}
	(q_{k+1}, p_{k+1}) =
\begin{cases}
	(\tilde{q}_{k+1}, \tilde{p}_{k+1}) \ \text{com probabilidade} \ P_k, \\
	(q_k, -\tilde{p}_{k}) \ \text{com probabilidade} \ 1-P_k; \\
\end{cases}
\label{Equation: Metropolis}
\end{equation}
De forma a garantir a conservação da energia para o sistema e otimizar a exploração do espaço de fase.