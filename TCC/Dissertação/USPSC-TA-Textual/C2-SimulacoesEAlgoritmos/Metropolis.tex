\section{Passo de Metropolis}
\label{Section: Metropolis}

Muitos algoritmos utilizam de um passo de seleção para estabilizar sua dinâmica e otimizar a convergência e a amostragem da variável de interesse \footnote{Como o \textit{Metropolis-Adjusted Langevin Algorith} (MALA) \cite[Anexo~C]{leimmolecular}}. Partindo dos esquemas da Seção \ref{Seção: Discretização}, consideraremos que temos uma proposta $\tilde{q}_{k+1}$ de estado. Para o método de Metropolis, um importante aspecto é manter a razão de rejeições baixa para não atrapalhar a eficiência do programa, o que influencia no tamanho do passo temporal decidido. Pode ser mostrado que $\Delta t$ é ideal quando é da ordem de $N^{-\frac{1}{4}}$ \cite{Chafa2018}, tornando o esquema interessante pela escalabilidade de $N$.

Propõe-se então que, a partir da proposição de estado $\tilde{q}_{k+1}$ gerada pelo esquema anterior, se calcule a probabilidade
\begin{equation}
P_k = 1 \wedge \frac{\K(\tilde{q}_{k+1}, q_k) \ee^{-\beta_N \Ee_N(\tilde{q}_{k+1})}}{\K(q_k, \tilde{q}_{k+1}) \ee^{-\beta_N \Ee_N(q_{k})}},
\label{Equation: Pk}
\end{equation}
onde o núcleo $K(x, y)$ é simétrico \cite{Chafa2018} para o caso do \textit{Langevin Monte Carlo} e, por se cancelar, não será discutido adiante. Atribua agora às novas coordenadas generalizadas $(q_{k+1}, p_{k+1})$ valor da seguinte forma
\begin{equation}
	(q_{k+1}, p_{k+1}) =
\begin{cases}
	(\tilde{q}_{k+1}, \tilde{p}_{k+1}) \ \text{com probabilidade} \ P_k, \\
	(q_k, -\tilde{p}_{k}) \ \text{com probabilidade} \ 1-P_k; \\
\end{cases}
\label{Equation: Metropolis}
\end{equation}
De forma a garantir a conservação da energia, que é uma preocupação na discretização da dinâmica, e otimizar a exploração do espaço de fase.