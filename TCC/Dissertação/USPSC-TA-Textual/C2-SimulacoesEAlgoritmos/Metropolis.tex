\section{Passo de Metropolis-Hastings}
\label{Section: Metropolis}

Muitos algoritmos utilizam de um passo de seleção para estabilizar sua dinâmica e otimizar a convergência e amostragem, usaremos dessa ideia para otimizar o algoritmo. Para o método de Metropolis-Hastings, é importante manter a razão de rejeições baixa para não atrapalhar a eficiência, o que influencia no tamanho do passo temporal decidido. Pode ser mostrado que $\Delta t$ é ideal quando é da ordem de $N^{-\frac{1}{4}}$, \cite{Chafa2018} tornando o esquema interessante pela escalabilidade de $N$.

Partindo dos esquemas da Seção \ref{Seção: Discretização}, consideraremos $(\tilde{q}_{k+1},\tilde{p}_{k+1})$ proposta de novo estado gerada pela dinâmica de $\Gl$, a partir do estado anterior $(q_{k},p_{k})$. Define-se
\begin{equation}
P_k = 1 \wedge \frac{\ee^{-\beta_N \Ee_N(\tilde{q}_{k+1}, \tilde{p}_{k+1})}}{\ee^{-\beta_N \Ee_N(q_{k}, \tilde{p}_{k})}},
\label{Equation: Pk}
\end{equation}
onde $\tilde{p}_{k}$ é dado por \eqref{Equation: Alg Mehler}, probabilidade de aceite tal que se atribua agora às novas coordenadas generalizadas $(q_{k+1}, p_{k+1})$ valor da seguinte forma
\begin{equation}
	(q_{k+1}, p_{k+1}) =
\begin{cases}
	(\tilde{q}_{k+1}, \tilde{p}_{k+1}) \ \text{com probabilidade} \ P_k, \\
	(q_k, -\tilde{p}_{k}) \ \text{com probabilidade} \ 1-P_k. \\
\end{cases}
\label{Equation: Metropolis}
\end{equation}
Assim, a proposta será aceita com probabilidade um se $\Ee_N(\tilde{q}_{k+1}, \tilde{p}_{k+1}) < \Ee_N(q_{k}, \tilde{p}_{k})$ e com probabilidade dada pela razão das medidas, caso contrário. Dessa forma garante-se a conservação da energia - preocupação na discretização da dinâmica - e otimiza-se a exploração do espaço de fase.