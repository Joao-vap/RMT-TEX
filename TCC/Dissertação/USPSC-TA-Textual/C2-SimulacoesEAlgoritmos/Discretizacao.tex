\section{Discretização}
\label{Seção: Discretização}

Para integrar $\Gl$, faremos separadamente a operação sobre $\Gl_{\Hf}$ e $\Gl_{\U}$. A dinâmica hamiltoniana é reversível, o que é importante no algoritmo para garantir que mantém-se a medida invariante. Ainda mais, preserva o volume do espaço de fase, de forma que não precisamos calcular o jacobiano da matriz que define a transformação da dinâmica. Essas duas propriedades podem ser mantidas quando discretizada a dinâmica pelo método de Verlet \cite{Chafa2018}\cite{leimmolecular}. A dinâmica deveria também manter o Hamiltoniano constante, contudo, discretizada, podemos garantir somente que ele se mantenha quase constante. Para lidar com esse fato, discute-se a implementação de um passo de Metropolis na próxima seção. Para $\Delta t > 0$, a partir do estado $(q_k, p_k)$, o esquema lê-se
\begin{equation}
\begin{cases}
	\tilde{p}_{k+\frac{1}{2}} = \tilde{p}_k - \nabla \Hf_N(q_k) \alpha_N \frac{\Delta t}{2}, \\
	\tilde{q}_{k+1} = q_k + \tilde{p}_{k + \frac{1}{2}} \alpha_N \Delta t, \\
	\tilde{p}_{k+1} = \tilde{p}_{k+\frac{1}{2}} - \nabla \Hf_N(q_{k+1}) \alpha_N \frac{\Delta t}{2}.
\end{cases}
\label{Equation: Verlet}
\end{equation}
Um esquema análogo é possível para energias cinéticas generalizadas \cite{Stoltz2018}. Outros métodos tais quais \textit{Euler-Maruyama} (EM) \cite[Capítulo~7]{leimmolecular} podem ser utilizados para o mesmo fim. Nos método que temos interesse o erro associado à discretização deve ir à zero quando $\Delta t$ vai à zero. Para EM, o erro por passo, local, é da ordem de $\Boh{(\Delta t^2)}$ e o erro final, global, $\Boh{(Delta t)}$, Já para o esquema escolhido, temos erro local de  $\Boh{(\Delta t^3)}$ e global de  $\Boh{(\Delta t^2)}$. Essa diferença vem do fato da discretização usada ser reversível \cite[Capítulo~5]{handbookmontecarlo}. 

Nos resta integrar $\Gl_{\U}$, o qual, para a energia cinética usual, consiste em um processo de Ornstein-Uhlenbeck de variância explícita $$dx_t = - \xi x_t dt + \sigma dB_t$$ onde $\xi, \sigma > 0$ são parâmetros e $B_t$ é processo browniano. Note que para $\alpha > 0$ substituiremos parcialmente o momento das variáveis, se $\alpha = 0$ retomaríamos \ref{Equação: Langevin Overdamped}. Este processo não é muito melhor, contudo, do que um \textit{Random Walk Metropolis} \cite[Capítulo~5]{handbookmontecarlo} já que o momento seria completamente substituído. Este processo pode ser resolvido a partir da fórmula de Mehler e obtêm-se
\begin{equation}
\tilde{p}_k = \eta p_k + \sqrt{\frac{1-\eta^2}{\beta_N}} G_k, \ \ \ \eta = \ee^{-\gamma_N \alpha_N \Delta t}.
\label{Equation: Mehler}
\end{equation}
Onde $G_k$ é variável aleatória Gaussiana usual.

