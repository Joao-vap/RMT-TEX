\section{Discretização - Verlet Integrator}

Para integrar $\Gl$, faremos separadamente a operação sobre $\Gl_{\Hf}$ e $\Gl_{\U}$. Sobre a primeira, a qual denomina-se dinâmica Hamiltoniana, o integrador de Verlet \cite{Chafa2018}\cite{leimmolecular}, para $\Delta t > 0$, a partir do estado $(q_k, p_k)$,
\begin{equation}
\begin{cases}
	\tilde{p}_{k+\frac{1}{2}} = \tilde{p}_k - \nabla \Hf_N(q_k) \alpha_N \frac{\Delta t}{2}, \\
	\tilde{q}_{k+1} = q_k + \tilde{p}_{k + \frac{1}{2}} \alpha_N \Delta t, \\
	\tilde{p}_{k+1} = \tilde{p}_{k+\frac{1}{2}} - \nabla \Hf_N(q_{k+1}) \alpha_N \frac{\Delta t}{2}.
\end{cases}
\label{Equation: Verlet}
\end{equation}
Que consiste em dois tipos de operações, uma de atualização do momento, uma de atualização da posição e novamente uma atualização do momento. Feito isso, nos resta integrar $\Gl_{\U}$, o qual, para a energia cinética usual supracitada, consiste em um processo de Ornstein-Uhlenbeck de variância explícita e pode ser resolvido a partir da fórmula de Mehler
\begin{equation}
\tilde{p}_k = \eta p_k + \sqrt{\frac{1-\eta^2}{\beta_N}} G_k, \ \ \ \eta = \ee^{-\gamma_N \alpha_N \Delta t}.
\label{Equation: Mehler}
\end{equation}
Onde $G_k$ é variável aleatória Gaussiana usual. Nota-se ainda que esse processo pode não ter medida invariante graças às singularidades das interações. Para evitar esse tipo de problema, introduz-se o passo de seleção de Metropolis.

