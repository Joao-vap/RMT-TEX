\section{Integração Numérica}
\label{Seção: Discretização}

%Um esquema ter invariância de $\Hf$ significa que, idealmente, a dinâmica preserva o volume do espaço de fase, de forma que não precisamos calcular o jacobiano da matriz que define a transformação da dinâmica. Essa propriedade pode ser mantida na discretização quando utilizamos do método de Verlet.  \cite{Chafa2018}\cite[Capítulo~2]{leimmolecular} A dinâmica deveria também manter o Hamiltoniano constante, contudo, discretizada, podemos garantir somente que ele se mantenha quase constante. 

Para integrar o Processo \eqref{Equação: EqDif - Dinamica Langevin} discretizaremos, para amostragem numérica, separadamente as dinâmicas associadas à $\Gl_{\Hf}$ e $\Gl_{\U}$. Naturalmente, $\Gl_{\Hf}$ descreve um processo hamiltoniano e deve preservar o volume do espaço de fase, de forma que não precisaremos calcular o jacobiano da transformação que dá esta dinâmica. Utilizando de um integrador simplético, tal como o de Verlet, podemos manter essa propriedade na discretização. A dinâmica é também reversível a menos de inversão do momento, importante no algoritmo para garantir que mantém-se a medida invariante. Contudo, é conhecido que a discretização não pode preservar a energia exatamente e, para lidar com esse fato, discute-se a implementação de um passo de Metropolis-Hastings na Seção \ref{Section: Metropolis}. Para $\Delta t > 0$, a partir do estado $(q_k, p_k)$, o esquema de Verlet lê-se
\begin{equation}
\begin{cases}
	p_{k+\frac{1}{2}} = p_k - \nabla \Hf_N(q_k) \alpha_N \frac{\Delta t}{2}, \\
	\tilde{q}_{k+1} = q_k + p_{k + \frac{1}{2}} \alpha_N \Delta t, \\
	\tilde{p}_{k+1} = p_{k+\frac{1}{2}} - \nabla \Hf_N(\tilde{q}_{k+1}) \alpha_N \frac{\Delta t}{2},
\end{cases}
\label{Equation: Verlet}
\end{equation}
onde $(\tilde{q}_{k+1}, \tilde{p}_{k+1})$ é estado seguinte da dinâmica. Outros métodos tais quais \textit{Euler-Maruyama} (EM) podem ser utilizados para o mesmo fim.  \cite[Capítulo~7]{leimmolecular} Nos métodos que temos interesse, o erro associado à discretização deve ir a zero quando $\Delta t$ vai a zero. Para EM, o erro local é da ordem de $\Boh{(\Delta t^2)}$ e o erro global $\Boh{(\Delta t)}$. Já para o esquema escolhido, devido à reversibilidade, o erro local é $\Boh{(\Delta t^3)}$ e o global $\Boh{(\Delta t^2)}$. \cite[Capítulo~5]{handbookmontecarlo} 

Nos resta tratar o processo de $\Gl_{\U}$, o qual, para a energia cinética usual, consiste em um processo de Ornstein-Uhlenbeck de variância explícita, ou ainda, da forma $$\dd p_t = - \alpha_N p_t \dd t + \sigma \dd B_t,$$ onde $\alpha_N, \sigma > 0$ são parâmetros e $B_t$ é processo browniano. Este processo também mantém a medida invariante e é reversível. Note que, para $\alpha_N > 0$ somente substituiremos parcialmente o momento das partículas e, se $\alpha_N, \gamma_N \rightarrow 0$ com $\alpha_N \gamma_N = 1$, retomaríamos a dinâmica da Equação \eqref{Equação: Langevin Overdamped}. Este processo não seria muito melhor, contudo, do que um \textit{Random Walk Metropolis} já que o momento seria completamente substituído. \cite[Capítulo~5]{handbookmontecarlo} De qualquer forma, sabemos existir solução analítica para o processo de Ornstein-Uhlenbeck a partir da fórmula de Mehler dada por
\begin{equation}
\tilde{p}_k = \eta p_k + \sqrt{\frac{1-\eta^2}{\beta_N}} G_k, \ \ \ \eta = \ee^{-\gamma_N \alpha_N \Delta t},
\label{Equation: Mehler}
\end{equation}
onde $G_k$ é variável aleatória gaussiana usual. \cite{Chafa2018}

