\section{Introdução - Dinâmica de Langevin}

O algoritmo apresentado é introduzido no artigo \cite{Chafa2018}. Denote a configuração do sistema por $(q, p)$, onde $q \in \R^d$ são as posições das partículas e $p \in \R^d$ são os momentos associados. Seja $\U_N \colon \R^d \rightarrow \R$ energia cinética generalizada tal que $\ee^{-\beta_N \U_N}$ seja Lebesgue integrável. Para uma energia da forma $\Ee(q,p) = \Hf(q) + \U(p)$, a dinâmica de Langevin para o processo de difusão em $\R^{dN} \cross \R^{dN}$ se descreve como a solução para a equação estocástica \cite{Stoltz2018} 
\begin{equation}
\begin{cases}
	\dd q_t = \alpha_N \nabla U_N (p_t) \dd t, \\
	\dd p_t = -\alpha_N \nabla H_N(p_t) \dd t - \gamma_N \alpha_N \nabla U_N(p_t) \dd t + \sqrt{2\frac{\gamma_N \alpha_N}{\beta_N}} \dd B_t.
\end{cases}
\label{Equação: EqDif - Dinamica Langevin}
\end{equation}
onde $(B_t)_{t>0}$ é movimento browniano em $\R^{dN}$ e $\gamma_N > 0$ constante de atrito. Já $\beta_N$, temperatura inversa e $\Hf$ são como em \ref{Equação: Medida Gas de Coulomb}. Essa dinâmica admite o gerador infinitesimal 
\[
	\Gl = \Gl_{\Hf} + \Gl_{\U},
\]
\[
 \Gl_{\Hf} = -\alpha_N \nabla\Hf_N(q) \cdot \nabla_p + \alpha_N \nabla \U_N(p) \cdot \nabla_q, \ \ \ \ \Gl_{\U} = \frac{\gamma_N\alpha_N}{\beta_N} \Delta_p - \gamma_N \alpha_N \nabla \U_N(p) \cdot \nabla_p 
\]
Aqui, denomina-se $\Gl_{\Hf}$ a parte Hamiltoniana e $\Gl_{\U}$ a parte de flutuação-dissipação. Toma-se $\U_N(p) = \frac{1}{2} |p|^2$ tal que $\U_N(p)$ é energia cinética. Para simular o processo $(p_t,q_t)$ temos agora que discretizar \ref{Equação: EqDif - Dinamica Langevin} e amostrar da trajetória obtida.