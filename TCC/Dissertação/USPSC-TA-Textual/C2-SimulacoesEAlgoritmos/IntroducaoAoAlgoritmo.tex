\section{Dinâmica de Langevin Monte Carlo}

Nosso objetivo com a simulação é determinar a esperança de uma função de interesse $\f(\vec{q})$ $$\langle f \rangle \approx \frac{1}{n} \sum_{i=0}^{n-1} f(\vec{q}_i),$$ onde $\vec{q}_i$ são obtidos por meio da simulação com dada distribuição de Gibbs-Boltzmann. Para fazer o modelo ergótico, ou seja, garantir que a dinâmica (e nossas amostras) não esteja restrita à um subconjunto do espaço de fase, tomaremos uma dinâmica, um termostato, estocástica. Isso usualmente garante que o sistema convirja para sua medida invariante (única). Um esquema comumente utilizado é o da dinâmica de Langevin\footnote{Poderíamos ter explorado quaisquer outras dinâmicas similares tais como as dinâmicas de \textit{Dissipative Particle} \cite{DPD} ou \textit{Nose-Hoover} \cite{Hoover}.}.

Denote $q$, com $q \in \R^d$ posição generalizada associada às $N$ partículas. Poderíamos enunciar a seguinte equação diferencial para a dinâmica
\begin{equation}
	\dd q_t = -\alpha_N \nabla H_N(q_t) \dd t + \sqrt{2\frac{\gamma_N \alpha_N}{\beta_N}} \dd W_t
	\label{Equação: Langevin Overdamped}
\end{equation}
onde $(W_t)_{t>0}$ é processo de Wiener, $\gamma_N > 0$ é constante de atrito e $\alpha_N$ é escala temporal. Isso seria suficiente e é chamado \textit{Overdamped Langevin}, contudo, tomaremos sua versão cinética. Usaremos $q$ como variável de interesse e $p$, com $p \in \R^d$ variável de momento generalizado, para flexibilizar a dinâmica. Considere $\U_N \colon \R^d \rightarrow \R$ energia cinética generalizada tal que $\ee^{-\beta_N \U_N}$ seja lebesgue integrável. Para uma energia da forma $\Ee(q,p) = \Hf_N(q) + \U_N(p)$, escreve-se \cite{Stoltz2018} a dinâmica de Langevin para o processo de difusão em $\R^{dN} \times \R^{dN}$ como a solução para a equação estocástica 
\begin{equation}
\begin{cases}
	\dd q_t = \alpha_N \nabla U_N (p_t) \dd t, \\
	\dd p_t = -\alpha_N \nabla \Hf_N(q_t) \dd t - \gamma_N \alpha_N \nabla U_N(p_t) \dd t + \sqrt{2\frac{\gamma_N \alpha_N}{\beta_N}} \dd B_t.
\end{cases}
\label{Equação: EqDif - Dinamica Langevin}
\end{equation}
onde $\beta_N$, temperatura inversa e $\Hf$ são como em \ref{Equation: Medida Log V}. Essa dinâmica admite o gerador infinitesimal 
\[
	\Gl = \Gl_{\Hf} + \Gl_{\U},
\]
\[
 \Gl_{\Hf} = -\alpha_N \nabla\Hf_N(q) \cdot \nabla_p + \alpha_N \nabla \U_N(p) \cdot \nabla_q, \ \ \ \ \Gl_{\U} = \frac{\gamma_N\alpha_N}{\beta_N} \Delta_p - \gamma_N \alpha_N \nabla \U_N(p) \cdot \nabla_p.
\]

Denomina-se $\Gl_{\Hf}$ a parte Hamiltoniana e $\Gl_{\U}$ a parte de flutuação-dissipação. Tomaremos $\U_N(p) = \frac{1}{2} |p|^2$ tal que $\U_N(p)$ é energia cinética usual. Além disso, $(B_t)_{t>0}$ é processo browniano. Para simular o processo $(p_t,q_t)$ integramos \ref{Equação: EqDif - Dinamica Langevin}, contudo, sabemos que isso pode não ser possível analiticamente, o que nos leva a recorrer a métodos numéricos para amostragem.