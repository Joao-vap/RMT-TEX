\chapter{Simulações e Algoritmos}
\label{Capitulo: Simulações}

A medida de Boltzmann-Gibbs descreve o denominado ensemble canônico. Médias sobre suas configurações, microestados, são usadas para inferir informações macroscópicas do sistema. Sistemas dinâmicos que amostrem desta medida são denominados termostatos e são notoriamente difíceis de se construir ergodicamente com processos dinâmicos determinísticos, portanto, uma teoria de equações diferenciais estocásticas foi desenvolvida. Usualmente, para o ensemble canônico, uma escolha natural de processo é a denominada \textit{Langevin Dynamics}, \cite[Capítulo~6]{leimmolecular} especialmente sua versão cinética. Muitas vezes as equações usadas não são diretamente integráveis e, por isso, se recorre a métodos numéricos. O caso cinético pode ser separado em duas dinâmicas. Para a integração da primeira, chamada hamiltoniana, utilizamos o esquema de Verlet. A segunda parte, denominada flutuação-dissipação, resolve-se analiticamente por se tratar de processo de Ornstein-Uhlenbeck de variância explícita. Apesar das qualidades dos métodos citados, a discretização pode introduzir instabilidade numérica e, para amenizar seus efeitos, introduz-se um passo de Metropolis-Hastings.  \cite[Apêndice~C]{leimmolecular} As escolhas supracitadas são descritas por Chafa\"{i} e Ferré \cite{Chafa2018} e são denominadas \textit{Langevin Monte Carlo}.


% -
% C2S1 - Introdução ao algoritmo
% - 

\section{Dinâmica de Langevin Monte Carlo}

Nosso objetivo com a simulação é determinar a esperança de uma função de interesse $\f(\vec{q})$ $$\langle f \rangle \approx \frac{1}{n} \sum_{i=0}^{n-1} f(\vec{q}_i),$$ onde $\vec{q}_i$ são obtidos por meio da simulação com dada distribuição de Gibbs-Boltzmann. Para fazer nosso modelo ergótico, ou seja, garantir que não restringiremos a dinâmica (e nossas amostras) à um subconjunto do espaço de fase, tomaremos uma dinâmica, um termostato, estocástica. Isso usualmente garante que o sistema convirja para sua medida invariante (única). Um esquema comumente utilizado é o da dinâmica de Langevin\footnote{Poderíamos ter explorado quaisquer outras dinâmicas similares tais como as dinâmicas de \textit{Dissipative Particle} \cite{DPD} ou \textit{Nose-Hoover} \cite{Hoover}.}.

Denote a configuração do sistema por $(q, p)$, onde $q,p \in \R^d$ são respectivamente as posições e momentos generalizados associados às $N$ partículas. Poderíamos enunciar a seguinte equação diferencial para a dinâmica
\begin{equation}
	\dd q_t = -\alpha_N \nabla H_N(q_t) \dd t + \sqrt{2\frac{\gamma_N \alpha_N}{\beta_N}} \dd W_t
	\label{Equação: Langevin Overdamped}
\end{equation}
onde $(W_t)_{t>0}$ é processo de Wiener, $\gamma_N > 0$ é constante de atrito e $\alpha_N$ é escala temporal. Isso seria suficiente e é chamado \textit{Overdamped Langevin}, contudo, tomaremos sua versão cinética. Usaremos $q$ como variável de interesse e $p$ para flexibilizar a dinâmica. Considere $\U_N \colon \R^d \rightarrow \R$ energia cinética generalizada tal que $\ee^{-\beta_N \U_N}$ seja lebesgue integrável. Para uma energia da forma $\Ee(q,p) = \Hf(q) + \U(p)$, escreve-se \cite{Stoltz2018} a dinâmica de Langevin para o processo de difusão em $\R^{dN} \cross \R^{dN}$ como a solução para a equação estocástica 
\begin{equation}
\begin{cases}
	\dd q_t = \alpha_N \nabla U_N (p_t) \dd t, \\
	\dd p_t = -\alpha_N \nabla H_N(q_t) \dd t - \gamma_N \alpha_N \nabla U_N(p_t) \dd t + \sqrt{2\frac{\gamma_N \alpha_N}{\beta_N}} \dd W_t.
\end{cases}
\label{Equação: EqDif - Dinamica Langevin}
\end{equation}
onde $\beta_N$, temperatura inversa e $\Hf$ são como em \ref{Equation: Medida Log V}. Essa dinâmica admite o gerador infinitesimal 
\[
	\Gl = \Gl_{\Hf} + \Gl_{\U},
\]
\[
 \Gl_{\Hf} = -\alpha_N \nabla\Hf_N(q) \cdot \nabla_p + \alpha_N \nabla \U_N(p) \cdot \nabla_q, \ \ \ \ \Gl_{\U} = \frac{\gamma_N\alpha_N}{\beta_N} \Delta_p - \gamma_N \alpha_N \nabla \U_N(p) \cdot \nabla_p.
\]

Denomina-se $\Gl_{\Hf}$ a parte Hamiltoniana e $\Gl_{\U}$ a parte de flutuação-dissipação. Tomaremos $\U_N(p) = \frac{1}{2} |p|^2$ tal que $\U_N(p)$ é energia cinética usual e $(W_t)_{t>0}$ é processo browniano. Para simular o processo $(p_t,q_t)$ integramos \ref{Equação: EqDif - Dinamica Langevin}, contudo, sabemos que isso pode não ser possível, o que nos leva a recorrer a métodos numéricos para amostragem.

% -
% C2S2 - Algoritmo Híbrido de Monte Carlo
% - 

%
\section{Algoritmo Híbrido de Monte Carlo}

O algoritmo híbrido de Monte Carlo é baseado no algoritmo anterior mas adicionando uma variável de momento para melhor explorar o espaço. Defina $E = \mathbb{R}^{\dd N}$ e deixe $U_N : E \rightarrow \mathbb{R}$ ser suave para que $\ee^{-\beta_N U_N}$ seja Lebesgue integrável. Seja ainda $(X_t, Y_t)_{t>0}$ o processo de difusão em $E \times E$ solução de

\[
\begin{cases}
	\dd X_t = \alpha_N \nabla U_N (Y_t) \dd t, \\
	\dd Y_t = \alpha_N \nabla H_N(X_t) \dd t - \gamma_N \alpha_N \nabla U_N(Y_t) \dd t + \sqrt{2\frac{\gamma_N \alpha_N}{\beta_N} \dd B_t},
\end{cases}
\]
onde $(B_t)_{t>0}$ é o movimento browniano em $E$ e $\gamma_N > 0$ parâmetro representando atrito.

Quando $U_N(y) = \frac{1}{2}|y|^2$ temos $Y_t = \dd X_t/\dd t$ e teremos que $X_t$ e $Y_t$ poderão ser interpretados como posição e velocidade do sistema de $N$ pontos em $S$ no tempo $t$. Nesse caso, $U_n$ é energia cinética

% -
% C2S2 - Discretização
% - 

\section{Discretização}
\label{Seção: Discretização}

Para integrar $\Gl$, faremos separadamente a operação sobre $\Gl_{\Hf}$ e $\Gl_{\U}$. A dinâmica hamiltoniana é reversível, o que é importante no algoritmo para garantir que mantém-se a medida invariante. Ainda mais, preserva o volume do espaço de fase, de forma que não precisamos calcular o jacobiano da matriz que define a transformação da dinâmica. Essas duas propriedades podem ser mantidas quando discretizada a dinâmica pelo método de Verlet \cite{Chafa2018}\cite{leimmolecular}. A dinâmica deveria também manter o Hamiltoniano constante, contudo, discretizada, podemos garantir somente que ele se mantenha quase constante. Para lidar com esse fato, discute-se a implementação de um passo de Metropolis na próxima seção. Para $\Delta t > 0$, a partir do estado $(q_k, p_k)$, o esquema lê-se
\begin{equation}
\begin{cases}
	\tilde{p}_{k+\frac{1}{2}} = \tilde{p}_k - \nabla \Hf_N(q_k) \alpha_N \frac{\Delta t}{2}, \\
	\tilde{q}_{k+1} = q_k + \tilde{p}_{k + \frac{1}{2}} \alpha_N \Delta t, \\
	\tilde{p}_{k+1} = \tilde{p}_{k+\frac{1}{2}} - \nabla \Hf_N(q_{k+1}) \alpha_N \frac{\Delta t}{2}.
\end{cases}
\label{Equation: Verlet}
\end{equation}
Um esquema análogo é possível para energias cinéticas generalizadas \cite{Stoltz2018}. Outros métodos tais quais \textit{Euler-Maruyama} (EM) \cite[Capítulo~7]{leimmolecular} podem ser utilizados para o mesmo fim. Nos método que temos interesse o erro associado à discretização deve ir à zero quando $\Delta t$ vai à zero. Para EM, o erro por passo, local, é da ordem de $\Boh{(\Delta t^2)}$ e o erro final, global, $\Boh{(Delta t)}$, Já para o esquema escolhido, temos erro local de  $\Boh{(\Delta t^3)}$ e global de  $\Boh{(\Delta t^2)}$. Essa diferença vem do fato da discretização usada ser reversível \cite[Capítulo~5]{handbookmontecarlo}. 

Nos resta integrar $\Gl_{\U}$, o qual, para a energia cinética usual, consiste em um processo de Ornstein-Uhlenbeck de variância explícita $$dx_t = - \xi x_t dt + \sigma dB_t$$ onde $\xi, \sigma > 0$ são parâmetros e $B_t$ é processo browniano. Note que para $\alpha > 0$ substituiremos parcialmente o momento das variáveis, se $\alpha = 0$ retomaríamos \ref{Equação: Langevin Overdamped}. Este processo não é muito melhor, contudo, do que um \textit{Random Walk Metropolis} \cite[Capítulo~5]{handbookmontecarlo} já que o momento seria completamente substituído. Este processo pode ser resolvido a partir da fórmula de Mehler e obtêm-se
\begin{equation}
\tilde{p}_k = \eta p_k + \sqrt{\frac{1-\eta^2}{\beta_N}} G_k, \ \ \ \eta = \ee^{-\gamma_N \alpha_N \Delta t}.
\label{Equation: Mehler}
\end{equation}
Onde $G_k$ é variável aleatória Gaussiana usual.



% -
% C2S3 - Discretização
% - 

\section{Passo de Metropolis}
\label{Section: Metropolis}

Muitos algoritmos utilizam de um passo de seleção para estabilizar sua dinâmica e otimizar a convergência e a amostragem da variável de interesse. Partindo dos esquemas da Seção \ref{Seção: Discretização}, consideraremos que temos uma proposta $\tilde{q}_{k+1}$ de estado. Para o método de Metropolis, um importante aspecto é manter a razão de rejeições baixa para não atrapalhar a eficiência do programa, o que influencia no tamanho do passo temporal decidido. Pode ser mostrado que $\Delta t$ é ideal quando é da ordem de $N^{-\frac{1}{4}}$ \cite{Chafa2018}, tornando o esquema interessante pela escalabilidade de $N$.

Propõe-se então que, a partir da proposição de estado $\tilde{q}_{k+1}$ gerada pelo esquema anterior, se calcule a probabilidade
\begin{equation}
P_k = 1 \wedge \frac{\K(\tilde{q}_{k+1}, q_k) \ee^{-\beta_N \Hf_N(\tilde{q}_{k+1})}}{\K(q_k, \tilde{q}_{k+1}) \ee^{-\beta_N \Hf_N(q_{k})}},
\label{Equation: Pk}
\end{equation}
onde o núcleo $K(x, y)$ é simétrico \cite{Chafa2018} para o caso do algoritmo de \textit{Langevin Monte Carlo} e, por se cancelar, não será discutido adiante. Atribua agora às novas coordenadas generalizadas $(q_{k+1}, p_{k+1})$ valor da seguinte forma
\begin{equation}
	(q_{k+1}, p_{k+1}) =
\begin{cases}
	(\tilde{q}_{k+1}, \tilde{p}_{k+1}) \ \text{com probabilidade} \ P_k, \\
	(q_k, -\tilde{p}_{k}) \ \text{com probabilidade} \ 1-P_k; \\
\end{cases}
\label{Equation: Metropolis}
\end{equation}
De forma a garantir a conservação da energia, que é uma preocupação na discretização da dinâmica, e otimizar a exploração do espaço de fase.



