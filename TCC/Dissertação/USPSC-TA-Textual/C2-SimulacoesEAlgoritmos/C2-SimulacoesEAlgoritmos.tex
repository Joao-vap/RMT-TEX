\chapter{Simulações e Algoritmos}
\label{Capitulo: Simulações}

Nos referenciaremos aqui aos desenvolvimento do artigo \cite{Chafa2018}. Compilemos o embasamento necessário e explicitaremos os resultados e métodos usados. Antes, notação. Tome um subespaço $S$ de dimensão $d$ em $\mathbb{R}^n$. O subespaço toma a métrica de Lebesgue, denotada $dx$. O campo externo é função $V : S \mapsto \mathbb{R}$ e a interação entre partículas é dada pelo núcleo $W : S \mapsto (-\infty, \infty]$. Para qualquer $N \geq 2$ consideramos $P_N$ em $S^N = S \times \cdots \times S$ definida $$ P_N(dx) = \frac{e^{-\beta_N H_N(x_1,\cdots,x_N)}}{Z_N} dx_1 \cdots dx_N, $$onde $\beta_n > 0$ é uma constante e $Z_N$ é contante de normalização. Note ainda $$ H_N(x_1, \cdots, x_N) = \frac{1}{N} \sum_{i=1}^{N} V(x_i) + \frac{1}{2N^2} \sum_{i\neq j} W(x_i - x_j).$$

Note que $P_N$ é invariante por permutação e que $H_N$ depende somente da medida empírica

\[
\mu_N = \frac{1}{N} \sum_{i=1}^{N} \delta_{x_i}
\]

Note que as partículas vivem em $S^N$ de dimensão $dN$.


% -
% C2S1 - Introdução ao algoritmo
% - 

\section{Dinâmica de \textit{Langevin Monte Carlo}}

Nosso objetivo com a simulação é determinar a esperança de uma função de interesse $\zeta(q,p)$, dado um ensemble. Pela teoria ergódica, sob algumas condições e no limite adequado, a média espacial $\langle \zeta \rangle_{\mu}$ é igual a média temporal $$\langle \zeta \rangle_t \approx \frac{1}{\tau} \sum_{k=1}^{\tau} \zeta(q_k, p_k),$$ onde $(q_k, p_k)$ podem ser obtidos por meio de uma dinâmica que preserve dada distribuição de Gibbs-Boltzmann. Para fazer o modelo ergódico, ou seja, garantir que a simulação - e nossas amostras - não esteja restrita a um subconjunto do espaço de fase, tomaremos uma dinâmica, um termostato, estocástica. Isso usualmente garante que o sistema possa convergir para sua medida invariante (única). Um esquema comumente utilizado é a dinâmica de Langevin\footnote{Poderíamos ter explorado outras dinâmicas similares tais como as dinâmicas de \textit{Dissipative Particle} \cite{DPD} ou \textit{Nose-Hoover} \cite{Hoover}.}.

Denote $q$, com $q \in \R^{(dN)}$, posição generalizada associada as $N$ partículas. A Equação \eqref{Equação: Medida Gas de Coulomb} é medida invariante do processo de difusão de Markov solução da equação diferencial estocástica
\begin{equation}
	\dd q_t = -\alpha_N \nabla \Hf_N(q_t) \dd t + \sqrt{2\frac{\gamma_N \alpha_N}{\beta_N}} \dd W_t,
	\label{Equação: Langevin Overdamped}
\end{equation}
onde $(W_t)_{t>0}$ é processo de Wiener, $\gamma_N > 0$ é constante de atrito e $\alpha_N$ é escala temporal. Isso seria suficiente e é chamado \textit{Overdamped Langevin}, contudo, tomaremos sua extensão cinética. Usaremos $q$ como variável de interesse e $p$, com $p \in \R^{(dN)}$, variável de momento generalizado, para flexibilizar a dinâmica. Considere $\U_N \colon \R^{(dN)} \rightarrow \R$ energia cinética generalizada tal que $\ee^{-\beta_N \U_N}$ seja Lebesgue integrável. Para uma energia da forma $\Ee_N(q,p) = \Hf_N(q) + \U_N(p)$, seja $(q_t, p_t)_{t\geq0}$ processo de difusão em $\R^{dN} \times \R^{dN}$ solução da equação diferencial estocástica
\begin{equation}
\begin{cases}
	\dd q_t = \alpha_N \nabla U_N (p_t) \dd t, \\
	\dd p_t = -\alpha_N \nabla \Hf_N(q_t) \dd t - \gamma_N \alpha_N \nabla U_N(p_t) \dd t + \sqrt{2\frac{\gamma_N \alpha_N}{\beta_N}} \dd B_t,
\end{cases}
\label{Equação: EqDif - Dinamica Langevin}
\end{equation}
onde $\beta_N$ é temperatura inversa e $\Hf_N \colon \R^{(dN)} \rightarrow \R$ é como na Distribuição \eqref{Equation: Medida Log V}. \cite{Stoltz2018} Esse processo deixa invariante $\p(q,p) = \p_q \otimes \p_p = \ee^{-\beta_N \Ee_N(q,p)}/Z'_N$ e admite o gerador infinitesimal 
\[
	\Gl = \Gl_{\Hf} + \Gl_{\U},
\]
\[
 \Gl_{\Hf} = -\alpha_N \nabla\Hf_N(q) \cdot \nabla_p + \alpha_N \nabla \U_N(p) \cdot \nabla_q, \ \ \ \ \Gl_{\U} = \frac{\gamma_N\alpha_N}{\beta_N} \Delta_p - \gamma_N \alpha_N \nabla \U_N(p) \cdot \nabla_p.
\]

Denomina-se $\Gl_{\Hf}$ a parte hamiltoniana e $\Gl_{\U}$ a parte de flutuação-dissipação. Tomaremos $\U_N(p) = \frac{1}{2} |p|^2$ tal que $\U_N(p)$ é energia cinética usual. Um esquema análogo é possível para energias cinéticas generalizadas. \cite{Stoltz2018} Além disso, $(B_t)_{t>0}$ é processo browniano. Para simular $(q_t,p_t)_{t \geq 0}$ integramos a Equação \eqref{Equação: EqDif - Dinamica Langevin}, contudo, isso pode não ser possível analiticamente, levando a recorrer a métodos numéricos para amostragem.

% -
% C2S2 - Algoritmo Híbrido de Monte Carlo
% - 


\section{Algoritmo Híbrido de Monte Carlo}

O algoritmo híbrido de Monte Carlo é baseado no algoritmo anterior mas adicionando uma variável de momento para melhor explorar o espaço. Defina $E = \mathbb{R}^{\dd N}$ e deixe $U_N : E \rightarrow \mathbb{R}$ ser suave para que $\ee^{-\beta_N U_N}$ seja Lebesgue integrável. Seja ainda $(X_t, Y_t)_{t>0}$ o processo de difusão em $E \times E$ solução de

\[
\begin{cases}
	\dd X_t = \alpha_N \nabla U_N (Y_t) \dd t, \\
	\dd Y_t = \alpha_N \nabla H_N(X_t) \dd t - \gamma_N \alpha_N \nabla U_N(Y_t) \dd t + \sqrt{2\frac{\gamma_N \alpha_N}{\beta_N} \dd B_t},
\end{cases}
\]
onde $(B_t)_{t>0}$ é o movimento browniano em $E$ e $\gamma_N > 0$ parâmetro representando atrito.

Quando $U_N(y) = \frac{1}{2}|y|^2$ temos $Y_t = \dd X_t/\dd t$ e teremos que $X_t$ e $Y_t$ poderão ser interpretados como posição e velocidade do sistema de $N$ pontos em $S$ no tempo $t$. Nesse caso, $U_n$ é energia cinética

% -
% C2S3 - Discretização
% - 

\section{Discretização}
\label{Seção: Discretização}

Para integrar $\Gl$, faremos separadamente a operação sobre $\Gl_{\Hf}$ e $\Gl_{\U}$. A dinâmica hamiltoniana é reversível, o que é importante no algoritmo para garantir que mantém-se a medida invariante. Ainda mais, preserva o volume do espaço de fase, de forma que não precisamos calcular o jacobiano da matriz que define a transformação da dinâmica. Essas duas propriedades podem ser mantidas quando discretizada a dinâmica pelo método de Verlet \cite{Chafa2018}\cite{leimmolecular}. A dinâmica deveria também manter o Hamiltoniano constante, contudo, discretizada, podemos garantir somente que ele se mantenha quase constante. Para lidar com esse fato, discute-se a implementação de um passo de Metropolis na próxima seção. Para $\Delta t > 0$, a partir do estado $(q_k, p_k)$, o esquema lê-se
\begin{equation}
\begin{cases}
	\tilde{p}_{k+\frac{1}{2}} = \tilde{p}_k - \nabla \Hf_N(q_k) \alpha_N \frac{\Delta t}{2}, \\
	\tilde{q}_{k+1} = q_k + \tilde{p}_{k + \frac{1}{2}} \alpha_N \Delta t, \\
	\tilde{p}_{k+1} = \tilde{p}_{k+\frac{1}{2}} - \nabla \Hf_N(q_{k+1}) \alpha_N \frac{\Delta t}{2}.
\end{cases}
\label{Equation: Verlet}
\end{equation}
Um esquema análogo é possível para energias cinéticas generalizadas \cite{Stoltz2018}. Outros métodos tais quais \textit{Euler-Maruyama} (EM) \cite[Capítulo~7]{leimmolecular} podem ser utilizados para o mesmo fim. Nos método que temos interesse o erro associado à discretização deve ir à zero quando $\Delta t$ vai à zero. Para EM, o erro por passo, local, é da ordem de $\Boh{(\Delta t^2)}$ e o erro final, global, $\Boh{(Delta t)}$, Já para o esquema escolhido, temos erro local de  $\Boh{(\Delta t^3)}$ e global de  $\Boh{(\Delta t^2)}$. Essa diferença vem do fato da discretização usada ser reversível \cite[Capítulo~5]{handbookmontecarlo}. 

Nos resta integrar $\Gl_{\U}$, o qual, para a energia cinética usual, consiste em um processo de Ornstein-Uhlenbeck de variância explícita $$dx_t = - \xi x_t dt + \sigma dB_t$$ onde $\xi, \sigma > 0$ são parâmetros e $B_t$ é processo browniano. Note que para $\alpha > 0$ substituiremos parcialmente o momento das variáveis, se $\alpha = 0$ retomaríamos \ref{Equação: Langevin Overdamped}. Este processo não é muito melhor, contudo, do que um \textit{Random Walk Metropolis} \cite[Capítulo~5]{handbookmontecarlo} já que o momento seria completamente substituído. Este processo pode ser resolvido a partir da fórmula de Mehler e obtêm-se
\begin{equation}
\tilde{p}_k = \eta p_k + \sqrt{\frac{1-\eta^2}{\beta_N}} G_k, \ \ \ \eta = \ee^{-\gamma_N \alpha_N \Delta t}.
\label{Equation: Mehler}
\end{equation}
Onde $G_k$ é variável aleatória Gaussiana usual.




