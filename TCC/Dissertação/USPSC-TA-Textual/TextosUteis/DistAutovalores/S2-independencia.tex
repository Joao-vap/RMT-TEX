Consideremos matrizes com entradas independentes. Qual a função densidade de probabilidade (F.P.D.) da matriz simétrica $\matriz{H_s}$? Devemos fazer separadamente a diagonal da seção triangular que formos usar e teremos

\[
\rho((\matriz{H_s})_{11}, \dots, (\matriz{H_s})_{NN}) = \prod_{i=1}^{N} \left[ \frac{e^{\frac{-(H_s)^2_{ii}}{2}}}{2\pi} \right] \prod_{i<j} \left[ \frac{e^{-(H_s)^2_{ij}}}{\sqrt{\pi}} \right]
\]

Que é a função distribuição da matriz. A desimetria da diagonal e da não diagonal nos permite escrever a distribuição como 

\begin{equation}
	\mathcal{P}(\mathcal{H}) = C_2 e^{-\frac{1}{2\sigma^2} \Tr{\mathcal{H}^2}}
\end{equation}

Mais geralmente

\begin{equation}
	\mathcal{P}(\mathcal{H}) = C_{\beta} e^{-\frac{\beta}{4\sigma^2} \Tr{\mathcal{H}^2}}
\end{equation}

Suponha que queremos derivar a equação da distribuição dos autovalores para tais matrizes. Começamos com $\mathcal{H}_{N\times N}$

\[
\begin{bmatrix}
	x_1 & x_{n+1} & \cdots & x_{2n} \\
	x_{n+1} & x_2 & \vdots & \vdots \\
	\vdots & \cdots & \ddots & \vdots \\
	x_{2n} & \cdots & \cdots & x_n
\end{bmatrix}
\]

Notemos que teremos $\mmany{x}{n} \sim \mathcal{N}(0,1)$ e $x_{n+1}, \cdots, x_{n^2} \sim \mathcal{N}(0,\frac{1}{2})$. Qual a distribuição $p(s)$ de $s = \lambda_2 - \lambda_1$? Sabemos os autovalores soluções do polinômio característico

\[
	\lambda^2 - \Tr{\mathcal{H}_s} \lambda + \det{\mathcal{H}_s}
\]

O resultado será uma função $s(\mmany{x}{{n^2}})$. Para a qual escrevemos

\[
p(s) = \int_{-\infty}^{\infty} dx_1 \cdots dx_{n^2} \prod_{i=1}^{n} \frac{e^{-\frac{x_i^2}{2}}}{\sqrt{2\pi}} \prod_{i=n+1}^{n^2} \frac{e^{-x_i^2}}{\sqrt{\pi}} \delta(s - s(\mmany{x}{{n^2}}))
\]

Que, desenvolvido, chega à

\[
	p(s) = -\frac{s}{2} e^{-f(s^2)}
\]

Ou seja, teremos um resultado em que as partículas parecem se repelir!