\chapter{Introdução}

\textcolor{red}{
\begin{itemize}
	\item Usaremos matrizes com entradas aleatórias;
	\item Porque matrizes;
	\item Contexto de criação na física - núcleos quânticos;
	\item Estamos interessados na distribuição de autovalores;
	\item Simulações e tempo de simulação.
\end{itemize}
}

Uma matriz aleatória é uma matriz cujas entradas são variáveis aleatórias, não necessariamente independentes tampouco de mesma distribuição. A princípio, de um ponto de vista puramente analítico, pode-se tratar uma matriz aleatória de tamanho $N\times N$ como um vetor aleatório de tamanho $N^2$. No entanto, as estruturas algébrico-geométricas presentes em matrizes, como sua multiplicação natural, interpretação como operadores e decomposições espectrais, trazem à matrizes aleatórias aplicações distintas. Em particular, a relevância de matrizes aleatórias deriva de um ponto comum compartilhado com variáveis aleatórias: descrever sistemas com modelagens estatísticas. 
	
É comum modelar com matrizes aleatórias, por exemplo, operadores com perturbações aleatórias. De um ponto de vista físico, autovalores de um dado operador descrevem espectros de energia do sistema descrito. Na quântica, por exemplo, autovalores são as medidas observadas. Surge uma pergunta naturalmente: dada uma matriz aleatória $M$, o que podemos dizer sobre estatísticas de seus autovalores? Essa resposta, claro, depende de maneira altamente não trivial das distribuições das entradas.  
