\chapter{Conclusão}
\label{Capitulo: Conclusão}

Descrevemos os fundamentos da Teoria de Matrizes Aleatórias, principalmente as ideias de ensembles e medida, explicitando as suas categorizações em invariantes por rotação e de entradas independentes. Com essas ideias, explicitamos os modelos gaussiano para $\beta = 1,2,4$, importantes em RMT pela sua característica única de pertencer à ambas categorias, de invariância e independência. Usando desse exemplo podemos entender os principais resultados sobre medidas dos ensembles invariantes e sobre a medida de autovalores destas matrizes.
 
Introduzimos a ideia de um Gás de Coulomb e como esta noção pode ser relacionada com alguns ensembles de matrizes aleatórias por uma escolha adequada de potencial e núcleo de interação. Usando da ideia de partículas para pensar na dinâmica dos autovalores intuímos as ideias de minimização da energia livre para identificar o equilíbrio. Com isso, mostramos os principais resultados que possibilitam o cálculo explicito da medida de autovalores para o caso gaussiano e mais dois ensembles que usaremos como exemplo nas simulações que seguem.

Com a analogia, percebemos que muitas vezes métodos numéricos são necessários para a solução das equações de movimento que descrevem a dinâmica das partículas modeladas. Discutimos os principais métodos e principalmente as características do algoritmo implementado, justificando seu uso e propondo sua forma final apresentada também neste trabalho.

Por fim, apresentamos os resultados. Simulações de medidas para os ensembles já descritos anteriormente, todos com autovalores reais, e um último ensemble com autovalores complexos, menos descrito. Para os efeitos deste trabalho, foi encontrada boa concordância das medidas simuladas com a descrição teórica atual e com os modelos de matriz testados. Principalmente para o ensemble de autovalores complexos, este resultado é de interesse visto que apresenta uma forma alternativa de simulação e, também, de análise numérica destes modelos para diversos ensembles explorados em literatura recente.

Finalmente, entendemos os resultados deste estudo como uma descrição e validação de métodos conhecidos de simulação e estudo de matrizes aleatórias, ainda que atuais. Contudo, vê-se extensões da utilização do método para estudo numérico de importantes resultados com menos descrição teórica.
