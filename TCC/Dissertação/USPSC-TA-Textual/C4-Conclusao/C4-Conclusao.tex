\chapter{Conclusão}
\label{Capitulo: Conclusão}

A Teoria de Matrizes Aleatórias é uma ferramenta matemática extremamente versátil. Suas aplicações são extensas e diversas, cobrindo ambos sistemas físicos e matemáticos de grande relevância. Partindo da hipótese que autoenergias de sistemas complexos se comportam localmente como autovalores de matrizes aleatórias adequadas, permite-se a caracterização estatística do núcleo atômico ou ainda a determinação de propriedades físicas de metais. Em matemática, além das clássicas aplicações estatísticas dos modelos, mostra-se que matrizes aleatórias tem importante papel na determinação dos zeros da função de Riemann. Consolidada assim sua importância, a teoria se desenvolve rapidamente e tem chamado atenção da comunidade científica-matemática. Introduzimos neste trabalho as ideias de medida de matrizes aleatórias e ensembles, essenciais à RMT, e descrevemos os clássicos ensembles gaussianos, que julgamos exemplares para o entendimento dos resultados sobre medida nos autovalores e equilíbrio.
 
A analogia de Gases de Coulomb surge naturalmente ao se explicitar a medida de matrizes de ensembles invariantes. Sua interpretação permite pensar na dinâmica dos autovalores como uma de partículas interagentes, da qual intuímos as ideias de minimização da energia livre para identificar o equilíbrio. Percebemos que muitas vezes métodos numéricos são necessários para a solução das equações de movimento estocásticas que descrevem a dinâmica das partículas modeladas. Apresentamos então os métodos de simulação numérica e discutimos as principais características do algoritmo implementado, denominado \textit{Langevin Monte Carlo}.

Além disso, apresentamos os resultados, que dividimos, em propósito, em duas partes. Os primeiros resultados são de medidas de autovalores na reta, bem explorados na teoria e relativamente simples. Para estes, incluímos explicitamente no trabalho as soluções. Qualitativamente observa-se que os resultados tem boa concordância com a teoria, mesmo em distribuições mais delicadas, como a Tracy-Widow. Isso nos dá boa indicação do bom comportamento dos métodos e implementação utilizados. Com isso, apresentamos um dos resultados obtidos em um Gás de Coulomb em duas dimensões. Isso de refere a um potencial complexo, explorado com mais afinco apenas em teoria recente. Mesmo aqui, mostra-se que é possível replicar características de resultados apontados em trabalhos recentes e indica uma possível direção para exploração numérica da teoria.

Entendemos este estudo como uma descrição e validação de métodos conhecidos de simulação e matrizes aleatórias, ainda que atuais. Contudo, veem-se extensões da utilização do método para estudo numérico de importantes resultados com menos descrição teórica, o que, até onde sabemos, é menos explorado.
