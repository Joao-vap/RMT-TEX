\chapter{Introdução}
\label{Capitulo: Intro}

Sistemas integráveis em física são descritos por equações diferenciais simples o suficiente tais que se pode determinar soluções explícitas. Seu comportamento é, em algum sentido, previsível e unicamente determinado pelas condições iniciais. Naturalmente, muitos sistemas de interesse não se enquadram nessa classe, são chamados caóticos ou não integráveis. Seja por complexidade ou instabilidade, não conseguimos expressar ou resolver significativamente os operadores associados à esses sistemas. 

De acordo com a mecânica quântica, níveis de energia de uma sistema são descritos pelos autovalores de seu operador hermitiano associado, o hamiltoniano $\Hf$. Para um modelo simples o suficiente, caracterizar o sistema físico é equivalente à resolver o problema de autoenergias $\Hf \Psi_i = E_i \Psi_i$. Contudo, para estado excitados de alta energia de núcleos atômicos pesados, por exemplo, esta abordagem se torna impeditiva, ou não se sabe o hamiltoniano ou sua solução é complicada. Wigner sugere uma abordagem alternativa, uma mecânica estatística para o problema de autovalores. Tal teoria descreveria as propriedades estatísticas da estrutura energética nucleica ao invés de detalhar seus níveis. Buscava-se, em algum sentido, uma universalidade, uma descrição que fosse, dada complexidade o suficiente, sensível às simetrias mas independente dos detalhes em $\Hf$. A teoria foi prontamente seguida por, dentre outros, Gaudin, Mehta \cite{MehtaGaudin}, e Dyson \cite{Dyson}, que avançaram na descrição dos principais ensembles. Esse desenvolvimento é o início do que chamamos hoje Teoria de Matrizes Aleatórias (RMT, \textit{Random Matrix Theory}).

Para alguns ensembles, chamados invariantes, uma importante analogia se apresenta, a de Gases de Coulomb. Pensando os autovalores como partículas de um gás interagente sob potencial externo, podemos usar de noções físicas para derivar, por exemplo, as densidades de autovalores no limite termodinâmico. A analogia permite, mudando a caracterização do gás, explorar ensembles de matrizes com entradas correlacionadas, de difícil construção direta. Contudo, nem sempre soluções analíticas são possível para as equações diferencias estocásticas que descrevem a dinâmica destes gases. Por isso, recorre-se à simulações numéricas. Mesmo estas, podem ser difíceis de tratas, a dinâmica tem alta complexidade temporal e as singularidades dificultam manter invariante a energia. Ainda assim, exploraremos abordagens que tornam a simulação da dinâmica suficientemente acurada e permitem, de forma direta, descrição numérica de casos analiticamente complicados e visualização de fenômenos, medidas e funções outrossim inacessíveis, ainda que em alguns casos, qualitativamente.


\chapter{Matrizes Aleatórias}
% -
% C1S1 - Distribuição de autovalores
% - 
\section{Distribuição de autovalores}

Seja $\Se$ um conjunto tal como $\R, \C, \He $ (Reais, Complexos e Quaterniônicos). Consideremos inicialmente uma matriz $\matriz{M} \in \mathcal{M}_{\Se}(N)$ espaço de matrizes $N \cross N$, ou seja, de $N^2$ entradas, sejam elas reais, complexas ou quaterniônicas. Se tomamos o elemento de matriz $M_{i,j}$ $\forall i, j \in \Z$, com $1 \leq i, j \leq N$, como variável aleatória de distribuição arbitrária, podemos expressar a densidade de probabilidade conjunta (jpdf, \textit{joint probability density function}) como $$\p(\hat{M}) \dd M = \p(M_{1,1}, \dots, M_{N,N}) \prod_{i,j=1}^{N} \dd M_{i,j}.$$

Não lidaremos, contudo, com uma classe tão ampla de matrizes. Considere a decomposição $\matriz{M} = \matriz{O} \matriz{D} \matriz{O}^{-1}$, onde $\matriz{D} = \diag(\mmany{\lambda}{N})$. Estamos especialmente interessados no caso onde $\matriz{O} \in V_N(\Se^N)$, espaço denominado variedade de Stiefel. Isso implica que $ \matriz{O} \matriz{O}^* = \Id$. Tomamos $\matriz{M}$ matriz ortogonal, unitária ou simplética, a depender de $\Se$, o que resulta em autovalores $\lambda \in \R$. Isto pode ser motivado fisicamente sabendo que, para sistemas quânticos invariantes reversíveis, o Hamiltoniano é matriz real simétrica; na presença de campo magnético, o Hamiltoniano é matriz complexa hermitiana; na presença de acoplamento spin-órbita, o Hamiltoniano é simplético \cite[Capítulo~2]{RMT-firstcourse-Potters}.

Para o subespaço tomado vale que $\cjgt{M_{i,j}} = M_{j, i}$. Este vínculo reflete na dimensão do subespaço escolhido, com valor dependente de $\Se$. A transformação tomada tem ainda Jacobiano $\J(\matriz{M} \rightarrow \{ \vec{\lambda}, \matriz{O} \} )$ tal que reescrevemos a jpdf como 
\begin{equation}
	 \p(\hat{M}) \dd M = \p \left( M_{1,1}(\vec{\lambda}, \matriz{O}), \cdots, M_{N,N}(\vec{\lambda}, \matriz{O}) | \J(\matriz{M} \rightarrow \{ \vec{\lambda}, \matriz{O} \} ) \right) \dd O \prod_{i=1}^{N} \lambda_i.
\label{Equation: p(lambda, O)}
\end{equation}

Aqui, ressalta-se que estamos interessados em distribuições de autovalores. Para calcular $\p(\mmany{\lambda}{N})$ devemos integrar os termos à direita da equação \ref{Equation: p(lambda, O)} sobre o subespaço $V_N(\Se^N)$. Isso nem sempre é fácil ou possível. Para garantir a integrabilidade, tomaremos \textit{ensembles} de matrizes aleatórias onde o jpdf de suas entradas pode ser escrito exclusivamente como função dos autovalores, ou seja $$\p(\mmany{\lambda}{N}, \matriz{O}) \equiv \p \left( M_{1,1}(\vec{\lambda}), \cdots, M_{N,N}(\vec{\lambda}) | \J(\matriz{M} \rightarrow \{ \vec{\lambda} \} ) \right).$$

Ensembles com esta propriedade são denominados invariantes por rotação. Esta escolha implica que quaisquer duas matrizes $\matriz{M}, \matriz{M'}$ que satisfaçam a relação de equivalência $\matriz{M} = \matriz{U} \matriz{M'} \matriz{U}^{-1}$ tem mesma probabilidade. Nesta relação, $\matriz{U}$ é simétrica, hermitiana ou simplética respectivamente quando $\Se = \R,\C,\He $. Considere o teorema \cite[Capítulo~3]{AlanThesis}.
\begin{thm}
	Tome $\matriz{M} \in M_{\R}(N),  M_{\C}(N),  M_{\He}(N)$ simétrica, hermitiana ou autodual, respectivamente. Se  $\matriz{M}$ tem jpdf da forma $\phi(\matriz{M})$, invariante sobre transformações de similaridade ortogonal, a jpdf dos $N$ autovalores ordenados de $\matriz{M}$, $\mcmany{\lambda}{N}{\geq}$, é $$ C_{N}^{(\beta)} \phi(\matriz{D}) \prod_{i < j} (\lambda_i - \lambda_j)^{\beta}$$ onde $C_{N}^{\beta}$ é constante e $\beta = 1, 2, 4$ corresponde respectivamente à $\matriz{M} \in M_{\R}(N),  M_{\C}(N),  M_{\He}(N)$. 
	\label{Teorema: Invariante}
\end{thm}

Logo, desde que tomemos um ensemble de matrizes aleatórias com a jpdf das entradas apropriado, podemos reescrever a distribuição em função dos autovalores com \ref{Teorema: Invariante}. Vale ainda observar que, pelo Lema de Weyl, uma jpdf invariante pode ser expressa totalmente por $\p(\matriz{M})= \phi \left(\Tr(F(M)) \right)$ com $F$ função polinomial, ou ainda
\begin{equation}
	\p_{ord}(\mmany{\lambda}{N}) = C_{N}^{\beta} \phi{\left( \sum_i^N F(\lambda_i) \right)} \prod_{i < j} (\lambda_i - \lambda_j)^{\beta}.
	\label{Equation: p-ord}
\end{equation}

%Essa expressão será usada em breve. Aqui, é mais natural entender o teorema quando se entende a constante $C_N^{\beta}$ como relacionada à integração $\int_{V_N(\Se^N)} \dd O$ e quando se enuncia o lema:

%\begin{lemma}
%	\[
%	\J(\matriz{M} \rightarrow \{ \vec{\lambda}, \matriz{O} \}) = \prod_{j > k} (\lambda_j - \lambda_k)^\beta
%	\]
%	Onde $\beta = 1,2,4$ respectivamente quando $M_{i,j} \in \R, \C, \He $.
%	\label{Lema: Jacobiano}
%\end{lemma}

% -
% C1S2 - Emsembles Gaussianos
% - 
\section{Ensembles Gaussianos}

Dentre os muitos ensembles da Teoria de Matrizes Aleatórias (RMT), os ensembles Gaussianos são notórios. São eles o \textit{Gaussian Orthogonal Ensemble (GOE)} \textit{Gaussian Unitary Ensemble (GUE)} e \textit{Gaussian Sympletic Ensemble (GSE)}. Notemos primeiramente que o nome é relacionado à escolha de $\Se$. Mais explicitamente, o nome é dado em relação à se $\matriz{O}$, tal que $\matriz{M} = \matriz{O}\matriz{D}\matriz{O}^*$, é ortogonal, unitário ou simplético. É natural então pensar nos ensembles \textit{GOE}, \textit{GUE} e \textit{GSE} como matrizes $\matriz{M} \in \mathcal{M}_{\Se}(N)$ onde 
$$ 
\mathcal{M}_{\Se}(N) \ni M_{i,j} \sim
\begin{cases}
	\mathcal{N}_{\R}(0,1/2) &  \ \text{para} \ i \neq j \ \text{se} \ \ \Se = \R \ (\beta = 1),\\
	\mathcal{N}_{\R}(0,1) & \ \text{para} \ i = j \ \text{se} \ \ \Se = \R \ (\beta = 1),\\
	\mathcal{N}_{\C}(0,1/2)  & \ \text{para} \ i \neq j \ \text{se} \ \ \Se = \C \ (\beta = 2),\\
	\mathcal{N}_{\C}(0,1) & \ \text{para} \ i = j \ \text{se} \ \ \Se = \C \ (\beta = 2),\\
	\mathcal{N}_{\He}(0,1/2) & \ \text{para} \ i \neq j \ \text{se} \ \ \Se = \He \ (\beta = 4), \\
	\mathcal{N}_{\He}(0,1) & \ \text{para} \ i = j \ \text{se} \ \ \Se = \He \ (\beta = 4).
\end{cases} $$


Os três ensembles gaussianos compartilham de uma propriedade exclusiva. Estes são os únicos ensembles tais que suas entradas são independentes e sua jpdf permanecem sendo rotacionalmente invariante. Para qualquer outro caso, apenas uma das propriedades pode ser esperada. Tomemos, por simplicidade, $\matriz{U} \in \mathcal{M}_{\R}(N)$, matriz real simétrica, do GOE. Para esta, sabendo as entradas independentes, podemos escrever $$\p(\matriz{U}) = \prod_{i=1}^{N}\frac{\exp{\frac{U_{i,i}^2}{2}}}{\sqrt{2\pi}} \prod_{i<j} \frac{\exp{U_{i,i}^2}}{\sqrt{\pi}} = 2^{-N/2} \pi^{-N(N + 1)/4} \exp{-\frac{1}{2} \Tr{U^2}}.$$

Note que essa jpdf satisfaz as condições do Teorema \ref{Teorema: Invariante} e, especialmente, é da forma que propomos na Equação \ref{Equation: p-ord}. Logo, utilizando o resultado, $$ \p_{ord}(\mmany{\lambda}{N}) = \frac{1}{Z_{N, \beta = 1}^{(ord)}} \exp{-\frac{1}{2} \sum_{i = 1}^{N} \lambda_i^2} \prod_{i < j} (\lambda_i - \lambda_j).$$ Concluímos notando que, se desordenarmos os autovalores, temos a relação $ Z_{N, \beta} = N! Z_{N, \beta}^{(ord)}$\footnote{Fator de contagem correta de Boltzmann}. Assim, $$ \p(\mmany{\lambda}{N}) = \frac{1}{ N! Z_{N, \beta = 1}^{(ord)}} \exp{- \left(\frac{1}{2} \sum_{i = 1}^{N} \lambda_i^2 + \sum_{i < j} \log\frac{1}{|\lambda_i - \lambda_j|} \right)}.$$

De forma análoga, podemos deduzir mais geralmente para os outros casos que

\begin{equation}
	\begin{split}
		\p(\mmany{\lambda}{N}) 
		&= \frac{1}{ N! Z_{N, \beta}^{(ord)}} \exp{- \left(\sum_{i = 1}^{N} \frac{\lambda_i^2}{2} - \sum_{i < j} \log{|\lambda_i - \lambda_j|^{\beta}} \right)} \\
		&= \frac{1}{Z_{N, \beta}} \ee^{-\beta \mathcal{H}_N(\vec{\lambda})}
	\end{split}
\label{Equation: medida Gaussian}
\end{equation}

Note que, por definição, $Z_{N, \beta}$, na equação \ref{Equation: medida Gaussian}, é função de partição canônica. O fator $\beta$ é pensado como a temperatura inversa. Definimos ainda o Hamiltoniano $\mathcal{H}_N(\vec{\lambda}) = \sum_{i = 1}^{N} \frac{\lambda_i^2}{2 \beta} + \sum_{i < j} \log{\frac{1}{|\lambda_i - \lambda_j|}}.$ Sabemos então, que a partir dessa função podemos retirar importantes propriedades estatísticas dos ensembles Gaussianos.



% -
% C1S3 - Gases de Coulomb (Log Gas?)
% - 
\section{Gases de Coulomb}
\label{Section: Gases de Coulomb}

Sob as devidas condições, o gás de coulomb $\p_N$ \cite{ChafaCoulombMeasure} é medida de probabilidade de Boltzmann-Gibbs dada em $(R^d)^N$. A medida $\p_N$ modela um gás interagente de partículas indistinguíveis sob potencial externo nas posições $\mmany{x}{N} \in \Se$ de dimensão $d$ em $\R^n$ \textit{ambient space}. A medida é dada por 
\begin{equation}
	\dd \p_N(\mmany{x}{N}) = \frac{e^{-\beta N^2 \Hf_N(\mmany{x}{N})}}{Z_{N,\beta}} \mcmany{\dd x}{N}{},
	\label{Equação: Medida Gas de Coulomb}
\end{equation}
onde $$\Hf_N(\vec{x}) = \frac{1}{N} \sum_{i = 1}^{N} \V(x) + \frac{1}{2N^2} \sum_{i \neq j} \g(x_i - x_j)$$ é usualmente chamado hamiltoniano\footnote{Note que $\p_N$ é um modelo de interações estáticas e não há campos magnéticos considerados.} ou energia do sistema. $\V \colon \Se \mapsto \R$ é potencial externo e $\g \colon \Se \mapsto (-\infty, \infty]$ núcleo de interação coulombiana solução da equação de Poisson dada por $- \nabla g(\vec{x}) = c_n\delta_0$. Além disso, $\beta N^2$ é chamado temperatura inversa. Assumiremos, para que valha a definição \ref{Equação: Medida Gas de Coulomb}, que $V, \ \g \ \text{e} \ \beta$ são tais que a constante de normalização (função partição) $Z_{N, \beta} < \infty \ \forall \ N$.

Se lembramos da expressão \ref{Equation: medida Gaussian}, perceberemos que, para o devido $\V \colon \R \rightarrow \R$, podemos tomar $d=1$ e $n = 2$ para recuperar a medida dos ensembles gaussianos 
\begin{equation}
	\p_N(\vec{x}) = \frac{e^{-\beta_N \Hf_N(\vec{x})}}{Z_{N,\beta}}, \ \ \Hf_N(\vec{x}) = \frac{1}{N} \sum_{i = 1}^{N} \V(x_i) + \frac{1}{N^2} \sum_{i < j} \log{\frac{1}{|x_i - x_j|}}.
	\label{Equation: Medida Log V}
\end{equation}
Estamos tratando de partículas no plano confinadas à uma reta neste caso. Para algum potencial arbitrário, além da devida escolha de $n$ e $d$, cairemos em outros ensembles de matrizes. Podemos, por exemplo, tomar partículas com suporte no plano tomando $d=2$ e $\V \colon \R^2 \rightarrow \R$. Outras extensões são admissíveis mas ficam fora do escopo deste trabalho.

% -
% C1S4 - Medidas de Equilíbrio
% - 
\section{Medidas de Equilíbrio}
\label{Seção: Medida}
O conjunto de pontos do espaço de fase, seus microestados, determinam um \textit{ensemble estatístico}\footnote{O nome 'Ensembles de Matrizes' não é coincidência.}. Não é difícil notar que o conjunto de microestados $\{\vec{\lambda}\}$ do sistema de $N$ autovalores descrito nesse trabalho caracteriza o ensemble canônico, com função partição $Z_{N, \beta}$, soma sobre os estados do sistema. Um argumento termodinâmico nos indica então que devemos minimizar a energia livre de Helmholtz $$F = -\frac{1}{\beta} \log{Z_{N, \beta}}.$$
Para todos os efeitos, consideraremos $\V, \ \g$ e $\beta$ tais que dada $\mu_{V,g}(\vec{\lambda})$ medida de probabilidade em $\Omega$, espaço das possíveis configurações de autovalores, e maximizada a função partição $Z_{N, \beta} = \int_{\Omega} \exp{-\beta \mathcal{H}_N(\vec{\lambda})}$, exista\footnote{Condições de Fisher} $$\mu_{V,g}^* = \arg \inf {\mathcal{H}_N(\vec{\lambda})}$$ medida de equilíbrio no limite termodinâmico $N, V \rightarrow \infty$ tal que $v = V/N$ constante. Para determinar a medida de equilíbrio \cite{RMT-firstcourse-Potters} de \ref{Equação: Medida Gas de Coulomb} com interação logarítmica, queremos satisfazer o sistema de equações
\begin{equation}
	\frac{\partial \mathcal{H}}{\partial \lambda_i} = 0 \ \implies \ \V'(\lambda_i) = \frac{1}{N} \sum_{1 = j \neq i}^{N} \frac{1}{\lambda_i - \lambda_j} \ \ \text{para} \ i = 1, \cdots, N.
	\label{Equação: Sistema minimizante}
\end{equation} 
Usaremos o denominado \textit{resolvent}. Considere a função complexa\footnote{Stieltjes transform} $$G_N(z) = \frac{1}{N} \Tr{\left(z\Id - \matriz{M}\right)^{-1}} = \frac{1}{N} \sum_{i=1}^{N} \frac{1}{z - \lambda_i},$$ onde $\matriz{M}$ é matriz aleatória com autovalores $\{\mmany{\lambda}{N}\}$. Note que $G_N(z)$ é uma função complexa aleatória com polos em $\lambda_i$. Não trivialmente, podemos reescrever \ref{Equação: Sistema minimizante} como $$\V'(z) G_N(z) - \Pi_N(z) = \frac{G_N^2(z)}{2} + \frac{G'_N(z)}{2N},$$ onde $\Pi_N(z) = \frac{1}{N} \sum_{i = 1}^{N} \frac{\V'(z) - \V'(\lambda_i)}{z - \lambda_i}$ é um polinômio de grau $k - 1 = \deg{\V'(z)} - 1$. 
Poderíamos tentar resolver explicitamente essa formula para qualquer $N$, isso é possível em alguns casos. Contudo, em geral, estaremos interessados em tirar o limite $N \to \infty$, de $<G_N(z)>$, média sobre a distribuição de $\matriz{M}$. Esta média, denomina-se \textit{resolvent}. Nesse limite,
\begin{equation}
	G^{(med)}_{\infty}(z) = \int \frac{\p(x)}{z - x} \dd x= \V'(z) \pm \sqrt{\V'(z)^2 - 2 \Pi_{\infty}(z) }.
	\label{Equation: Resolvent}
\end{equation}
Como consequência da fórmula de Sokhotski-Plemeji, é enunciado o resultado 
\begin{equation}
	\p(x) = \frac{1}{\pi} \lim_{\epsilon \to 0^+} \Im{G_{\infty}^{(med)}(x - \ii\epsilon)}.
	\label{Equation: p(lambda)}
\end{equation}
Podemos ir um passo além, desde que o potencial $\V(x)$ seja convexo. Neste caso, teremos uma medida de equilíbrio $\p(x)$ não nula apenas no intervalo $(\lambda_{-}, \lambda_{+})$. Sabemos que o comportamento não analítico deve surgir da raiz quadrada, tal que se definirmos $\Df(z) := \V'(z)^2 - 2 \Pi_{\infty}(z)$ polinômio de grau $2k$, $\{\lambda_{-}, \lambda_{+}\}$ são suas raízes e o polinômio tem valor negativo em algum intervalo. Equivalentemente $$D(z) = (z-\lambda_{-})(z - \lambda_{+}) \Qf^2(z),$$ onde $\Qf(z)$ é polinômio de grau $k-1$. Com essas definições podemos escrever que $$G_{\infty}^{(med)}(z) = \V'(z) \pm \Qf(z) \sqrt{(z - \lambda_{-})(z - \lambda_{+})}$$ e, principalmente, por \ref{Equation: p(lambda)},
\begin{equation}
	\p(x) =\frac{\Qf(x)}{\pi} \sqrt{(\lambda_{+} - x)(x - \lambda_{-})}, \ \ \text{para} \ \  \lambda_{-} \leq x \leq \lambda_{+}
\end{equation}
Restaria, para cada potencial, dada a condição que $G_{\infty}^{(med)}(z) \sim 1/z$ para $z \rightarrow \infty$, resolver um sistema de $k+2$ equações balanceando os coeficientes dos polinômios $\V'$ e $\Qf$ e os valores $\{\lambda_{-}, \lambda_{+}\}$ que tomaremos simétrico $a = \lambda_{+} = \lambda_{-}$, em
\[
\frac{1}{\pi \ii} \int_{\lambda_{-}}^{\lambda_{+}} \frac{\sqrt{x^2 - a^2}\Qf(x)}{z-x} dx = \V'(z) \pm \sqrt{z^2 - a^2}\Qf(z) 
\]




% -
% C1S5 - Potenciais notáveis
% - 
\section{Potenciais notáveis}
\label{Section: Potencias}

 O desenvolvimento feito na seção \ref{Seção: Medida} é suficiente para resolver os casos exemplificados aqui, salvo detalhes. Explicitar a conta não elucidaria a teoria e por isso foi omitido. Retome a medida \ref{Equation: Medida Log V} e considere os seguintes potencias.


%Um resultado importante enuncia \cite{deiftorthogonal}:

%\begin{thm}
%	Para $V(x) = t x^{2m}$ com $t>0$, vale que $$ \p_V(x) = - \frac{m t}{\pi} \sqrt{x^2 - a^2} + h(x) $$ no suporte $\supp(-a, a)$. Onde, $$ a = \left( mt \prod_{l=1}^{m} \frac{2l - 1}{2l} \right)$$ e $$h(x) = x^{2m-2} + \sum_{j=1}^{m-1} x^{2m - 2 - 2j} a^{2j} \prod_{l=1}^{j}.$$
%	\label{Teorema: Medida V(x)}
%\end{thm}


\subsection{Potenciais Quadráticos}
Considere o potencial $$\V(x) = \frac{x^2}{2}.$$ Neste caso, resolvemos o sistema para descobrir que
\begin{equation}
	 \supp{\mu_V(x)} = [-\sqrt{2}, \sqrt{2}], \ \ \ \text{e} \ \ \ \mu_V(x) = \frac{1}{\pi} \sqrt{2 - x^2}.
	 \label{Equação: Quadrático}
\end{equation}
Esse resultado é bem conhecido e a medida encontrada é denominada Semi-Círculo de Wigner. Note que isso vale para qualquer $\beta$, a diferença é notada somente quando $N$ é suficientemente pequeno.

\subsection{Potencial Quártico}

Considere o potencial $$\V(x) = \frac{x^4}{4} + t \frac{x^2}{2}.$$ Aqui observaremos, a depender de $t$, pela primeira vez a separação do suporte da função. Teremos um ponto crítico em $t=-2$ onde o suporte se separa nos intervalos $[-b_t, -a_t]$ e $[a_t, b_t]$ para $t < -2$. Para $t \geq -2$ o suporte é um único intervalo $[-b_t, b_t]$. Definiremos a medida nos dois casos,
\begin{itemize}
	\item \(t \geq -2\)
	\begin{equation}
	\supp \mu_V(x) = [-b_t, b_t], \ \ \mu_V(x) = \frac{1}{2\pi} (x^2 + c_t^2) \sqrt{b_t^2 - x^2},\label{Equação: Quartico +}
	\end{equation}
	com $c_t^2 \deff\frac{1}{2} b_t^2 + t \deff \frac{1}{3} (2t + \sqrt{t^2 + 12})$.
	\item \(t < -2\)
	\begin{equation}
	\supp \mu_V(x) = [-b_t, -a_t] \cup [a_t, b_t], \ \ \mu_V(x) = \frac{1}{2\pi} |x| \sqrt{(x^2 - a_t^2)(b_t^2 - x^2)},
	\label{Equação: Quartico -}
	\end{equation}
	com $ a_t \deff \sqrt{-2-t}, b_t \deff \sqrt{2-t}$.
\end{itemize}

%\subsection{Potencial Mônico}

%Considere o potencial

%\[
%V(x) = \frac{t}{2\alpha} x^{2\alpha},
%\]
%onde $t > 0$ é escala e $\alpha \in \Z$. A medida de equilíbrio para $\alpha = 1$ é o semi-círculo de Wigner podemos validar na figura com a distribuição em vermelho. Sabemos também que o suporte $[-a, a]$ da densidade é dado por

%\[
%a = \left( \frac{t}{2} \prod_{j=1}^{\alpha} \frac{2j-1}{2j} \right)^{-\frac{1}{2\alpha}}.
%\]

\subsection{Potencial Mônico}
 
 Por último, tome $$\V(x) = t x^{2m}.$$ Com o mesmo processo, apesar de mais geral, determinamos sua medida 
 \begin{equation}
 	\supp \mu_V(x) = [-a, a], \ \ \mu_V(x) = \frac{mt}{\pi} \sqrt{a^2 - x^2} \h(x),
 	\label{Equação: Mônico}
 \end{equation}
com $ a \deff \left( mt \prod_{l=1}^{m} \frac{2l-1}{2l}\right)$ e $$\h(x) = x^{2m-2} + \sum_{j=1}^{m-1} x^{2m-2-2j} a^{2j} \prod_{l=1}^{j} \frac{2l-1}{2l}.$$

