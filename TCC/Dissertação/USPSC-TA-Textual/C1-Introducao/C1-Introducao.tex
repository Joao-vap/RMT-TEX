\chapter{Introdução}
\label{Capitulo: Intro}

De acordo com a mecânica quântica, níveis de energia de uma sistema são descritos pelos autovalores de seu operador hermitiano associado, o hamiltoniano $\Hf$. Em situações mais complexas, $\Hf$ não é completamente descrito pela teoria ou é complicado. Por isto, somado à necessidade do cálculo explícito de grandezas, considera-se usualmente truncamentos do espaço de Hilbert onde opera $\Hf$, representado agora por matriz de dimensão finita. Caracterizar o sistema físico é resolver o problema de autovalores $\Hf \Psi_i = E_i \Psi_i$. Wigner, em seu estudo de núcleos atômicos, foi um dos primeiros a sugerir uma solução alternativa, uma mecânica estatística para o problema de autovalores. Tal teoria descreveria o perfil da estrutura energética nucleica ao invés de detalhar seus níveis, impondo perguntas estocásticas ao invés de determinísticas, que comumente seriam impossíveis de responder. Buscava-se, em algum sentido, uma universalidade, uma resposta que fosse, dada complexidade o suficiente, independente de $\Hf$. A teoria foi prontamente seguida por, dentre outros, Porter e Rosenzweig \cite{PoterRosen}, que procederam a validar com dados experimentais as ideias postuladas e por Gaudin, Mehta \cite{MehtaGaudin}, e Dyson \cite{Dyson}, que avançaram na descrição, dentre outros, dos importantes ensembles gaussianos. Esse desenvolvimento e seus desdobramentos veio a ser o que chamamos hoje de Teoria de Matrizes Aleatórias (RMT, \textit{Random Matrix Theory}). Hoje, suas aplicações são extensas em campos de alta complexidade ou com descrição matricial, principalmente quando há estrutura, como matrizes de correlação ou operadores físicos.

Para os ensembles que chamamos invariantes, comuns na física, ao calcular a densidade de autovalores, uma importante analogia se mostra disponível, a de Gases de Coulomb. Pensando os $N$ autovalores das matrizes como partículas de um gás com o devido núcleo de interação e potencial externo, podemos usar de noções físicas para intuir seu comportamento. Usando das estabelecidas leis termodinâmicas é possível ainda derivar, por exemplo, as densidades de autovalores no limite termodinâmico ($N \rightarrow \infty$). Generalizando o potencial aplicado podemos ainda recuperar um ensemble de matrizes que pode não diretamente disponível. Usar de simulações de gases para extrair medidas tem algumas dificuldades. Nem sempre uma solução analítica é possível para as equações diferencias que descrevem sua dinâmica, que deve ser ergótica. Por isso, recorre-se à simulações numéricas que podem também se mostrar delicadas de tratar. A dinâmica tem alta complexidade temporal pela quantidade de interações entre partículas e as singularidades dificultam a invariância do hamiltoniano na simulação. Ainda assim, essa abordagem permite explorar ensembles exóticos em RMT e será discutida nesse trabalho.

% -
% C1S1 - Distribuição de autovalores
% - 
\section{Distribuição de autovalores}

Seja $\Se$ um conjunto tal como $\R, \C, \He $ (Reais, Complexos e Quaterniônicos). Consideremos inicialmente uma matriz $\matriz{M} \in \mathcal{M}_{\Se}(N)$ espaço de matrizes $N \cross N$, ou seja, de $N^2$ entradas, sejam elas reais, complexas ou quaterniônicas. Se tomamos o elemento de matriz $M_{i,j}$ $\forall i, j \in \Z$, com $1 \leq i, j \leq N$, como variável aleatória de distribuição arbitrária, podemos expressar a densidade de probabilidade conjunta (jpdf) como $$\p(\hat{M}) \dd M = \p(M_{1,1}, \dots, M_{N,N}) \prod_{i,j=1}^{N} \dd M_{i,j}.$$

Não lidaremos, contudo, com uma classe tão ampla de matrizes. Considere a decomposição $\matriz{M} = \matriz{O} \matriz{D} \matriz{O}^{-1}$, onde $\matriz{D} = \diag(\mmany{\lambda}{N})$. Estamos especialmente interessados no caso onde $\matriz{O} \in V_N(\Se^N)$, espaço denominado variedade de Stiefel. Isso implica que $ \matriz{O} \matriz{O}^* = \Id$. Tomamos $\matriz{O}$ matriz ortogonal, unitária ou simplética, a depender de $\Se$, o que resulta em autovalores $\lambda \in \R$. Isso pode ser motivado fisicamente, por exemplo, se pensarmos que queremos tomar grandezas reais.

Para o subespaço tomado vale que $\cjgt{M_{i,j}} = M_{j, i}$. Este vínculo reflete na dimensão do subespaço escolhido, com valor dependente de $\Se$. A transformação tomada tem ainda Jacobiano $\J(\matriz{M} \rightarrow \{ \vec{\lambda}, \matriz{O} \} )$. Com estes fatos podemos reescrever a jpdf como 
\begin{equation}
	 \p(\hat{M}) \dd M = \p \left( M_{1,1}(\vec{\lambda}, \matriz{O}), \cdots, M_{N,N}(\vec{\lambda}, \matriz{O}) | \J(\matriz{M} \rightarrow \{ \vec{\lambda}, \matriz{O} \} ) \right) \dd O \prod_{i=1}^{N} \lambda_i.
\label{Equation: p(lambda, O)}
\end{equation}

Aqui, ressalta-se que estamos interessados em distribuições de autovalores. Para calcular $\p(\mmany{\lambda}{N})$ devemos integrar os termos à direita da Equação \ref{Equation: p(lambda, O)} sobre o subespaço $V_N(\Se^N)$. Isso nem sempre é fácil ou possível. Para garantir a integrabilidade, tomaremos \textit{ensembles} de matrizes aleatórias onde o jpdf de suas entradas pode ser escrito exclusivamente como função dos autovalores, ou seja $$\p(\mmany{\lambda}{N}, \matriz{O}) \equiv \p \left( M_{1,1}(\vec{\lambda}), \cdots, M_{N,N}(\vec{\lambda}) | \J(\matriz{M} \rightarrow \{ \vec{\lambda} \} ) \right).$$

Ensembles com esta propriedade são denominados invariantes por rotação. Esta escolha implica que quaisquer duas matrizes $\matriz{M}, \matriz{M'~}$ que satisfaçam a relação de equivalência $\matriz{M} = \matriz{U} \matriz{M'} \matriz{U}^{-1}$ tem mesma probabilidade. Nesta relação, $\matriz{U}$ é simétrica, hermitiana ou simplética respectivamente quando $\Se = \R,\C,\He $. Considere o teorema \cite{AlanThesis}.
\begin{thm}
	Tome $\matriz{M} \in M_{\R}(N),  M_{\C}(N),  M_{\He}(N)$ simétrica, hermitiana ou autodual, respectivamente. Se  $\matriz{M}$ tem jpdf da forma $\phi(\matriz{M})$, invariante sobre transformações de similaridade ortogonal, a jpdf dos $N$ autovalores ordenados de $\matriz{M}$, $\mcmany{\lambda}{N}{\geq}$, é $$ C_{N}^{(\beta)} \phi(\matriz{D}) \prod_{i < j} (\lambda_i - \lambda_j)^{\beta}$$ onde $C_{N}^{\beta}$ é constante e $\beta = 1, 2, 3$ corresponde respectivamente à $\matriz{M} \in M_{\R}(N),  M_{\C}(N),  M_{\He}(N)$. 
	\label{Teorema: Invariante}
\end{thm}

Logo, desde que tomemos um ensemble de matrizes aleatórias com a jpdf das entradas apropriado, podemos reescrever a distribuição em função dos autovalores com \ref{Teorema: Invariante}. Vale ainda observar que, pelo Lema de Weyl \cite{weyl1946classical}, uma jpdf invariante pode ser expressa totalmente por $\p(\matriz{M})= \phi \left( \Tr(\matriz{M}), \Tr(\matriz{M}^2), \cdots, \Tr(\matriz{M}^N) \right)$. Tomada esta expressão, podemos escrever
\begin{equation}
	\p_{ord}(\mmany{\lambda}{N}) = C_{N}^{\beta} \phi{\left( \sum_i^N \lambda_i, \cdots, \sum_i^N \lambda_i^N \right)} \prod_{i < j} (\lambda_i - \lambda_j)^{\beta}
	\label{Equation: p-ord}
\end{equation}

%Essa expressão será usada em breve. Aqui, é mais natural entender o teorema quando se entende a constante $C_N^{\beta}$ como relacionada à integração $\int_{V_N(\Se^N)} \dd O$ e quando se enuncia o lema:

%\begin{lemma}
%	\[
%	\J(\matriz{M} \rightarrow \{ \vec{\lambda}, \matriz{O} \}) = \prod_{j > k} (\lambda_j - \lambda_k)^\beta
%	\]
%	Onde $\beta = 1,2,4$ respectivamente quando $M_{i,j} \in \R, \C, \He $.
%	\label{Lema: Jacobiano}
%\end{lemma}

% -
% C1S2 - Emsembles Gaussianos
% - 
\section{Ensembles Gaussianos}
\label{Section: Ensembles Gaussianos}

Dentre os muitos ensembles em RMT, os Gaussianos são notórios. São eles o \textit{Gaussian Orthogonal Ensemble (GOE)} ($\beta=1$), \textit{Gaussian Unitary Ensemble (GUE)} ($\beta=2$) e \textit{Gaussian Sympletic Ensemble (GSE)} ($\beta=4$). Notemos primeiramente que o nome é relacionado à escolha de $\Se$. Mais explicitamente, o nome é dado em relação à se $\matriz{O}$, tal que $\matriz{M} = \matriz{O}\matriz{D}\matriz{O}^*$, é ortogonal, unitário ou simplético. É natural então pensar nos ensembles \textit{GOE}, \textit{GUE} e \textit{GSE} como matrizes $\matriz{M} \in \mathcal{M}_{\Se}(N)$ onde 
$$
\mathcal{M}_{\Se}(N) \ni M_{i,j} \sim
\begin{cases}
	\mathcal{N}_{\Se}(0,1/2) &  \ \text{para} \ i \neq j,\\
	\mathcal{N}_{\Se}(0,1) & \ \text{para} \ i = j.
\end{cases}
$$

Os três ensembles gaussianos compartilham de uma propriedade exclusiva - são os únicos ensembles com entradas independentes e, simultaneamente, jpdf rotacionalmente invariante. Tomemos, por simplicidade, $\matriz{U} \in \mathcal{M}_{\R}(N)$, matriz real simétrica, do GOE. Para esta, sabendo as entradas independentes, podemos escrever $$\p(\matriz{U}) = \prod_{i=1}^{N}\frac{\exp{\frac{U_{i,i}^2}{2}}}{\sqrt{2\pi}} \prod_{i<j} \frac{\exp{U_{i,i}^2}}{\sqrt{\pi}} = 2^{-N/2} \pi^{-N(N + 1)/4} \exp{-\frac{1}{2} \Tr{U^2}}.$$

Note que essa jpdf satisfaz as condições do Teorema \ref{Teorema: Invariante} e, especialmente, é da forma que propomos na Equação \ref{Equation: p-ord}, logo, $$ \p_{ord}(\mmany{\lambda}{N}) = \frac{1}{Z_{N, \beta = 1}^{(ord)}} \exp{-\frac{1}{2} \sum_{i = 1}^{N} \lambda_i^2} \prod_{i < j} (\lambda_i - \lambda_j).$$ 
De forma análoga, podemos deduzir mais geralmente para $\beta = 1,2,4$ que
\begin{equation}
	\begin{split}
		\p(\mmany{\lambda}{N}) 
		&= \frac{1}{ N! Z_{N, \beta}^{(ord)}} \exp{- \left(\sum_{i = 1}^{N} \frac{\lambda_i^2}{2} - \sum_{i < j} \log{|\lambda_i - \lambda_j|^{\beta}} \right)}, \\
		&= \frac{1}{Z_{N, \beta}} \ee^{-\beta_N \mathcal{H}_N(\vec{\lambda})},
	\end{split}
\label{Equation: medida Gaussian}
\end{equation}
onde $Z_{N, \beta}$ é função de partição canônica para autovalores desordenados\footnote{Usa-se do fator de contagem de Boltzmann \cite[Capítulo~3]{landau2013statistical} para escrever $ Z_{N, \beta} = N! Z_{N, \beta}^{(ord)}$.}, normalizante da expressão \ref{Equation: medida Gaussian}. O fator $\beta_N = \beta N^2$ é pensado como a temperatura inversa. Definimos ainda o Hamiltoniano $$\mathcal{H}_N(\vec{\lambda}) = \frac{1}{N}\sum_{i = 1}^{N} \frac{\lambda_i^2}{2} + \frac{1}{N^2} \sum_{i < j} \log{\frac{1}{|\lambda_i - \lambda_j|}}, \ \ \  \lambda_i \mapsto \lambda_i \sqrt{\beta N}.$$
% Sabemos então, que a partir dessa função podemos retirar importantes propriedades estatísticas (macroscópicas) do sistema de autovalores dos ensembles Gaussianos.



% -
% C1S3 - Gases de Coulomb (Log Gas?)
% - 
\section{Gases de Coulomb}
\label{Section: Gases de Coulomb}

Sob as devidas condições, o gás de coulomb $\p_N$ \cite{ChafaCoulombMeasure} é a medida de probabilidade de Boltzmann-Gibbs dada em $(R^d)^N$ por 
\begin{equation}
	\dd \p_N(\mmany{x}{N}) = \frac{e^{-\beta N^2 \Hf_N(\mmany{x}{N})}}{Z_{N,\beta}} \mcmany{\dd x}{N}{},
	\label{Equação: Medida Gas de Coulomb}
\end{equation}
onde $\Hf_N(\vec{x}) = \frac{1}{N} \sum_{i = 1}^{N} \V(x) + \frac{1}{2N^2} \sum_{i \neq j} \g(x_i - x_j)$ é usualmente chamado hamiltoniano ou energia do sistema.

A medida $\p_N$ modela um gás de partículas indistinguíveis com carga nas posições $\mmany{x}{N} \in \Se$ de dimensão $d$ em $\R^n$ \textit{ambient space}. As partículas estão sujeitas a um potencial externo $\V \colon \Se \mapsto \R$ e interagem por $\g \colon \Se \mapsto (-\infty, \infty]$. A temperatura inversa é $\beta N^2$. Assumiremos, para que valha a definição \ref{Equação: Medida Gas de Coulomb}, que $V, \ \g \ \text{e} \ \beta$ são tais que a constante de normalização (função partição) $Z_{N, \beta} < \infty \ \forall \ N$\footnote{Note que $\p_N$ é um modelo de interações estáticas e não há campos magnéticos considerados.}. Tome $\R^n$ com $n \geq 2$, sabemos que, para $x \neq 0$ o núcleo de interação coulombiana (função de Green) vale $$
	g(\vec{x}) =
	\begin{cases}
			\log \frac{1}{|\vec{x}|} \ \ \text{se} \ n = 2,\\
			\frac{1}{|x|^{n-2}} \ \ \text{se} \ n \geq 3.
	\end{cases}
$$ onde $g$ é solução da equação de Poisson dada por $$
	- \nabla g(\vec{x}) = c\delta_0 \ \ \text{com} \ c = 
	\begin{cases}
		2\pi \ \ \text{para} \ n = 2,\\
		(n-2) |S^{n-1}| \ \ \text{para} \ n \geq 3.
	\end{cases}
$$

Se lembramos da expressão \ref{Equation: medida Gaussian}, perceberemos que, para o devido $\V(x)$, podemos tomar $d=1$ e $n = 2$ para recuperar a medida dos ensembles gaussianos 
\begin{equation}
	\p_N(\vec{x}) = \frac{e^{-\beta_N \Hf_N(\vec{x})}}{Z_{N,\beta}}, \ \ \Hf_N(\vec{x}) = \frac{1}{N} \sum_{i = 1}^{N} \V(x_i) + \frac{1}{N^2} \sum_{i < j} \log{\frac{1}{|x_i - x_j|}}.
	\label{Equation: Medida Log V}
\end{equation}
Estamos tratando de partículas no plano confinadas à uma reta. Para algum potencial arbitrário, além da devida escolha de $n$ e $d$, cairemos em outros ensembles de matrizes.

% -
% C1S4 - Medidas de Equilíbrio
% - 
\section{Medidas de Equilíbrio}
\label{Seção: Medida}
O conjunto de pontos no espaço de fase ou ainda, os microestados, determinam um \textit{ensemble estatístico}. De mesma forma, um conjunto de matrizes determina um ensemble em RMT. Podemos relacionar o conjunto de microestados dos autovalores $\{\vec{\lambda}\}$ com as configurações do sistema de $N$ partículas descrito na Seção \ref{Section: Gases de Coulomb}. Notando que tratamos do ensemble canônico, um argumento termodinâmico nos indica então que devemos minimizar a energia livre $E^V_{N,\beta} = \log{Z_{N, \beta}}.$

Consideraremos $\V$ sob condições tais que seja denominado um potencial admissível \cite{ChafaCoulombMeasure}. Com isso, se $\mu_{V}(\vec{\lambda})$ é medida de probabilidade no espaço das possíveis configurações de autovalores, $Z_{N, \beta}$ será finita e existirá $\mu_{V}^* = \arg \inf {\mathcal{H}_N(\vec{\lambda})}$ medida de equilíbrio única no limite termodinâmico $N \rightarrow \infty$. Para determinar a medida de equilíbrio de \ref{Equation: Medida Log V} \cite{RMT-firstcourse-Potters}, queremos satisfazer o sistema de equações
\begin{equation}
	\frac{\partial \mathcal{H}}{\partial \lambda_i} = 0 \ \implies \ \V'(\lambda_i) = \frac{1}{N} \sum_{1 = j \neq i}^{N} \frac{1}{\lambda_i - \lambda_j} \ \ \text{para} \ i = 1, \cdots, N.
	\label{Equação: Sistema minimizante}
\end{equation} 
Usaremos o denominado \textit{resolvent}. Considere a função complexa $$G_N(z) = \frac{1}{N} \Tr{\left(z\Id - \matriz{M}\right)^{-1}} = \frac{1}{N} \sum_{i=1}^{N} \frac{1}{z - \lambda_i},$$ onde $\matriz{M}$ é matriz aleatória com autovalores $\{\mmany{\lambda}{N}\}$. Note que $G_N(z)$ é uma função complexa aleatória com polos em $\lambda_i$. Não trivialmente, podemos reescrever \ref{Equação: Sistema minimizante} como $$\V'(z) G_N(z) - \Pi_N(z) = \frac{G_N^2(z)}{2} + \frac{G'_N(z)}{2N},$$ onde $$\Pi_N(z) = \frac{1}{N} \sum_{i = 1}^{N} \frac{\V'(z) - \V'(\lambda_i)}{z - \lambda_i}$$ é um polinômio de grau $\deg{\V'(z)} - 1 = k - 1$. Resolver explicitamente para $N$ constante pode não ser simples ou mesmo possível. Em geral, tomaremos a assintótica $N \to \infty$ de $G_N(z)$, nesse limite temos a transformada de Stieltjes\footnote{Também chamada transformada de Cauchy.}
\begin{equation}
	S^{\mu_V}(z) = \int \frac{\mu^*_V(\lambda)}{z - \lambda} \dd \lambda= \V'(z) \pm \sqrt{\V'(z)^2 - 2 \Pi_{\infty}(z) }.
	\label{Equation: Resolvent}
\end{equation}
com $$\Pi_{\infty}(z) = \int \frac{\V'(z) - \V'(\lambda)}{z - \lambda} \mu^*_V(\lambda) d\lambda.$$ Como consequência da fórmula de Sokhotski-Plemeji, é enunciado o resultado 
\begin{equation}
	\mu^{*}_{V}(x) = \frac{1}{2\pi \ii} \left( S^{\mu_V}_{+} -  S^{\mu_V}_{-}\right) = \frac{1}{\pi} \lim_{\epsilon \to 0^+} \Im{S^{\mu_V}_{+}(x + \ii\epsilon)}.
	\label{Equation: p(lambda)}
\end{equation}
%Considere ainda um potencial $\V(x)$ convexo. Neste caso, tomamos naturalmente $\mu^*_V(x)$ não nula apenas em um intervalo $(\lambda_{-}, \lambda_{+})$. Sabemos que o comportamento não analítico surge da raiz quadrada, tal que se definirmos $\Df(z) := \V'(z)^2 - 2 \Pi_{\infty}(z)$ polinômio de grau $2k$, $\{\lambda_{-}, \lambda_{+}\}$ são suas raízes e o polinômio tem valor negativo em algum intervalo. Equivalentemente $D(z) = (z-\lambda_{-})(z - \lambda_{+}) \Qf^2(z),$ onde $\Qf(z)$ é polinômio de grau $k-1$. Com essas definições podemos escrever que, por \ref{Equation: p(lambda)},
%\begin{equation}
%	\mu^*_V(x) =\frac{\Qf(x)}{\pi} \sqrt{(\lambda_{+} - x)(x - \lambda_{-})}, \ \ \text{para} \ \  \lambda_{-} \leq x \leq \lambda_{+}.
%\end{equation}
Em geral, restaria para cada potencial, balancear o sistema de $k+2$ equações dada por
\[
\left( S^{\mu_{V}} - \V' \right)^2 = \left( V' \right)^2 - 2 \Pi_{\infty}
\]




% -
% C1S5 - Potenciais notáveis
% - 
\section{Potenciais notáveis}

Consideraremos a mudança de variável $\V(x) \mapsto \beta N \V(x)$ tal que possamos escrever $\p(\{ \lambda_i\}) \propto \ee^{-\beta N \mathcal{H}_N(\{ \lambda_i\})}$ com $$\mathcal{H}_N(\{ \lambda_i\}) = \sum_{i = 1}^{N} V(x) + \frac{1}{2N} \sum_{i \neq j} \log{|\lambda_i - \lambda_j|}.$$ Com essa mudança, consideremos os seguintes potencias.


%Um resultado importante enuncia \cite{deiftorthogonal}:

%\begin{thm}
%	Para $V(x) = t x^{2m}$ com $t>0$, vale que $$ \p_V(x) = - \frac{m t}{\pi} \sqrt{x^2 - a^2} + h(x) $$ no suporte $\supp(-a, a)$. Onde, $$ a = \left( mt \prod_{l=1}^{m} \frac{2l - 1}{2l} \right)$$ e $$h(x) = x^{2m-2} + \sum_{j=1}^{m-1} x^{2m - 2 - 2j} a^{2j} \prod_{l=1}^{j}.$$
%	\label{Teorema: Medida V(x)}
%\end{thm}


\subsection{Potenciais Quadráticos}

O caso de potencial quadrático $$V(x) = \frac{x^2}{2}$$ descreve o caso dos ensembles gaussianos, onde é fácil determinar que $$V'(z) = z \ \implies \ \Pi(z) = 1$$ e, por isso, 

% $$\mathcal{H}(\vec{\lambda}) = \frac{1}{2}\sum_{i = 1}^{N} \lambda_i^2 - \frac{1}{2N}\sum_{i \neq j} \log{|\lambda_i - \lambda_j|}.$$ É fácil determinar que $$V'(z) = z \ \implies \ \Pi(z) = 1$$ e, por isso, 
%Note o fator $N\beta$. Tomaremos, eventualmente, $N$ suficientemente grande para notar efeitos assintóticos em N. Com isso, estaremos explorando o limite de temperatura zero. Minimizemos $\mathcal{H}(\vec{\lambda})$\footnote{Seguindo passos de \cite{IntroRM}}. Para cada $\lambda_i$, escrevemos

%\begin{equation}
%	\frac{\partial \mathcal{H}(\vec{\lambda})}{\partial \lambda_i} = 0 \implies \lambda_i = \frac{1}{N} \sum_{i \neq j} \frac{1}{\lambda_i - \lambda_j}.
%	\label{Equation: Saddle}
%\end{equation}

%\noindent Multiplicando \ref{Equation: Saddle} por  $1/(N (z - \lambda_i))$, onde $z \in \C \setminus \{\lambda_i\}$ e somando sobre todos autovalores, teremos

%\begin{equation*}
%	\frac{1}{N} \sum_{i=1}^{N} \frac{\lambda_i}{z - \lambda_i} = \frac{1}{N} \sum_{i=1}^{N} \sum_{i \neq j} \frac{1}{\lambda_i - \lambda_j} \frac{1}{N(z - \lambda_i)}.
%	\label{Equation: Post-Saddle}
%\end{equation*}

%\noindent é possível reescrever ainda (não trivialmente) a expressão \ref{Equation: Post-Saddle} retomando a definição \ref{Equation: def G}. Ficaremos com:

%\begin{equation}
%	\frac{1}{2} G_N^2(z) + \frac{1}{2N} G_N'(z) = -1 + z G_N(z)
%	\label{Equation: S1 G(z)}
%\end{equation}

%Temos uma equação diferencial nas mãos. Contudo, o termo em \ref{Equation: S1 G(z)} com a derivada está sendo dividido por $N$. Lembremos que, pela ordem de $\lambda_i$ devemos ter também que $G_N(z)$ tem ordem $\Boh(1)$. Logo, sua derivada divida por $N$ não terá a ordem dominante. Naturalmente, quando tomamos o limite $N \rightarrow \infty$ ficaremos com $${G_{\infty}^{(med)}}^2 (z) - 2z G_{\infty}^{(med)}(z) + 2 = 0.$$ Equação algébrica que pode ser resolvida diretamente, resultando

\begin{equation}
	G_{\infty}^{(med)}(z) = z \pm \sqrt{z^2 - 2}
	\label{Equation: G gauss}.
\end{equation}

Nosso problema chega ao fim pois definimos o \textit{resolvent}. Resta agora invocar a Equação \ref{Equation: p(lambda)} utilizando de \ref{Equation: G gauss} para descobrir que

\begin{equation*}
	\p(x) = \pm \frac{\sign(-x)}{\pi \sqrt{2}} \sqrt{|x^2 - 2| - x^2 + 2}.
\end{equation*}

\noindent Ou ainda, no suporte $\supp(-\sqrt{2}, \sqrt{2})$,

\begin{equation}
	\p(x) = \frac{1}{\pi} \sqrt{2 - x^2}.
\end{equation}

Esse resultado é bem conhecido e a medida encontrada denominada Semi-Círculo de Wigner. Note que isso vale para qualquer $\beta$, a diferença é notada somente quando $N$ é suficientemente pequeno..


\subsection{Potencial Mônico}

Considere o potencial

\[
V(x) = \frac{t}{2\alpha} x^{2\alpha},
\]
onde $t > 0$ é escala e $\alpha \in \Z$. A medida de equilíbrio para $\alpha = 1$ é o semi-círculo de Wigner podemos validar na figura com a distribuição em vermelho. Sabemos também que o suporte $[-a, a]$ da densidade é dado por

\[
a = \left( \frac{t}{2} \prod_{j=1}^{\alpha} \frac{2j-1}{2j} \right)^{-\frac{1}{2\alpha}}.
\]


\subsection{Potencial Quártico}

Para Considere o potencial

\begin{equation}
	V(x) = \frac{x^4}{4} + t \frac{x^2}{2}.
	\label{Equação: Quartico}	
\end{equation}

\noindent Aqui observaremos, a depender de $t$, pela primeira vez a separação do suporte da função. Teremos um ponto crítico em $t=-2$ onde o suporte se separa nos intervalos $[-b_t, -a_t]$ e $[a_t, b_t]$ para $t < -2$. Para $t > -2$ o suporte é um único intervalo $[-b_t, b_t]$. Definiremos a medida nos dois casos,

\begin{itemize}
	\item \(t > -2\)
	\[
	\supp \mu_V = [-b_t, b_t], \ \ \frac{\dd \mu_V}{\dd x}(x) = \frac{1}{2\pi} (x^2 + c_t^2) \sqrt{b_t^2 - x^2},
	\]
	
	com $c_t^2 \deff\frac{1}{2} b_t^2 + t \deff \frac{1}{3} (2t + \sqrt{t^2 + 12})$.
	
	\item \(t < -2\)
	\[
	\supp \mu_V = [-b_t, -a_t] \cup [a_t, b_t], \ \ \frac{\dd \mu_V}{\dd x}(x) = \frac{1}{2\pi} |x| \sqrt{(x^2 - a_t^2)(b_t^2 - x^2)},
	\]
	
	com $ a_t \deff \sqrt{-2-t}, b_t \deff \sqrt{2-t}$.
\end{itemize}