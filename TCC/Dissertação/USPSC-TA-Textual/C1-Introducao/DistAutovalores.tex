\section{Distribuição de autovalores}

Seja $\Se$ um conjunto tal como $\R, \C, \He $ (Reais, Complexos e Quaterniônicos). Consideremos inicialmente uma matriz $\matriz{M} \in \mathcal{M}_{\Se}(N)$ espaço de matrizes $N \times N$, de entradas reais, complexas ou quaterniônicas. Se tomamos o elemento de matriz $M_{i,j}$ $\forall i, j \in \Z$, com $1 \leq i, j \leq N$, como variável aleatória de distribuição arbitrária, podemos expressar a densidade de probabilidade conjunta de $\matriz{M}$ (jpdf, \textit{joint probability density function}) como $$\p(\hat{M}) \dd M = \p(M_{1,1}, \dots, M_{N,N}) \prod_{i,j=1}^{N} \dd M_{i,j}.$$

Considere a decomposição $\matriz{M} = \matriz{O} \matriz{D} \matriz{O}^{-1}$, com $\matriz{O} \in V_N(\Se^N)$ variedade de Stiefel e $\matriz{D} = \diag(\mmany{\lambda}{N})$. Esta decomposição vale quase certamente mas, especialmente, quando tomado $\matriz{M}$ matriz simétrica, hermitiana ou autodual, que implica autovalores $\lambda \in \R$ sem degeneração. Isto pode ser motivado fisicamente sabendo que, para sistemas quânticos invariantes reversíveis, o Hamiltoniano é matriz real simétrica; na presença de campo magnético, o Hamiltoniano é matriz complexa hermitiana; na presença de acoplamento spin-órbita, o Hamiltoniano é simplético \cite[Capítulo~2]{RMT-firstcourse-Potters}. Consideremos os ensembles com esta simetria. Se a transformação tem Jacobiano $\J(\matriz{M} \rightarrow \{ \vec{\lambda}, \matriz{O} \} )$, reescreve-se a jpdf em função dos autovalores e $\matriz{O}$ tal que:
\begin{equation}
	 \p(\hat{M}) \dd M = \p \left( M_{1,1}(\vec{\lambda}, \matriz{O}), \cdots, M_{N,N}(\vec{\lambda}, \matriz{O}) | \J(\matriz{M} \rightarrow \{ \vec{\lambda}, \matriz{O} \} ) \right) \dd O \prod_{i=1}^{N} \lambda_i.
\label{Equation: p(lambda, O)}
\end{equation}

Aqui, ressalta-se que estamos interessados em distribuições de autovalores. Para calcular $\p(\mmany{\lambda}{N})$ devemos integrar os termos à direita da equação \ref{Equation: p(lambda, O)} sobre o subespaço $V_N(\Se^N)$, o que nem sempre é fácil ou possível. Para garantir integrabilidade, tomaremos \textit{ensembles} de matrizes aleatórias onde o jpdf de suas entradas pode ser escrito exclusivamente como função dos autovalores, ou seja $$\p(\mmany{\lambda}{N}, \matriz{O}) \equiv \p \left( M_{1,1}(\vec{\lambda}), \cdots, M_{N,N}(\vec{\lambda}) | \J(\matriz{M} \rightarrow \{ \vec{\lambda} \} ) \right).$$

Ensembles com esta propriedade são denominados invariantes (por rotação). Considere $\matriz{U}$ é ortogonal, unitária ou simplética respectivamente quando $\Se = \R,\C,\He $. A escolha de ensemble implica que quaisquer duas matrizes $\matriz{M}, \matriz{M'}$ que satisfaçam a relação de equivalência $\matriz{M} = \matriz{U} \matriz{M'} \matriz{U}^{-1}$, de mesmos autovalores, tem mesma probabilidade. Considere o teorema \cite[Capítulo~3]{AlanThesis}.
\begin{thm}
	Tome $\matriz{M} \in M_{\R}(N),  M_{\C}(N),  M_{\He}(N)$ simétrica, hermitiana ou autodual, respectivamente. Se  $\matriz{M}$ tem jpdf da forma $\phi(\matriz{M})$, invariante sobre transformações de similaridade ortogonal, a jpdf dos $N$ autovalores ordenados de $\matriz{M}$, $\mcmany{\lambda}{N}{\geq}$, é $$ C_{N, \beta}^{(ord)} \phi(\matriz{D}) \prod_{i < j} (\lambda_i - \lambda_j)^{\beta}$$ com $C_{N, \beta}^{(ord)}$ constante e $\beta = 1, 2, 4$ correspondente à $\matriz{M} \in M_{\R}(N),  M_{\C}(N),  M_{\He}(N)$, respectivamente. 
	\label{Teorema: Invariante}
\end{thm}
Logo, desde que tomemos um ensemble invariante, podemos reescrever a distribuição em função dos autovalores pelo Teorema \ref{Teorema: Invariante}. Vale ainda observar que, pelo Lema de Weyl, uma jpdf invariante pode ser expressa totalmente por $\p(\matriz{M})= \phi \left(\Tr(F(M)) \right)$ com $F$ função polinomial. Ou seja, se unirmos os resultados anteriores, podemos escrever
\begin{equation}
	\p_{ord}(\mmany{\lambda}{N}) = C_{N, \beta}^{(ord)} \phi{\left( \sum_i^N F(\lambda_i) \right)} \prod_{i < j} (\lambda_i - \lambda_j)^{\beta}.
	\label{Equation: p-ord}
\end{equation}

%Essa expressão será usada em breve. Aqui, é mais natural entender o teorema quando se entende a constante $C_N^{\beta}$ como relacionada à integração $\int_{V_N(\Se^N)} \dd O$ e quando se enuncia o lema:

%\begin{lemma}
%	\[
%	\J(\matriz{M} \rightarrow \{ \vec{\lambda}, \matriz{O} \}) = \prod_{j > k} (\lambda_j - \lambda_k)^\beta
%	\]
%	Onde $\beta = 1,2,4$ respectivamente quando $M_{i,j} \in \R, \C, \He $.
%	\label{Lema: Jacobiano}
%\end{lemma}