\section{Distribuição de Autovalores}

Seja $\Se$ um conjunto tal como $\R, \C, \He $ (Reais, Complexos e Quaterniônicos). Consideremos inicialmente uma matriz $\matriz{M} \in \mathcal{M}_{\Se}(N)$ espaço de matrizes $N \times N$, de entradas reais, complexas ou quaterniônicas. Se tomamos o elemento de matriz $M_{i,j}$ $\forall i, j \in \Z$, com $1 \leq i, j \leq N$, como variável aleatória de distribuição arbitrária, podemos expressar a densidade de probabilidade conjunta de $\matriz{M}$ (jpdf, \textit{joint probability density function}) como $$\p(\hat{M}) \dd \matriz{M} = \p(M_{1,1}, \dots, M_{N,N}) \prod_{i,j=1}^{N} \dd M_{i,j}.$$

Considere a decomposição nas coordenadas espectrais $\matriz{M} = \matriz{P} \matriz{\Lambda} \matriz{P}^{-1}$ onde $\matriz{P}$ é matriz invertível e $\matriz{\Lambda} = \diag(\mmany{\lambda}{N})$. Para que valha a decomposição tomaremos $\matriz{M}$ matriz simétrica, hermitiana ou hermitiana quaterniônica. Esta escolha é feita em importantes ensembles em RMT e motivada fisicamente sabendo que, para sistemas quânticos invariantes reversíveis, o Hamiltoniano é matriz real simétrica; na presença de campo magnético, o Hamiltoniano é matriz complexa hermitiana; na presença de acoplamento spin-órbita, o Hamiltoniano é matriz hermitiana quaterniônica. \cite[Capítulo~2]{RMT-firstcourse-Potters} Devemos atentar ainda pela escolha de mapa $\matriz{M} \mapsto \matriz{P} \matriz{\Lambda} \matriz{P^{-1}}$ bijetivo\footnote{Injetividade é garantida desconsiderando fase e sinal dos autovetores e ordenando os autovalores. Restringe-se ao subconjunto de matrizes sem multiplicidade de autovalores - denso, aberto e de medida completa tal que seu complemento é irrelevante na integração conseguinte.}. Se a mudança de variáveis tem Jacobiano $\J(\matriz{M} \rightarrow \{ \matriz{\Lambda}, \matriz{P} \} )$, reescreve-se a jpdf em função de $\matriz{\Lambda}$ e $\matriz{P}$ tal que
\begin{equation}
	 \p(\hat{M}) \dd \matriz{M} = \p \left( M_{1,1}(\matriz{\Lambda}, \matriz{P}), \cdots, M_{N,N}(\matriz{\Lambda}, \matriz{P}) | \J(\matriz{M} \rightarrow \{ \matriz{\Lambda}, \matriz{P} \} ) \right) \dd \matriz{\Lambda} \dd \matriz{P}.
\label{Equation: p(lambda, O)}
\end{equation}

Estamos especialmente interessados na distribuição de autovalores, logo, devemos integrar a Equação \eqref{Equation: p(lambda, O)} sobre $\dd \matriz{P}$, o que nem sempre é fácil ou possível. Por isso, tomaremos ensembles denominados invariantes (por rotação), isto é, tais que quaisquer duas matrizes $\matriz{M}$ e $\matriz{M'}$ que satisfaçam a relação de equivalência $\matriz{M} = \matriz{U} \matriz{M'} \matriz{U}^{-1}$, sendo $\matriz{U}$ uma rotação, tem mesma probabilidade. Com isso, a jpdf de suas entradas pode ser escrita exclusivamente como função dos autovalores, ou seja, $$\p(\matriz{\Lambda}, \matriz{P}) \dd \matriz{\Lambda} \dd \matriz{P} \coloneqq \p \left( M_{1,1}( \matriz{\Lambda}), \cdots, M_{N,N}(\matriz{\Lambda}) | \J(\matriz{M} \rightarrow \{\matriz{\Lambda}\} ) \right) \dd \matriz{\Lambda} \dd \matriz{P}.$$

Pelo Lema de Weyl, uma jpdf invariante pode ser expressa totalmente por $\p(\matriz{M}) \coloneqq \phi \left(\Tr(V(\matriz{M})) \right)$ com $V$ função polinomial. Além disso, o jacobiano $\J(\matriz{M} \rightarrow \{\matriz{\Lambda}, \matriz{P}\} )$ desta transformação pode ser expresso pelo determinante de matriz de Vandermonde tal que $$\p(\matriz{\Lambda}, \matriz{P}) \dd \matriz{\Lambda} \dd \matriz{P} = \phi \left(\Tr(V(\matriz{M})) \right) \prod_{i<j}^{N} |\lambda_i - \lambda_j|^{\beta} \dd \matriz{\Lambda} \dd \matriz{P},$$ onde $\beta = 1,2,4$ quando tomado $\Se = \R, \C, \He$, respectivamente. Com esta expressão, sabendo medida uniforme sobre os autovetores, podemos explicitar pela primeira vez a jpdf para os autovalores ordenados destes ensembles de matrizes aleatórias como 
\begin{equation}
	\p_{ord}(\mmany{\lambda}{N}) = \frac{1}{Z_{N, \beta}^{(ord)}} \phi{\left( \sum_i^N V(\lambda_i) \right)} \prod_{i < j}^{N} |\lambda_i - \lambda_j|^{\beta}.
	\label{Equation: p-ord}
\end{equation}

Note que, graças ao jacobiano, autovalores destas matrizes apresentam repulsão mútua, expressa pelo produtório na Equação \eqref{Equation: p-ord}. Este fato naturaliza a analogia da Seção \ref{Section: Gases de Coulomb} e é central a muitos resultados em RMT. É possível fazer desenvolvimento análogo para matrizes normais de ensembles associados a $\beta = 2$ com autovalores $\lambda_i \in \C$ - extensão explorada nos resultados. Outros casos fogem ao escopo do trabalho.
%Essa expressão será usada em breve. Aqui, é mais natural entender o teorema quando se entende a constante $C_N^{\beta}$ como relacionada à integração $\int_{V_N(\Se^N)} \dd O$ e quando se enuncia o lema:

%\begin{lemma}
%	\[
%	\J(\matriz{M} \rightarrow \{ \vec{\lambda}, \matriz{P} \}) = \prod_{j > k} (\lambda_j - \lambda_k)^\beta
%	\]
%	Onde $\beta = 1,2,4$ respectivamente quando $M_{i,j} \in \R, \C, \He $.
%	\label{Lema: Jacobiano}
%\end{lemma}