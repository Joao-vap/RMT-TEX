\section{Gases de Coulomb}
\label{Section: Gases de Coulomb}

Sob as devidas condições, o gás de coulomb $\p_N$ \cite{ChafaCoulombMeasure} é medida de probabilidade de Boltzmann-Gibbs dada em $(R^d)^N$. A medida $\p_N$ modela um gás interagente de partículas indistinguíveis sob potencial externo nas posições $\mmany{x}{N} \in \Se$ de dimensão $d$ em $\R^n$ \textit{ambient space}. A medida é dada por 
\begin{equation}
	\dd \p_N(\mmany{x}{N}) = \frac{e^{-\beta N^2 \Hf_N(\mmany{x}{N})}}{Z_{N,\beta}} \mcmany{\dd x}{N}{},
	\label{Equação: Medida Gas de Coulomb}
\end{equation}
onde $$\Hf_N(\vec{x}) = \frac{1}{N} \sum_{i = 1}^{N} \V(x) + \frac{1}{2N^2} \sum_{i \neq j} \g(x_i - x_j)$$ é usualmente chamado hamiltoniano\footnote{Note que $\p_N$ é um modelo de interações estáticas e não há campos magnéticos considerados.} ou energia do sistema. $\V \colon \Se \mapsto \R$ é potencial externo e $\g \colon \Se \mapsto (-\infty, \infty]$ núcleo de interação coulombiana solução da equação de Poisson dada por $- \nabla g(\vec{x}) = c_n\delta_0$. Além disso, $\beta N^2$ é chamado temperatura inversa. Assumiremos, para que valha a definição \ref{Equação: Medida Gas de Coulomb}, que $V$ é tal que a constante de normalização (função partição) $Z_{N, \beta} < \infty \ \forall \ N$ e o suporte da medida é compacto.

Se lembramos da expressão \ref{Equation: medida Gaussian}, perceberemos que, para o devido $\V \colon \R \rightarrow \R$, podemos tomar $d=1$ e $n = 2$ para recuperar a medida dos ensembles gaussianos 
\begin{equation}
	\p_N(\vec{x}) = \frac{e^{-\beta_N \Hf_N(\vec{x})}}{Z_{N,\beta}}, \ \ \Hf_N(\vec{x}) = \frac{1}{N} \sum_{i = 1}^{N} \V(x_i) + \frac{1}{N^2} \sum_{i < j} \log{\frac{1}{|x_i - x_j|}}.
	\label{Equation: Medida Log V}
\end{equation}
Estamos tratando de partículas no plano confinadas à uma reta neste caso. Esta medida aceita uma extensão natural para um potencial admissível arbitrário. Isso, junto à devida escolha de $n$ e $d$ adequados, leva à medida de outros ensembles de matrizes.