\section{Gases de Coulomb}

Sob as devidas condições, o Gás de Coulomb $\p_N$ é \cite{CoulombGaschafai} a medida de probabilidade de Boltzmann-Gibbs dada em $(R^d)^N$ por 
\begin{equation}
	\dd \p_N(\mmany{x}{N}) = \frac{e^{-\beta N^2 \Hf_N(\mmany{x}{N})}}{Z_{N,\beta}} \mcmany{\dd x}{N}{},
	\label{Equação: Medida Gas de Coulomb}
\end{equation}
onde $\Hf_N(\vec{\lambda}) = \frac{1}{N} \sum_{i = 1}^{N} \V(x) + \frac{1}{2N^2} \sum_{i \neq j} \g(x_i - x_j)$ é usualmente chamado hamiltoniano ou energia do sistema.

A medida modela um gás de partículas indistinguíveis com carga nas posições $\mmany{x}{N} \in S$ de dimensão $d$ em $R^n$ \textit{ambient space}. As partículas estão sujeitas a um potencial externo $\V \colon S \mapsto \R$ e interagem por $\W \colon S \mapsto (-\infty, \infty]$. A temperatura inversa é $\beta N^2$. Assumiremos, para que valha a expressão, que $V, \ \W \ \text{e} \ \beta$ são tais a constante de normalização (função partição) $Z_{N, \beta} < \infty \ \forall \ N$. Reforço que $\p_N$ é um modelo de interações estáticas e não há campos magnéticos considerados. Tome $\R^n$ com $n \geq 2$. Sabemos que, para $x \neq 0$ o núcleo de interação coulombiana (função de Green) vale $$
	g(\vec{x}) =
	\begin{cases}
			\log \frac{1}{|\vec{x}|} \ \ \text{se} \ n = 2,\\
			\frac{1}{|x|^{n-2}} \ \ \text{se} \ n \geq 3.
	\end{cases}
$$ onde $g$ é solução da equação de Poisson dada por $$
	- \nabla g(\vec{x}) = c\delta_0 \ \ \text{com} \ c = 
	\begin{cases}
		2\pi \ \ \text{para} \ n = 2,\\
		(n-2) |\mathbb{S}^{n-1}| \ \ \text{para} \ n \geq 3.
	\end{cases}
$$

Se lembramos da expressão \ref{Equation: medida Gaussian}, perceberemos que, para um potencial devidamente escolhido, podemos tomar $d=1$ e $n = 2$ e recuperar a medida dos ensembles gaussianos. Estamos tratando de partículas no plano confinadas à uma reta. Para algum potencial arbitrário, além da devida escolha de $n$ e $d$, cairemos em outros ensembles de matrizes. Exploraremos melhor esse fato no Capítulo \ref{Capitulo: Simulações}.