\section{Gases de Coulomb}
\label{Section: Gases de Coulomb}

Sob as devidas condições, o Gás de Coulomb $\p_N$ é medida de probabilidade de Boltzmann-Gibbs dada em $(\R^d)^N$. \cite{ChafaCoulombMeasure} A medida $\p_N$ modela um gás interagente de partículas indistinguíveis sob potencial externo nas posições $\mmany{x}{N} \in \Se$ de dimensão $d$ no espaço ambiente $\R^n$. A medida é dada por 
\begin{equation}
	\dd \p_N(\mmany{x}{N}) = \frac{e^{-\beta N^2 \Hf_N(\vec{x})}}{Z_{N,\beta}} \mcmany{\dd x}{N}{},
	\label{Equação: Medida Gas de Coulomb}
\end{equation}
onde $$\Hf_N(\vec{x}) = \frac{1}{N} \sum_{i = 1}^{N} \V(x) + \frac{1}{2N^2} \sum_{i \neq j}^{N} \g(x_i - x_j)$$ é usualmente chamado Hamiltoniano\footnote{Note que $\p_N$ é um modelo de interações estáticas e não há campos magnéticos considerados.} ou energia do sistema. $\V \colon \Se \mapsto \R$ é potencial externo e $\g \colon \Se \mapsto (-\infty, \infty]$ núcleo de interação coulombiana solução da equação de Poisson dada por $- \nabla g(\vec{x}) = c_n\delta_0$. Além disso, $\beta N^2$ é chamado temperatura inversa. Tomaremos $V$, para que valha a Definição \eqref{Equação: Medida Gas de Coulomb}, tal que a constante de normalização (função partição) $Z_{N, \beta}$ seja finita para todo $N$ e o suporte da medida seja compacto.

Se lembrarmos da Expressão \eqref{Equation: medida Gaussian}, perceberemos que, para o devido $\V \colon \R \rightarrow \R$, podemos tomar $d=1$ e $n = 2$ para recuperar a medida dos ensembles gaussianos 
\begin{equation}
	\p_N(\vec{x}) = \frac{e^{-\beta_N \Hf_N(\vec{x})}}{Z_{N,\beta}}, \ \ \Hf_N(\vec{x}) = \frac{1}{N} \sum_{i = 1}^{N} \V(x_i) + \frac{1}{N^2} \sum_{i < j}^{N} \log{\frac{1}{|x_i - x_j|}}.
	\label{Equation: Medida Log V}
\end{equation}
Estamos tratando de partículas no plano confinadas à reta, neste caso. Contudo, a medida dos gases aceita uma extensão natural para potenciais admissíveis arbitrários. Junto à escolha adequada de $n$ e $d$ estaremos lidando, ao explorar estas extensões, com a medida de outros diversos ensembles de matrizes aleatórias.