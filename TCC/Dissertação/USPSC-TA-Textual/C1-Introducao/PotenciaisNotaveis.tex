\section{Potenciais notáveis}
\label{Section: Potencias}

Consideraremos a mudança de variável $\V(x) \mapsto \beta N^2 \V(x)$ tal que possamos escrever $\p(\{ \lambda_i\}) \propto \ee^{-\beta N^2 \mathcal{H}_N(\{ \lambda_i\})}$ com $$\mathcal{H}_N(\{ \lambda_i\}) = \frac{1}{N} \sum_{i = 1}^{N} V(x) + \frac{1}{2N^2} \sum_{i \neq j} \log{|\lambda_i - \lambda_j|}.$$ O desenvolvimento feito na seção \ref{Seção: Medida} é suficiente para resolver os casos exemplificados aqui, salvo detalhes. Explicitar a conta não elucidaria a teoria e por isso omitido. Consideremos os seguintes potencias.


%Um resultado importante enuncia \cite{deiftorthogonal}:

%\begin{thm}
%	Para $V(x) = t x^{2m}$ com $t>0$, vale que $$ \p_V(x) = - \frac{m t}{\pi} \sqrt{x^2 - a^2} + h(x) $$ no suporte $\supp(-a, a)$. Onde, $$ a = \left( mt \prod_{l=1}^{m} \frac{2l - 1}{2l} \right)$$ e $$h(x) = x^{2m-2} + \sum_{j=1}^{m-1} x^{2m - 2 - 2j} a^{2j} \prod_{l=1}^{j}.$$
%	\label{Teorema: Medida V(x)}
%\end{thm}


\subsection{Potenciais Quadráticos}
Considere o potencial $$\V(x) = \frac{x^2}{2}.$$ Neste caso, resolvemos o sistema para descobrir que
\begin{equation}
	 \supp{\p(x)} = [-\sqrt{2}, \sqrt{2}], \ \ \ \text{e} \ \ \ \p(x) = \frac{1}{\pi} \sqrt{2 - x^2}.
	 \label{Equação: Quadrático}
\end{equation}
Esse resultado é bem conhecido e a medida encontrada é denominada Semi-Círculo de Wigner. Note que isso vale para qualquer $\beta$, a diferença é notada somente quando $N$ é suficientemente pequeno.

\subsection{Potencial Quártico}

Considere o potencial $$\V(x) = \frac{x^4}{4} + t \frac{x^2}{2}.$$ Aqui observaremos, a depender de $t$, pela primeira vez a separação do suporte da função. Isso muda um puco o procedimento dado que temos que considerar um suporte composto da união de dois fechados disjuntos (ainda com simetria em relação ao zero) mas não introduz nenhum empecilho. Teremos um ponto crítico em $t=-2$ onde o suporte se separa nos intervalos $[-b_t, -a_t]$ e $[a_t, b_t]$ para $t < -2$. Para $t \geq -2$ o suporte é um único intervalo $[-b_t, b_t]$. Definiremos a medida nos dois casos,
\begin{itemize}
	\item \(t \geq -2\)
	\begin{equation}
	\supp \p(x) = [-b_t, b_t], \ \ \p(x) = \frac{1}{2\pi} (x^2 + c_t^2) \sqrt{b_t^2 - x^2},\label{Equação: Quartico +}
	\end{equation}
	com $c_t^2 \deff\frac{1}{2} b_t^2 + t \deff \frac{1}{3} (2t + \sqrt{t^2 + 12})$.
	\item \(t < -2\)
	\begin{equation}
	\supp \p(x) = [-b_t, -a_t] \cup [a_t, b_t], \ \ \p(x) = \frac{1}{2\pi} |x| \sqrt{(x^2 - a_t^2)(b_t^2 - x^2)},
	\label{Equação: Quartico -}
	\end{equation}
	com $ a_t \deff \sqrt{-2-t}, b_t \deff \sqrt{2-t}$.
\end{itemize}

%\subsection{Potencial Mônico}

%Considere o potencial

%\[
%V(x) = \frac{t}{2\alpha} x^{2\alpha},
%\]
%onde $t > 0$ é escala e $\alpha \in \Z$. A medida de equilíbrio para $\alpha = 1$ é o semi-círculo de Wigner podemos validar na figura com a distribuição em vermelho. Sabemos também que o suporte $[-a, a]$ da densidade é dado por

%\[
%a = \left( \frac{t}{2} \prod_{j=1}^{\alpha} \frac{2j-1}{2j} \right)^{-\frac{1}{2\alpha}}.
%\]

\subsection{Potencial Mônico}
 
 Por último, tome $$\V(x) = t x^{2m}.$$ Com o mesmo processo, apesar de mais geral, determinamos sua medida 
 \begin{equation}
 	\supp \p(x) = [-a, a], \ \ \p(x) = \frac{mt}{\pi} \sqrt{a^2 - x^2} \h(x),
 	\label{Equação: Mônico}
 \end{equation}
com $ a \deff \left( mt \prod_{l=1}^{m} \frac{2l-1}{2l}\right)$ e $\h(x) = x^{2m-2} + \sum_{j=1}^{m-1} x^{2m-2-2j} a^{2j} \prod_{l=1}^{j} \frac{2l-1}{2l}$

