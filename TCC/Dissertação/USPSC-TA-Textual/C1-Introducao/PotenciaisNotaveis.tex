\section{Potenciais Notáveis}
\label{Section: Potencias}

 Sabemos que para um potencial arbitrário podemos usar o desenvolvimento da Seção \ref{Seção: Medida}  para determinar $\Pi$ e, consequentemente, $S^{\mu_V}$ e $\mu^{*}_{V}$. Enunciemos então alguns potencias notáveis. Retome a Medida \eqref{Equation: Medida Log V} e considere primeiramente o potencial quadrático $$\V(x) = \frac{x^2}{2}.$$ Neste caso, teremos que
\begin{equation}
	 \supp{\mu^*_V(x)} = [-\sqrt{2}, \sqrt{2}], \ \ \ \mu^*_V(x) = \frac{1}{\pi} \sqrt{2 - x^2}.
	 \label{Equação: Quadrático}
\end{equation}
Esse resultado é bem conhecido e a medida encontrada é denominada Semi-Círculo de Wigner. Especialmente é medida de equilíbrio para os ensembles gaussianos e vale para qualquer $\beta$. A diferença é notada somente para $N$ suficientemente pequeno.

Agora, considere o potencial quártico $$\V(x) = \frac{x^4}{4} + t \frac{x^2}{2}.$$ Aqui observaremos, a depender de $t$, pela primeira vez, a separação do suporte de $\mu^*_V$. Teremos um ponto crítico em $t=-2$, onde, com $t < -2$, este se separa do intervalo $[-b_t, b_t]$ para $[-b_t, -a_t] \bigcup [a_t, b_t]$. Considere a medida nos dois casos
\begin{itemize}
	\item \(t \geq -2\)
	\begin{equation}
	\supp \mu^*_V(x) = [-b_t, b_t], \ \ \mu^*_V(x) = \frac{1}{2\pi} (x^2 + c_t^2) \sqrt{b_t^2 - x^2},\label{Equação: Quartico +}
	\end{equation}
	com $c_t^2 \deff\frac{1}{2} b_t^2 + t \deff \frac{1}{3} (2t + \sqrt{t^2 + 12})$.
	\item \(t < -2\)
	\begin{equation}
	\supp \mu^*_V(x) = [-b_t, -a_t] \cup [a_t, b_t], \ \ \mu^*_V(x) = \frac{1}{2\pi} |x| \sqrt{(x^2 - a_t^2)(b_t^2 - x^2)},
	\label{Equação: Quartico -}
	\end{equation}
	com $ a_t \deff \sqrt{-2-t}, b_t \deff \sqrt{2-t}$.
\end{itemize}

%\subsection{Potencial Mônico}

%Considere o potencial

%\[
%V(x) = \frac{t}{2\alpha} x^{2\alpha},
%\]
%onde $t > 0$ é escala e $\alpha \in \Z$. A medida de equilíbrio para $\alpha = 1$ é o semi-círculo de Wigner podemos validar na figura com a distribuição em vermelho. Sabemos também que o suporte $[-a, a]$ da densidade é dado por

%\[
%a = \left( \frac{t}{2} \prod_{j=1}^{\alpha} \frac{2j-1}{2j} \right)^{-\frac{1}{2\alpha}}.
%\]
 
 Por último, tome o potencial mônico $$\V(x) = t x^{2m}.$$ Com o mesmo processo, apesar de mais geral, determinamos sua medida 
 \begin{equation}
 	\supp \mu^*_V(x) = [-a, a], \ \ \mu^*_V(x) = \frac{mt}{\pi} \sqrt{a^2 - x^2} \h(x),
 	\label{Equação: Mônico}
 \end{equation}
com $$ a \deff \left( mt \prod_{l=1}^{m} \frac{2l-1}{2l}\right) \ \ \ \text{e} \ \ \ \h(x) \deff x^{2m-2} + \sum_{j=1}^{m-1} x^{2m-2-2j} a^{2j} \prod_{l=1}^{j} \frac{2l-1}{2l}.$$

Essas medidas de equilíbrio nos servirão no Capítulo \ref{Capitulo: Resultados} quando quisermos assegurar o bom comportamento das simulações implementadas.
