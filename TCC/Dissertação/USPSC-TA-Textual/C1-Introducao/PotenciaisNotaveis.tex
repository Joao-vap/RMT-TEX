\section{Potenciais notáveis}
\label{Section: Potencias}

 O desenvolvimento feito na seção \ref{Seção: Medida} é suficiente para resolver os casos exemplificados aqui, salvo detalhes. Explicitar a conta não elucidaria a teoria e por isso foi omitido. Retome a medida \ref{Equation: Medida Log V} e considere os seguintes potencias.


%Um resultado importante enuncia \cite{deiftorthogonal}:

%\begin{thm}
%	Para $V(x) = t x^{2m}$ com $t>0$, vale que $$ \p_V(x) = - \frac{m t}{\pi} \sqrt{x^2 - a^2} + h(x) $$ no suporte $\supp(-a, a)$. Onde, $$ a = \left( mt \prod_{l=1}^{m} \frac{2l - 1}{2l} \right)$$ e $$h(x) = x^{2m-2} + \sum_{j=1}^{m-1} x^{2m - 2 - 2j} a^{2j} \prod_{l=1}^{j}.$$
%	\label{Teorema: Medida V(x)}
%\end{thm}


\subsection{Potenciais Quadráticos}
Considere o potencial $$\V(x) = \frac{x^2}{2}.$$ Neste caso, resolvemos o sistema para descobrir que
\begin{equation}
	 \supp{\mu_V(x)} = [-\sqrt{2}, \sqrt{2}], \ \ \ \text{e} \ \ \ \mu_V(x) = \frac{1}{\pi} \sqrt{2 - x^2}.
	 \label{Equação: Quadrático}
\end{equation}
Esse resultado é bem conhecido e a medida encontrada é denominada Semi-Círculo de Wigner. Note que isso vale para qualquer $\beta$, a diferença é notada somente quando $N$ é suficientemente pequeno.

\subsection{Potencial Quártico}

Considere o potencial $$\V(x) = \frac{x^4}{4} + t \frac{x^2}{2}.$$ Aqui observaremos, a depender de $t$, pela primeira vez a separação do suporte da função. Teremos um ponto crítico em $t=-2$ onde o suporte se separa nos intervalos $[-b_t, -a_t]$ e $[a_t, b_t]$ para $t < -2$. Para $t \geq -2$ o suporte é um único intervalo $[-b_t, b_t]$. Definiremos a medida nos dois casos,
\begin{itemize}
	\item \(t \geq -2\)
	\begin{equation}
	\supp \mu_V(x) = [-b_t, b_t], \ \ \mu_V(x) = \frac{1}{2\pi} (x^2 + c_t^2) \sqrt{b_t^2 - x^2},\label{Equação: Quartico +}
	\end{equation}
	com $c_t^2 \deff\frac{1}{2} b_t^2 + t \deff \frac{1}{3} (2t + \sqrt{t^2 + 12})$.
	\item \(t < -2\)
	\begin{equation}
	\supp \mu_V(x) = [-b_t, -a_t] \cup [a_t, b_t], \ \ \mu_V(x) = \frac{1}{2\pi} |x| \sqrt{(x^2 - a_t^2)(b_t^2 - x^2)},
	\label{Equação: Quartico -}
	\end{equation}
	com $ a_t \deff \sqrt{-2-t}, b_t \deff \sqrt{2-t}$.
\end{itemize}

%\subsection{Potencial Mônico}

%Considere o potencial

%\[
%V(x) = \frac{t}{2\alpha} x^{2\alpha},
%\]
%onde $t > 0$ é escala e $\alpha \in \Z$. A medida de equilíbrio para $\alpha = 1$ é o semi-círculo de Wigner podemos validar na figura com a distribuição em vermelho. Sabemos também que o suporte $[-a, a]$ da densidade é dado por

%\[
%a = \left( \frac{t}{2} \prod_{j=1}^{\alpha} \frac{2j-1}{2j} \right)^{-\frac{1}{2\alpha}}.
%\]

\subsection{Potencial Mônico}
 
 Por último, tome $$\V(x) = t x^{2m}.$$ Com o mesmo processo, apesar de mais geral, determinamos sua medida 
 \begin{equation}
 	\supp \mu_V(x) = [-a, a], \ \ \mu_V(x) = \frac{mt}{\pi} \sqrt{a^2 - x^2} \h(x),
 	\label{Equação: Mônico}
 \end{equation}
com $ a \deff \left( mt \prod_{l=1}^{m} \frac{2l-1}{2l}\right)$ e $$\h(x) = x^{2m-2} + \sum_{j=1}^{m-1} x^{2m-2-2j} a^{2j} \prod_{l=1}^{j} \frac{2l-1}{2l}.$$

