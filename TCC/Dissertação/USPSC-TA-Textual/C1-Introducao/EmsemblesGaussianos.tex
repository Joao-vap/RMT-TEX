\section{Ensembles Gaussianos}
\label{Section: Ensembles Gaussianos}

Dentre os muitos ensembles em RMT, os gaussianos são notórios. São eles o \textit{Gaussian Orthogonal Ensemble (GOE)} ($\beta=1$), \textit{Gaussian Unitary Ensemble (GUE)} ($\beta=2$) e \textit{Gaussian Sympletic Ensemble (GSE)} ($\beta=4$). Notemos primeiramente que o nome é relacionado a escolha de $\Se$. Mais explicitamente, o nome é dado em relação à se $\matriz{P}$, tal que $\matriz{M} = \matriz{P}\matriz{\Lambda}\matriz{P^{-1}}$, é ortogonal, unitário ou simplético. Pensamos então, a menos de escala, nos ensembles \textit{GOE}, \textit{GUE} e \textit{GSE} como matrizes simétricas $\matriz{M} \in \mathcal{M}_{\Se}(N)$ onde 
$$
M_{i,j} \sim
\begin{cases}
	\mathcal{N}_{\Se}(0,1/2) &  \ \text{para} \ i \neq j,\\
	\mathcal{N}_{\Se}(0,1) & \ \text{para} \ i = j,
\end{cases}
\ \ \text{onde} \ i \leq j.
$$

Os três ensembles gaussianos compartilham de uma propriedade exclusiva - são os únicos ensembles com entradas independentes e, simultaneamente, jpdf rotacionalmente invariante. Tomemos, por simplicidade, $\matriz{G} \in \mathcal{M}_{\R}(N)$, matriz real simétrica do GOE. Para esta, sabendo as entradas independentes, podemos escrever $$\p(\matriz{G}) = \prod_{i=1}^{N}\frac{\exp{-\frac{G_{i,i}^2}{2}}}{\sqrt{2\pi}} \prod_{i<j}^{N} \frac{\exp{-G_{i,i}^2}}{\sqrt{\pi}} = 2^{-N/2} \pi^{-N(N + 1)/4} \exp{-\frac{1}{2} \Tr{\matriz{G}^2}}.$$

Note que essa jpdf é função do traço de um polinômio em $\matriz{G}$ da forma que exige o Lema de Weyl, logo, pela Equação \eqref{Equation: p-ord}, $$ \p_{ord}^{G}(\mmany{\lambda}{N}) = \frac{1}{Z_{N, \beta = 1}^{(ord)}} \exp{-\frac{1}{2} \sum_{i = 1}^{N} \lambda_i^2} \prod_{i < j}^{N} |\lambda_i - \lambda_j|.$$ 
De forma análoga, podemos deduzir mais geralmente para $\beta = 1,2,4$ que
\begin{equation}
	\begin{split}
		\p(\mmany{\lambda}{N}) 
		&= \frac{1}{ N! Z_{N, \beta}^{(ord)}} \exp{- \left(\sum_{i = 1}^{N} \frac{\lambda_i^2}{2} - \sum_{i < j}^{N} \log{|\lambda_i - \lambda_j|^{\beta}} \right)}, \\
		&= \frac{1}{Z_{N, \beta}} \ee^{-\beta_N \mathcal{H}_N(\vec{\lambda})},
	\end{split}
\label{Equation: medida Gaussian}
\end{equation}
onde $Z_{N, \beta}$ é função de partição canônica para autovalores desordenados\footnote{Usa-se do fator de contagem de Boltzmann para escrever $ Z_{N, \beta} = N! Z_{N, \beta}^{(ord)}$.}, normalizante da Expressão \eqref{Equation: medida Gaussian}. O fator $\beta_N = \beta N^2$ é pensado como a temperatura inversa. Definimos ainda o Hamiltoniano $$\mathcal{H}_N(\vec{\lambda}) = \frac{1}{N}\sum_{i = 1}^{N} \frac{\lambda_i^2}{2} + \frac{1}{N^2} \sum_{i < j}^{N} \log{\frac{1}{|\lambda_i - \lambda_j|}}, \ \ \  \lambda_i \mapsto \lambda_i \sqrt{\beta N},$$ onde é aplicado a devida escala na magnitude dos autovalores.
% Sabemos então, que a partir dessa função podemos retirar importantes propriedades estatísticas (macroscópicas) do sistema de autovalores dos ensembles Gaussianos.

