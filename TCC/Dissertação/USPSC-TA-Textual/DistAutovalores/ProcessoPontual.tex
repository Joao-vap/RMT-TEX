Um processo pontual pode ser interpretado como um conjunto aleatório de pontos ou como a medida de probabilidade associada a esse conjunto. Um processo pontual possui $n$ pontos se

\[
\mathcal{P}(\# X=n) = 1
\]

Onde $X$ é um conjunto enumerável de $\mathcal{X}$ ($\mathbb{R}$, $\mathbb{Z}$ ou um subconjunto destes). O conjunto de todas configurações possíveis é denominado $Conf(\mathcal{X})$. Se $P(x_1, \dots, x_n)$ é uma função de densidade de probabilidade em $\mathbb{R}^n$ invariante por permutações

\[
	\mathbb{R}^n \rightarrow Conf(\mathbb{R})
\]
\[
	(x_1, \dots, x_n) \mapsto X = {x_1, \dots, x_n}
\]

define naturalmente um processo pontual com $n$ pontos.
