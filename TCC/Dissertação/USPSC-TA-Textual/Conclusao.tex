\chapter[Conclusão]{Conclusão}
\label{conclusao}

O objetivo desse trabalho foi revisar estruturas fundamentais do conceito de auto-dualidade, e mostrar algumas de suas principais aplicações em fenômenos não lineares em grande parte da física.

Primeiramente, foram descritas as principais estruturas envolvidas na aplicação do conceito de auto-dualidade generalizada, em que foram definidas a carga topológica ($\mathcal{Q}$) e as equações de auto-dualidade (\ref{bps}). Observou-se que, com a combinação das identidades, vindas da invariância homotópica da carga topológica, com as equações de auto-dualidade, foram obtidas as equações de Euler-Lagrange do funcional de energia (\ref{energia}). Além disso, foi possível demonstrar que as soluções auto-duais são aquelas que saturam o limite inferior de energia encontrado para as soluções gerais do funcional. Posteriormente, os conceitos discutidos foram generalizados com a matriz $\eta$, que pode introduzir novos campos na teoria; com essa modificação, foram obtidas as novas equações de auto-dualidade (\ref{bps2}) e o novo funcional de energia (\ref{energia2}).

Em seguida, foram desenvolvidos exemplos conhecidos de aplicações do conceito de auto-dualidade, esses sendo \textit{Kinks} em $(1+1)$ dimensões, \textit{Lumps} em $(2+1)$ dimensões, um modelo básico de \textit{Skyrmions} e monopolos em $(3+1)$ dimensões e por último a teoria de \textit{Instantons} em quatro dimensões euclidianas. Destacou-se o desenvolvimento de um paralelo geométrico para a teoria de \textit{Kinks}, onde se provou que essas curvas no espaço dos campos devem escalar para cima ou para baixo o pré-potencial e que as soluções com energia finita são as curvas que começam e terminam em seus vácuos. Além disso, desenvolveu-se um exemplo concreto, em $(1+1)$ dimensões, da criação de uma teoria a partir do pré-potencial, usando um método baseado em teoria de grupos, especificamente a estrutura de pesos da álgebra de $SU(3)$.

No geral, o presente trabalho expôs as principais estruturas do conceito de auto-dualidade e algumas de suas aplicações de maior impacto na física contemporânea. Ainda assim, como dito em \cite{laf(1+1)}, há muito a ser explorado no conceito de auto-dualidade e novos desenvolvimentos impactantes são esperados no futuro.