\chapter[\textit{Lumps} em $(2+1)$ dimensões]{\textit{Lumps} em $(2+1)$ dimensões}
\label{Lumps (2+1)}

Como um exemplo de uma teoria com um setor auto-dual, será considerado o modelo $\mathbb{CP}^{N-1}$ em (2+1) dimensões. $\mathbb{CP}^{N-1}$ é o espaço projetivo complexo de ($N-1$) dimensões, ou seja, o espaço de todas as classes equivalentes de vetores complexos $z = \left(z_1 \ , \ z_2 \ , \ ... \ , \ z_N\right)$, tal que dois vetores $z$ e $z'$ são equivalentes se $z' = \lambda z$, sendo $\lambda$ um número complexo \cite{sigmamodelinst, lowsigma}. Serão considerados os representantes dessas classes como os vetores unitários
\begin{equation}
    z = (z_1,z_2,...,z_N) \ ; \qquad \qquad \qquad z^{*}_a z_a = 1 \ .
\end{equation}

$\mathbb{CP}^{N-1}$ é isomórfico a $SU(N)/SU(N-1) \otimes U(1)$. De fato, o grupo $SU(N)$ age transitivamente nos vetores $z$ por meio da sua representação definidora $N-\text{dimensional}$. Como essa representação é unitária, sua ação preserva o módulo dos vetores $z$, e um dado vetor é mantido invariante por matrizes $(N-1)\times(N-1)$ unitárias, ou seja, o subgrupo $U(N-1) = SU(N-1)\otimes U(1)$.

O segundo grupo de homotopia de $\mathbb{CP}^{N-1}$ é isomórfico aos inteiros sobre adição, ou seja, $\pi_2\left(SU(N)/SU(N-1)\otimes U(1)\right) = \mathbb{Z}$. A carga topológica associada possui uma representação integral da forma
\begin{equation}
    \mathcal{Q} = \dfrac{1}{2\pi}\int \dd^2 x \ \varepsilon_{\mu \nu}\partial_\mu A_\nu \ ,
    \label{Q(2+1)}
\end{equation}
onde
\begin{equation}
    A_\mu = \dfrac{i}{2}\left(z^{\dag}\partial_\mu z - \partial_\mu z^{\dag} z\right) \ .
\end{equation}
A integração em (\ref{Q(2+1)}) é no plano bidimensional $(x_1 , x_2)$, que, identificando o infinito espacial, é isomórfico à $S^2$.\footnote{Superfície da esfera tridimensional.} Sobre a transformação local $z \rightarrow e^{i\alpha}z$, será obtido $A_\mu \rightarrow A_\mu -\partial_\mu \alpha$. Definindo a derivada covariante $D_\mu \equiv \partial_\mu + i A_\mu$, (\ref{Q(2+1)}) pode ser reescrita como
\begin{equation}
    \mathcal{Q} = \dfrac{i}{2\pi} \int \dd^2 x \ \varepsilon_{\mu\nu}\left(D_\mu z \right)^{\dag}\left(D_\nu z\right) = \dfrac{1}{4\pi} \int \dd^2 x \left[\left(D_\mu z \right)^{\dag} i \, \varepsilon_{\mu\nu} \, D_\nu z + \left( i \,  \varepsilon_{\mu\nu} \, D_\nu z\right)^{\dag} D_\mu z \right].
    \label{carga(2)}
\end{equation}
Seguindo (\ref{trasnf_geral}), são definidas as quantidades:
\begin{equation}
    \mathcal{A}^a_\mu = \left( D_\mu z\right)_b k_{ba} \ ; \qquad \qquad \Tilde{\mathcal{A}}_{\mu}^{a} = \left( k^{-1}_{ab} \right)^* i \, \varepsilon_{\mu\nu} \, \left(D_\nu z \right)_b \ , 
\end{equation}
logo a carga (\ref{carga(2)}) pode ser escrita na forma (\ref{cargatop}). De (\ref{bps2}), as equações de auto-dualidade são
\begin{equation}
    \left(D_\mu z\right)_b h_{ba} = \pm \, i \, \varepsilon_{\mu\nu} \left(D_\nu z\right )_a \ .
    \label{bps2d}
\end{equation}
De acordo com (\ref{energia2}), o funcional da energia se torna
\begin{equation}
    E = \dfrac{1}{2} \int \dd^2x \left[\left(D_\mu z\right)_a^* h_{ab} \left(D_\mu z\right)_b + \left(D_\mu z\right)_a^* h^{-1}_{ab} \left(D_\mu z\right)_b  \right] \ .
    \label{E2d}
\end{equation}
Contraindo os dois lados de (\ref{bps2d}) com $\varepsilon_{\rho \mu}$, obtém-se
\begin{equation}
    \left(D_\mu z\right)_a = \pm \, i \, \varepsilon_{\mu\nu} \, \left(D_\nu z\right)_b h_{ba} \ .
    \label{relacao_eps}
\end{equation}
Portanto, (\ref{bps2d}) e (\ref{relacao_eps}) implicam
\begin{equation}
    \left(D_\mu z \right)_b \left(h_{ba} - h^{-1}_{ba} \right) = 0 \qquad \rightarrow \qquad h^2 = \mathbb{1} \ ,
    \label{4.9}
\end{equation}
mas uma matriz hermitiana pode ser diagonalizada por uma transformação unitária, $h = Uh_DU^{\dag}$, com $h_D$ diagonal. Logo
\begin{equation}
    h^2 = \mathbb{1} \qquad \qquad \rightarrow \qquad\qquad h^2_D = \mathbb{1} \ ,
\end{equation}
assim $\lambda_a^2 = 1$. Porém, para a energia $E$ (\ref{E2d}) ser positiva definida, é necessário que todos autovalores de $h$ tenham o mesmo sinal. Então
\begin{equation}
    h = \mathbb{1} \ .
    \label{4.11}
\end{equation}
Nesse caso, (\ref{bps2d}) reduz-se às equações de auto-dualidade para o modelo $\mathbb{CP}^{N-1}$ usual \cite{sigmamodelinst, lowsigma},
\begin{equation}
    \left(D_\mu z\right)_a = \pm \, i \, \varepsilon_{\mu\nu} \left(D_\nu z\right )_a \ ,
    \label{bps2d_2}
\end{equation}
e (\ref{E2d}) à energia usual do modelo $\mathbb{CP}^{N-1}$,
\begin{equation}
    E = \int \dd^2x \left(D_\mu z \right)^{\dag} D_\mu z \ .
\end{equation}

Para construir soluções auto-duais da equação (\ref{bps2d_2}), é conveniente introduzir novos campos complexos $u_a$ como
\begin{equation}
    (u_1^{(j)}, u_2^{(j)}, ... , u_{N-1}^{(j)}, u_N^{(j)}) = \dfrac{1}{z_j}(z_1,z_2,...,z_{N-1},z_N) \ ,
\end{equation}
em que $u_j^{(j)} = 1$, ou seja, as coordenadas são $u_\alpha^{(j)}, \alpha\neq j$. Pode ser escolhida qualquer componente $z_a$ para construir o campo $u_a$; nesse caso, será escolhida a componente $z_N$, logo $u_\alpha , \alpha = 1,...,N-1$ são as coordenadas. Podemos reescrever a equação (\ref{bps2d_2}) em termo das novas coordenadas da seguinte forma: \cite{sigmamodelinst}
\begin{equation}
    \partial_\mu u_\alpha = \pm \, i \, \varepsilon_{\mu\nu} \, \partial_\nu u_\alpha \ ; \qquad \qquad \alpha = 1,2,...,N-1.
\end{equation}

Essas são as equações de Cauchy-Riemann para os campos $u$. De fato, o sinal $(+)$ implica que $u$ é holomórfico, ou seja, $u_\alpha = u_\alpha(\omega)$, e o sinal $(-)$ que $u$ é anti-holomórfico, ou seja, $u_\beta = u_\beta(\omega^*)$, em que $\omega = x_1 + i x_2$.