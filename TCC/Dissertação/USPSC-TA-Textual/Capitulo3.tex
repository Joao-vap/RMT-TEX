\chapter[\textit{Kinks} com vários campos em (1+1) dimensões]{\textit{Kinks} com vários campos em (1+1) dimensões}
\label{(1+1) dimensões}

Setores auto-duais para teorias em $(1+1)$ dimensões, contendo somente um campo escalar, como modelos de \textit{sine-Gordon} e $\lambda\phi^4$, são conhecidos há bastante tempo. A aplicação das ideias discutidas nas seções anteriores levou a construção de setores auto-duais em teorias em $(1+1)$ dimensões com qualquer número de campos escalares \cite{laf(1+1)}. Nesta seção serão considerados campos escalares reais. A carga topológica nesse caso é
\begin{equation}
    \mathcal{Q} = \int^{\infty}_{-\infty} \dd x \dv{U}{x} = \int^{\infty}_{-\infty}\dd x \fdv{U}{\varphi_a}\dv{\varphi_a}{x} = U(\varphi_a(x=\infty)) - U(\varphi_a(x=-\infty)) \ ,
\end{equation}
em que $U$ é um funcional real arbitrário dos campos $\varphi_a$, $a = 1,2,...,r$, mas não de suas derivadas.\footnote{Isso será importante para que a carga topológica não seja função de derivadas de mais do que primeira ordem dos campos.} A equação acima está na mesma forma de (\ref{cargatop}), assim é possível realizar as identificações
\begin{equation}
    \mathcal{A}_\alpha \equiv k_{\alpha\beta}\dv{\varphi_\beta}{x} \ ; \ \ \ \ \ \ \Tilde{\mathcal{A}}_\alpha \equiv \fdv{U}{\varphi_\beta}k^{-1}_{\beta\alpha} \ ,
\end{equation}
onde os funcionais $\mathcal{A}_\alpha$, $\Tilde{\mathcal{A}}_\alpha$ e $k$ são reais, e a matriz $k$ também é inversível e arbitrária. De acordo com (\ref{bps2}), as equações auto-duais são
\begin{equation}
    \eta_{ab}\dv{\varphi_b}{x} = \pm\fdv{U}{\varphi_a} \ ; \ \ \ \ \ \ \ \ \eta = k^{T}k \ .
    \label{bps1d}
\end{equation}

Então, $\eta$ é uma matriz simétrica e inversível. Seguindo (\ref{energia2}), a energia estática da nossa teoria se torna
\begin{equation}
    E = \int^{\infty}_{-\infty} \dd x \left[\dfrac{1}{2}\eta_{ab}\dv{\varphi_a}{x}\dv{\varphi_b}{x} + V \right] \ ,
    \label{funcE}
\end{equation}
onde a forma do potencial é
\begin{equation}
    V = \dfrac{1}{2} \ \eta^{-1}_{ab} \ \fdv{U}{\varphi_a}\fdv{U}{\varphi_b} \ .
    \label{potencial}
\end{equation}

Portanto, dos argumentos da seção anterior, segue que as soluções de (\ref{bps1d}) são também soluções das equações de Euler-lagrange do funcional (\ref{funcE}), onde a quantidade $U$ desempenha o papel de um pré-potencial. Note que, dado a escolha de um pré-potencial $U$, é possível determinar o potencial $V$ e também uma teoria de campos escalares com um setor auto-dual. Entretanto, dado um potencial, não é trivial encontrar o pré-potencial $U$; tendo isso em vista, será discutida a construção de teorias auto-duais por meio da escolha do pré-potencial. Nesse sentido, a análise será restringida para casos em que os campos escalares $\varphi_a$, o pré-potencial $U$ e a matriz $\eta$ sejam reais. Além disso, é imposto que o funcional $E$ (\ref{funcE}) seja positivo definido, dessa forma os autovalores de $\eta$ também devem ser positivos.

Para que as soluções auto-duais de (\ref{bps1d}) tenham energia finita, é necessário que a densidade de energia em (\ref{funcE}) desapareça para infinitos espaciais quando evaluado nessas soluções, logo é necessário que
\begin{equation}
    \dv{\varphi_a}{x} \rightarrow 0 \ ; \ \ \ \ \ \ \fdv{U}{\varphi_a} \rightarrow 0 \ ; \ \ \ \ \ \ \text{com} \ \ \ \ \ \ x \rightarrow \pm \infty.
    \label{phi_U_zero}
\end{equation}

Portanto, as equações de auto-dualidade (\ref{bps1d}) devem possuir soluções constantes de vácuo $\varphi_a^{(vac)}$ que sejam zeros para todas derivadas do pré-potencial, ou seja,
\begin{equation}
    \eval{\fdv{U}{\varphi_c}}_{\varphi_a = \varphi_a^{(vac)}} = 0.
    \label{dU_zero}
\end{equation}

De (\ref{potencial}), essas soluções de vácuo também são zeros do potencial $V$ e de suas primeiras primeiras derivadas, ou seja,
\begin{equation}
    V\left(\varphi_a^{(vac)}\right) = 0 \ ; \qquad \qquad \eval{\fdv{V}{\varphi_c}}_{\varphi_a = \varphi_a^{(vac)}} = 0 \ .
\end{equation}

Ademais, é desejável que as teorias contruídas tenham diversas soluções tipo sóliton, e que tenham um sistema de vácuos tão degenerados quanto possível para manter as estruturas topológicas não triviais de $\mathcal{Q}$. Existem diversas formas de obter esse sistema de vácuos; nesse trabalho, será adotado o mesmo procedimento que em \cite{laf(1+1)}, baseado na teoria de grupos. Não será discutido o procedimento de criação dos pré-potenciais; para detalhes, consultar \cite{laf(1+1)}.

\section{Interpretação mecânica de soluções auto-duais.}

Tendo como base os desenvolvimentos em (\ref{phi_U_zero}) e (\ref{dU_zero}), soluções da equação de auto-dualidade (\ref{bps1d}) com energia finita devem tender a soluções constantes de vácuo quando $x\rightarrow \pm \infty$. Assim, cada uma dessas soluções conecta dois vácuos da teoria. Para desenvolver uma visualização geométrica dessas soluções, serão reescritas as equações de auto-dualidade da seguinte forma:
\begin{equation}
    \Vec{v} = \pm \Vec{\nabla}_{\eta}U \ ; \quad \quad \quad \text{com} \quad \quad \quad (\Vec{v})_a = \dv{\varphi_a}{x} \ ; \qquad \left(\Vec{\nabla}_{\eta}U \right)_a = \eta^{-1}_{ab} \fdv{U}{\varphi_b} \ .
\end{equation}

Dado o potencial $U$ e a métrica $\eta$, que é real, constante e positiva, $\Vec{\nabla}_\eta U$ define curvas no espaço dos campos ($\varphi_1, \varphi_2, \ ... \ , \varphi_r$),\footnote{O índice utilizado ($r$) a príncipio não possui qualquer significado, mas o uso se deu devido ao conceito de \textit{rank} de uma álgebra de Lie. No caso das construções utilizando teoria de grupos: $r = \textit{rank} \ (\mathcal{G}).$} fazendo o papel de vetor tangente a essas curvas. Essas curvas não se intersectam; para manter $\Vec{\nabla}_\eta U$ bem definido em qualquer ponto do espaço dos campos, no máximo elas podem se tocar tangencialmente ou se encontrar em pontos em que $\Vec{\nabla}_\eta U$ zera. A equação de auto-dualidade (\ref{bps1d}) é uma equação diferencial de primeira ordem, logo uma solução é determinada pelo valor dos campos $\varphi_a$ em um ponto $x = x_0$. 

A visão geométrica é a de uma partícula viajando no espaço dos $\varphi_a$ com \textit{x-velocidade} $\Vec{v}$ e com a coordenada espacial $x$ desempenhando o papel do tempo. Portanto, o problema de resolver a equação de auto-dualidade se reduz ao de construir curvas no espaço dos campos determinadas por $\Vec{\nabla}_\eta U$. As soluções de energia finita correponderão às curvas que começam e terminam nos extremos do pré-potencial $U$, ou seja, nos pontos em que $\Vec{\nabla}_\eta U = 0$.

Considere agora uma curva $\gamma$ no espaço dos campos, parametrizada por $x$, ou seja, $\varphi_a(x)$, que seja solução das equações de auto-dualidade (\ref{bps1d}). Associada a essa curva é definida a quantidade
\begin{equation}
    \Tilde{\mathcal{Q}}(\gamma) = \int_{\gamma}\dd x \ \Vec{v}\cdot\Vec{\nabla}U = \int_{\gamma}\dd x \ \dv{\varphi_a}{x}\fdv{U}{\varphi_a} = U(x_f) - U(x_i),
    \label{Qdiff}
\end{equation}
em que $x_f$ e $x_i$ são os pontos final e inicial, respectivamente, da curva $\gamma$. Perceba que o vetor tangente à curva é $\Vec{\nabla}_\eta U$ e não $\Vec{\nabla}U$, uma vez que a curva é solução das equações (\ref{bps1d}). Utilizando as equações de auto-dualiadade é possível reescrever a equação (\ref{Qdiff}) da seguinte forma:
\begin{equation}
    \Tilde{\mathcal{Q}}(\gamma) = \pm \int_{\gamma} \dd x \ \eta_{ab}\dv{\varphi_a}{x}\dv{\varphi_b}{x} = \pm \int_{\gamma} \dd x \ \omega_a \left(\dv{\tilde{\varphi_a}}{x} \right)^2,
\end{equation}
em que a matriz $\eta$ foi diagonalizada, ou seja, 
\begin{equation}
    \eta = \Lambda^{T}\eta^D\Lambda \ ; \qquad \quad \Lambda^T\Lambda = \mathbb{1} \ ; \quad\qquad \eta_{ab}^D = \omega_a\delta_{ab} \ ; \quad \qquad \omega_a > 0 \ ,
\end{equation}
onde foi assumido que todos autovalores de $\eta$ são positivos, e foi definido $\Tilde{\varphi_a} = \Lambda_{ab}\varphi_b$. Mantendo $\eta$ como positiva definida, a quantidade $\Tilde{\mathcal{Q}}(\gamma)$ só pode assumir o valor zero se os campos forem constantes por toda curva, ou seja, a curva teria que se reduzir a um ponto. Logo, as soluções das equações de auto-dualidade (\ref{bps1d}) não podem começar e acabar em pontos no espaço dos campos em que o pré-potencial $U$ possui o mesmo valor. Além disso, conforme alguém \textit{anda} pela curva a diferença do valor do pré-potencial de um particular ponto e do ponto inicial sempre aumenta em módulo. Portanto, a curva, que é solução das equações (\ref{bps1d}), sempre \textit{escala} o pré-potencial $U$, para cima ou para baixo dependendo do sinal tomado na equação (\ref{bps1d}), sem nunca retornar a uma altitude já atingida. 

\section{Exemplo -- $SU(3)$}

Nesta seção será apresentado um exemplo concreto dos conceitos discutidos nas seções (\ref{autodulidade generalizada}) e (\ref{(1+1) dimensões}). No exemplo que segue, a matriz $\eta$ será constante\footnote{Isto é, não irá depender dos campos $\varphi_a$ da teoria ou de outros campos externos.}, real e positiva definida. Mesmo com essas restrições ainda é possível construir teorias interessantes.

O \textit{rank} de $SU(3)$ é dois, portanto a teoria possuirá dois campos $\varphi_1$ e $\varphi_2$. A matriz $\eta$ é escolhida tal que\footnote{Note que $ \eta|_{\lambda = 1} = K$, com $K$ sendo a matriz de Cartan de $SU(3)$. \cite{lafGRUPOS}}
\begin{equation}
    \eta = \mqty(2 & -\lambda \\ -\lambda & 2) \ ; \qquad\qquad \eta^{-1} = \dfrac{1}{4-\lambda^2} \mqty(2 & \lambda \\ \lambda & 2) \ ,
\end{equation}
onde foi introduzido o parâmetro real $\lambda$. Os autovalores de $\eta$ são $2\pm \lambda$, portanto $\lambda$ deve se manter no intervalo $2 < \lambda < -2$, para manter $\eta$ positiva definida e inversível. Por meio de um procedimento utilizando teoria de grupos\footnote{A construção se baseia nos pesos das representações da algebra. Para mais detalhes consultar \cite{laf(1+1)}.}, é obtido o pré-potencial
\begin{equation}
    U = \gamma_1 \cos\qty(\varphi_1) + \gamma_2 \cos\qty(\varphi_2) + \gamma_3 \cos\qty(\varphi_1 - \varphi_2) \ ,
    \label{U_SU3}
\end{equation}
em que $\gamma_1$, $\gamma_2$ e $\gamma_3$ são constantes arbitrárias.

A energia estática (\ref{funcE}) se torna
\begin{equation}
    E = \int^{\infty}_{-\infty} \dd x \ \qty[\qty(\partial_x \varphi_1)^2 + \qty(\partial_x \varphi_2)^2 - \lambda\partial_x\varphi_1\partial_x\varphi_2 + V\qty(\varphi_1, \varphi_2)] \ ,
\end{equation}
onde o potencial (\ref{potencial}) é dado por
\begin{equation}
\begin{aligned}
 	V &= \dfrac{1}{\lambda^2 - 4}[-\gamma_1^2\sin^2(\varphi_1) + \gamma_1\sin(\varphi_1)\qty(\gamma_3\qty(\lambda -2)\sin (\varphi_1 - \varphi_2) - \gamma_2\lambda\sin(\varphi_2)) \\
	   & - \gamma_2^2\sin^2(\varphi_2) - \gamma_2\gamma_3\qty(\lambda-2)\sin(\varphi_2)\sin(\varphi_1-\varphi_2)+\gamma_3^2\qty(\lambda -2)\sin^2(\varphi_1-\varphi_2)] \ .
    \label{V_SU3}
\end{aligned}
\end{equation}

As equações de auto-dualidade (\ref{bps1d}) são da forma
\begin{equation}
\begin{aligned}
    &\partial_x\varphi_1 = \pm \dfrac{\qty[2\gamma_1\sin(\varphi_1) + \lambda\gamma_2\sin(\varphi_2) - \gamma_3\qty(\lambda-2)\sin(\varphi_1-\varphi_2)]}{\lambda^2 - 4}, \\
    &\partial_x\varphi_2 = \pm \dfrac{\qty[2\gamma_2\sin(\varphi_2) + \lambda\gamma_1\sin(\varphi_1) + \gamma_3\qty(\lambda-2)\sin(\varphi_1-\varphi_2)]}{\lambda^2 - 4}.
\end{aligned}
\end{equation}

Os vácuos são determinados pelas condições (\ref{dU_zero}), que implicam
\begin{equation}
    \gamma_1\sin\qty(\varphi_1^{(vac)})=-\gamma_3\sin\qty(\varphi_1^{(vac)}-\varphi_2^{(vac)}) = - \gamma_2 \sin\qty(\varphi_2^{(vac)}).
\end{equation}

Certamente, a equação acima é satisfeita se
\begin{equation}
    \varphi_a^{(vac)} = n_a \pi \ , \qquad\qquad n_a \in \mathbb{Z}, \qquad\qquad a=1,2 \ ,
    \label{vac_normal}
\end{equation}
para qualquer valor para os coeficientes $\gamma$. Entretanto, existem outros tipos de vácuo que dependem do valor escolhido para cada $\gamma$, como 
\begin{equation}
\begin{aligned}
    & \qty(\varphi_1^{(vac)}, \varphi_2^{(vac)}) = \qty(\dfrac{2\pi}{3} + 2\pi \, n_1 \, , \ \dfrac{4\pi}{3} + 2\pi \, n_2) \ ; \qquad \qquad \gamma_1=\gamma_2=\gamma_3 = 1, \\
    & \qty(\varphi_1^{(vac)}, \varphi_2^{(vac)}) = \qty(\dfrac{4\pi}{3} + 2\pi \, n_1 \, , \ \dfrac{2\pi}{3} + 2\pi \, n_2) \ ; \qquad\qquad n_1 \, , \, n_2 \in \mathbb{Z} \ .
    \label{vac_especial}
\end{aligned}
\end{equation}


\begin{figure}[ht] 
  \begin{subfigure}[b]{0.5\linewidth}
    \centering
    \includegraphics[width=0.95\linewidth]{USPSC-img/U_SU(3)_(3D).png} 
    \caption{} 
    \label{fig7:a} 
    \vspace{4ex}
  \end{subfigure}%% 
  \begin{subfigure}[b]{0.5\linewidth}
    \centering
    \includegraphics[width=0.95\linewidth]{USPSC-img/V_SU(3)_Lambda=1.0_(3D).png} 
    \caption{} 
    \label{fig7:b} 
    \vspace{4ex}
  \end{subfigure} 
  \begin{subfigure}[b]{0.5\linewidth}
    \centering
    \includegraphics[width=0.95\linewidth]{USPSC-img/U_SU(3)_(2D).png} 
    \caption{} 
    \label{fig7:c} 
  \end{subfigure}%%
  \begin{subfigure}[b]{0.5\linewidth}
    \centering
    \includegraphics[width=0.95\linewidth]{USPSC-img/V_SU(3)_Lambda=1.0_(2D).png} 
    \caption{} 
    \label{fig7:d} 
  \end{subfigure} 
  \caption{(a) e (c) são representações do pré-potencial (\ref{U_SU3}), e (b) e (c) do potencial (\ref{V_SU3}). Os vácuos do tipo (\ref{vac_normal}) são os pontos pretos, enquanto os do tipo (\ref{vac_especial}) são denotados por quadrados em branco. As linhas representam o fluxo de $\grad U$ e $\grad V$, em que $\grad = \qty(\partial_{\varphi_1}, \partial_{\varphi_2})$. Nos gráficos, foi utilizado $\lambda = 1$ e $\gamma_1 = \gamma_2 = \gamma_3 = 1$.}
  \label{fig7} 
\end{figure}
