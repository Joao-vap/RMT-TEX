Uma matriz aleatória é uma matriz cujas entradas são variáveis aleatórias, não necessariamente independentes tampouco de mesma distribuição. A princípio, de um ponto de vista puramente analítico, pode-se tratar uma matriz aleatória de tamanho $N\times N$ como um vetor aleatório de tamanho $N^2$. No entanto, as estruturas algébrico-geométricas presentes a matrizes, como multiplicação natural, interpretação como operadores, ou decomposições espectrais, trazem à matrizes aleatórias aplicações múltiplas. Em particular, sua relevância estende um ponto comum que compartilham com variáveis aleatórias: permitir descrições estatísticas a sistemas e fenômenos. 
	
É comum modelar com matrizes aleatórias, por exemplo, operadores com perturbações aleatórias. De um ponto de vista físico, autovalores de um dado operador descrevem espectros de energia do sistema descrito. Na quântica, por exemplo, autovalores são as medidas observadas. Surge uma pergunta naturalmente: dada uma matriz aleatória $M$, o que podemos dizer sobre estatísticas de seus autovalores? Essa resposta, claro, depende de maneira altamente não trivial das distribuições das entradas.  
