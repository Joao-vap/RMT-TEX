% General Configs

\documentclass[10pt, compress]{beamer}

%---------------------------------------------------------
%---------------------------------------------------------
\usepackage[alf]{abntex2cite}
\usepackage[utf8]{inputenc}
\usepackage[portuguese]{babel}
\usepackage{graphicx}			% Inclusão de gráficos
\usepackage{float} 				% Fixa tabelas e figuras no local exato
\usepackage{physics}
\usepackage{bbold}
\usepackage{subcaption}

%------------------Tikz
\usepackage{tikz}
\usetikzlibrary{shapes,positioning,calc,quotes}
\tikzstyle{subrotina} = [rectangle, draw,text centered, minimum width=5em, inner sep=5pt]
\tikzstyle{funcao} = [rounded rectangle, draw, text centered, minimum width=7em, inner sep=5pt]
\tikzstyle{random} = [rectangle, text centered, inner sep=5pt, minimum width=5em]
\tikzstyle{loop} = [rectangle, draw, align=left]
\tikzstyle{fluxo} = [draw, thick, -latex]
\tikzstyle{meiofluxo} = [draw, -]
\tikzstyle{chamada} = [draw, dashed, <->]
%-----------------------

%------------------------------------------------------------
% Configure Theme like things

\usetheme{Madrid}
\usefonttheme{structuresmallcapsserif}
\useoutertheme{miniframes} % Alternatively: miniframes, infolines, split
\useinnertheme{circles}
\setbeamercovered{transparent}

\definecolor{UBCblue}{rgb}{0.04706, 0.13725, 0.26667} % UBC Blue (primary)
\usecolortheme[named=UBCblue]{structure}

%------------ proportions in footer
\makeatletter
\setbeamertemplate{footline}
{
	\leavevmode%
	\hbox{%
		\begin{beamercolorbox}[wd=.333333\paperwidth,ht=2.25ex,dp=1ex,center]{author in head/foot}%
			\usebeamerfont{author in head/foot}\insertshortauthor\expandafter\beamer@ifempty\expandafter{\beamer@shortinstitute}{}{~~(\insertshortinstitute)}
		\end{beamercolorbox}%
		\begin{beamercolorbox}[wd=.4\paperwidth,ht=2.25ex,dp=1ex,center]{title in head/foot}%
			\usebeamerfont{title in head/foot}\insertshorttitle
		\end{beamercolorbox}%
		\begin{beamercolorbox}[wd=.266666\paperwidth,ht=2.25ex,dp=1ex,right]{date in head/foot}%
			\usebeamerfont{date in head/foot}\insertshortdate{}\hspace*{2em}
			\insertframenumber{} / \inserttotalframenumber\hspace*{2ex} 
	\end{beamercolorbox}}%
	\vskip0pt%
}
\makeatother
%------------- background image title page
\addtobeamertemplate{title page}{
	\begin{tikzpicture}[remember picture,overlay]
		\node[anchor=south west,inner sep=0pt] at (-4,-2.2)
		{
			\includegraphics[width=1.5\paperwidth]{./media/aesthetics/croppedCircle.png}
		};
	\end{tikzpicture}
}{}
%------------

%TOC
\setbeamertemplate{section in toc}[sections numbered]
\setbeamertemplate{subsection in toc}[subsections numbered]
\setbeamerfont{subsection in toc}{size=\small, shape=\itshape}

%Templetae
\setbeamertemplate{navigation symbols}{}
\setbeamertemplate{note page}[compress]

%------------------------------------------------------------
%This block of code defines the information to appear in the
%Title page
\title[Matrizes Aleatórias e Simulação de Gases de Coulomb] %optional
{Matrizes Aleatórias e Simulação de Gases de Coulomb}

\subtitle{}

\author[João V. A. Pimenta] % (optional)
{João V. A. Pimenta\inst{1} \and Guilherme Silva\inst{2}}

\institute[IFSC] % (optional)
{
	\inst{1}%
	Instituto de Física de São Carlos - IFSC\\
	Universidade de São Paulo - USP
	\and
	\inst{2}%
	Insitituto de Ciências Matemáticas e de Computação - ICMC\\
	Universidade de São Paulo - USP
}

\date[TCC 2024] % (optional)
{Defesa de TCC em Física Computacional, Julho 2024}

\logo{\includegraphics[height=0.5cm]{media/aesthetics/logotipoifsc}}

%End of title page configuration block
%------------------------------------------------------------

%------------------------------------------------------------
%The next block of commands puts the table of contents at the 
%beginning of each section and highlights the current section:

\AtBeginSection[]
{
	{
	\usebackgroundtemplate{\includegraphics[width=\paperwidth]{./media/aesthetics/AstecDiamond.png}}
	\begin{frame}
		\frametitle{Sumário}
		\tableofcontents[currentsection]
	\end{frame}
	}
}
%------------------------------------------------------------

\newcommand\underrel[2]{\mathrel{\mathop{#2}\limits_{#1}}}

\newcommand{\matriz}[1]{\hat#1}

\newcommand{\many}[2]{$#1_1, #1_2, \dots, #1_#2$}

\newcommand{\cmany}[3]{$#1_1 #3 #1_2 #3 \dots #3 #1_#2$}

\newcommand{\mmany}[2]{ #1_1, #1_2, \dots, #1_#2 }

\newcommand{\mcmany}[3]{#1_1 #3 #1_2 #3 \dots #3 #1_#2}

\newcommand{\set}[1]{\{#1\}}

\newcommand{\cjgt}[1]{\overline{#1}}
\DeclareMathOperator{\diag}{diag}
\DeclareMathOperator{\sign}{sign}
\DeclareMathOperator{\ai}{Ai}
\DeclareMathOperator{\re}{Re}
\DeclareMathOperator{\im}{Im}
\DeclareMathOperator{\Df}{D}
\DeclareMathOperator{\Ee}{E}
\DeclareMathOperator{\h}{h_1}
\DeclareMathOperator{\f}{f}
\DeclareMathOperator{\U}{U}
\DeclareMathOperator{\W}{W}
\DeclareMathOperator{\K}{K}
\DeclareMathOperator{\Hf}{\mathcal{H}}
\DeclareMathOperator{\Qf}{Q}
\DeclareMathOperator{\Gl}{\mathcal{L}}
\DeclareMathOperator{\g}{g}
\DeclareMathOperator{\V}{V}
\DeclareMathOperator{\Glin}{GL}
\newcommand{\iu}{\mathrm{i}\mkern1mu}
\renewcommand{\Im}{\mathop{\textrm Im}}
\DeclareMathOperator{\ee}{e}
\DeclareMathOperator{\supp}{supp}
\newcommand{\N}{\mathbb{N}}
\newcommand{\C}{\mathbb{C}}
\newcommand{\R}{\mathbb{R}}
\newcommand{\Z}{\mathbb{Z}}
\newcommand{\D}{\mathbb{D}}
\newcommand{\Q}{\mathbb{Q}}
\newcommand{\J}{J} %Jacobiano
\newcommand{\Id}{\mathbb{1}}
\newcommand{\p}{p} %medida
\newcommand{\E}{\mathbb{E}}
\newcommand{\Se}{\mathbb{S}}
\newcommand{\He}{\mathbb{H}}
\newcommand{\boh}{\mathit{o}}
\newcommand{\Boh}{\mathcal{O}}
\newcommand{\bbp}{\bm K_{\mathrm{BBP}}}
\newcommand{\ii}{\mathrm{i}}
\newcommand*{\deff}{\mathrel{\vcenter{\baselineskip0.5ex \lineskiplimit0pt
			\hbox{\scriptsize.}\hbox{\scriptsize.}}}%
	=}
\newcommand*{\revdeff}{=\mathrel{\vcenter{\baselineskip0.5ex \lineskiplimit0pt
			\hbox{\scriptsize.}\hbox{\scriptsize.}}}%
}


% MATH DECLARATION
\numberwithin{equation}{section} %numeracao dentro de secoes