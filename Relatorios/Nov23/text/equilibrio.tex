Assim como na física, para resolver nossa distribuição, vamos precisar minimizar a energia livre do nosso ensemble. De alguma forma recuperamos a noção da energia livre de Helmholtz,

\[
F = -\frac{1}{k_b} \ln{(\mathcal{Z}_N)}
\]

Teremos que os estados mais prováveis serão aqueles em que for maximizada a expressão

\begin{equation}\
	\exp{\left[-N^2 \left( \frac{1}{N^2}\sum_{i\neq j}\log{\frac{1}{|\lambda_i - \lambda_j|}} + \frac{1}{N^2} \sum_{i=1}^{N} \tilde{V}(\lambda_i)  \right)\right]}
\end{equation}

Onde identificamos o Hamiltoniano e escrevemos $\exp{(-N^2\mathcal{\tilde{H}}_N(\lambda))}$. Precisamos minimizar o Hamiltoniano do sistema $\mathcal{\tilde{H}}_N(\lambda)$. Vamos introduzir uma função contagem para facilitar o tratamento do conjunto de pontos em $\mathbb{R}$. Definimos

\begin{equation}
	\upsilon_\lambda = \frac{1}{N} \sum_1^N \delta_{\lambda_i}
\end{equation}

De forma que

\begin{equation}
	\mathcal{H}_N(\upsilon_\lambda) = 	\int \int_{x\neq y} \log{\frac{1}{|\lambda_i - \lambda_j|}}  \upsilon_\lambda(x) \upsilon_\lambda(y) dx dy + \frac{1}{N} \int \tilde{V}(x) \upsilon_\lambda(x) dx
	\label{eq::CoulombGas:: hamilton}
\end{equation}

O que acontece quando tratamos do limite termodinâmico? Ou seja, quando $N\to\infty$. Estaremos transicionando da nossa função $\upsilon_\lambda$ para uma densidade $\mu_V(x) dx$, ou seja,

\[
\int f(x) \upsilon_\lambda(x) dx =  \int f(x) \mu_V(x) dx
\]

Precisamos garantir ainda um potencial $V(x) = 
N\tilde{V}(x)$ para trabalharmos a assintótica e mantermos a integrabilidade. Escreve-se

\[
\frac{1}{\mathcal{Z}_N} \prod_{i<j} (\lambda_i - \lambda_j)^\beta \prod_{i=1}^{N} e^{-NV(\lambda_i)} d\lambda = 	\frac{1}{\mathcal{Z}_N}  e^{-N^2 \mathcal{H}_N(\lambda)}
\]

E finalmente poderemos expressar de forma correta

\begin{equation}
	\mathcal{H}_N(\upsilon_\lambda) \to \int \int_{x\neq y} \log{\frac{1}{|\lambda_i - \lambda_j|}}  d\mu_V(dx) \mu_V(dy) + \int V(x) \mu_V(dx) \equiv \epsilon^V(\mu_V)
\end{equation}

Onde $\mu_V(x)$ será medida de probabilidade não aleatória tal que na assintótica,

\[
\mu_V^* = \arg \inf {\epsilon}
\]

sendo o minimizante único da função 'energia' $\epsilon$ convexa e semi continua.
