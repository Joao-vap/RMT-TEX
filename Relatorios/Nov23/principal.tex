\documentclass[12pt]{report}
\usepackage[a4paper]{geometry}
\usepackage[utf8]{inputenc}
\usepackage[english,portuguese]{babel}
\usepackage[myheadings]{fullpage}
\usepackage[T1]{fontenc}
\usepackage{fancyhdr}
\usepackage{graphicx, setspace}
\usepackage{sectsty}
\usepackage{url}
\usepackage{pdfpages}
\usepackage{subcaption}
\usepackage{amsmath}
\usepackage{multirow}
\usepackage{tikz}
\usepackage{minted}
\usepackage{hyperref}
\usepackage{amsfonts}
\usepackage{mathtools, amsmath}

%%------ 
%% Comandos gerais
%% Observação: o arquivo "comandos.tex" tem que estar presente.
%%------
%%%%%%%%%%%%%%%%%%%%%%%%%%%%%%%%%%%%%%%%%%%%%%%%%%%%%%%%%%%%%%%%%%%%%
% In English:
%    This is a list of commands specification for FAPESP reports.
%
% In Portuguese:
%    Esta é uma lista de especificação de comandos para relatórios
% da Fundação de Amparo à pesquisa do Estado de São Paulo (FAPESP).
%
% Author/Autor: André Leon Sampaio Gradvohl, Dr.
% Email:        andre.gradvohl@gmail.com
% Lattes CV:    http://lattes.cnpq.br/9343261628675642
% 
% Last update/Última versão: 11/Sep/2016
%%%%%%%%%%%%%%%%%%%%%%%%%%%%%%%%%%%%%%%%%%%%%%%%%%%%%%%%%%%%%%%%%%%%%%

\def\checkmark{\tikz\fill[scale=0.4](0,.35) -- (.25,0) -- (1,.7) -- (.25,.15) -- cycle;}

\DeclareMathOperator{\diag}{diag}
\DeclareMathOperator{\ai}{Ai}
\DeclareMathOperator{\re}{Re}
\DeclareMathOperator{\im}{Im}
\DeclareMathOperator{\ee}{\rm e}
\DeclareMathOperator{\supp}{supp}
\renewcommand{\Re}{\mathop{\rm Re}}
\newcommand{\res}{\mathop{\rm Res}}
\renewcommand{\Im}{\mathop{\rm Im}}
\newcommand{\N}{\mathbb{N}}
\newcommand{\C}{\mathbb{C}}
\DeclareMathOperator{\Tr}{Tr}
\newcommand{\R}{\mathbb{R}}
\newcommand{\Z}{\mathbb{Z}}
\newcommand{\D}{\mathbb{D}}
\newcommand{\Q}{\mathbb{Q}}
\newcommand{\boh}{\mathit{o}}
\newcommand{\Boh}{\mathcal{O}}
\newcommand{\bbp}{\bm K_{\mathrm{BBP}}}
\newcommand{\ii}{\mathrm{i}}
\newcommand{\dd}{\mathrm{d}}
\newcommand*{\deff}{\mathrel{\vcenter{\baselineskip0.5ex \lineskiplimit0pt
			\hbox{\scriptsize.}\hbox{\scriptsize.}}}%
	=}
\newcommand*{\revdeff}{=\mathrel{\vcenter{\baselineskip0.5ex \lineskiplimit0pt
			\hbox{\scriptsize.}\hbox{\scriptsize.}}}%
}

\newcommand{\HRule}[1]{\rule{\linewidth}{#1}}
\setcounter{tocdepth}{3}
\setcounter{secnumdepth}{3}

\newcommand{\titulo}[1]{\def\meuTitulo{#1}}
\newcommand{\tituloIngles}[1]{\def\meuTituloIngles{#1}}
\newcommand{\numProjeto}[1]{\def\numFAP{#1}}
\newcommand{\tipoRelatorio}[1]{\def\tipoRelat{#1 }} %o espaço depois do #1 é importante
\newcommand{\modalidadeProjeto}[1]{\def\modProjeto{#1}} 
\newcommand{\agFomento}[2]{\def\agFom{#1} \def\siglaAgFom{#2}} %extenso Sigla
\newcommand{\autor}[1]{\def\nomeAutor{#1}}
\newcommand{\cidade}[1]{\def\nomeCidade{#1}}
\newcommand{\universidade}[1]{\def\nomeUniversidade{#1}}
\newcommand{\faculdade}[1]{\def\nomeFaculdade{#1}}
\newcommand{\periodoVigencia}[1]{\def\periodVig{#1}}
\newcommand{\periodoRelatorio}[1]{\def\periodRelat{#1}}

\author{}
\date{}

%Definição de membros da equipe de pesquisas
\newcommand{\membroA}[1]{\def\nomeMembroA{#1}}
\newcommand{\membroB}[1]{\def\nomeMembroB{#1}}
\newcommand{\membroC}[1]{\def\nomeMembroC{#1}}
\newcommand{\membroD}[1]{\def\nomeMembroD{#1}}
\newcommand{\membroE}[1]{\def\nomeMembroE{#1}}
\newcommand{\membroF}[1]{\def\nomeMembroF{#1}}

\newcommand{\Figure}[1]{Figura~\ref{fig:#1}}
\newcommand{\Table}[1] {Tabela~\ref{#1}}
\newcommand{\Equation}[1] {Equa\c{c}\~ao~\ref{#1}}
\newcommand{\addFigure}[3] { %Parametros scale, fig_name, caption 
    \begin{figure}[!hbt]
      \centering
      \includegraphics[scale=#1]{figures/}
      \caption{#3}\label{fig:#2}
    \end{figure}
}

\newcommand{\geraTitulo}{
\clearpage
\begin{titlepage}
  \begin{center}
      \vspace*{-3cm}
       { \setstretch{.5} 
         \textsc{\nomeUniversidade} \\
         \HRule{.2pt}\\
         \textsc{\nomeFaculdade}
       }

       \vspace{5.5cm}

       \Large \textbf{\textsc{\meuTitulo}}
 	  \HRule{1.5pt} \\ [0.5cm]
       \linespread{1}
       \large Relatório Científico 
       \ifdefined\tipoRelat
            \tipoRelat
       \fi
       do projeto 
       \ifdefined\modProjeto
           na modalidade \modProjeto,
       \fi
       fomentado pela \agFom. \\ 
   	   \HRule{1.5pt} \\ [0.5cm]

       \ifdefined\numFAP
          Projeto \siglaAgFom~\texttt{\#\numFAP}
          \\ [0.5cm]
       \fi
        Pesquisador Responsável: \nomeAutor
        
        \hspace{2cm}
        
        \includegraphics[scale=0.7]{Assets/CuteCircleWhite}
       
        \vfill
       
        {\normalsize  \nomeCidade, \today}
 \end{center}
 \end{titlepage}
}

\usepackage{titlesec}
\titleformat{\chapter}{\normalfont\LARGE\bfseries}{\thechapter}{1em}{}
\titlespacing*{\chapter}{0pt}{3.5ex plus 1ex minus .2ex}{2.3ex plus .2ex}

%----------------------------------------------------------------------
% Cabeçalho e rodapé
%----------------------------------------------------------------------
\pagestyle{fancy}
\fancyhf{} % Limpa todos os campos de header and footer fields
\renewcommand{\headrulewidth}{0pt}
\fancyfoot[R]{\thepage}

\addto\captionsportuguese{\renewcommand{\contentsname}{Sumário}}
\addto\captionsportuguese{\renewcommand{\bibname}{Referências Bibliográficas}}

%------
% Resumo e Abstract
%------
\newcommand{\Resumo}[1]{
   \begin{otherlanguage}{portuguese}
       \addcontentsline{toc}{chapter}{Resumo}
       \begin{abstract} \thispagestyle{plain} \setcounter{page}{2}
          #1
        \end{abstract}
   \end{otherlanguage} 
} %end \Resumo

\newcommand{\Abstract}[1]{
   \begin{otherlanguage}{english}
      \addcontentsline{toc}{chapter}{Abstract}
      \begin{abstract} \thispagestyle{plain} \setcounter{page}{3}
       #1
      \end{abstract}    
    \end{otherlanguage} 
} %end \abstract

%------
% Folha de rosto
%------
\newcommand{\folhaDeRosto}{
   \chapter*{Informações Gerais do Projeto}
   \addcontentsline{toc}{chapter}{Informações Gerais do Projeto}
   \begin{itemize}
      \item Título do projeto: 
            \begin{itemize}\item[] \textbf{\meuTitulo} \end{itemize}
      \item Nome do pesquisador responsável: 
            \begin{itemize}\item[]\textbf{\nomeAutor}\end{itemize}
      \item Instituição sede do projeto: 
            \begin{itemize}
               \item[]\textbf{\nomeFaculdade \ da \nomeUniversidade} 
            \end{itemize}
      \item Equipe de pesquisa:
            \begin{itemize}
               \ifdefined\nomeMembroA
                 \item[]\textbf{\nomeMembroA}
               \else 
                 \item[]\textbf{\nomeAutor}
               \fi
               \ifx\nomeMembroB\undefined\else \item[]\textbf{\nomeMembroB}\fi
               \ifx\nomeMembroC\undefined\else \item[]\textbf{\nomeMembroC}\fi
               \ifx\nomeMembroD\undefined\else \item[]\textbf{\nomeMembroD}\fi
               \ifx\nomeMembroE\undefined\else \item[]\textbf{\nomeMembroE}\fi
               \ifx\nomeMembroF\undefined\else \item[]\textbf{\nomeMembroF}\fi
             \end{itemize}
       
          \ifdefined \numFAP
             \item Número do projeto de pesquisa:
             \begin{itemize}
                 \item[]\textbf{\numFAP} 
             \end{itemize}
          \fi  
       \item Período de vigência:
            \begin{itemize}
               \item[]\textbf{\periodRelat} 
            \end{itemize}
       \item Período coberto por este relatório científico:
            \begin{itemize}
               \item[]\textbf{\periodVig} 
            \end{itemize}
   \end{itemize}
   \clearpage
}


\newcommand\underrel[2]{\mathrel{\mathop{#2}\limits_{#1}}}

\newcommand{\matriz}[1]{\hat#1}

\newcommand{\many}[2]{$#1_1, #1_2, \dots, #1_#2$}

\newcommand{\cmany}[3]{$#1_1 #3 #1_2 #3 \dots #3 #1_#2$}

\newcommand{\mmany}[2]{ #1_1, #1_2, \dots, #1_#2 }

\newcommand{\mcmany}[3]{#1_1 #3 #1_2 #3 \dots #3 #1_#2}

\newcommand{\set}[1]{\{#1\}}

\newcommand{\cjgt}[1]{\overline{#1}}
\DeclareMathOperator{\sign}{sign}
\DeclareMathOperator{\Df}{D}
\DeclareMathOperator{\Ee}{E}
\DeclareMathOperator{\h}{h_1}
\DeclareMathOperator{\f}{f}
\DeclareMathOperator{\U}{U}
\DeclareMathOperator{\W}{W}
\DeclareMathOperator{\K}{K}
\DeclareMathOperator{\Hf}{\mathcal{H}}
\DeclareMathOperator{\Qf}{Q}
\DeclareMathOperator{\Gl}{\mathcal{L}}
\DeclareMathOperator{\g}{g}
\DeclareMathOperator{\V}{V}
\newcommand{\iu}{\mathrm{i}\mkern1mu}
\renewcommand{\Im}{\mathop{\textrm Im}}
\newcommand{\J}{J} %Jacobiano
\newcommand{\Id}{\mathds{1}}
\newcommand{\p}{\mathcal{P}} %medida
\newcommand{\Se}{\mathbb{S}}
\newcommand{\He}{\mathbb{H}}
 \newcommand{\E}{\mathbb{E}}

% MATH DECLARATIONS
\newtheorem{lemma}{Lema}[section]
\newtheorem{thm}[lemma]{Teorema}
\newtheorem{claim}[lemma]{Afirmação}
\newtheorem{cor}[lemma]{Corolário}
\newtheorem{definition}[lemma]{Definição}
\newtheorem{conjecture}[lemma]{Conjectura}
\newtheorem{prop}[lemma]{Proposição}
\newtheorem{assumption}[lemma]{Assumpção}
\numberwithin{equation}{section} %numeracao dentro de secoes

% PROOF ENV
\makeatletter
\newenvironment{proof}[1][Demonstração]{\par
	\pushQED{\qed}%
	\normalfont \topsep6\p@\@plus6\p@\relax
	\trivlist
	\item\relax
	{\itshape
		#1\@addpunct{.}}\hspace\labelsep\ignorespaces
}{%
	\popQED\endtrivlist\@endpefalse
}
\makeatother
%
%%-----
%% Página de título
%% Observação: As definições que aparecem a seguir comporão a
%%             página de título e a folha de rosto.
%%-----
%% Define o nome da universidade onde o projeto foi desenvolvido.
\universidade{Universidade de São Paulo}
%
%% Define o nome da faculdade onde o projeto foi desenvolvido.
\faculdade{Instituto de Ciências Matemáticas e de Computação (ICMC)}
%
%% Define o título do projeto.
\titulo{Análise Assintótica de Sistemas de Partículas e Matrizes Aleatórias}
%
%% Define a agencia de Fomento e a abreviatura. O primeiro argumento é o 
%% nome por extenso e o segundo a abreviatura.
%% Ambos os argumentos são obrigatórios
\agFomento{Fundação de Amparo à Pesquisa do Estado de São Paulo}{FAPESP}
%
%% Define o tipo de relatório. Pode ser Anual ou Final.
%% Não é obrigatório definir o tipo de relatório.
\tipoRelatorio{Anual}
%
%% Define a modalidade de Projeto. Pode ser temático, regular, etc.
\modalidadeProjeto{Auxílio à Pesquisa Regular}
%
%% Define o número do projeto.
%% Não é obrigatório definir o número do projeto.
\numProjeto{2023/02674-0} 
%
%% Define o autor do relatório.
\autor{Guilherme L. F. Silva}
%
%% Define a equipe do projeto (incluindo o pesquisador responsável no comando \membroA{}
\membroA{Guilherme L. F. Silva}
%% Inclua os demais membros do grupo (máximo +5)
\membroB{João Victor Alcantara Pimenta}
%\membroC{Francisco}
%\membroD{Joao}
%\membroE{Antonio}
%\membroF{José}
%
%% Define o período da vigência do Projeto.
\periodoVigencia{01/06/2023 a 31/05/2024}
%
%% Define o período coberto pelo relatório.
\periodoRelatorio{01/06/2023 a 10/11/2023}
%
%% Define a cidade onde o projeto foi desenvolvido.
\cidade{São Carlos}

%%-----
%% Página de título
%% Observação: Os comandos a seguir não devem ser mudados, 
%%             exceto caso necessário.
%%-----
\begin{document}
	%
	%% Define a numeração em romanos.
	\pagenumbering{roman}
	%
	%% Gera a folha de título.
	\geraTitulo
	%
	%% Gera a folha de rosto.
	\folhaDeRosto
	%
	%% Escreva aqui o resumo em português.
	\Resumo{O estudo de Matrizes Aleatórias demonstra aplicabilidade em uma gama diversa de áreas, com destaque no estudo de mecânica estatística, principalmente na simulação de gases. Estudando a densidade espectral de sistemas de matrizes Gaussianas pode-se desenvolver uma analogia que possibilita a simulação de sistemas de gases diversos, como o de Coulomb. Algumas dificuldades surgem na implementação de simulações baseadas nesta teoria, principalmente em escalabilidade do sistema e no tratamento de possíveis singularidades. Para resolver estes problemas, abordou-se na simulação na literatura, dentre outros, o Algoritmo Híbrido de Monte Carlo, de ótimo comportamento numérico. Nosso objetivo é explorar este assunto, as simulações de gases e o algoritmo citado acima além de expandir os potenciais em que foi-se bem documentado o comportamento destas simulações.}
	%
	%% Escreva aqui o resumo em inglês.
	% \Abstract{
		%	The study of Random Matrices demonstrates applicability in a diverse range of areas, empha-
		%	sizing the study of statistical mechanics, mainly in the simulation of gases. By studying the
		%	spectral density of Gaussian matrix systems, one can develop the simulation of Coulomb gas
		%	systems in analogy. Some difficulties arise in implementing simulations based on this theory,
		%	mainly in system scalability and the treatment of possible singularities. To solve these pro-
		%	blems, the literature has simulated these gases, with excellent numerical behavior, using the
		%	Monte Carlo Hybrid Algorithm. We wish to explore this problem, the simulations, and the
		%	proposed algorithm along with extending the potentials to wich it has been tested.
		%}
	%
	%% Adicionará o sumário.
	%% Mantenha o \thispagestyle{empty} e \clearpage
	\thispagestyle{empty}
	\clearpage
	%
	%% Define a numeração em arábicos.
	\pagenumbering{arabic}
	
	%%-----
	%% Formatação do título da seção
	%%-----
	\sectionfont{\scshape}
	
	%%-----
	%% Corpo do texto
	%%-----
	\chapter{Introdução à Matrizes Aleatórias}\label{chp:resumoProj} 
	
	Uma matriz aleatória é uma matriz cujas entradas são variáveis aleatórias, não necessariamente independentes tampouco de mesma distribuição. A princípio, de um ponto de vista puramente analítico, pode-se tratar uma matriz aleatória de tamanho $N\times N$ como um vetor aleatório de tamanho $N^2$. No entanto, as estruturas algébrico-geométricas presentes a matrizes, como multiplicação natural, interpretação como operadores, ou decomposições espectrais, trazem à matrizes aleatórias aplicações múltiplas. Em particular, sua relevância estende um ponto comum que compartilham com variáveis aleatórias: permitir descrições estatísticas a sistemas e fenômenos. 
	
	É comum modelar com matrizes aleatórias, por exemplo, operadores com perturbações aleatórias. De um ponto de vista físico, autovalores de um dado operador descrevem espectros de energia do sistema descrito. Na quântica, por exemplo, autovalores são as medidas observadas. Surge uma pergunta naturalmente: dada uma matriz aleatória $M$, o que podemos dizer sobre estatísticas de seus autovalores? Essa resposta, claro, depende de maneira altamente não trivial das distribuições das entradas.  
	
	\section{Distribuição dos Autovalores}
	
	Consideremos inicialmente um espaço de matrizes com $N^2$ entradas independentes, sejam elas reais, complexas ou simpléticas. Se tivéssemos interesse de expressar a medida desse espaço poderíamos escrever

\begin{equation}
	p(\hat{M}) dM = p(M_{1,1}, \dots, M_{N,N}) \prod_{i,j=1}^{N} dM_{i,j}
\end{equation}

Contido neste espaço temos um espaço de maior interesse correspondente ao espaço das matrizes \textit{simétricas} ou \textit{hermitianas}. A escolha do subespaço está relacionada com o fato  de que essas matrizes são diagonalizáveis. Podemos escrever nossa matriz $\matriz{H}$ como 

\[
\matriz{H} = \matriz{U} \matriz{\Lambda} \matriz{U}^{-1} \ , \ \matriz{\Lambda} = diag(\lambda_1, \dots, \lambda_N) \ , \ \matriz{U}\cdot\matriz{U}^* = I
\]

 onde, claro, $\matriz{\Lambda}$ é matriz diagonal e $\matriz{U}$ é matriz unitária. Em geral, o conjunto de matrizes degeneradas tem medida nula e não é uma preocupação.  Um cuidado deve ser tomado. A correspondência $\matriz{H} \implies (\matriz{U} \ U(N), \matriz{\Lambda})$ não é injetora, podemos tomar $\matriz{U}_1 \matriz{\Lambda} \matriz{U}_1^{-1} = \matriz{U}_2 \matriz{\Lambda} \matriz{U}_2^{-1}$ se $\matriz{U}_1^{-1} \matriz{U}_2 = diag(e^i\phi_1, \dots, e^i\phi_N)$ para qualquer escolha de fases $(\phi_1, \dots, \phi_N)$. Para restringir nosso problema e tornar a função injetiva será necessário considerar as matrizes unitárias ao espaço de coset  $U(N) / U(1) \times \dots \times U(1)$.  Outra restrição necessária é ordenar os autovalores, ou seja, $\lambda_1 < \dots < \lambda_n$, isso deverá introduzir uma constante de normalização $N!$ à expressão. Podemos assim reescrever a medida $d\mu(\matriz{H})$ em função dos autovalores. Nesse subespaço escreveríamos
 
\[
	p(M_{1,1}, \dots, M_{N,N}) \prod_{i<j} dM_{i,j} = p(\lambda_1, \dots, \lambda_N, \hat{U}) dU \prod_{i=1}^N d\lambda_{i}
\]

Onde, é claro, sendo $J(\hat{M} \rightarrow {\lambda_i, U})$ o jacobiano da transformação

\[
		p(M_{1,1}, \dots, M_{N,N}) J(\hat{M} \rightarrow {\lambda_i, U})= p(\lambda_1, \dots, \lambda_N, \hat{U})
\]

Neste caso, podemos expressar o jacobiano como um determinante de Vandermonde

\begin{equation}
	J(\hat{M} \rightarrow \{\lambda_i, U\}) = \prod_{j>k} (\lambda_j - \lambda_{k})^\beta
\end{equation}

Onde $\beta > 0$ e depende da entradas da matriz. Se quisermos expressar a medida somente em termos dos autovalores poderíamos integrar no espaço dos autovetores. Isso nem sempre é simples ou possível. Por simplicidade temos ocultado a dependência das entradas que deveriam ser expressas $M_{i,j}(\lambda, U)$.  Para efeitos deste trabalho tomaremos \textit{ensembles} ortogonalmente invariantes, ou seja, tais que $M_{i,j}(\lambda)$. Basta então, definir o volume do espaço dos autovetores que nos dará uma constante na expressão.

\begin{equation}
	p(\hat{M}) dM =  \frac{1}{Z_N} p(\lambda_1, \dots, \lambda_N) \prod_{j>k} (\lambda_j - \lambda_{k})^\beta
\end{equation}


Notemos um ponto importante. Ao restringir o espaço das matrizes para o espaço das hermitianas introduzimos o determinante de Vandermonde. Este desempenha importante papel na caracterização da medida, note que agora, realizações com autovalores próximos são improváveis. Isso se expressa como uma repulsão de autovalores distintos quando introduzimos uma dinâmica.
	
	\section{Ensembles Gaussianos}
	
	Especial interesse é disposto em uma classe específica de matrizes aleatórias, denominada \textit{Gaussian Ensemble} (abreviadame, em qualquer de suas três formas tradicionais:  \textit{Gaussian Orthogonal Ensemble} (abreviadamente GOE), \textit{Gaussian Unitary Ensemble} (GUE) e \textit{Gaussian Symplectic Ensemble} (GSE). As três formas se distinguem essencialmente pelo tipo de matrizes consideradas, a saber simétricas, unitárias ou unitárias auto-duais, com seus grupos de simetria associados, matrizes ortogonais, unitárias ou unitárias-simpléticas, respectivamente. Estes tipos de matrizes são especialmente interessantes pois possuem uma propriedade única: a distribuição conjunta das entradas é invariante pela ação do seu grupo de simetria e, simultaneamente, possuem entradas independentes, neste caso Gaussianas no corpo real, complexo ou quaterniônico para o GOE, GUE e GSE, respectivamente. Para seus autovalores autovalores, a distribuição assume a forma

\begin{equation*}
	p(\lambda_1, \ldots, \lambda_N) \prod_{i}^{N}\dd \lambda_i = \frac{1}{Z_N} \ee^{-\beta \sum_{k}\lambda_k^2}\prod_{j<k}|\lambda_k - \lambda_j|^\beta \prod_{i}^{N}\dd \lambda_i , \quad \lambda_1,\cdots, \lambda_N\in R,
\end{equation*}
%
onde $\beta$ tem relação com o tipo de matriz Gaussiana utilizada, tendo valor $1$ para o GOE, $2$ para o GUE e $4$ para o GSE, $Z_N$ é a constante de normalização, também chamada de função de partição, e $\dd\lambda_j$ é a medida de Lebesgue unidimensional. Essa expressão pode ser reescrita de forma mais interessante como uma medida de Gibbs, a saber
%
\begin{equation}\label{eigdist1}
	P(\lambda_1, \ldots, \lambda_N)  = \frac{1}{Z_N} \ee^{-\beta H(\lambda_1,\cdots,\lambda_N)},\quad H(\lambda_1,\cdots,\lambda_N)\deff \sum_{j<k}\log\frac{1}{|\lambda_k - \lambda_j|}+\sum_{k}\lambda_k^2.
\end{equation}
%
Desta forma, fator $\beta$ pode - e deve - ser interpretado como temperatura inversa. Nesta representação, ele é o peso de Boltzmann e nossa distribuição é uma analogia à distribuição do sistema canônico da mecânica estatística. Nesta analogia, $H$ seria o Hamiltoniano do sistema que determina o potencial, energia cinética e da interação entre partículas.

A medida de Gibbs \eqref{eigdist1} admite uma extensão natural
%
\begin{equation}\label{eq:Gibbsgeneral}
	P_V(\lambda_1,\hdots,\lambda_n)\deff \frac{1}{Z^V_N}\ee^{-\beta H_V(\lambda_1,\cdots, \lambda_n)},\quad H_V(\lambda_1,\cdots,\lambda_N)\deff  \sum_{j<k}\log\frac{1}{|\lambda_k - \lambda_j|}+\sum_{k}V(\lambda_k),
\end{equation}
%
onde $V:R\to R$ é uma função suficientemente regular. A escolha $V(x)=x^2$, obviamente, recupera \eqref{eigdist1}. A distribuição \eqref{eq:Gibbsgeneral} também descreve autovalores de modelos de matrizes aleatórias apropriados, mas agora as entradas já não são mais independentes.

	\section{A medida de equilíbrio}
	
	Assim como na física, para resolver nossa distribuição, vamos precisar minimizar a energia livre do nosso ensemble. De alguma forma recuperamos a noção da energia livre de Helmholtz,

\[
F = -\frac{1}{k_b} \ln{(\mathcal{Z}_N)}
\]

Teremos que os estados mais prováveis serão aqueles em que for maximizada a expressão

\begin{equation}\
	\exp{\left[-N^2 \left( \frac{1}{N^2}\sum_{i\neq j}\log{\frac{1}{|\lambda_i - \lambda_j|}} + \frac{1}{N^2} \sum_{i=1}^{N} \tilde{V}(\lambda_i)  \right)\right]}
\end{equation}

Onde identificamos o Hamiltoniano e escrevemos $\exp{(-N^2\mathcal{\tilde{H}}_N(\lambda))}$. Precisamos minimizar o Hamiltoniano do sistema $\mathcal{\tilde{H}}_N(\lambda)$. Vamos introduzir uma função contagem para facilitar o tratamento do conjunto de pontos em $\mathbb{R}$. Definimos

\begin{equation}
	\upsilon_\lambda = \frac{1}{N} \sum_1^N \delta_{\lambda_i}
\end{equation}

De forma que

\begin{equation}
	\mathcal{H}_N(\upsilon_\lambda) = 	\int \int_{x\neq y} \log{\frac{1}{|\lambda_i - \lambda_j|}}  \upsilon_\lambda(x) \upsilon_\lambda(y) dx dy + \frac{1}{N} \int \tilde{V}(x) \upsilon_\lambda(x) dx
	\label{eq::CoulombGas:: hamilton}
\end{equation}

O que acontece quando tratamos do limite termodinâmico? Ou seja, quando $N\to\infty$. Estaremos transicionando da nossa função $\upsilon_\lambda$ para uma densidade $\mu_V(x) dx$, ou seja,

\[
\int f(x) \upsilon_\lambda(x) dx =  \int f(x) \mu_V(x) dx
\]

Precisamos garantir ainda um potencial $V(x) = 
N\tilde{V}(x)$ para trabalharmos a assintótica e mantermos a integrabilidade. Escreve-se

\[
\frac{1}{\mathcal{Z}_N} \prod_{i<j} (\lambda_i - \lambda_j)^\beta \prod_{i=1}^{N} e^{-NV(\lambda_i)} d\lambda = 	\frac{1}{\mathcal{Z}_N}  e^{-N^2 \mathcal{H}_N(\lambda)}
\]

E finalmente poderemos expressar de forma correta

\begin{equation}
	\mathcal{H}_N(\upsilon_\lambda) \to \int \int_{x\neq y} \log{\frac{1}{|\lambda_i - \lambda_j|}}  d\mu_V(dx) \mu_V(dy) + \int V(x) \mu_V(dx) \equiv \epsilon^V(\mu_V)
\end{equation}

Onde $\mu_V(x)$ será medida de probabilidade não aleatória tal que na assintótica,

\[
\mu_V^* = \arg \inf {\epsilon}
\]

sendo o minimizante único da função 'energia' $\epsilon$ convexa e semi continua.

	
	\section{Exemplos}
	
	Tratemos agora de alguns exemplos pertinentes à teoria desenvolvida. AS figuras aqui utilizadas são resultados das simulações realizadas na seção \ref{sec: simul}.

\subsection{Potencial Gaussiano}

Em geral, para os ensembles gaussianos estaremos interessados no potencial quadrático. Salvo uma escala, se define o potencial

\[
	V(x) = x^2.
\]
Para estes ensembles vale o clássico resultado da medida de equilíbrio dada pela Lei do Semi-Círculo de Wigner. Se expressa,
\[
\supp \mu_V = [-\sqrt{2}, \sqrt{2}], \quad \frac{\dd \mu_V}{\dd x}(x) = \frac{1}{\pi} \sqrt{2 - x^2},
\]
e teremos das simulações para os três ensembles e suas distribuições esperadas na Figura \ref{fig: semicircle}.

\begin{figure}[ht!]
	\centering
	\includegraphics[scale=0.45]{Assets/validationArticleAlg}
	\caption{Validação para ensembles clássicos, utilizamos $200000$ passos registrando a cada $500$ a partir da metade dos passos. $\Delta t = 0.1$, $\gamma = 1$, $\alpha = 1.0$.}
	\label{fig: semicircle}
\end{figure}

Podemos considerar agora outros potenciais. Lembre que isso implica que nossas entradas são correlacionadas, os ensembles de matrizes para este caso não são tão diretos quanto para entradas independentes. Consideraremos para os exemplos $\beta = 2$.

\subsection{Potencial Mônico}

Considere o potencial

\[
	V(x) = \frac{t}{2\alpha} x^{2\alpha},
\]
onde $t > 0$ é escala e $\alpha \in \Z$. A medida de equilíbrio para $\alpha = 1$ é o semi-círculo de Wigner podemos validar na figura com a distribuição em vermelho. Sabemos também que o suporte $[-a, a]$ da densidade é dado por

\[
	a = \left( \frac{t}{2} \prod_{j=1}^{\alpha} \frac{2j-1}{2j} \right)^{-\frac{1}{2\alpha}}.
\]

 Podemos observar os resultados das simulações para este potencial e notar o comportamento obtido e a distribuição teórica em vermelho para o semicírculo em \ref{fig: quarticmonic}.


\subsection{Potencial Quártico}

Considere o potencial

\[
	V(x) = \frac{x^4}{4} + t \frac{x^2}{2}.
\]
Aqui observaremos pela primeira vez uma transição de estado. Teremos um ponto crítico em $t=-2$ onde a medida separa em dois intervalos $[-b_t, -a_t]$ e $[a_t, b_t]$ para $t < -2$ e um único intervalo $[-b_t, b_t]$ para $t > -2$. Ou seja:

\begin{itemize}
	\item \(t > -2\)
	\[
	\supp \mu_V = [-b_t, b_t], \frac{\dd \mu_V}{\dd x}(x) = \frac{1}{2\pi} (x^2 + c_t^2) \sqrt{b_t^2 - x^2} 
	\]
	
	com
	
	\[
		c_t^2 \deff\frac{1}{2} b_t^2 + t \deff \frac{1}{3} (2t + \sqrt{t^2 + 12})
	\]
	
	\item \(t < -2\)
	\[
	\supp \mu_V = [-b_t, -a_t] \cup [a_t, b_t], \frac{\dd \mu_V}{\dd x}(x) = \frac{1}{2\pi} |x| \sqrt{(x^2 - a_t^2)(b_t^2 - x^2)} 
	\]
	
	com
	
	\[
	a_t \deff \sqrt{-2-t}, b_t \deff \sqrt{2-t}
	\]
\end{itemize}

Observamos o comportamento obtido em \ref{fig: quarticmonic}. 

\begin{figure}[ht!]
	\centering
	\includegraphics[scale=0.8]{Assets/validationArticleMonicQuarticTheo}
	\caption{Para o potencial quártico, à esquerda, vale $V(x) = \frac{1}{4} x^4 + \frac{t}{2} x^2$. Utilizamos $1000000$ passos registrando a cada $500$ a partir da metade dos passos. Com parâmetros $\Delta t = 0.1$, $\gamma = 10$, $\alpha = 0.1$. Já para o potencial mônico, à direita, vale $V(x) = \frac{t}{2 \alpha} x^{2\alpha}$. Utilizamos $1000000$ passos registrando a cada $500$ a partir da metade dos passos. Com parâmetros $t = 1$, $\Delta t = 0.1$, $\gamma = 10$, $\alpha = 0.1$. Em linha sólida, as distribuições teóricas.}
	\label{fig: quarticmonic}
\end{figure}

	
	\section{Simulações}
	
	\input{text/simulacoes}
	
	{\let\clearpage\relax \chapter{Realizações do período}}
	\label{chp:realizacoes}
	
	\section{Graduação}
	
	Durante o período referente ao relatório o aluno completou as matérias do primeiro semestre e iniciou as matérias do segundo semestre listadas na tabela abaixo.
	
	\hspace{1cm}
	
	\begin{center}
		\begin{tabular}{|c|c|c|c|}
			\hline
			Disciplina & Sigla & Nota & Semestre \\
			\hline
			Mecânica Estatística Avançada & 7600041 & 10.0 & 1 - 2023 \\
			\hline
			Introdução aos Sistemas de Computação & 7600056 & 9.2 & 1 - 2023 \\
			\hline
			Física Estatística Computacional & 7600073 & 9.7 & 1 - 2023 \\
			\hline
			Teoria Espectral de Matrizes & SME0243 & 10.0 & 1 - 2023 \\
			\hline
			Mecânica Quântica & 7600022 & - & 2 - 2023 \\
			\hline
			Física Matemática Avançada & 7600034 & - & 2 - 2023 \\
			\hline
			Noções Básicas de Fabricação Mecânica & 7600134 & - & 2 - 2023 \\
			\hline
			Espaços Métricos & SMA0343 & - & 2 - 2023 \\
			\hline
			Trabalho de Conclusão de Curso & 7600039 & - & 2 - 2023 \\
			\hline
		\end{tabular}
	\end{center}
	
	\section{Pesquisa}
	
	Durando os meses passados no período referente a esse relatório grande parte do esforço foi no estudo da bibliografia e conteúdo de interesse dentro da teoria de matrizes aleatórias. Para isso, permitiu-se exploração ampla de conceitos relacionados e implementação de algoritmos especiais em interesse ao aluno. Todos os resultados computacionais e implementações realizadas podem ser encontrados em \href{https://github.com/Joao-vap/RMT-Code/tree/main}{GitHub - Repositório Geral}. A outra atividade principal realizada foi a implementação do algoritmo descrito em \cite{Chafa__2018}, onde podemos simular algumas medidas de probabilidade relacionadas à ensembles clássicos para validar seu funcionamento. Aqui, como é possível ver em \ref{fig: semicircle}, simulamos a distribuição para a GOE, GUE e GSE. Podemos ver nas imagens a concordância das simulações com o método clássico utilizando autovalores de ensembles de matrizes aleatórias.

	{\let\clearpage\relax \chapter{Plano de atividades}}\label{chp:plano}
	
	O projeto tem como base um planejamento de 12 meses, que se deram início em junho de 2023, até o momento foram realizados 5 meses de projeto.
	
	\section{Atividades Desenvolvidas}
	\label{section:atividadesdesenvolvidas}
	
	A execução do projeto foi dividida nas seguintes etapas:
	
	\begin{enumerate}
		\item \textbf{Revisão da Literatura em RMT, e estudo de teoria básica do GUE}, é necessário fazer vasta revisão de literatura no tema para que o aluno tenha domínio das ferramentas e métodos utilizados para o tratamento de matrizes aleatórias e suas implicações em mecânica estatística. Para isso, durante esse período será realizado o estudo da bibliografia adequada;
		
		\item \textbf{Estudo dos métodos de Simulação}, como mencionado, uma das aplicações importantes da teoria de matrizes aleatórias reside em sua conexão com gases de Coulomb. Em 2018 publicou-se o \cite{Chafa__2018}, artigo que é base para o estudo de métodos de simulação desses gases;
		
		\item \textbf{Implementação dos algoritmos}, implementa-se os métodos descritos no artigo e tenta-se estender seu uso em condições diferentes das utilizadas no artigo, como por exemplo em outros potenciais;
		
		\item \textbf{Redação dos Relatórios Científicos}, quando serão escritos os relatórios exigidos pelas normas da \textit{FAPESP}.
		
	\end{enumerate}
	
	\section{Cronograma}
	
	Com base nas tarefas enumeradas na Seção \ref{section:atividadesdesenvolvidas}, é mostrado na Tabela \ref{tab:cronograma1ano} o cronograma atual de desenvolvimento do projeto. Em especial, os métodos de simulação puderam ser adiantados no desenvolvimento para o mês 4, previamente previsto para o mês 5 e consequentemente as implementações também puderam ser iniciadas.


\begin{table}[ht!]
	\centering
	\begin{tabular}{|c|c|c|c|c|c|c|c|c|c|c|c|c|}
		\hline
		\multirow{2}{*}{{\bf Fases}} & \multicolumn{12}{c|}{{\bf Meses}}
		\\ \cline{2-13}
		& 1 & 2 & 3 & 4 & 5 & 6 & 7 & 8 & 9 & 10 & 11 & 12
		\\ \hline
		{\bf 1. Revisão Literatura RMT} & \checkmark & \checkmark & \checkmark & \checkmark & \checkmark & & & & & & &
		\\ \hline
		{\bf 2. Métodos de Simulação} &  &  &  & \checkmark & \checkmark & x & x & x & & & &
		\\ \hline
		{\bf 3. Implementação algoritmos} & & & & & \checkmark & x & x & x & x & x & x &
		\\ \hline
		{\bf 4. Redação Relatórios} & & & & \checkmark & \checkmark & & & & & & x & x 
		\\ \hline
	\end{tabular}
	\caption{Cronograma das atividades.}
	\label{tab:cronograma1ano}
\end{table}
	
	{\let\clearpage\relax \chapter{Participação em eventos científicos}}\label{chp:particEvento}
	
		O bolsista apresentou em dois eventos no período em que se refere o presente relatório. O Colóquio Brasileiro de Matemática (CBM) e a Semana Integrada da Física de São Carlos (SIFSC). Apenas para o primeiro, realizado no Rio de Janeiro, foi necessário o uso da reserva técnica. Por isso, segue o pôster apresentado na página que se segue. O trabalho é complementar aos estudos assintóticos e de probabilidade realizados nos meses cobertos por este relatório.

\hspace{1cm}

\begin{tabular}{|c|c|c|c|c|c|}
	\hline
	Evento &  Sede & Data & Modalidade & Apresentação & Reserva Técnica \\
	\hline
	CBM & IMPA & 24-28/07/23 & Presencial & Pôster - Oral & Sim \\
	\hline
	SIFSC & IFSC-USP & 21-25/08/23 & Presencial & Pôster - Oral & Não \\
	\hline
\end{tabular}

\hspace{1cm}

O trabalho apresentado no CBM foi apresentado oralmente por meio de pôsteres no evento científico Colóquio Brasileiro de Matemática ocorrido de 24-28 de Julho no IMPA, Rio de Janeiro. Foram utilizadas duas diárias da reserva para a participação do colóquio.


\includepdf[pages={1}]{Assets/posterwhite.pdf}
	
	%%-----
	%% Referências bibliográficas
	%%-----
	\addcontentsline{toc}{chapter}{\bibname}
	\bibliographystyle{abntex2-num}
	\bibliography{bibliografia}
	
	%%-----
	%% Fim do documento
	%%-----
	
	%\appendix
	%\chapter{Implementação Algoritmo}
	
	%\inputminted[
	%frame=lines,
	%framesep=2mm,
	%baselinestretch=1.2,
	%bgcolor=white,
	%fontsize=\footnotesize,
	%linenos
	%]{FORTRAN}{Assets/HKMC.f}
	
\end{document}