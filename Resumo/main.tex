\documentclass[11pt,oneside,a4paper]{report}
%%%%%%%%%%%%%% General %%%%%%%%%%%%%%%%%%%%%%%

\usepackage[utf8]{inputenc}

%language
\usepackage[portuguese]{babel}

%page layout
\usepackage[a4paper,width=150mm,top=25mm,bottom=25mm]{geometry}

% footers and headers
\usepackage{fancyhdr}
\pagestyle{fancy}

%citations
\usepackage{cite}

%%%%%%%%%%%%%%%% graphicx %%%%%%%%%%%%%%%%%%%%%

\usepackage{graphicx}

% include graphics path
\graphicspath{ {images} }

%%%%%%%%%%%%%%%%%%%%%%%%%%%%%%%%%%%%%%%%%%%%%%%%
%%%%%%%%%%%%%%%% mathematics %%%%%%%%%%%%%%%%%%%%%

\usepackage{mathtools, amsmath}

\DeclareMathOperator{\Tr}{Tr}

%%%%%%%%%%%%%%%%%%%%%%%%%%%%%%%%%%%%%%%%%%%%%%%%
%%%%%%%%%%%%%%%% notation %%%%%%%%%%%%%%%%%%%%%%

\newcommand{\matriz}[1]{\hat#1}

\newcommand{\many}[2]{$#1_1, #1_2, \dots, #1_#2$}

\newcommand{\cmany}[3]{$#1_1 #3 #1_2 #3 \dots #3 #1_#2$}


% Title Page
\title{
	{No estudo de Matrizes Aleatórias}\\
}
\author{PIMENTA, J. V. A.}


\begin{document}
\maketitle

\begin{abstract}
	Resumo dos estudos de Iniciação Científica em Matrizes Aleatórias e Simulação de Gases de Coulomb.
\end{abstract}

\tableofcontents

\chapter{Introdução}
\input{chapters/introduction}

\section{Física e Mecânica Estatística}
Estaremos lidando o tempo todo no nosso estudo com funções partições, de alguma forma, compartilhada com a física. Entenderemos um pouco mais sobre os desenvolvimentos dos ensembles na termodinâmica.

Em termos gerais os ensembles são sistemas aos quais se impõe vínculos arbitrários. Tanto matematicamente, quanto fisicamente. Da entropia de Shannon podemos deduzir as funções partições apenas adicionando relações de vínculo com multiplicadores de Lagrange. Dos sistemas físicos, usaremos um banho que possa atuar como vínculo.

\subsection{Entropia de Shannon}
Podemos fazer algo um pouco mais matemático. Definiremos a entropia de Shannon, que dá origem às expressões entrópicas gerais para qualquer ensemble.

\[
\mathcal{S} = -k_b \sum_{i} p_i \log{p_i}
\]

\subsubsection{Um vínculo}

Se impormos um vínculo do tipo

\[
\sum_{i} p_i = 1
\]

Podemos usar os multiplicadores de Lagrange para realizar a maximização da nossa função. Pela definição, escrevemos:

\[
S({p_i}, \lambda) \equiv k_b \sum_{i=1}^{N} p_i \log{(p_i)} - \lambda \left( \sum_{i=1}^{N}p_i -1 \right) 
\]

Com diferencial

\[
\delta S = - k_b \sum_{i=1}^{N} \left( p_i \log{p_i} + \frac{p_i}{p_i} \delta p_i \right) - \lambda \sum_{i=1}^{N}\delta p_i
\]

Resolveremos agora o sistema

\[
\begin{cases}
	\delta S = 0 \\
	\sum_{i} p_i = 1
\end{cases}
\]

ou seja,

\[
\begin{cases}
	-k_b (\log{p_i} + 1) - \lambda \\
	\sum_{i} p_i = 1
\end{cases}
\]

Note que $p_i = cte \ \forall i$. Teremos então $p_i = \frac{1}{N}$. A entropia neste caso será 

\[
S({p_i}^*) = - k_b \sum_{i=1}^{N} \left( \frac{1}{N} \log{\frac{1}{N}} \right) = k_b \log{N}
\]

\subsubsection{Dois vínculos}

Faremos agora a implicação de um novo vínculo

\[
\sum_{\sigma} p_\sigma E_\sigma = U
\]

Desenvolvendo a equação

\[
S({p_i}, \lambda_1, \lambda_2) \equiv k_b \sum_{i=1}^{N} p_i \log{(p_i)} - \lambda_1 \left( \sum_{i=1}^{N}p_i -1 \right)  - \lambda_2 \left( \sum_\sigma  p_\sigma E_\sigma - U \right) 
\]

Chegaremos no sistema


\[
\begin{cases}
	-k_b (\log{p_i} + 1) - \lambda_1 - \lambda_2 E_\sigma  = 0\\
	\sum_{\sigma} p_i = 1 \\
	\sum_{\sigma} p_\sigma E_\sigma = U
\end{cases}
\]

E finalmente, da primeira equação tiramos

\[
p_\sigma = e^{A E_\sigma + B} = M e^{-\beta E_\sigma}
\]

De forma que podemos reescrever

\[
1 = \sum_\sigma M e^{-\beta E_\sigma} = M \sum_\sigma e^{-\beta E_\sigma}
\]

Onde nomearemos $M = \frac{1}{\sum_{\sigma}e^{-\beta E_\sigma}} = \frac{1}{\mathcal{Z}}$ \textbf{função partição}. Falta apenas definir $\beta$ em

\[
p_\sigma = \frac{e^{-\beta E_\sigma}}{Z}
\]

Para $\beta$ podemos aplicar o último vínculo (da temperatura térmica)

\[
U = \sum_\sigma p_\sigma E_\sigma = \sum_\sigma \left( \frac{1}{Z} e^{-\beta E_\sigma} \right) E_\sigma = \frac{1}{Z} \sum_\sigma E_\sigma e^{-\beta E_\sigma}
\]

\[
-\frac{1}{Z} \frac{\partial }{ \partial \beta} \left( \sum_\sigma e^{-\beta E_\sigma} \right) = -\frac{1}{\mathcal{Z}} \frac{\partial Z}{\partial \beta} = - \frac{\partial (\log{Z})}{\partial \beta}
\]

De alguma forma esta equação transcendental nos define $\beta$.

\subsubsection{Três vínculos}

Introduziremos um terceiro novo vínculo

\[
	\sum_\sigma p_\sigma N_\sigma = N
\]

Da mesma forma, teremos que desenvolver a equação

\[
S({p_i}, \lambda_1, \lambda_2, \lambda_3) \equiv k_b \sum_{i=1}^{N} p_i \log{(p_i)} - \lambda_1 \left( \sum_{i=1}^{N}p_i -1 \right)  - \lambda_2 \left( \sum_\sigma  p_\sigma E_\sigma - U \right) - \lambda_3 \left( \sum_\sigma  p_\sigma N_\sigma - N \right) 
\]

Do sistema associado aos vínculos de Lagrange

\[
\begin{cases}
	-k_b (\log{p_i} + 1) - \lambda_1 - \lambda_2 E_\sigma  - \lambda_3 N_\sigma = 0\\
	\sum_{\sigma} p_i = 1 \\
	\sum_{\sigma} p_\sigma E_\sigma = U \\
	\sum_{\sigma} p_\sigma N_\sigma = N
\end{cases}
\]

Da primeira equação tiramos

\[
p_\sigma = e^{A E_\sigma + B N_\sigma + C} = M e^{-\beta E_\sigma +\beta \mu N_\sigma}
\]

De forma que podemos reescrever

\[
1 = \sum_\sigma M e^{-\beta E_\sigma +\beta \mu N_\sigma} = M \sum_\sigma e^{-\beta E_\sigma +\beta \mu N_\sigma}
\]

Novamente nomearemos $Z$ em 

\[
	M = \frac{1}{\sum_\sigma e^{-\beta E_\sigma +\beta \mu N_\sigma}} = \frac{1}{Z}
\]

a função partição. As outras duas equações nos definem $\beta$ e $\mu$.

\subsection{O ensemble Micro-Canônico}
O ensemble Micro-Canônico devera ser o mais simples que desenvolveremos. Para este caso a energia é constante e todo microestados devem ser igualmente prováveis pela hipótese ergótica.

\begin{center}
	\begin{tikzpicture}
		\fill[green!10!white] (0,0) rectangle (2,3);
		\draw (0,0) rectangle (2,3);
		\node at (1,2) {$\mathcal{S}$};
		\node at (1,0.5) {$E$};
	\end{tikzpicture}
\end{center}

Sabemos então que vamos querer minimizar a entropia usual com

\begin{equation}
	U(S,V,N)  
\end{equation}

Com 

\[
	dU = TdS - PdV + \mu N
\]

e, especialmente

\[
\rho_E = \sum_{\sigma} \rho_\sigma = \Omega(E) \rho_\sigma
\]

Ou ainda, se $\Omega(E)$ é a quantidade de microestados,

\[
	 \rho_\sigma = \frac{1}{\Omega(E)}
\]

De forma que nossa função partição será

\[
	Z = \sum_\sigma \frac{1}{\Omega(E)} = 1
\]

E em relação a energia

\[
	Z = \sum_E \frac{1}{\Omega(E)} \Omega(E) = \exp{\left\lbrace \beta T k_b \log{\frac{1}{\Omega(E)}} \right\rbrace } \Omega(E) 
\]

Finalmente

\[
 \log{(Z)} = 0 = -\frac{S}{k_b} + \log(\Omega(E))
\]

Ou melhor

\begin{equation}
	S = k_b \log{(\Omega(E))}
\end{equation}

\subsection{O ensemble Canônico}
Quando tratamos destes sistemas no ensemble canônico a energia interna não mais será minimizada no nosso sistema termalizado. Introduzimos uma nova grandeza chamada Energia Livre de Helmholtz ($F$), definida como a transformada de Legendre (discutida no Apêndice \ref{apdx: legendre}) da energia interna em relação à entropia, ou seja

\begin{equation}
	U(S,V,N) \mapsto F(T, V, N)
\end{equation}

ou ainda mais especificamente

\[
F(T, V, N) = U(S(T,V,N), V, N) - T S(T,V,N) 
\]

Note ainda as diferenciais

\begin{align*}
	& dU = TdS - PdV + \mu dN \\
	& dF = -SdT - PdV + \mu dN
\end{align*}

Para o desenvolvimento da função partição estamos atentos aos requisitos do ensemble canônico. Um sistema termalizado por um banho térmico. Nosso sistema completo pode ser representado

\begin{center}
	\begin{tikzpicture}
		\fill[blue!10!white] (0,0) rectangle (8,3);
		\draw (0,0) rectangle (8,3);
		\fill[green!10!white] (0,0) rectangle (2,3);
		\draw (0,0) rectangle (2,3);
		\node at (1,2) {$\mathcal{S}$};
		\node at (1,0.5) {$E$};
		\node at (5,2) {$\mathcal{S}_r$};
		\node at (7.5,0.5) {$T$};
		\node at (3,0.5) {$E_t - E$};
	\end{tikzpicture}
\end{center}

Considere que os sistemas estão em contato térmico e o sistema completo, junto com o banho é um sistema isolado de energia $E_t$. Sabemos que a probabilidade de uma energia no nosso sistema termalizado ($\rho_E$) é expressa:

\[
\rho_E = \sum_{\sigma} \rho_\sigma = \Omega(E)\rho_\sigma
\]

Onde note, $\rho_\sigma$ é a probabilidade de, dada uma temperatura, um microestado possível. Note por outro lado que podemos expressar:

\[
\rho_E = \frac{\Omega(E) \Omega_r (E_t - E)}{\sum_{i} \Omega(E_i) \Omega_r(E_t-E)}
\]

Note que isso é nada mais do que dizer que os microestados são equiprováveis e basta uma contagem (normalizada) para definir a probabilidade. Neste sentido podemos também escrever:

\[
\rho_\sigma \propto \Omega_r(E_t - E)
\]

Isso é, quanto mais formas o reservatório possa se organizar para determinada energia, mais provável é o microestado associado à energia de $\mathcal{S}$.

\begin{align*}
	\rho_\sigma & \propto e^{\beta T K_b \log{(\Omega_r(E_t - E))}} \\
	& \propto e^{\beta T \left( \mathcal{S}_r(E_t) - \frac{E}{T}\right) } \\
	& \propto e^{-\beta E} e^{\beta T \mathcal{S}_r(E_t)} \\
	& \propto e^{-\beta E}
\end{align*}

Onde fizemos a expansão de $k_b \log{(\Omega_r(E_t - E))}$ (entropia) em Taylor (apesar de que, em realidade, apenas $\frac{E}{E_t}$ é pequeno, não necessariamente $E$) e obtivemos

\[
\mathcal{S}_r(E_t - E) \approx \mathcal{S}_r(E_t) + \left( \frac{\partial \mathcal{S}_r}{\partial E}\right)_{E=E_t} (-E)
\]

Onde 

\[
\left( \frac{\partial \mathcal{S}_r}{\partial E} \right)_{E=E_t} = \frac{1}{T}
\]

Um sistema termalizado vai querer minimizar essa nova grandeza da energia livre de Helmholtz. Em todo caso iniciaremos com a expressão já deduzida da equação de partição

\[
\mathcal{Z} = \sum_{\sigma} e^{-\beta E_\sigma}
\]

Que pode ser reescrito em termos de uma soma na energia

\begin{align*}
	& \mathcal{Z} = \sum_{E} e^{-\beta E} \Omega(E) \\
	& \mathcal{Z} = \sum_{E} e^{-\beta T K_b \log{(\Omega(E))}} \Omega(E)
\end{align*}

Podemos argumentar que $ K_b \log{(\Omega(E))} = S(E)$ e usaremos o logaritmo de somas assintóticas (discutido no Apêndice \ref{apdx: somaassin}) para terminar o desenvolvimento. Note

\begin{align*}
	& \log{(\mathcal{Z})} \approx \log{\left( \max_x{[e^{-\beta(E - T S(E))}]} \right)} \\
	& \log{(\mathcal{Z})} \approx \log{\left(e^{-\beta \min_E{((E - T S(E)))}} \right)}
\end{align*}

Onde podemos reconhecer pela transformada de Legendre o termo referente à Energia livre de Helmholtz. Tendo assim

\[
\log{(\mathcal{Z})} \approx -\beta F
\]	

\begin{equation}
	F = - K_b T \log{(\mathcal{Z})}
\end{equation}


\subsection{O ensemble Grão Canônico}
Supomos agora nosso sistema novamente controlado por um banho térmico. Desta vez permitiremos a troca de temperatura e partículas. Definiremos uma energia livre tal que $U(S,V,N) \mapsto \Phi(T,V,\mu)$. Chamaremos esta energia livre de Grão Potencial ou Potencial de Landau. Como sempre, definiremos o potencial como uma transformada de Legendre sob a Energia

\begin{equation}
	\Phi = U - TS - \mu N
\end{equation}

\[
	\Phi(T,V,\mu) = U(S(T,V,\mu), V, N(T,V,\mu)) - TS(T,V,\mu) - \mu N(T,V,\mu)
\]

e é claro, na diferencial

\[
	d\Phi = dU - Tds - SdT - \mu dN - Nd\mu
\]

Onde lembramos que $d\Phi = TdS - PdV + \mu dN$ de forma que

\[
	d\Phi = -SdT - PdV - Nd\mu
\]

Ou seja,

\[
	S = - \left| \frac{\partial \Phi}{\partial T} \right|_{V,N} 
\]
\[
	N = - \left| \frac{\partial \Phi}{\partial \mu} \right|_{T,V} 
\]

Tratamos do seguinte sistema,

\begin{center}
	\begin{tikzpicture}
		\fill[blue!10!white] (0,0) rectangle (8,3);
		\fill[green!10!white] (0,0) rectangle (2,3);
		\draw (0,0) rectangle (8,3);
		\draw[dashed] (2,0) -- (2,3);
		\node at (1,2) {$\mathcal{S}$};
		\node at (1,0.5) {$E,N$};
		\node at (5,2) {$\mathcal{S}_r$};
		\node at (7.5,0.5) {$T,\mu$};
		\node at (3,0.7) {$E_t - E$};
		\node at (3,0.3) {$N_t - N$};
	\end{tikzpicture}
\end{center}

Sabemos afirmar que $p_{\sigma} \propto \Omega_R (E_t - E, N_t - N)$, ou seja, cada microestado do nosso sistema é proporcional às formas que o banho pode se arranjar dado $(E,N)$. Ou seja

\begin{align*}
	p_{\sigma} & = \exp{ \lbrace \log{[\Omega_R(E_t - E, N_t - N)]} } \rbrace \\
	& \propto exp{\left\lbrace \frac{1}{k_b} \left[  k_b \log{(\Omega_R(E_T, N_T))} - E \frac{\partial}{\partial E'} (k_b log{\Omega_R(E', N')})\middle|_{N'=N_T}^{E'=E_T} -  N \frac{\partial}{\partial N'} (k_b log{\Omega_R(E', N')})\middle|_{N'=N_T}^{E'=E_T} \right]  \right\rbrace} \\
	& \propto \exp{\left\lbrace  \frac{1}{k_b} \left[  - E \frac{\partial}{\partial E'} (k_b log{\Omega_R(E', N')})\middle|_{N'=N_T}^{E'=E_T} -  N \frac{\partial}{\partial N'} (k_b log{\Omega_R(E', N')})\middle|_{N'=N_T}^{E'=E_T} \right] \right\rbrace}
\end{align*}

Onde retiramos o termo que não depende do nosso sistema de interesse e será constante, agregando ele na proporcionalidade. Agora faremos uso da ideia da entropia como $S = k_b \log{(\Omega_R(E',N'))}$ para escrever a relação acima em termos da temperatura e potencial químico. Usando das derivadas parciais da entropia em $dS = \frac{1}{T} dU + \frac{P}{T} dV - \frac{\mu}{T} dN$,

\begin{align*}
	p_{\sigma} & \propto \exp{\left\lbrace  \frac{1}{k_b} \left[  -E \frac{1}{T} - N \frac{\mu}{T} \right]  \right\rbrace} \\
	& \propto e^{-\beta E + \beta \mu N}
\end{align*}

e

\begin{equation}
	p_\sigma = \frac{1}{\Xi} e^{-\beta E + \beta \mu N}
\end{equation}

Onde 

\begin{equation}
	\Xi = \sum_\sigma e^{-\beta E + \beta \mu N}
\end{equation}

O log da nossa função partição deve resultar em uma expressão de energia livre.

\begin{align*}
	\log{\Xi} & = \log{ \left( \sum_\sigma e^{-\beta E_\sigma + \beta \mu N_\sigma} \right) } \\
	& = \log{ \left( \sum_{E,N}  \Omega(E,N) e^{-\beta E + \beta \mu N} \right) } \\
	& = \log{ \left( \sum_{E,N}  \exp{\left\lbrace \frac{k_b}{k_b} \log{\Omega(E,N)}\right\rbrace}  \exp{\left\lbrace (-\beta E + \beta N \mu) \right\rbrace} \right) } \\
	& = \log{ \left( \sum_{E,N}  \exp{\left\lbrace \frac{T}{k_b T} S - \beta E + \beta N \mu \right\rbrace} \right) } \\
	& = \log{ \left( \sum_{E,N}  e^{ \beta(E - TS - N \mu)} \right) } \\
\end{align*}

Para a aproximação desta expressão vamos considerar que o logaritmo de uma somatória pode ser aproximada por seu termo máximo,

\[
	\approx \log{ e^{ \beta(E^* - TS(E^*,N^*) - N^* \mu)}} = \beta(E^* - TS(E^*,N^*) - N^* \mu)
\]

Ou seja,

\[
\log{\Xi} = \beta(E^* - TS(E^*,N^*) - N^* \mu) = -\beta \Phi
\]

e finalmente,

\begin{equation}
	\Phi = -k_b T \log{\Xi}
\end{equation}

\chapter{Procura-se autovalores}
\input{chapters/distribuicoesS}

\section{Porquê exponencial?}
\input{sections/introduction/exponential}

\section{Independência ou Morte}
Consideremos matrizes com entradas independentes. Qual a função densidade de probabilidade (F.P.D.) da matriz simétrica $\matriz{H_s}$? Devemos fazer separadamente a diagonal da seção triangular que formos usar e teremos

\[
\rho((\matriz{H_s})_{11}, \dots, (\matriz{H_s})_{NN}) = \prod_{i=1}^{N} \left[ \frac{e^{\frac{-(H_s)^2_{ii}}{2}}}{2\pi} \right] \prod_{i<j} \left[ \frac{e^{-(H_s)^2_{ij}}}{\sqrt{\pi}} \right]
\]

Podemos também definir a distribuição para os autovalores de uma matriz Gaussiana de dimensão $N$ como \footnote{Esse resultado é não óbvio e deve ser discutido em breve.}

\begin{equation}
	\rho(x_1, \dots, x_N) = \frac{1}{\mathcal{Z_{N, \beta}}} e^{-\frac{1}{2} \sum_{i=1}^{N} x_i^2} \prod_{j<k} | x_j - x_k |^{\beta}
\end{equation}

onde a constante de normalização é dada por

\[
\mathcal{Z_{N, \beta}} = (2\pi)^\frac{N}{2} \prod_{j=1}^{N} \frac{\Gamma(1+ j\frac{\beta}{2})}{\Gamma(1+ \frac{\beta}{2})}
\]

Para ressaltar um pouco do jargão, $\beta$ é denominado \textit{Dyson Index} que em suma se refere à "dimensão" das suas entradas na matriz. $1$ para GOE, $2$ para GUE e $4$ para GSE.

Algumas observações sobre essa expressão são interessantes. Note que o fator exponencial deve matar qualquer chance de uma matriz com autovalor alto. Ao mesmo tempo o fator de dependência deve matar qualquer configuração com autovalores muito próximos entre si. Existe um efeito de repelência entre autovalores na expressão.

\section{Uma medida à Hermitiana}
Consideremos inicialmente um espaço de matrizes com $N^2$ entradas independentes, sejam elas reais, complexas ou simpléticas. Se tivéssemos interesse de expressar a medida desse espaço poderíamos escrever

\begin{equation}
	p(\hat{M}) dM = p(M_{1,1}, \dots, M_{N,N}) \prod_{i,j=1}^{N} dM_{i,j}
\end{equation}

Contido neste espaço temos um espaço de maior interesse correspondente ao espaço das matrizes \textit{simétricas} ou \textit{hermitianas}. A escolha do subespaço está relacionada com o fato  de que essas matrizes são diagonalizáveis. Podemos escrever nossa matriz $\matriz{H}$ como 

\[
\matriz{H} = \matriz{U} \matriz{\Lambda} \matriz{U}^{-1} \ , \ \matriz{\Lambda} = diag(\lambda_1, \dots, \lambda_N) \ , \ \matriz{U}\cdot\matriz{U}^* = I
\]

 onde, claro, $\matriz{\Lambda}$ é matriz diagonal e $\matriz{U}$ é matriz unitária. Em geral, o conjunto de matrizes degeneradas tem medida nula e não é uma preocupação.  Um cuidado deve ser tomado. A correspondência $\matriz{H} \implies (\matriz{U} \ U(N), \matriz{\Lambda})$ não é injetora, podemos tomar $\matriz{U}_1 \matriz{\Lambda} \matriz{U}_1^{-1} = \matriz{U}_2 \matriz{\Lambda} \matriz{U}_2^{-1}$ se $\matriz{U}_1^{-1} \matriz{U}_2 = diag(e^i\phi_1, \dots, e^i\phi_N)$ para qualquer escolha de fases $(\phi_1, \dots, \phi_N)$. Para restringir nosso problema e tornar a função injetiva será necessário considerar as matrizes unitárias ao espaço de coset  $U(N) / U(1) \times \dots \times U(1)$.  Outra restrição necessária é ordenar os autovalores, ou seja, $\lambda_1 < \dots < \lambda_n$, isso deverá introduzir uma constante de normalização $N!$ à expressão. Podemos assim reescrever a medida $d\mu(\matriz{H})$ em função dos autovalores. Nesse subespaço escreveríamos
 
\[
	p(M_{1,1}, \dots, M_{N,N}) \prod_{i<j} dM_{i,j} = p(\lambda_1, \dots, \lambda_N, \hat{U}) dU \prod_{i=1}^N d\lambda_{i}
\]

Onde, é claro, sendo $J(\hat{M} \rightarrow {\lambda_i, U})$ o jacobiano da transformação

\[
		p(M_{1,1}, \dots, M_{N,N}) J(\hat{M} \rightarrow {\lambda_i, U})= p(\lambda_1, \dots, \lambda_N, \hat{U})
\]

Neste caso, podemos expressar o jacobiano como um determinante de Vandermonde

\begin{equation}
	J(\hat{M} \rightarrow \{\lambda_i, U\}) = \prod_{j>k} (\lambda_j - \lambda_{k})^\beta
\end{equation}

Onde $\beta > 0$ e depende da entradas da matriz. Se quisermos expressar a medida somente em termos dos autovalores poderíamos integrar no espaço dos autovetores. Isso nem sempre é simples ou possível. Por simplicidade temos ocultado a dependência das entradas que deveriam ser expressas $M_{i,j}(\lambda, U)$.  Para efeitos deste trabalho tomaremos \textit{ensembles} ortogonalmente invariantes, ou seja, tais que $M_{i,j}(\lambda)$. Basta então, definir o volume do espaço dos autovetores que nos dará uma constante na expressão.

\begin{equation}
	p(\hat{M}) dM =  \frac{1}{Z_N} p(\lambda_1, \dots, \lambda_N) \prod_{j>k} (\lambda_j - \lambda_{k})^\beta
\end{equation}


Notemos um ponto importante. Ao restringir o espaço das matrizes para o espaço das hermitianas introduzimos o determinante de Vandermonde. Este desempenha importante papel na caracterização da medida, note que agora, realizações com autovalores próximos são improváveis. Isso se expressa como uma repulsão de autovalores distintos quando introduzimos uma dinâmica.

\chapter{Movimento Browniano}
\input{chapters/brownian}

\section{Processo Pontual}
Um processo pontual pode ser interpretado como um conjunto aleatório de pontos ou como a medida de probabilidade associada a esse conjunto. Um processo pontual possui $n$ pontos se

\[
\mathcal{P}(\# X=n) = 1
\]

Onde $X$ é um conjunto enumerável de $\mathcal{X}$ ($\mathbb{R}$, $\mathbb{Z}$ ou um subconjunto destes). O conjunto de todas configurações possíveis é denominado $Conf(\mathcal{X})$. Se $P(x_1, \dots, x_n)$ é uma função de densidade de probabilidade em $\mathbb{R}^n$ invariante por permutações

\[
	\mathbb{R}^n \rightarrow Conf(\mathbb{R})
\]
\[
	(x_1, \dots, x_n) \mapsto X = {x_1, \dots, x_n}
\]

define naturalmente um processo pontual com $n$ pontos.

\subsection{Poisson \& fries}

Tome $\set{N(t)}$ o número de eventos no intervalo de tempo $]0,t]$. $\set{N(t)}$  é um processo estocástico (de contagem). Se o processo de Poisson possui $\lambda > 0$, para um elemento fiox do espaço amostral a variável aleatória $N$ assume valor $k$ no tempo $t$ com probabilidade

\begin{equation}
	\mathcal{P}[N(t) = k] = \frac{(\lambda t)^k e^{-\lambda t}}{k!}
\end{equation}

Onde $\lambda$ é o número esperado de chagadas por unidade de tempo. Agora, como um processo pontual, a probabilidade de n eventos no intervalo $]a,b]$ é

\[
	\mathcal{P}(N]a,b] = n) =  \frac{(\lambda (b-a))^n e^{-\lambda (b-a)}}{n!}
\]

Podemos usar a independência de cada evento de Poisson em intervalos disjuntos para escrever

\[
	\mathcal{P}(N]a_1,b_1] = n_1, \dots, N]a_k,b_k] = n_k) = \prod_{i=1}^{k} \frac{(\lambda (b_i-a_i))^n_i e^{-\lambda (b_i-a_i)}}{n_i!}
\]

Podemos escrever para uma função $f$ mensurável em $\mathbb{R}$

\[
	\sum_{x_i \in \mathcal{X}} f(x_i) = \int f(x) dN(x)
\]

Onde a medida dN é

\[
	dN(x) = \sum_{x_i \in \mathcal{X}} \delta_{x_i} (x)
\]

Onde notamos que podemos interpretar tanto quanto uma soma de um processo pontual quanto uma medida de probabilidade.


\subsection{Funcão Correlação}

Definimos uma variável aleatória $N$ anteriormente. Naturalmente, poderíamos estar interessados em sua esperança. Mais especificamente, podemos procurar a esperança do número de pontos de uma ocnfiguração dentro de um intervalo $A \subset \mathbb{R}$.

\[
	A \mapsto \mathbb{E}[N(A)] = \mathbb{E}[\#(A \cap X)]	
\]

Que pode ser interpretada como uma medida com densidade $p_1$

\begin{equation}
	\mathbb{E}[\#(A \cap X)] = \int_{A} p_1(x) dx
	\label{eq: p1}
\end{equation}

A equação \ref{eq: p1} é conhecida como \textit{função de correlação de 1 ponto}. Em grosso modo, $p_1(x)$ é a probabilidade de haver um ponto da configuração entre $x$ e $x+dx$. Seja um configuração simples $X = \set{x_1, \dots, x_n}$ e intervalos disjuntos na reta \many{A}{n},

\[
	\int_A \dots \int_A \rho_n(x_1, \dots, x_n) dx_1, \dots, dx_n = \mathbb{E} \left( \prod_{j=1}^{k} \# (X \cap A_j) \right) 
\]

é o número esperado de n-uplas (\many{x}{n}) $\in A_1 \times \dots \times A_n$ tais que $x_i \in A_i, i=1,\dots,n$. Seja $\mathbb{P}(x_1, \dots, x_n)$ uma densidade de probabilidade em $\mathbb{R}^n$, então o processo pontual de n pontos gerado possui funções de correlação dadas por

\[
	p_k(x_1, \dots, x_k) = \frac{n!}{(n-k)!} \int \dots \int \mathbb{P}(x_1, \dots, x_n) dx_{k+1}\dots dx_n
\]


\subsection{Pontual Determinantal}

Um processo pontual vai ser chamado determinantal se dada uma função de correlação $\rho_n$, existe um núcleo $K(x, y)$ conhecido como núcleo de correlação tal que

\begin{equation}
	\rho_n(x_1, \dots, x_n) = det[K(x_i, x_j)]_{i,j=1}^{n}
	\label{eq: pontualdet}
\end{equation}

onde

\[
[K(x_i,x_j)]_{i,j=1}^{n} = 
\begin{bmatrix}
	K(x_1, x_1) & K(x_1, x_2) & \dots & K(x_n, x_n) \\
	K(x_2, x_1) & K(x_2, x_2) & \dots & K(x_n, x_n) \\
	\vdots & \vdots & \ddots & \vdots \\
	K(x_n, x_1) & K(x_n, x_2) & \dots & K(x_n, x_n)
\end{bmatrix}
\]

Nos casos de baixa dimensão

\[
p_1(x_1) = K(x_1,x_1), \quad \quad p_2(x_1,x_2) =
\begin{vmatrix}
	K(x_1, x_1) & K(x_1, x_2) \\
	K(x_2, x_1) & K(x_2, x_2)
\end{vmatrix}
\]

Para que o núcleo satisfaça a equação \ref{eq: pontualdet} enunciaremos um resultado de interesse.

\begin{theorem}
	Seja K um núcleo tal que
	\begin{enumerate}[label=(\alph*)]
		\item $\int K(x,x) = n \in \mathbb{N}$,
		\item Para todo \many{x}{n} $\in \mathbb{R}$, o determinante é não negativo
		\item K possui a propriedade de \textbf{núcleo reprodutor}, isto é;
		\[
		K(x,y) = \int_{-\infty}^{\infty} K(x,s) K(s,y) ds
		\]
		Então
		\[
		P(x_1,\dots, x_n) = \frac{1}{n!} det[K(x_i, x_j)]_{i,j=1}^{n}
		\]
		será uma densidade de probabilidade em $\mathbb{R}$ cujo processo de n pontos associado é determinantal.
	\end{enumerate}
\end{theorem}



\section{Emsemble Biortogonal}
\input{sections/brownian/biortogonal}

\section{Karlin-McGregor}
\subsection{O teorema}

Exploraremos os caminhos não cruzantes providos por processos de Markov. Considere uma partícula de movendo com uma regra qualquer, vamos descrever esse movimento de forma que denotaremos $p_t(a;x)$ a densidade de probabilidade de transição; isto é, a chance uma partícula em $a$ ir para $x$ em um próximo momento. Um teorema clássico enuncia a probabilidade de um certo número de caminhos não se intersectarem passado um tempo $t$.

O teorema diz: Considere $X_1(t), \dots, X_n(t)$ cópias independentes de um processo forte de Markov com caminhos condicionados tais que

\[
	X_j(0) = a_j
\] 

onde \cmany{a}{n}{<} são valores dados. Notamos novamente $p_t(x, y)$ ser a densidade do processo de transição. Vamos definir regiões \many{E}{n} onde $E$'s vizinhos não se intersectam. Temos

\[
	\int_{E_1} \dots \int_{E_n} \det{[p_t(a_i, x_j)]^{n}_{i,j=1}} dx_1 \dots dx_n
\]

vai ser a probabilidade de que os caminhos não tenham se intersectados no intervalo de tempo $[0, t]$ e $X_j(t)$ nos intervalos correspondentes. A demonstração está em \cite{ArnoLectureNotes}. Note que temos

\[
\int_{E_1} \dots \int_{E_n} \det{[p_t(a_i, x_j)]^{n}_{i,j=1}}  dx_1 \dots dx_n
\]

\begin{align}
	& = \int_{E_1} \dots \int_{E_n}
	\begin{vmatrix}
		p_t(a_1, x_1) 	& p_t(a_2, x_1) 	 & \dots	& p_t(a_{n-1}, x_1) 	& p_t(a_n, x_1) \\
		p_t(a_1, x_2) 	& p_t(a_2, x_2) 	 & \dots 	&  p_t(a_{n-1}, x_2)				&  p_t(a_n, x_2) \\
		\vdots 			& \vdots 			 & \vdots 	& \vdots 				& \vdots \\
		p_t(a_1, x_{n-1}) & p_t(a_2, x_{n-1})& \dots 	&  	p_t(a_{n-1}, x_{n-1})	& p_t(a_n, x_{n-1}) \\
		p_t(a_1, x_n) 	& p_t(a_2, x_n) 	 & \dots  	& p_t(a_{n-1}, x_n) 	& p_t(a_n, x_n)
	\end{vmatrix} dx_1 \dots dx_n \\ 
	& = \sum_{\sigma}sgn(\sigma) \prod_{j=1}^{n} p_t(a_j, E_{\sigma(j)}) \\
	& = \sum_{\sigma} sgn(\sigma) \mathcal{P}(A_\sigma)
	\label{eq: detInd}
\end{align}

Onde denotamos

\[
	p_t(a_j, E_{\sigma(j)}) = \int_{E_j} p_t(a_i, x_j) dx_j
\]

$\sigma$ é uma permutação de ${1, \dots, n}$ e $A_{\sigma}$ é o evento que $X_j(t) \in E_{\sigma(j)}$ para todo $j$. Os caminhos devem ser independentes para \ref{eq: detInd}.

De alguma forma o determinada permuta os caminhos em todas ordens possíveis e calcula a probabilidade de todos se manterem nos intervalos adequados. Um exemplo de baixas dimensões pode mostrar que


\begin{align}
	&
	\begin{vmatrix}
		p_t(a_1, x_1) & p_t(a_2, x_1) & p_t(a_3, x_1) \\
		p_t(a_1, x_2) & p_t(a_2, x_2) & p_t(a_3, x_2) \\
		p_t(a_1, x_3) & p_t(a_2, x_3) & p_t(a_3, x_3)
	\end{vmatrix} =\\
	&
	+ p_t(a_1, x_1) p_t(a_2, x_2) p_t(a_3, x_3)  \\
	&
	+ p_t(a_2, x_1) p_t(a_3, x_2) p_t(a_1, x_3) \\
	&
	+ p_t(a_3, x_1) p_t(a_1, x_2) p_t(a_2, x_3) \\
	& 
	- p_t(a_3, x_1) p_t(a_2, x_2) p_t(a_1, x_3) \\
	&
	-  p_t(a_2, x_1) p_t(a_1, x_2) p_t(a_3, x_3) \\
	&
	- p_t(a_1, x_1) p_t(a_3, x_2) p_t(a_2, x_3)
\end{align}

Logo

\begin{align}
	\int_{E_1} \dots \int_{E_n} \det{[p_t(a_i, x_j)]^{n}_{i,j=1}}  dx_1 \dots dx_n  = 
	&
	+ p_t(a_1, E_1) p_t(a_2, E_2) p_t(a_3, E_3)  \\
	&
	+ p_t(a_2, E_1) p_t(a_3, E_2) p_t(a_1, E_3) \\
	&
	+ p_t(a_3, E_1) p_t(a_1, E_2) p_t(a_2, E_3) \\
	& 
	- p_t(a_3, E_1) p_t(a_2, E_2) p_t(a_1, E_3) \\
	&
	-  p_t(a_2, E_1) p_t(a_1, E_2) p_t(a_3, E_3) \\
	&
	- p_t(a_1, E_1) p_t(a_3, E_2) p_t(a_2, E_3)
\end{align}

Onde somamos os casos onde as partículas se matém ordenadas e subtraímos os casos onde elas se cruzam.


\subsection{Consequências}

Considere $n$ cópias do processo de Markov condicionado para começar em $t=0$ nas determinadas posições \cmany{a}{n}{<}. Se condicionarmos estes processos para não intersectar no intervalo $[0,t]$, o teorema vai nos dizer que os caminhos em um tempo $t$ vão ter uma densidade de probabilidade conjunta

\[
	\frac{1}{\mathcal{Z}_n} \det{[p_t(a_i, x_j)]^{n}_{i,j=1}}
\]

Mas este não pode ser considerado um processo pontual determinado. Não é expresso por um produto de determinantes. Isso pode ser ajeitado se considerarmos um tempo $T > t$ no nosso processo. Tomaremos \many{b}{n} posições finais e condicionaremos os caminhos a não intersectar no intervalo $[0, T]$ com $X_j(0) = a_j$ e $X_j(T) = b_j$ para todos. É possível mostrar que a distribuição conjunta deles será

\[
	\frac{1}{\mathcal{Z}_n'} \det{[p_t(a_i, x_j)]^{n}_{i,j=1}} \det{[p_{T-t}(x_i, b_j)]^{n}_{i,j=1}}
\]

Que será biortogonal com as funções

\[
	f_j = p_t(a_j, x) \ ; \ g_j = p_{T-t}(x, b_j)
\]

E nosso caso de interesse é quando $a_j \rightarrow a$ e $b_j \rightarrow b$. Note que usando as duas funções podemos forçar que o movimento browniano se inicie em um ponto e encerre em outro determinado. Em uma, reverteremos o tempo e, nos limites $0$ e $T$, forçaremos que apenas uma das funções seja predominante de forma que a posição inicial de cada uma prevaleça. Podemos impor a posição inicial e final do movimento. No caso browniano teremos

\[
	p_t(a, x) = \frac{1}{\sqrt{2\pi t}} e^{-\frac{(x-a)^2}{2t}}
\]

No caso dos limites de $a$ e $b$ ficamos com

\[
	f_j = F_{j-1}(x)e^{-\frac{(x-a)^2}{2t}} \ ; \ g_j = G_{j-1}(x)e^{-\frac{(x-b)^2}{2(T-t)}}
\]

onde $F$ e $G$ são polinômios em $x$  de grau $j-1$. Este processo podemos escalar e transladar para uma versão do GUE $n \times n$.





\section{Simulações Gerais}
\input{sections/brownian/simulacoes}

\chapter{Coulombolas! Gases Aleatórios}
\input{chapters/coulumbGas}

\section{Hamiltonianozinho}
\input{sections/coulombGas/hamilton}

\chapter{Simulações e o Artigo}
Nos referenciaremos aqui aos desenvolvimento do artigo citado em \cite{Chafa__2018}. Vamos compilar algum desenvolvimento teórico necessário e explicitar os resultados e métodos do artigo.

\section{Introdução Teórica}
\input{sections/article/intro}

\section{Simulando Coulomb-Log Gases}
\input{sections/article/simulating}

\bibliography{references}{}
\bibliographystyle{plain}

\appendix
\chapter{Transformação Legendre}
\label{apdx: legendre}
Transformações de Legendre tem ampla aplicação e interpretação. Faremos uma breve dissertação de duas possíveis visualizações do processo. 

\section{Tangentes}

Uma primeira interpretação do processo representa uma mudança clara de variável a partir de tangentes de uma função de concavidade bem definida. Faremos 

\[
	y = f(x) \mapsto \psi(\rho) = f(x(\rho)) - \rho x(\rho)
\]

onde 

\[
	\rho \equiv \frac{y - f(x(\rho))}{x - x(\rho)} = \frac{\psi(\rho) - f(x(\rho))}{0 - x(\rho)}
\]


Que podemos visualizar

\begin{center}
	\begin{tikzpicture}[declare function={f(\x)=0.6*\x*\x - 5*\x + 13;}]
		\draw[help lines, color=gray!30, dashed] (-1,-0.9) grid (5.9,4.9);
		\draw[->,thick] (-1,0)--(6,0) node[right]{$x$};
		\draw[->,thick] (0,-1)--(0,5) node[above]{$y$};
		
		\draw[scale=0.5, domain=1:7.3, smooth, variable=\x, blue] plot ({\x}, {f(\x)});
		\draw (0,0.5) -- (2.5,1.4);
		\filldraw[black] (0,0.5) circle (2pt) node[anchor=east]{$\psi(\rho)$};
		\draw[gray, dash dot] (0,1.3) -- (2.3,1.3);
		\filldraw[black] (0,1.3) circle (2pt) node[anchor=east]{$f(x(\rho))$};
		\draw[gray, dash dot] (2.3,0) -- (2.3,1.3);
		\filldraw[black] (2.3,0) circle (2pt) node[anchor=north]{$x(\rho)$};
	\end{tikzpicture}
\end{center}

De forma que a transformada é criada pela projeção desta tangente no eixo x=0.


\section{Otimização}
Outra forma de visualizar a transformada é por um problema mais conveniente de otimização. Definiremos

\[
	\psi(\rho) = \max_x[\rho x - f(x)]
\]

Onde teremos

\begin{center}
	\begin{tikzpicture}[declare function={f(\x)=0.2*\x*\x;}]
		\draw[help lines, color=gray!30, dashed] (-1,-0.9) grid (5.0,4);
		\draw[->,thick] (-1,0)--(5,0) node[right]{$x$};
		\draw[->,thick] (0,-1)--(0,4) node[above]{$y$};
		
		\draw[scale=0.5, domain=0:6, smooth, variable=\x, blue] plot ({\x}, {f(\x)});
		\draw (0,0) -- (4,3);
		\node[above left] at (3.0,0.5) {\footnotesize $y=f(x)$};
		\node[below right] at (0.3,1.5) {\footnotesize $y=\rho x$};
	\end{tikzpicture}
\end{center}

Se definirmos 

\[
	g(x) = \rho x - f(x)
\]

Minimizaremos $g(x)$ e teremos a condição $f'(x^*) = \rho$. Ou seja

\[
	f'(x(\rho))=\rho \implies \max_x[\rho x - f(x)] = \rho x(\rho) - f(x(\rho))
\]

Que é equivalente à dizer

\[
\psi(\rho) = \min_x[f(x) - \rho x] = f(x(\rho)) - \rho x(\rho)
\]


\chapter{Soma Assintóticas}
\label{apdx: somaassin}
Existe uma aproximação a se fazer no logaritmo de somas assintóticas que pode ser de interesse. Assuma

\[
	S = \sum_{i=1}^{M} a_i
\]  

Se $a_i(N) \sim e^{\phi_i N}$ ou $\log{(a_i)} \approx \phi_i N$, podemos afirmar

\[
	\log{(S)} \sim \log{(a_{max})}
\]

Para demonstrar isso, notamos

\[
	a_{max} < S < M_{a_max}
\]
\[
	\frac{\log{(a_{max})}}{N} < \frac{\log{S}}{N} < \frac{\log{(a_{max})}}{N} + \frac{\log{M}}{N}
\]

Ou seja, desde que $\frac{\log{M}}{N} \rightarrow 0$. Isto ocorrerá desde que $M$ seja sub-exponencial. Contudo \textbf{NOTE} que dizer

\[
	\log{(n!)} \sim n \log{n} - n
\]

Não implica que

\[
	n! \sim ! \left( \frac{n}{e}\right)^n 
\]

Em algumas situações é possível afirmar contudo

\[
	n! \sim \left( \frac{n}{e}\right)^n \sqrt{2\pi n}
\]

\chapter{Det Vandermonde}
\label{apdx: vandermonde}
A formulação nos diz

\begin{equation}
	|V| = \prod_{1\leq i < j \leq n} (\alpha_j - \alpha_i)
\end{equation}

Note que podemos iniciar com uma matriz $n\cdot n$. Seja $c_i$ a coluna $i$, multiplicamos a coluna $c_i$ por $-\alpha_1$ e somamos com a coluna $c_{i+1}$

\[
V = 
\begin{bmatrix}
	1 & \alpha_1 & \alpha_1^2 & \dots & \alpha_1^{n-1} \\
	1 & \alpha_2 & \alpha_2^2 & \dots & \alpha_2^{n-1} \\
	1 & \alpha_3 & \alpha_3^3 & \dots & \alpha_3^{n-1} \\
	\vdots & \vdots & \vdots & \ddots & \vdots \\
	1 & \alpha_n & \alpha_n^2 & \dots & \alpha_n^{n-1}
\end{bmatrix}
=
\begin{bmatrix}
	1 & 0 & 0 & \dots & 0 \\
	1 & \alpha_2 - \alpha_1 & \alpha_2(\alpha_2 - \alpha_1) & \dots & \alpha_2^{n-2}(\alpha_2 - \alpha_1) \\
	1 & \alpha_3 - \alpha_1& \alpha_3(\alpha_3 - \alpha_1) & \dots & \alpha_3^{n-2} (\alpha_3 - \alpha_1)\\
	\vdots & \vdots & \vdots & \ddots & \vdots \\
	1 & \alpha_n- \alpha_1 & \alpha_n(\alpha_n - \alpha_1) & \dots & \alpha_n^{n-2}(\alpha_n - \alpha_1)
\end{bmatrix}
\]

Utilizando do Teorema de Laplace, o determinante vai ser definido simplesmente por

\[
|V| =
	\begin{vmatrix}
		\alpha_2 - \alpha_1 & \alpha_2(\alpha_2 - \alpha_1) & \dots & \alpha_2^{n-2}(\alpha_2 - \alpha_1) \\
		\alpha_3 - \alpha_1& \alpha_3(\alpha_3 - \alpha_1) & \dots & \alpha_3^{n-2} (\alpha_3 - \alpha_1)\\
		\vdots & \vdots & \ddots & \vdots \\
		\alpha_n- \alpha_1 & \alpha_n(\alpha_n - \alpha_1) & \dots & \alpha_n^{n-2}(\alpha_n - \alpha_1)
	\end{vmatrix}
\]

De onde é claro, podemos fatorar os coeficientes e ter

\[
|V| = (\alpha_2 - \alpha_1)(\alpha_3 - \alpha_1)\dots(\alpha_n - \alpha_1)
\begin{vmatrix}
	1 & \alpha_2 & \alpha_2^2 & \dots & \alpha_2^{n-2} \\
	1 & \alpha_3 & \alpha_3^2 & \dots & \alpha_3^{n-2} \\
	1 & \alpha_4 & \alpha_4^3 & \dots & \alpha_4^{n-2} \\
	\vdots & \vdots & \vdots & \ddots & \vdots \\
	1 & \alpha_n & \alpha_n^2 & \dots & \alpha_n^{n-2}
\end{vmatrix}
\]

E assim por diante.


\end{document}          
