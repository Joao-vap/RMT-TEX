A formulação nos diz

\begin{equation}
	|V| = \prod_{1\leq i < j \leq n} (\alpha_j - \alpha_i)
\end{equation}

Note que podemos iniciar com uma matriz $n\cdot n$. Seja $c_i$ a coluna $i$, multiplicamos a coluna $c_i$ por $-\alpha_1$ e somamos com a coluna $c_{i+1}$

\[
V = 
\begin{bmatrix}
	1 & \alpha_1 & \alpha_1^2 & \dots & \alpha_1^{n-1} \\
	1 & \alpha_2 & \alpha_2^2 & \dots & \alpha_2^{n-1} \\
	1 & \alpha_3 & \alpha_3^3 & \dots & \alpha_3^{n-1} \\
	\vdots & \vdots & \vdots & \ddots & \vdots \\
	1 & \alpha_n & \alpha_n^2 & \dots & \alpha_n^{n-1}
\end{bmatrix}
=
\begin{bmatrix}
	1 & 0 & 0 & \dots & 0 \\
	1 & \alpha_2 - \alpha_1 & \alpha_2(\alpha_2 - \alpha_1) & \dots & \alpha_2^{n-2}(\alpha_2 - \alpha_1) \\
	1 & \alpha_3 - \alpha_1& \alpha_3(\alpha_3 - \alpha_1) & \dots & \alpha_3^{n-2} (\alpha_3 - \alpha_1)\\
	\vdots & \vdots & \vdots & \ddots & \vdots \\
	1 & \alpha_n- \alpha_1 & \alpha_n(\alpha_n - \alpha_1) & \dots & \alpha_n^{n-2}(\alpha_n - \alpha_1)
\end{bmatrix}
\]

Utilizando do Teorema de Laplace, o determinante vai ser definido simplesmente por

\[
|V| =
	\begin{vmatrix}
		\alpha_2 - \alpha_1 & \alpha_2(\alpha_2 - \alpha_1) & \dots & \alpha_2^{n-2}(\alpha_2 - \alpha_1) \\
		\alpha_3 - \alpha_1& \alpha_3(\alpha_3 - \alpha_1) & \dots & \alpha_3^{n-2} (\alpha_3 - \alpha_1)\\
		\vdots & \vdots & \ddots & \vdots \\
		\alpha_n- \alpha_1 & \alpha_n(\alpha_n - \alpha_1) & \dots & \alpha_n^{n-2}(\alpha_n - \alpha_1)
	\end{vmatrix}
\]

De onde é claro, podemos fatorar os coeficientes e ter

\[
|V| = (\alpha_2 - \alpha_1)(\alpha_3 - \alpha_1)\dots(\alpha_n - \alpha_1)
\begin{vmatrix}
	1 & \alpha_2 & \alpha_2^2 & \dots & \alpha_2^{n-2} \\
	1 & \alpha_3 & \alpha_3^2 & \dots & \alpha_3^{n-2} \\
	1 & \alpha_4 & \alpha_4^3 & \dots & \alpha_4^{n-2} \\
	\vdots & \vdots & \vdots & \ddots & \vdots \\
	1 & \alpha_n & \alpha_n^2 & \dots & \alpha_n^{n-2}
\end{vmatrix}
\]

E assim por diante.
