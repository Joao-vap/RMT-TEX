Existe uma aproximação a se fazer no logaritmo de somas assintóticas que pode ser de interesse. Assuma

\[
	S = \sum_{i=1}^{M} a_i
\]  

Se $a_i(N) \sim e^{\phi_i N}$ ou $\log{(a_i)} \approx \phi_i N$, podemos afirmar

\[
	\log{(S)} \sim \log{(a_{max})}
\]

Para demonstrar isso, notamos

\[
	a_{max} < S < M_{a_max}
\]
\[
	\frac{\log{(a_{max})}}{N} < \frac{\log{S}}{N} < \frac{\log{(a_{max})}}{N} + \frac{\log{M}}{N}
\]

Ou seja, desde que $\frac{\log{M}}{N} \rightarrow 0$. Isto ocorrerá desde que $M$ seja sub-exponencial. Contudo \textbf{NOTE} que dizer

\[
	\log{(n!)} \sim n \log{n} - n
\]

Não implica que

\[
	n! \sim ! \left( \frac{n}{e}\right)^n 
\]

Em algumas situações é possível afirmar contudo

\[
	n! \sim \left( \frac{n}{e}\right)^n \sqrt{2\pi n}
\]