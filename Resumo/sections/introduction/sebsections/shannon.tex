Podemos fazer algo um pouco mais matemático. Definiremos a entropia de Shannon, que dá origem às expressões entrópicas gerais para qualquer ensemble.

\[
\mathcal{S} = -k_b \sum_{i} p_i \log{p_i}
\]

\subsubsection{Um vínculo}

Se impormos um vínculo do tipo

\[
\sum_{i} p_i = 1
\]

Podemos usar os multiplicadores de Lagrange para realizar a maximização da nossa função. Pela definição, escrevemos:

\[
S({p_i}, \lambda) \equiv k_b \sum_{i=1}^{N} p_i \log{(p_i)} - \lambda \left( \sum_{i=1}^{N}p_i -1 \right) 
\]

Com diferencial

\[
\delta S = - k_b \sum_{i=1}^{N} \left( p_i \log{p_i} + \frac{p_i}{p_i} \delta p_i \right) - \lambda \sum_{i=1}^{N}\delta p_i
\]

Resolveremos agora o sistema

\[
\begin{cases}
	\delta S = 0 \\
	\sum_{i} p_i = 1
\end{cases}
\]

ou seja,

\[
\begin{cases}
	-k_b (\log{p_i} + 1) - \lambda \\
	\sum_{i} p_i = 1
\end{cases}
\]

Note que $p_i = cte \ \forall i$. Teremos então $p_i = \frac{1}{N}$. A entropia neste caso será 

\[
S({p_i}^*) = - k_b \sum_{i=1}^{N} \left( \frac{1}{N} \log{\frac{1}{N}} \right) = k_b \log{N}
\]

\subsubsection{Dois vínculos}

Faremos agora a implicação de um novo vínculo

\[
\sum_{\sigma} p_\sigma E_\sigma = U
\]

Desenvolvendo a equação

\[
S({p_i}, \lambda_1, \lambda_2) \equiv k_b \sum_{i=1}^{N} p_i \log{(p_i)} - \lambda_1 \left( \sum_{i=1}^{N}p_i -1 \right)  - \lambda_2 \left( \sum_\sigma  p_\sigma E_\sigma - U \right) 
\]

Chegaremos no sistema


\[
\begin{cases}
	-k_b (\log{p_i} + 1) - \lambda_1 - \lambda_2 E_\sigma  = 0\\
	\sum_{\sigma} p_i = 1 \\
	\sum_{\sigma} p_\sigma E_\sigma = U
\end{cases}
\]

E finalmente, da primeira equação tiramos

\[
p_\sigma = e^{A E_\sigma + B} = M e^{-\beta E_\sigma}
\]

De forma que podemos reescrever

\[
1 = \sum_\sigma M e^{-\beta E_\sigma} = M \sum_\sigma e^{-\beta E_\sigma}
\]

Onde nomearemos $M = \frac{1}{\sum_{\sigma}e^{-\beta E_\sigma}} = \frac{1}{\mathcal{Z}}$ \textbf{função partição}. Falta apenas definir $\beta$ em

\[
p_\sigma = \frac{e^{-\beta E_\sigma}}{Z}
\]

Para $\beta$ podemos aplicar o último vínculo (da temperatura térmica)

\[
U = \sum_\sigma p_\sigma E_\sigma = \sum_\sigma \left( \frac{1}{Z} e^{-\beta E_\sigma} \right) E_\sigma = \frac{1}{Z} \sum_\sigma E_\sigma e^{-\beta E_\sigma}
\]

\[
-\frac{1}{Z} \frac{\partial }{ \partial \beta} \left( \sum_\sigma e^{-\beta E_\sigma} \right) = -\frac{1}{\mathcal{Z}} \frac{\partial Z}{\partial \beta} = - \frac{\partial (\log{Z})}{\partial \beta}
\]

De alguma forma esta equação transcendental nos define $\beta$.

\subsubsection{Três vínculos}

Introduziremos um terceiro novo vínculo

\[
	\sum_\sigma p_\sigma N_\sigma = N
\]

Da mesma forma, teremos que desenvolver a equação

\[
S({p_i}, \lambda_1, \lambda_2, \lambda_3) \equiv k_b \sum_{i=1}^{N} p_i \log{(p_i)} - \lambda_1 \left( \sum_{i=1}^{N}p_i -1 \right)  - \lambda_2 \left( \sum_\sigma  p_\sigma E_\sigma - U \right) - \lambda_3 \left( \sum_\sigma  p_\sigma N_\sigma - N \right) 
\]

Do sistema associado aos vínculos de Lagrange

\[
\begin{cases}
	-k_b (\log{p_i} + 1) - \lambda_1 - \lambda_2 E_\sigma  - \lambda_3 N_\sigma = 0\\
	\sum_{\sigma} p_i = 1 \\
	\sum_{\sigma} p_\sigma E_\sigma = U \\
	\sum_{\sigma} p_\sigma N_\sigma = N
\end{cases}
\]

Da primeira equação tiramos

\[
p_\sigma = e^{A E_\sigma + B N_\sigma + C} = M e^{-\beta E_\sigma +\beta \mu N_\sigma}
\]

De forma que podemos reescrever

\[
1 = \sum_\sigma M e^{-\beta E_\sigma +\beta \mu N_\sigma} = M \sum_\sigma e^{-\beta E_\sigma +\beta \mu N_\sigma}
\]

Novamente nomearemos $Z$ em 

\[
	M = \frac{1}{\sum_\sigma e^{-\beta E_\sigma +\beta \mu N_\sigma}} = \frac{1}{Z}
\]

a função partição. As outras duas equações nos definem $\beta$ e $\mu$.